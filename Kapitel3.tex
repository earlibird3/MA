% !TEX root = MA.tex
\section{Analye der Erfolgsfaktoren des Bühler Projektmanagements}\label{drei}
Basierend auf der vorangehende Literaturrecherche kann mit Leichtigkeit der Eindruck gewonnen werden, dass eine weitere Analyse der Faktoren, welche den Erfolg von Projekten beeinflussen, keine neuen Erkenntnisse liefern würde. Für eine Vielzahl der berücksichtigen Studien gründen weitergehende statistische Analysen auf einer anfänglichen Wertung von Erfolgsattributen durch Personen, die in der entsprechenden Industrie oder Projektmanagement tätig sind. Die erforschten Faktoren wurden vorgängig jeweils aus früheren Studien extrahiert. Der Zusammenhang zwischen dem Erfolg von Projekten, der entweder als binäre Ausprägung oder als indexiertes Kriterium repräsentiert war, und den unabhängigen Erfolgsattributen wurde mittels der entsprechender Regressionsanalysen erforscht. Die gewählte Methode der Likert-Skala führte jeweils dazu, dass die erforderlichen Annahmen für eine Regression oder Faktorenanalyse gegeben waren. Da sich die Ergebnisse zu einem Teil überschneiden, kann postuliert werden, dass unabhängig vom gewählten Performancekriterium ein gewisser Konsens bezüglich der Erfolgsfaktoren existiert. Diese Aussage ist mit Vorsicht zu geniessen, da die Studien nicht eins zu eins miteinander verglichen werden können, weshalb sie als Annahme formuliert wurde. Ausserdem lässt sich aus den betrachteten Forschungsberichten schliessen, dass keine unternehmensspezifische Daten respektive unternehmensbezogene Daten zu den Erfolgsattributen erhoben wurden beziehungsweise für die Analyse herangezogen wurde.\newline
Die nachfolgende Analyse wird sich aufgrund der Daten und Fixierung eines bestimmten Performancekriteriums sowie dem Fokus dieser Arbeit von bisherigen Analysen unterscheiden. 
\subsection{Daten und statistische Methoden}
In Kapitel \ref{zweizwei} wurde beschrieben, dass die Bühler AG die zu evaluierenden Faktoren auf Basis bisheriger Erfahrungen aus einer finanziellen Perspektive identifiziert und auch neue Indikatoren geschaffen hat. Der Rahmen für die Datenerhebung bildete die Datenverfügbarkeit des BPM-Cockpits und des SAP. In der Folge wurde für die Datenextraktion ein eigene Query geschaffen, die sämtliche Faktoren pro Projekt abbildet. Trotz mehrfacher Validierung der Daten, konnte nach Erreichung des Fertigstellungstermin keine vollständige Korrektheit vor allem einiger berechneter Indikatoren gewährleistet werden. Die Stichprobe enthält alle abgeschlossenen Projekte im Zeitraum zwischen 2013 und 2015. Das eindeutige Abgrenzungskriterium bilden hierbei der Projektstatus und das Datum des MS11 Start und Übergabe. Zuerst wurden alle Projekte mit einem MS11-Datum zwischen dem 1.1.2013 und dem 31.12.2015 eingegrenzt. Der Projektstatus stellte sicher, dass das Projekt auch aus finanzieller Sicht als abgeschlossen betrachtet werden kann, da gewisse Projekte MS11 bereits erreicht haben können, aber fehlende Rechnungen noch zu verbuchen sind.
\newline
Das ursprüngliche beabsichtigte Analysemodell orientierte sich an den bisherigen Studien und hätte sich aus einer Faktorenanalyse zur Reduktion der Anzahl Faktoren mit anschliessender Regressionsanalyse zur Bestimmung der Abhängigkeiten, zusammengesetzt. Allerdings konnte bei der Prüfung der Modellvoraussetzungen die zwingende Linearitätsannahme zwischen der abhängigen und den unabhängigen Variablen nicht zufriedenstellend erfüllt werden. Selbst eine entsprechende lineare Variablentransformationen hätte die Linearitätsannahme nicht besser erfüllt. Dementsprechend mussten lineare statistische Modelle von den möglichen Analysemethoden ausgeschlossen werden. Aus diesem Grund und der Tatsache, dass Erfolgsfaktoren bereits sehr gut erforscht wurden, hat dazu beigetragen, dass sich die Analyse im Bereich der deskriptiven Statistik bewegt. Darüber hinaus sollen soll mittels explorativer Analysen, ein Teilgebiet der Datenanalyse, Strukturen respektive neue Hypothesen zu möglichen Erfolgsfaktoren formulieren werden. Die Aussagekraft der Ergebnisse wird mit dem Verzicht auf die Anwendung der Inferenzstatistik insofern eingeschränkt, da keine Rückschlüsse auf die Grundgesamtheit (sämtliche Projekte der Bühler AG) gemacht werden können. Allerdings können Aussagen und Vermutungen bezüglich der Stichprobe gemacht werden. Demzufolge kann die nachfolgenden Analyse auch als expost-Analyse betrachtet werden. Ziel dieser expost-Betrachtung aus finanzieller Perspektive ist einerseits einen genauerer Untersuchung des monetären Margenverlusts. Hierbei ist die Ausprägung der Faktoren der nicht erfolgreichen Projekte von zentralem Interesse. Das zweite Ziele ist die Faktoren zu beschreiben, um anschliessend Hypothesen für mögliche Erfolgsfaktoren der Projekte der Bühler AG zu formulieren. Als Basis dienen die Vermutungen und Einschätzungen pro Faktor der Bühler AG, welche zusammen mit den Faktoren ergründet wurden. Diese Vorgehensweise ermöglicht das bisherige Datenmodell zu prüfen und ergänzende Faktoren zu finden. Da diese Analyse die erste ihrer Art für die Bühlerprojekte ist, kann sie zudem wertvolle Hinweise zu Projekten und allenfalls möglichen Problemfeldern liefern.
\newline Wie bereits in Kapitel \ref{zweizwei} erwähnt wurde, bildet das Erfolgskriterium die Abweichung des realisierten vom budgetierten Deckungsbeitrag eines Projekts. Sie stellt die finanzielle Perspektive eines Projekts dar und hat direkten Einfluss auf das Ergebnis eines Geschäftsbereich. Ausserdem hängt die variable Vergütung der Projektmanager und Verkaufsmanager vom realisierten DB1 ab. In erster Linie wurde die Logik Erfolg (über Budget) Fail (unter Budget) angewandt. Allerdings wurde für einige Analysen und Darstellungen die Ampellogik der Kosten des BPM-Cockpits angewandt (s. Kapitel \ref{zweizwei}).
\newline
\newline\textbf{Plausibilität, fehlende Werte und Ausreisser}
\newline	Die Daten wurden vor der Anwendung deskriptiver Statistik auf Plausibilität, fehlende Werte und Ausreisser hin untersucht. Dabei wurde der Fokus darauf gelegt, möglichst viele Datensätze in der Stichprobe zu erhalten. Bei 96 Variablen kann nicht erwartet werden, dass für jedes Projekt alle Werte verfügbar sind. Folglich wurden im Zusammenhang mit fehlenden Werten auch einige Variablen von der Analyse ausgeschlossen. Somit wurde zu Lasten einer Grossen Stichprobe ein gewisser Datenverlust hingenommen. Ausserdem macht es statistisch wenig Sinn, eine Determinante zu evaluieren, wenn für die Hälfte der Datensätze kein interpretierbarer Wert vorhanden ist. Dies würde verzerrte Schlussfolgerungen nach sich ziehen.
\newline Der Plausibilitätstest erfolgte basierend auf der Interpretation des Faktors. Die Stichprobe betrug ursprünglich 1497 Projekte $N = 1497$. Die Anzahl unabhängiger Faktoren belief sich auf 96 $x = 97$, davon wurden 21 aus unterschiedlichen Gründen entfernt. Der ursprüngliche Datensatz enthielt neun Berechnungsspalten, die Bestandteil andere Faktoren waren. Fünf weitere Faktoren wurden entfernt, da sie bereits durch einen andere Determinante im Datensatz enthalten waren oder lediglich der Identifikation des Projektes dienten. Im Anschluss wurde der Plausibilitätstest durchgeführt, der direkten Einfluss auf die Stichprobengrösse hat. Bei der Prüfung stellte sich heraus, dass folgende Anpassungen gemacht werden mussten, damit die Plausibilität der Daten gewährleistet werden konnte. (Tabelle EINFÜGEN).
\newline Nach dem Plausibilitäts betrug $n = 1055$ und die Anzahl Variablen $x = 83$. Im Anschluss an die Plausibilitätsanalyse wurde die Anzahl fehlender Werte pro unabhängiger Variable gemessen. Auf Basis dieser Kalkulation wurde entschieden, einige Variablen von der Analyse systematisch auszuschliessen, da unter deren Berücksichtigung eine erhebliche Anzahl Datensätze verloren gegangen wäre. (Tabelle Einfügen)! Die CostMostnegFCajd-Variablen haben einen weiteren Werte $y = 1'111'111$, der angibt, dass das Projekt nur positive FC-Anpassungen gehabt hat. Dies impliziert, dass der Forecast für die Kosten gesunken sind und somit weniger Kosten erwartet wurden, wobei der Umsatz konstant gehalten wurde. Die HOM-YellowStatus und die HOM-RedStatus haben ebenfalls den Wert $y = 1'111'111$, der in diesem Fall angibt, dass entweder kein Yellow-Status respektive RedStatus gegeben hat. Die Interpretation wäre somit, dass bei den HOMYellow-Status Variablen der Status immer grün war oder zuerst respektive direkt den roten Status hatte. Eine ähnliche Interpretation gilt für HOMRedStatus-Variablen, somit hätte dieses Projekt, den roten Status gar nicht erst erreicht. Da diese Interpretationen valide sind und somit keine fehlende Werte darstellen, wurden sie vor der Plausibilitätsanalyse nicht als NA markiert. Die folgende Tabelle zeigt alle Variablen mit ihrer Anzahl an fehlender Datensätze. Sämtliche Indikatoren mit weniger als 100 fehlender Datensätze wurden in der Analyse mitberücksichtigt. Der restliche Informationsverlust wurde dementsprechend hingenommen. 
\newline
\begin{table}
	\centering
	\caption{Anzahl NA's per variable}
	\begin{tabular} {| l| r | l |}
			\textbf{Variable Code} & \textbf{Anzahl NA's} & \textbf{Action}\\\hline
			AMNo & 98 & von der Analyse ausgeschlossen\\
			CostMostnegFCadj & 126 & von der Analyse ausgeschlossen\\
			CostMostnegFCadjPA & 500 & von der Analyse ausgeschlossen\\
			CostMostnegFCadjIS & 358 & von der Analyse ausgeschlossen\\
			CostFirstadj & 43 &\\
			PrTimeDelayMS2 & 35 &\\
			PrTimeDelayMS5 & 277 & von der Analyse ausgeschlossen\\
			PrTimeDelayMS8 & 15 &\\
			PrTimeDelayMS10 & 37 &\\
			AMAge & 308  & von der Analyse ausgeschlossen\\
			PMAge & 57 & \\		
	\end{tabular}
\end{table}
Nach der Entfernung der fehlenden Datensätze betrug die Stichprobe $n = 900$ und die Anzahl unabhängiger Variablen $ x = 77$. Nachdem 
\newline
Gemäss verschiedener Quellen ist die Identifikation und Eliminierung der Ausreisser ein zentraler Aspekt vor der deskriptiven Analyse, da ansonsten auch die deskriptiven Erhebungsmethoden wenig aussagekräftig bleiben. Der angewendet Ansatz basiert auf dem 'Interquartile Range'. Gemäss diesem Ansatz werden sämtliche Daten von der Analyse ausgeschlossen, welche folgende Werte über- respektive unterschreiten.
\newline\newline
\begin{centering}
		$ all avlues \leq Q1 - 1.5 * IQR$
		\newline
		$ all values \geq Q3 + 1-5 * IQR$
\end{centering}
\newline
\newline
\begin{centering}
		$ all avlues \leq Q1 - 3 * IQR$
		\newline
		$ all values \geq Q3 + 3 * IQR$
\end{centering}
\newline\newline
Obwohl dieser Ansatz als Orientierungshilfe dient, gestalte sich die Entscheidung ob ein Datensatz ein Ausreiser ist, eher schwierig. Denn der Ausreisser ist nur in Relation zu den übrigen Ausprägungen zu bestimmen und folglich vom Datensatz und dem Kontext der Daten abhängige. Beispielsweise ist das Umsatzvolumen in der Bühler AG sehr variable und ab und zu realisiert die Bühler AG auch Projekte, die ein extraordinäres Umsatzvolumen aufweisen. Im Zusammehang mit der Analyse, welche Projekte erfolgreich waren und welche nicht, würde es Sinn machen auch Projekte, welche quasi ein ausserordentliches Umsatzvolumen haben zu analysieren, da dieselben Projektmanagementmethodik zur Anwendung kommt. Deshalb wurde unter dem Gesichtspunkt möglichst viele Datensätze zu erhalten und um die Projektdiversität der Bühler AG zu erfassen sehr sparsam mit der Entfernung von Ausreissern umgegangen. Die vorangehend erwähnte Ansatz wurde als Orientierungshilfe angewandt. Die Nachfolgende Tabelle zeigt alle Variablen, welche um Ausreisser bereinigt wurden an. Der DB1Act entsprechend auf den 3IQR angepasst, da die durchschnittliche Performance des Anlagengeschäft der Bühler AG ca. 28\% beträgt. Die Anpassung dieser Variable hat direkten Einfluss auf DB1BudDev, welche somit nicht mehr zusätzlich bereinigt wurde. Die relativen CostActBud-Variablen für die einzelnen Kostenkategorien wurden jeweils um die höchsten Kostenüberschreitungen relativ zu den nächst höheren Werten bereinigt. Hohe Kostenüberschreitungen in den einzelnen Kostenkategorien treten beispielsweise auf, wenn der Budget-Anteil der entsprechenden Kostenkategorie relativ gross respektive klein ist. Aus diesem Grund ist es schwierig zu identifizieren, ob es sich um einen Ausreisser handelt oder nicht. Es könnte auch ein Fehler in der Budgetierungsphase erfolgt sein, bei der das entsprechende Kostenbudget zu tief eingeschätzt wurde. Allerdings mussten extreme Kostenabweichungen eliminiert werden. Hierbei wurden die maximalen Werte mit den darauffolgende 10 Werten verglichen. Darauf basierend wurde bei allen relativen Kostenvergleichen die höchsen 1-4 Werte eliminiert. Bei der relativen Wichtigkeit eines Projektes für den Geschäftsbereich wurde ein Wert mitunter eliminiert, da der maximale Wert ca. drei mal höher war als der zweithöchste Wert. Dieser Wert hätte die Aussagekraft des Indikators massgebend beeinflusst und zu verzerrten Interpretationen geführt. Nach der Bereinigung der Daten um die Ausreisser beträgt die Stichprobnezahl $N = 883 $ und die Anzahl Variablen $ x = 70$.
\subsection{Finanzieller Impact von nicht erfolgreichen Projekten}
\subsection{Ergebnisse und Interpretation}
\subsection{Kritische Würdigung der Ergebnisse}
\newpage	
	


