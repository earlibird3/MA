% !TEX root = MA.tex
\section{Analye der Erfolgsfaktoren des Bühler Projektmanagements}\label{drei}
Basierend auf der vorangehende Literaturrecherche kann mit Leichtigkeit der Eindruck gewonnen werden, dass eine weitere Analyse der Faktoren, welche den Erfolg von Projekten beeinflussen, keine neuen Erkenntnisse liefern würde. Für eine Vielzahl der berücksichtigen Studien gründen weitergehende statistische Analysen auf einer anfänglichen Wertung von Erfolgsattributen durch Personen, die in der entsprechenden Industrie oder Projektmanagement tätig sind. Die erforschten Faktoren wurden vorgängig jeweils aus früheren Studien extrahiert. Der Zusammenhang zwischen dem Erfolg von Projekten, der entweder als binäre Ausprägung oder als indexiertes Kriterium repräsentiert war, und den unabhängigen Erfolgsattributen wurde mittels der entsprechender Regressionsanalysen erforscht. Die gewählte Methode der Likert-Skala führte jeweils dazu, dass die erforderlichen Annahmen für eine Regression oder Faktorenanalyse gegeben waren. Da sich die Ergebnisse zu einem Teil überschneiden, kann postuliert werden, dass unabhängig vom gewählten Performancekriterium ein gewisser Konsens bezüglich der Erfolgsfaktoren existiert. Diese Aussage ist mit Vorsicht zu geniessen, da die Studien nicht eins zu eins miteinander verglichen werden können, weshalb sie als Annahme formuliert wurde. Ausserdem lässt sich aus den betrachteten Forschungsberichten schliessen, dass keine unternehmensspezifische Daten respektive unternehmensbezogene Daten zu den Erfolgsattributen erhoben wurden beziehungsweise für die Analyse herangezogen wurde.\newline
Die nachfolgende Analyse wird sich aufgrund der Daten und Fixierung eines bestimmten Performancekriteriums sowie dem Fokus dieser Arbeit von bisherigen Analysen unterscheiden. 
\subsection{Daten und statistische Methoden}
In Kapitel \ref{zweizwei} wurde beschrieben, dass die Bühler AG die zu evaluierenden Faktoren auf Basis bisheriger Erfahrungen aus einer finanziellen Perspektive identifiziert und auch neue Indikatoren geschaffen hat. Der Rahmen für die Datenerhebung bildete die Datenverfügbarkeit des BPM-Cockpits und des SAP. In der Folge wurde für die Datenextraktion ein eigene Query geschaffen, die sämtliche Faktoren pro Projekt abbildet. Trotz mehrfacher Validierung der Daten, konnte nach Erreichung des Fertigstellungstermin keine vollständige Korrektheit vor allem einiger berechneter Indikatoren gewährleistet werden. Die Stichprobe enthält alle abgeschlossenen Projekte im Zeitraum zwischen 2013 und 2015. Das eindeutige Abgrenzungskriterium bilden hierbei der Projektstatus und das Datum des MS11 Start und Übergabe. Zuerst wurden alle Projekte mit einem MS11-Datum zwischen dem 1.1.2013 und dem 31.12.2015 eingegrenzt. Der Projektstatus stellte sicher, dass das Projekt auch aus finanzieller Sicht als abgeschlossen betrachtet werden kann, da gewisse Projekte MS11 bereits erreicht haben können, aber fehlende Rechnungen noch zu verbuchen sind.
\newline Das ursprüngliche beabsichtigte Analysemodell orientierte sich an den bisherigen Studien und hätte sich aus einer Faktorenanalyse zur Reduktion der Anzahl Faktoren mit anschliessender Regressionsanalyse zur Bestimmung der Abhängigkeiten, zusammengesetzt. Allerdings konnte bei der Prüfung der Modellvoraussetzungen die zwingende Linearitätsannahme zwischen der abhängigen und den unabhängigen Variablen nicht zufriedenstellend erfüllt werden. Selbst eine entsprechende lineare Variablentransformationen hätte die Linearitätsannahme nicht besser erfüllt. Dementsprechend mussten lineare statistische Modelle von den möglichen Analysemethoden ausgeschlossen werden. Aus diesem Grund und der Tatsache, dass Erfolgsfaktoren bereits sehr gut erforscht wurden, hat dazu beigetragen, dass sich die Analyse im Bereich der deskriptiven Statistik bewegt. Darüber hinaus sollen soll mittels explorativer Analysen, ein Teilgebiet der Datenanalyse, Strukturen respektive neue Hypothesen zu möglichen Erfolgsfaktoren formulieren werden. Die Aussagekraft der Ergebnisse wird mit dem Verzicht auf die Anwendung der Inferenzstatistik insofern eingeschränkt, da keine Rückschlüsse auf die Grundgesamtheit (sämtliche Projekte der Bühler AG) gemacht werden können. Allerdings können Aussagen und Vermutungen bezüglich der Stichprobe gemacht werden. Demzufolge kann die nachfolgenden Analyse auch als expost-Analyse betrachtet werden. Ziel dieser expost-Betrachtung aus finanzieller Perspektive ist einerseits einen genauerer Untersuchung des monetären Margenverlusts. Hierbei ist die Ausprägung der Faktoren der nicht erfolgreichen Projekte von zentralem Interesse. Das zweite Ziele ist die Faktoren zu beschreiben, um anschliessend Hypothesen für mögliche Erfolgsfaktoren der Projekte der Bühler AG zu formulieren. Als Basis dienen die Vermutungen und Einschätzungen pro Faktor der Bühler AG, welche zusammen mit den Faktoren ergründet wurden. Diese Vorgehensweise ermöglicht das bisherige Datenmodell zu prüfen und ergänzende Faktoren zu finden. Da diese Analyse die erste ihrer Art für die Bühlerprojekte ist, kann sie zudem wertvolle Hinweise zu Projekten und allenfalls möglichen Problemfeldern liefern.
\newline Wie bereits in Kapitel \ref{zweizwei} erwähnt wurde, bildet das Erfolgskriterium die Abweichung des realisierten vom budgetierten Deckungsbeitrag eines Projekts. Sie stellt die finanzielle Perspektive eines Projekts dar und hat direkten Einfluss auf das Ergebnis eines Geschäftsbereich. Ausserdem hängt die variable Vergütung der Projektmanager und Verkaufsmanager vom realisierten DB1 ab. In erster Linie wurde die Logik Erfolg (über Budget) Fail (unter Budget) angewandt. Allerdings wurde für einige Analysen und Darstellungen die Ampellogik der Kosten des BPM-Cockpits angewandt (s. Kapitel \ref{zweizwei}).
\newline
\newline\textbf{Rohdaten}
\newline Die Rohdaten setzen sich aus drei verschiedenen Datensätzen zusammen. Im grössten Datensatz sind sämtliche Faktoren enthalten, welche mittels BPM Cockpit erhoben wurden. Dieser Datensatz wurde um Umsatz- und Margenwerten und die Regionenzuordnung eines Projektes ergänzt. Des weiteren wurden Daten zum Alter und der Betriebszugehörigkeit der Projekt- und Verkaufsmanager manuell auf Basis der intern ermittelten Daten hinzugefügt, da dieser Datensatz die Unterscheidung zwischen fehlenden Werten und Nullwerten zuliess. Von diesem Datensatz wurden einerseits Berechnungsvariabeln und anderseits doppelte Variablen sowie fehlerhafte Variabeln entfernt (siehe Anhang). Der Rohdatensatz ist folglich bereits ein korrigierter Datensatz, der einen Stichgrössenumfang von $N = 1471$ und $i = 93$ Indikatoren enthält. Vor der deskriptiven Analyse wurde auf Basis dieses Rohdatensatzes, fehlende Werte, die Datenplausibilität sowie Ausreisser untersucht. Da bei der grossen Anzahl von Variablen, die Komplettheit eines Datensatzes, das heisst, dass alle Faktoren konnten für ein Projekt erhoben werden konnte, schwierig zu erreichen ist, wurde der Fokus darauf gelegt, möglichst viele Datensätze in der Stichprobe zu erhalten. In der nachfolgenden Tabelle sind sämtliche Variablen mit fehlenden Werten inklusiver der Anzahl fehlender Daten aufgelistet. Grundsätzliche wurde sämtliche Indikatoren, welche mehr als $200$ NA's aufweisen von der Analyse ausgeschlossen. AMNo wurde zudem ausgeschlossen, da ausser einem direkten Personenbezug keine weitere Informationen bezüglich der Verkaufsmanager zu erwarten ist, da bereits AMAge2 und AMTen2 ausgeschlossen werden mussten. Einzig könnte eine separate Analyse der Verkaufsmanager gemacht werden, um sie auf ihre Performance und Auslastung hin untersuchen zu können. CostMostnegFCadj wurde ebenso von der Analyse ausgeschlossen, da eine Einzelauswertung auf Kostenträgerebene sinnvoller erscheint. Die Eliminierung der fehlenden Daten führt dazu, dass die Stichprobe von $N = 1471$ auf $N = $ sank und die Variablenanzahl sich von $i = 93$ auf $i = $ einschränkte.
\newline
\begin{table}[h]
	\centering
	\caption{Anzahl NA's per variable}
	\begin{tabular} {| l| l | p{6cm} |}
		\textbf{Variable Code} & \textbf{Anz. NA} & \textbf{Handhabung} \\\hline
		CostMostnegFCadjPA & 749   & von der Analyse ausgeschlossen \\
		PrTimeDelayMS5 & 538   & von der Analyse ausgeschlossen \\
		CostMostnegFCadjIS & 537   & von der Analyse ausgeschlossen \\
		AMAge2 & 444   & von der Analyse ausgeschlossen \\
		AMTen2 & 444   & von der Analyse ausgeschlossen \\
		PrTimeDelay & 254   &  \\
		PrTimeDelayMS11 & 227   &  \\
		PrTimeDelayMS10 & 214   &  \\
		PrTimeAct & 192   &  \\
		CostMostnegFCadj & 177   & von der Analyse ausgeschlossen \\
		PrTimeDelayMS2 & 156   &  \\
		PrTimeDelayMS8 & 139   &  \\
		AMNo  & 132   & von der Analyse ausgeschlossen \\
		PrTimeBase & 118   &  \\
		PMAge2 & 98    &  \\
		PMTen2 & 98    &  \\
		CostFirstadj & 61    &  \\
		PrStartDate & 13    &  \\
		PMNo  & 6     &  \\
		BA    & 6     &  \\
		BU    & 6     &  \\
		TOAct & 6     &  \\
		DB1Budabs & 6     &  \\
		DB1Actabs & 6     &  \\
		EquLoc & 2     &  \\	
	\end{tabular}
\end{table}
\newline
\newline\textbf{Plausibilität:} Die Plausibilitätsüberlegungen basieren auf der logischen Interpretation und Herleitung der Indikatoren. Die Tabelle im Anhang zeigt, welche Werte für die Indikatoren möglich sind. Die Garantierung der Plausibilität hat direkten Einfluss auf die Stichprobengrössen. Diese schrumpfte infolge der Anpassungen auf $N = $  (Tabelle im Anhang EINFÜGEN).
\newline Im Anschluss wurden die Daten auf mögliche Aussreisser hin untersucht. Bei dieser Untersuchung wurde das Augenmerk auch auf den Erhalt möglichst vieler Datensätze gelegt. Allerdings ist die Entfernung der Ausreisser auch in der deskreptiven Statistik von grosser Bedeutung, da die Aussagekraft der Ergebnisse beeinträchtgt werden kann. Die Ausreisser wurden anhand Boxplots, Histogramme und der 'Interquartile Ranges' der numerischen Variablen identifiziert.  Vorab muss erläutert werden, dass der Wert $1'111'111$ bei den CostMostnegFCajd-Variablen keinen Ausreisser darstellt sondern angibt, dass das Projekt nur positive FC-Anpassungen gehabt hat. Dies impliziert, dass der Forecast für die Kosten gesunken sind und somit weniger Kosten erwartet wurden, wobei der Umsatz konstant geblieben oder gestiegen ist. Bei den HOM-YellowStatus und die HOM-RedStatus drückt der Wert $1'111'111$ aus, dass der entsprechende Status nicht als erstes oder gar nicht aufgetreten ist. Die Interpretation wäre somit, dass bei den HOMYellow-Status Variablen der Status immer grün war oder zuerst respektive direkt den roten Status hatte. Eine ähnliche Interpretation gilt für HOMRedStatus-Variablen, somit hätte dieses Projekt, den roten Status gar nicht erst erreicht. Da diese Interpretationen valide sind und somit keine fehlende Werte darstellen, verbleiben sie im Datensatz. 
\newline
Die Bestimmung von extremen Werten ist stichprobenabhängig und je nach Zweck der Analyse und untersuchten Objekten sind Ausreisser unterschiedlich einzustufen. Ziel der nachfolgenden Analyse ist es, sämtliche Projekte der Bühler AG auf ihre Erfolgsfaktoren und finanzielle Performance untersuchen. Die Geschäftsbereiche der Bühler AG verkaufen unterschiedliche Anlangen, weshalb die Datenbereiche der Faktoren stark variieren können. Da aufgrund der unterschiedlichen Anlagen, kein direkter Vergleich zwischen den Daten gemacht werden kann, sind Ausreisser schwierig zu bestimmen. Methodisch wurde der einfache und doppelte Interquartile Range für alle numerischen Variablen berechnet und das folgende Entscheidungskalkül zur Bestimmung von Extremalwerten herangezogen. Anschliessend wurde pro Variable ausgewertet, ob Werte ausserhalb der Grenzwerte des doppelten IQR liegen. In einem solchen Fall wurde je Variable entschieden, ob eine Korrektur um die Ausreisser sinnvoll erscheint. Die argumentative Erklärung pro Variable befindet sich im Anhang. Einzig die realisierte Marge, DB1Act, wurde auf die Werte des doppelten IQR berichtigt, da extreme negative Margen auf sogenannte Crash-Projects schliessen lassen, welche bereits mittels internem Audit untersucht wurden und die Stichprobenergebnisse unnötige verzerren. Extrem positive DB1 sind bei einer durchschnittlichen Projektmarge von ca. 30\% relativ unwahrscheinlich und lassen Zweifel zur Richtigkeit der Kostenverbuchung zu. Bei den  relativen Kostenabweichungen für PA und IS wurden jeweils diejenigen Daten entfernt, welche extrem hoch waren und kein entsprechendes Budget geplant wurde. Hierbei wurde davon ausgegangen, dass die Budgetierung der Projektkosten nicht korrekt verlaufen ist, was letztendlich zu extremalen relativen Kostenabweichung geführt hat. Es wurden keine weiteren Ausreisser eliminiert, selbst wenn einige Werte ausserhalb des Entscheidungskalküls lagen. Die Erklärung pro Faktor sowie die Tabelle mit den IQR befindet sich im Anhang. Die Stichprobengrösse hat sich auf $N = $ veringert. Die Anzahl Faktoren entspricht $ i = $. Im Anschluss folgt die finanzielle Analyse und Untersuchung der Erfolgsfaktoren. 
\newline\newline
\begin{centering}
		$ all avlues \leq Q1 - 1.5 * IQR$
		\newline
		$ all values \geq Q3 + 1-5 * IQR$
\end{centering}
\newline
\newline
\begin{centering}
		$ all avlues \leq Q1 - 3 * IQR$
		\newline
		$ all values \geq Q3 + 3 * IQR$
\end{centering}
\newline\newline
 $N = 883 $ und die Anzahl Variablen $ x = 70$.
\subsection{Ergebnisse und Interpretation}
Im ersten Schritt wurde versucht, der finanzielle Verlust nicht erfolgreicher Projekte zu evaluieren und sukzessive weitere Kriterien in die Analyse einfliessen zu lassen um dadurch Hinweise zu möglichen Einflussfaktoren zu erhalten. Bei der nachfolgenden Analyse handelt es sich um eine post-Analyse der Projekt, deren Kostperformance in Abhängigkeit der identifizieren Erfolgsfaktoren. Es werden einige statistische Methoden zur Analyse hinzugezogen. Es sind Rückschlüsse auf die zugrundeliegende Stichprobe mögliche, jedoch keine Generalisierungen auf zukünftige oder laufende Projekte. Allerdings können Hinweise auf mögliche Erfolgsfaktoren gewonnen werden, deren Relevanz jedoch in einem weiteren Schritt untersucht werden müsste. Die Analyse erfolgt anhand der in Kapitel 2.2 definierten Kategorien. Wobei sich die mögliche Anzahl zu evaluierenden Faktoren aufgrund Plausibilitätkriterien und fehlender Daten verkürzt hat.
\newline Modell mit Faktoren einfügen\newline
\subsubsection{Finanzielle Analyse}
Die Finanzielle Analyse basiert auf dem binären Erfolgskriterium, die prozentuale Abweichung der realisierten von der budgetierten Projektmarge. 
\newline\newline
$Success = (DB1Act-DB1Bud)/DB1Bud \geq 0$\newline\newline
$Fail = (DB1Act-DB1Bud)/DB1Bud < 0$
\newline\newline
Als Vergleich wurde der Erfolgsschlüssel des BPM-Cockpits der Bühler AG hinzugezogen. Das interne Ampelsystem  legt für die Kosten folgenden Schlüssel fest:\newline\newline
$ green = (DB1Act-DB1Bud)/DB1Bud > -4\%$\newline\newline
$ yellow = -4\% \leq (DB1Act-DB1Bud)/DB1Bud > -10\%$\newline\newline
$ red = (DB1Act-DB1Bud)/DB1Bud  \leq -10\%$\newline\newline
Das Ampelsystem wird während des Projektverlaufs angewandt, um die Abweichung zwischen dem FC und dem Budget der Projektmarge zu evaluieren. Der gleiche Schlüssel wurde für die Abweichung zwischen dem DB1 Act und Bud angewandt, um zu evaluieren wie sich die Performance in monetärer Hinsicht und in Bezug auf die Häufigkeit von erfolgreichen und nicht erfolgreichen Projekten verändert. Erfolgreiche Projekte haben demzufolge den Status grün wohingegen gescheiterte Projektee die Status yellow und red haben. Die Anzahl erfolgreicher Projekte wird zwangsläufig ansteigen sowie auch die finanzielle Performance beeinflussen. Insbesondere ist die Veränderung des Margenverlust zwischen erfolgreichen und gescheiterten Projekten ein Interessenspunkt.\newline
Die Stichprobe setzt sich wie folgt zusammen:
\begin{table}[htbp]
	\centering
	\caption{Übersicht Stichprobe}
	\begin{tabular} {l|r|r}
		\textbf{Stichprobe} & \textbf{absolut} & \textbf{relativ} \\
		\textbf{Total} & 928 & 100\% \\
		\textbf{Success} & 624 & 67\% \\
		\textbf{Fail} & 304 & 33\% \\
	\end{tabular}
\end{table}
\newline
Die Erfolgsquote $Anzahl Projekte (Success)/Anzahl Projekte(Fail)$ beträgt 2.05. Daraus folgt, dass jedes dritte Projekte ein Erfolg wird. Gemäss subjektiver Einschätzung ist diese Erfolgsquote relativ gut. Auf Basis dieser Daten wurde nachfolgend der Verlust über diese Projekte berechnet. 
2 Table einfügen mit TO Bud, Cost Bud DB1 Bud, DB1Bud, DB1 Act, Cost Act, TOBudabs, DB1Budabs, CostBudDevabs.
\begin{table}[htbp]
	\centering
	\caption{Finanzieller Verlust: Binäres Erfolgskriterium }
	\begin{tabular}{lrrrrrrrrrrrr}
		\textbf{Erfolgskriterium} & \multicolumn{1}{l}{\textbf{TO Bud}} & \multicolumn{1}{l}{\textbf{Cost Bud}} & \multicolumn{1}{l}{\textbf{DB1 Bud}} & \multicolumn{1}{l}{\textbf{DB1 Bud [\%]}} & \multicolumn{1}{l}{\textbf{TO Act}} & \multicolumn{1}{l}{\textbf{Cost Act}} & \multicolumn{1}{l}{\textbf{DB1 Act}} & \multicolumn{1}{l}{\textbf{DB1 Act [\%]}} & \multicolumn{1}{l}{\textbf{TO Dev}} & \multicolumn{1}{l}{\textbf{Cost Dev.}} & \multicolumn{1}{l}{\textbf{DB1 Dev.}} & \multicolumn{1}{l}{\textbf{DB1 Dev. [\%]}} \\
		SUCCESS & 1'515'433 & -1'129'938 & 385'494 & 25.4\% & 1'522'924 & -1'018'031 & 504'893 & 33.2\% & 7'491 & 111'908 & 119'399 & 7.7\% \\
		FAIL  & 615'509 & -463'513 & 151'996 & 24.7\% & 629'118 & -525'333 & 103'785 & 16.5\% & 13'609 & -61'820 & -48'211 & -8.2\% \\
		Grand Total & 2'130'942 & -1'593'451 & 537'490 & 25.2\% & 2'152'042 & -1'543'364 & 608'679 & 28.3\% & 21'101 & 50'088 & 71'188 & 13.2\% \\
	\end{tabular}%
	\label{tab:addlabel}%
\end{table}%
\newline
\begin{table}[htbp]
	\centering
	\caption{Finanzieller Verlust: Erfolgskriterium ist Erfolg gemäss Ampel}
	\begin{tabular}{lrrrrrrrrrrrr}
		\textbf{Erfolgskrit} & \multicolumn{1}{l}{\textbf{TO Bud}} & \multicolumn{1}{l}{\textbf{Cost Bud}} & \multicolumn{1}{l}{\textbf{DB1 Bud}} & \multicolumn{1}{l}{\textbf{DB1 Bud [\%]}} & \multicolumn{1}{l}{\textbf{TO Act}} & \multicolumn{1}{l}{\textbf{Cost Act}} & \multicolumn{1}{l}{\textbf{DB1 Act}} & \multicolumn{1}{l}{\textbf{DB1 Act [\%]}} & \multicolumn{1}{l}{\textbf{TO Dev}} & \multicolumn{1}{l}{\textbf{Cost Dev.}} & \multicolumn{1}{l}{\textbf{DB1 Dev.}} & \multicolumn{1}{l}{\textbf{DB1 Dev. [\%]}} \\
		green & 1'796'310 & -1'344'946 & 451'364 & 25.1\% & 1'808'320 & -1'241'767 & 566'553 & 31.3\% & 12'010 & 103'179 & 115'189 & 6.2\% \\
		red   & 175'002 & -131'684 & 43'318 & 24.8\% & 183'152 & -172'966 & 10'185 & 5.6\% & 8'149.862 & -41'283 & -33'133 & -19.2\% \\
		yellow & 159'630 & -116'821 & 42'809 & 26.8\% & 160'571 & -128'630 & 31'941 & 19.9\% & 941   & -11'809 & -10'868 & -6.9\% \\
		Grand Total & 2'130'942 & -1'593'451 & 537'490 & 25.2\% & 2'152'042 & -1'543'364 & 608'679 & 28.3\% & 21'101 & 50'088 & 71'188 & 3.1\% \\
	\end{tabular}%
	\label{tab:addlabel}%
\end{table}%
\newline\newline


\subsubsection{Erfolgsfaktoren}
\subsection{Kritische Würdigung der Ergebnisse}
\newpage	
	


