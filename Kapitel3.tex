% !TEX root = MA.tex
\section{Analye der Erfolgsfaktoren des Bühler Projektmanagements}\label{drei}
In diesem Kapitel wird zuerst das analytische Vorgehen erläutert und anschliessend die Ergebnisse präsentiert sowie kritisch gewürdigt. Die Literaturrecherche zeigt, dass die Erfolgsfaktoren des Projektmanagement mehrfach erforscht wurden, wobei mehrheitlich sich überschneidende Resultat in Abhängigkeit unterschiedlicher Erfolgskriterien und Projektmanagementmethoden ergründet wurden. Das gewählte Forschungsdesign basierte für eine Vielzahl von Studien auf Einschätzungen der Wichtigkeit für den Projekterfolg pro Attribut mittels einer Likert-Skale. Die befragten Personen waren entweder direkt im Projektmanagement oder in der entsprechenden Industrie tätig. Die Attribute wurde aus früheren Studien extrahiert. Die gewählte Methode führte dazu, dass eine numerische Datenbasis, welche die Kriterien für lineare statistische Modelle, zu Analysezwecken vorhanden war. Da sich die Ergebnisse zu einem Teil überschneiden, kann postuliert werden, dass unabhängig vom gewählten Performancekriterium ein gewisser Konsens bezüglich der Erfolgsfaktoren existiert. Diese Aussage ist mit Vorsicht zu geniessen, da die Studien nicht direktr miteinander verglichen werden können, weshalb sie als Annahme formuliert wurde. Die nachfolgende Analyse unterscheidet sich insofern, da die zugrundeliegende Stichprobe unternehmensspezifische Daten sind. Daraus folgt, dass keine Rückschlüsse auf Projekterfolg und Projektmanagementerfolg im Allgemeinen möglich sind.\newline 
\subsection{Daten und statistische Methoden}
Dieses Kapitel dient dazu, das Forschungsdesign, die Datengrundlage und das Vorgehen genauer zu erläutern. Die Datenerhebung erfolgte mittels einer extra programmierten Query im Bühler System. Trotz mehrfacher Prüfung der Datenplausibilität konnte nach Beendigung der Programmierarbeiten die Korrektheit einiger berechneter Indikatoren nicht gewährleistet werden. Somit war es denkbar, dass einige Faktoren aufgrund mangelnder Plausibilität oder fehlender Daten von der Analyse ausgeschlossen werden mussten.
\newline
Die untersuchten Projekte kurz die Stichprobe enthält alle im Zeitraum zwischen 2013 und 2105 abgeschlossenen Projekte. Die eindeutigen Abgrenzungskriterien bilden der Projektstatus und das Datum des Project Closure (MS11). Zuerst wurden alle Projekte mit einem MS11-Datum zwischen dem 1.1.2013 und dem 31.12.2015 eingegrenzt. Anschliessend wurde mittles dem Projektstatus sichergestellt, dass das Projekt auch aus finanzieller Sicht abgeschlossen war. Denn gewisse Projekte sind zwar operativ bereits beendet, gelten aber aufgrund ausstehender Rechnungen aus finanzieller Sicht als 'nicht abgeschlossen'.
\newline Nach der ersten Datenexploration und Prüfung der Annahmen für lineare statistische Modelle, wurde festgestellt, dass die ursprünglich geplante Methodenwahl nicht angewendet werden konnte. Denn die unabhängigen Daten hatten geringe bis keine Korrelation mit der abhängigen Variable. Die lineare Variablentransformationen und andere Methoden um eine Verteilungskurve zu simulieren führten nur zu kleineren Verbesserung. Dieser Umstand und die Tatsache, dass Erfolgsfaktoren bereits sehr gut erforscht wurden, hat die Entscheidung auf Inferenzstatistik zu verzichten bestärkt. Die nachfolgende Analyse ist deshalb deskriptiver Natur und legt mehr Fokus auf die Analyse der finanziellen Einbussen von sogenannten nicht erfolgreichen Projekten. Die Aussagekraft der Ergebnisse wird eingeschränkt, so dass da keine Rückschlüsse auf die Grundgesamtheit (sämtliche Projekte der Bühler AG) gemacht werden können. Es kann jedoch Aussagen über die untersuchte Stichprobe gemacht und neue Hypothesen abgeleitet werden. Diese müssten mittels anderer, geeigneter statistischer Methoden analysiert werden. Da die Auswertung der Projektdaten eine erstmalige Analyse darstellt, können wertvolle Hinweise zu möglichen Problemfeldern aufgedeckt werden.
\newline Das Erfolgskriterium (DB1BudDev)ist die prozentuale Abweichung der realisierten Projektmarge vom Budget. Die Projektmarge errechnet sich aus Umsatz minus Kosten und wird in Relation zum Umsatz gestellt. Die Differenz zur budgetierten Projektmarge wurde in Prozenten gemessen. Dieses Erfolgskriterium wurde in Zusammenarbeit mit der Bühler AG festgelegt. Der DB1 ist Erfolgskriterium der Projekte und Managementziel der Geschäftsbereiche und Projektmanager zugleich. Dies unterstützt die Wahl dieses Kriteriums, da durch die Incentivierung, sämtliche Parteien ein Interesse an einer überdurchschnittlichen DB1-Performance haben. Der DB1BudDev wurde zu Analysezwecken in eine binäre Variable transformiert. Daraus folgt, dass alle positiven (negativen) Differenzen als erfolgreiche (nicht-erfolgreiche) Projekte betrachtet werden. Im Folgenden werden erfolgreiche Projekte und Success-Projekte sowie erfolgreiche Projekte und Fail-Projekte als Synonyme verwendet. Aus finanzieller Sicht kann das Projekt bei nicht Erreichung des Budgets als gescheitert (failed) betrachtet werden. Dabei werden retrospektive Erkenntnisse und Erfahrungen aufgrund des Projekts nicht berücksichtigt, obwohl diese ein Gewinn für das Unternehmen darstellen können und somit das Projekt nicht als gescheitert betrachtet werden könnte.
\newline\newline\textbf{Rohdaten}
\newline Die Rohdatensatz enthält sämtliche Daten zu den Faktoren der untersuchten Projekte (Stichprobe). Er setzt sich aus drei Datensätzen zusammen, die separat aus den Bühler-System extrahiert wurden. Das Alter und die Betriebszugehörigkeit der Projekt- und Areamanager mussten korrigiert werden, da der ursprüngliche Datensatz die Unterscheidung zwischen fehlenden Werten und Nullwerten nicht zu liess.
\newline\newline $Stichprobenumfang N = 1471$ und $Anzahl Faktoren i = 93$.
\newline\newline
Nach der Elimination doppelt vorhandener Faktoren und Berechnungsfaktoren, wurden sämtliche Determinanten mit fehlenden Werten von über 200 von der Analyse ausgeschlossen. Dazu wurde die Anzahl NA's pro Faktor ausgewertet anstatt pro Projekt. Dieser Schritt war notwendig, da auf diese Weise die Mehrheit der Datensätze erhalten blieb. Die nachfolgende Tabelle zeigt die Anzahl fehlender Daten pro Faktor. Zusätzlich wurde die Variablen AMNo, da einerseits wenig Informationsgehalt erwartet wurde und anderseits durch den Ausschluss der verbundenen Variablen (AMTen und AMAge)  AMNo an Relevanz verlor. Ausserdem mussten alle Variablen, welche die Zeitdifferenz zwischen dem letzten Kostenforecast und dem Projektende messen, aufgrund fragwürdiger Plausibilität und Korrektheit der Daten von der Analyse ausgeschlossen werden. Dieses Vorgehen reduzierte einerseits den Stichprobenumfang und die Anzahl zu untersuchender Faktoren. 
\newline\newline $Stichprobenumfang N = 1076$ und $Anzahl Faktoren i = 71$
\newline\newline
\begin{table}[h]
	\centering
	\caption{Anzahl NA's per variable}
	\begin{tabular} {| l| l | p{6cm} |}
		\textbf{Variable Code} & \textbf{Anz. NA} & \textbf{Handhabung} \\\hline
		CostMostnegFCadjPA & 749   & von der Analyse ausgeschlossen \\
		PrTimeDelayMS5 & 538   & von der Analyse ausgeschlossen \\
		CostMostnegFCadjIS & 537   & von der Analyse ausgeschlossen \\
		AMAge2 & 444   & von der Analyse ausgeschlossen \\
		AMTen2 & 444   & von der Analyse ausgeschlossen \\
		PrTimeDelay & 254   &  \\
		PrTimeDelayMS11 & 227   &  \\
		PrTimeDelayMS10 & 214   &  \\
		PrTimeAct & 192   &  \\
		CostMostnegFCadj & 177   & von der Analyse ausgeschlossen \\
		PrTimeDelayMS2 & 156   &  \\
		PrTimeDelayMS8 & 139   &  \\
		AMNo  & 132   & von der Analyse ausgeschlossen \\
		PrTimeBase & 118   &  \\
		PMAge2 & 98    &  \\
		PMTen2 & 98    &  \\
		CostFirstadj & 61    &  \\
		PrStartDate & 13    &  \\
		PMNo  & 6     &  \\
		BA    & 6     &  \\
		BU    & 6     &  \\
		TOAct & 6     &  \\
		DB1Budabs & 6     &  \\
		DB1Actabs & 6     &  \\
		EquLoc & 2     &  \\	
	\end{tabular}
\end{table}
\newline
\newline\textbf{Plausibilität:} Die Plausibilitätsüberlegungen basieren auf der logischen Interpretation und Herleitung der Indikatoren. Die Tabelle im Anhang enthält die Begründung der unplausiblen Werte. Dies hatte zur Folge, dass die Stichprobe auf $N = $ sank. (Tabelle im Anhang EINFÜGEN).
\newline Im Anschluss wurden die Daten auf mögliche Aussreisser hin untersucht. Bei dieser Untersuchung wurde das Augenmerk auch auf den Erhalt möglichst vieler Datensätze gelegt. Allerdings ist die Entfernung der Ausreisser auch in der deskreptiven Statistik von grosser Bedeutung, da die Aussagekraft der Ergebnisse beeinträchtgt werden kann. Die Ausreisser wurden anhand Boxplots, Histogramme und der 'Interquartile Ranges' der numerischen Variablen identifiziert. Vorab muss erläutert werden, dass der Wert $1'111'111$ bei den CostMostnegFCajd-Variablen keinen Ausreisser darstellt sondern angibt, dass das Projekt nur positive FC-Anpassungen gehabt hat. Dies impliziert, dass der Forecast für die Kosten gesunken sind und somit weniger Kosten erwartet wurden, wobei der Umsatz konstant geblieben oder gestiegen ist. Bei den HOM-YellowStatus und die HOM-RedStatus drückt der Wert $1'111'111$ aus, dass der entsprechende Status nicht als erstes oder gar nicht aufgetreten ist. Die Interpretation wäre somit, dass bei den HOMYellow-Status Variablen der Status immer grün war oder zuerst respektive direkt den roten Status hatte. Eine ähnliche Interpretation gilt für HOMRedStatus-Variablen, somit hätte dieses Projekt, den roten Status gar nicht erst erreicht. Da diese Interpretationen valide sind und somit keine fehlende Werte darstellen, verbleiben sie im Datensatz. 
\newline
Die Bestimmung von extremen Werten ist stichprobenabhängig und je nach Zweck der Analyse und untersuchten Objekten sind Ausreisser unterschiedlich einzustufen. Ziel der nachfolgenden Analyse ist es, sämtliche Projekte der Bühler AG auf ihre Erfolgsfaktoren und finanzielle Performance untersuchen. Die Geschäftsbereiche der Bühler AG verkaufen unterschiedliche Anlangen, weshalb die Datenbereiche der Faktoren stark variieren können. Da aufgrund der unterschiedlichen Anlagen, kein direkter Vergleich zwischen den Daten gemacht werden kann, sind Ausreisser schwierig zu bestimmen. Methodisch wurde der einfache und doppelte Interquartile Range für alle numerischen Variablen berechnet und das folgende Entscheidungskalkül zur Bestimmung von Extremalwerten herangezogen. Anschliessend wurde pro Variable ausgewertet, ob Werte ausserhalb der Grenzwerte des doppelten IQR liegen. In einem solchen Fall wurde je Variable entschieden, ob eine Korrektur um die Ausreisser sinnvoll erscheint. Die argumentative Erklärung pro Variable befindet sich im Anhang. Einzig die realisierte Marge, DB1Act, wurde auf die Werte des doppelten IQR berichtigt, da extreme negative Margen auf sogenannte Crash-Projects schliessen lassen, welche bereits mittels internem Audit untersucht wurden und die Stichprobenergebnisse unnötige verzerren. Extrem positive DB1 sind bei einer durchschnittlichen Projektmarge von ca. 30\% relativ unwahrscheinlich und lassen Zweifel zur Richtigkeit der Kostenverbuchung zu. Bei den  relativen Kostenabweichungen für PA und IS wurden jeweils diejenigen Daten entfernt, welche extrem hoch waren und kein entsprechendes Budget geplant wurde. Hierbei wurde davon ausgegangen, dass die Budgetierung der Projektkosten nicht korrekt verlaufen ist, was letztendlich zu extremalen relativen Kostenabweichung geführt hat. Es wurden keine weiteren Ausreisser eliminiert, selbst wenn einige Werte ausserhalb des Entscheidungskalküls lagen. Die Erklärung pro Faktor sowie die Tabelle mit den IQR befindet sich im Anhang. Die Stichprobengrösse hat sich auf $N = $ veringert. Die Anzahl Faktoren entspricht $ i = $. Im Anschluss folgt die finanzielle Analyse und Untersuchung der Erfolgsfaktoren. 
\newline\newline
\begin{centering}
		$ all avlues \leq Q1 - 1.5 * IQR$
		\newline
		$ all values \geq Q3 + 1-5 * IQR$
\end{centering}
\newline
\newline
\begin{centering}
		$ all avlues \leq Q1 - 3 * IQR$
		\newline
		$ all values \geq Q3 + 3 * IQR$
\end{centering}
\newline\newline
 $N = 883 $ und die Anzahl Variablen $ x = 70$.
\newline
Nach dem Datenbereinigungsprozess wurde zu analytischen Zwecken zusätzliche Variablen hinzugefügt. Die nachfolgende Tabelle zeigt sämtliche verbleibende (s. Kapitel 2 für alle Faktoren) inklusive der hinzugefügten Faktoren gemäss ihre Kategorie strukturiert. Im Bereich der finanziellen Analyse wurde versucht mittels der Rahmenbedingungen den Verlust der Projektmarge zu lokalisieren. Es wurde vorerst darauf verzichtet, diesem Szenario weitere vor allem kategoriale Variablen hinzuzufügen. Anschliessend wurde für jeden Faktor ein Histogram erstellt, um zu prüfen, ob die Verteilung zwischen Success- und Fail-Projekten auffallende Unterschiede aufweist. Mittels Häufigkeitstabellen, Mittelwerten wurde beabsichtigt die Charakteristiken der erfolgreichen und nicht-erfolgreichen Projekte zu beschreiben und deren numerischen Ausprägung zu unterscheiden. Basierend auf diese Ergebnissen wurde versucht entweder neue oder weiterführende Hypothesen zu formulieren. Zur Evaluation von kategorialen Variablen wurde ein weiteres Kriterium die Erfolgsquote hinzugezogen, um beispielsweise Geschäftsbereiche oder Region untereinander vergleichen zu können. 
\newline\newline $Erfolgsquote = Anzahl erfolgreicher Projekte/Anzahl nicht-erfolgreicher Projekte$
\newline\newline
\subsection{Ergebnisse und Interpretation}
In diesem Unterkapitel werden die Ergebnisse der finanziellen Analyse und der Untersuchung der Einflussfaktoren getrennt dargestellt. Die untersuchte Stichprobe enthält 966 Projekte, wovon 654 erfolgreich abgeschlossen wurden. Die Erfolgsquote beträgt 2.1.
\begin{table}[htbp]
	\centering
	\caption{Übersicht Stichprobe}
	\begin{tabular} {l|r|r}
		\textbf{Stichprobe} & \textbf{absolut} & \textbf{relativ} \\\hline
		\textbf{Total} & 966 & 100\% \\
		\textbf{Success} & 654 & 68\% \\
		\textbf{Fail} & 312 & 32\% \\
	\end{tabular}
\end{table}
\newpage
\subsubsection{Finanzielle Performance Analyse}
Dieses Kapitel dient dazu die finanzielle Performance der erfolgreichen und nicht-erfolgreichen Projekte zu ergründen. Dafür wurden die Umsatz-, Kosten- und Margenabweichungen der beiden Projektgruppen untersucht.
\newline
\begin{table}[htbp]
	\centering
	\caption{Übersicht Budget [TCHF]}
	\begin{tabular}{lrrrr}
		\textbf{Erfolgskrit} & \textbf{TO Bud} & \textbf{Cost Bud} &
		\textbf{DB1 Bud} & \textbf{DB1 Bud [\%]} \\
	SUCCESS & 1'552'450 & -1'156'598 & 395'851 & 25.5\% \\
	FAIL  & 618'013 & -465'066 & 152'947 & 24.7\% \\
	Grand Total & 2'170'463 & -1'621'664 & 548'799 & 25.3\% \\
	\end{tabular}%
\label{tab:addlabel}%
\end{table}%
\begin{table}[htbp]
	\centering
	\caption{Übersicht Actuals [TCHF]}
	\begin{tabular}{lrrrr}
		\textbf{Erfolgskrit} & \textbf{TO Act} & \textbf{Cost Act} & \textbf{DB1 Act}&
		\textbf{DB1 Act-Bud [\%]} \\
			SUCCESS & 1'560'001 & -1'041'728 & 518'273 & 33.2\% \\
			FAIL  & 631'346 & -526'600 & 104'746 & 16.6\% \\
			Grand Total & 2'191'347 & -1'568'328 & 623'018 & 28.4\% \\
	\end{tabular}
\label{tab:addlabel}%
\end{table}%
\begin{table}[htbp]
\centering
\caption{Übersicht Abweichungen [TCHF] ($Act-Bud$)}
\begin{tabular}{lrrrr}
	\textbf{Erfolgskrit} & \textbf{TO} & \textbf{Cost} & \textbf{DB1}&
	\textbf{DB1 [\%]} \\
	SUCCESS & 7'551 & 114'870 & 122'421 & 7.7\% \\
	FAIL  & 13'333 & -61'534 & -48'202 & -8.2\% \\
	Grand Total & 20'884 & 53'336 & 74'220 & 3.1\% \\
\end{tabular}
\label{tab:addlabel}%
\end{table}%
\newline Das budgetierte Umsatzvolumen der nicht-erfolgreichen Projekte beträgt ca 28\%. Der höhere realisierte Umsatz kann auf Zusatzverkäufe oder die Verrechnung der Mehrkosten an den Kunden zurückgeführt werden. Der Margenverlust der Fail-Projekte beträgt 48 Mio. CHF. oder (-32\%) CHF und. Demgegenüber stehen 122 MCHF (31\%) DB1-Gewinn auf einen Umsatz von 1.6 Milliarden CHF. Die positive Abweichung der Kosten für Success-Projekte kann mittels der realisierten Kostenreserve, welche nicht erschöpft wurde, erklärt werden. Denn pro Projekt werden je nach Geschäftsbereich zwischen 4\% und 9\% Kostenreserven budgetiert. Nach dem die Kostenreserve aufgebraucht ist, wird die Abweichung vom Kostenbudget negativ. Demzufolge wurde bei Fail-Projekten die Kostenreserve kumulativ vollständig ausgeschöpft. Da die Reserve in dieser Betrachtung nicht ersichtlich ist, wäre die effektive Differenz für Fail-Projekte (Success-Projekte) tiefer (höher). Die realisierte Marge über alle Success-Projekte beträgt 33\% und liegt 7.7\% über der budgetierten Marge von 25.4\%. Demgegenüber beträgt der DB1Act der Fail-Projekte 16.6\% und liegt 8.2\% unter dem DB1 Bud von 24.7\%.
\newpage
In Bezug auf die Kategorisierung der Projekte gemäss ihrem budgetierten Umsatzvolumen, erscheint es sinnvoll zu evaluieren, ob wenige grosse Projekte oder viele kleinere Projekten für den finanziellen Verlust gemacht werden können. Gleichzeit wird analysiert, wie sich die absoluten Kostenabweichungen voneinander unterscheiden. \newline
Die gesamte Kostenabweichung für Fail-Projekte beträgt ca. -61 Mio. CHF und für die Success-Projekte 111 Mio. CHF. Die Kostenabweichung von Fail-Projekten ist zu je einem Drittel auf Installtationskosten und die Mechanical Supply-Kosten zurückuführen. Das Restliche Drittel ist auf die ME-Kosten und PA-Kosten zurückzuführen, wobei PA die geringste Kostenabweichung aufweist. Bei den Success-Projekten ist die IS-Kostenüberschreitung mit ca. 7 MCHF erwähnenswert, wird jedoch vollständig durch die positive Abweichung der MS-Kosten kompensiert. Bei den Fail-Projekten ist die Kostendifferenz zwischen Act und Bud über alle Fail-Projekte negativ, was impliziert, dass die Kostenreserve vollständig aufgebraucht wurde und die Kosten tatsächlich über dem Budget liegen. Sie wird zudem fast vollständig durch die vier Kostenarten erklärt. Bei den Success-Projekten fehlt ein Teil, welcher die 111 Mio. CHF erklärt. Die Realisierung der Kostenreserve bei der Kostenart WA (Warranty) fehlt im Datensatz, erklärt aber höchst wahrscheinlich die Differenz zwischen der Summe der Kostenabweichungen der Kostenarten und der gesamten Kostenabweichung.
\newline\textbf{Einfügung Kostenperformance pro TOBud\_Cat + Kommentierung}
\subsubsection{Erfolgsfaktoren}
In diesem Kapitel werden die Ergebnisse pro Variablenkategorie präsentiert und kurz erläutert sowie mögliche Erklärungsansätze ergründet.
\newline\newline\textbf{Rahmenbedingungen:} Die Analyse der Rahmenbedingungen eines Projekts geben Hinweise darauf, in welchen Geschäftsbereiche und Regionen und mit welchen Kunden nicht-erfolgreiche gemäss dem Erfolgskriterium realisiert wurden. Da die Bühler AG in einer Matrix-Organisation organsiert ist, wurde nebst den Einzelauswertungen für die Region und die Business Area, der Regionen-BA Split für die Häufigkeit der Success- und Fail-Projekte erstellt.
\begin{table}[htbp]
	\centering
	\caption{Erfolgsquote pro Region}
	\begin{tabular}{lrrrrrr}
		\textbf{Region} & \multicolumn{1}{l}{\textbf{Erfolgsquote}} & \multicolumn{1}{l}{\textbf{Success}} & \multicolumn{1}{l}{\textbf{Fail}} & \multicolumn{1}{l}{\textbf{Fail [\%]}} & \multicolumn{1}{l}{\textbf{Total}} & \multicolumn{1}{l}{\textbf{Total [\%]}} \\
		East\_Asia & 6.7   & 20    & 3     & 13.0\% & 23    & 2.4\% \\
		EU    & 1.7   & 240   & 145   & 37.7\% & 385   & 39.9\% \\
		MEA\_Afr & 2.7   & 112   & 42    & 27.3\% & 154   & 15.9\% \\
		North\_Ame & 1.4   & 54    & 38    & 41.3\% & 92    & 9.5\% \\
		SAS\_BCHI & 2.9   & 119   & 41    & 25.6\% & 160   & 16.6\% \\
		South\_Ame & 1.9   & 58    & 31    & 34.8\% & 89    & 9.2\% \\
		South\_Asia & 4.3   & 51    & 12    & 19.0\% & 63    & 6.5\% \\
		\textbf{Total} & \textbf{2.1} & \textbf{654} & \textbf{312} & \textbf{32.3\%} & \textbf{966} & \textbf{100.0\%} \\
	\end{tabular}%
	\label{tab:addlabel}%
\end{table}% 
\begin{table}[htbp]
	\centering
	\caption{Erfolgsquote pro Geschäftsbereich}
	\begin{tabular}{lrrrrrr}
		BA    & \multicolumn{1}{l}{\textbf{Erfolgsquote}} & \multicolumn{1}{l}{\textbf{Success}} & \multicolumn{1}{l}{\textbf{Fail}} & \multicolumn{1}{l}{\textbf{Fail [\%]}} & \multicolumn{1}{l}{\textbf{Total}} & \multicolumn{1}{l}{\textbf{Total [\%]}} \\
		CF    & 2.8   & 118   & 42    & 26.3\% & 160   & 16.6\% \\
		DC    & 5.6   & 96    & 17    & 15.0\% & 113   & 11.7\% \\
		GD    & 2.3   & 7     & 3     & 30.0\% & 10    & 1.0\% \\
		GL    & 1.2   & 39    & 32    & 45.1\% & 71    & 7.3\% \\
		GM    & 1.9   & 226   & 122   & 35.1\% & 348   & 36.0\% \\
		LO    & 1.4   & 30    & 21    & 41.2\% & 51    & 5.3\% \\
		SR    & 5.0   & 35    & 7     & 16.7\% & 42    & 4.3\% \\
		TP    &       & 8     & 0     & 0.0\% & 8     & 0.8\% \\
		VN    & 1.4   & 95    & 68    & 41.7\% & 163   & 16.9\% \\
		\textbf{Total } & \textbf{2.1} & \textbf{654} & \textbf{312} & \textbf{32.3\%} & \textbf{966} & \textbf{100.0\%} \\
	\end{tabular}%
	\label{tab:addlabel}%
\end{table}%
\newline Europa ist der grösste Absatzmarkt und GM die grösste Business Unit  der Bühler AG. Diese Tatsache wiederspiegelt sich in den absoluten und relativen Häufigkeitsverteilung. Die Erfolgsquote ist in Europa und Nordamerika am niedrigsten und in Ostasien sowie Südasien am höchsten. Der Geschäftsbereiche VN, GL und GM haben die niedrigsten Erfolgsquoten.
\newpage
Die kleineren Geschäftseinheiten von GM haben eine relativ tiefe Erfolgsquote. Die Chance, dass es erfolgreich oder nicht-erfolgreich entspricht in etwas 50\%. Jedoch wird die Erfolgsquote von GM fast ausschliesslich durch die grösste Geschäftseinheit Industrial Milling bestimmt. Bei VN sind die Erfolgsquote mit Ausnahme von PN und OL sehr niedrig und beeinflussen die Erfolgsbilanz des gesamten Geschäftsbereichs. Etwas mehr als die Hälfte der Projekte von Grain Storage, der grösste Geschäftsbereich von Grain Logistics, haben erfolgreich abgeschlossen. 
\begin{table}[htbp]
	\centering
	\caption{Erfolgsquote: Geschäftseinheit}
	\begin{tabular}{llrrrrr}
		\textbf{BA} & \textbf{BU} & \multicolumn{1}{l}{\textbf{Erfolgsquote}} & \multicolumn{1}{l}{\textbf{Success}} & \multicolumn{1}{l}{\textbf{Fail}} & \multicolumn{1}{l}{\textbf{Fail\_per}} & \multicolumn{1}{l}{\textbf{Total}} \\
		GL    & GC    & \#DIV/0! & 1     & 0     & 0.0\% & 1 \\
		GL    & GS    & 1.2   & 36    & 29    & 44.6\% & 65 \\
		GL    & MT    & 0.7   & 2     & 3     & 60.0\% & 5 \\\hline
		GM    & BA    & 1.5   & 17    & 11    & 39.3\% & 28 \\
		GM    & BR    & 0.9   & 12    & 14    & 53.8\% & 26 \\
		GM    & IM    & 2.1   & 185   & 87    & 32.0\% & 272 \\
		GM    & SM    & 1.2   & 12    & 10    & 45.5\% & 22 \\\hline
		VN    & AG    & 0.8   & 11    & 14    & 56.0\% & 25 \\
		VN    & FE    & 1.2   & 27    & 22    & 44.9\% & 49 \\
		VN    & NU    & 1.3   & 27    & 21    & 43.8\% & 48 \\
		VN    & OL    & 2.0   & 6     & 3     & 33.3\% & 9 \\
		VN    & PN    & 3.0   & 24    & 8     & 25.0\% & 32 \\
	\end{tabular}%
	\label{tab:addlabel}%
\end{table}%
\newline
Im Regionen-BA Split für diejenigen Regionen mit den niedrigsten Erfolgsquoten, EU und NAM, sind jene BA's mit den niedrigsten Erfolgsquoten zu finden. 
\begin{table}[htbp]
	\centering
	\caption{Add caption}
	\begin{tabular}{llrrrrr}
		Region & BA    & \multicolumn{1}{l}{Erfolgsquote} & \multicolumn{1}{l}{Dummy\_Success} & \multicolumn{1}{l}{Dummy\_Fail} & \multicolumn{1}{l}{Fail\_per} & \multicolumn{1}{l}{Total} \\
		EU    & CF    & 1.9   & 58    & 31    & 34.8\% & 89 \\
		EU    & DC    & 5.0   & 45    & 9     & 16.7\% & 54 \\
		EU    & GD    & \#DIV/0! & 2     & 0     & 0.0\% & 2 \\
		EU    & GL    & 1.0   & 24    & 23    & 48.9\% & 47 \\
		EU    & GM    & 1.2   & 58    & 50    & 46.3\% & 108 \\
		EU    & LO    & 2.5   & 10    & 4     & 28.6\% & 14 \\
		EU    & SR    & 2.5   & 5     & 2     & 28.6\% & 7 \\
		EU    & VN    & 1.5   & 38    & 26    & 40.6\% & 64 \\
		North\_Ame & CF    & 2.0   & 10    & 5     & 33.3\% & 15 \\
		North\_Ame & DC    & 1.0   & 2     & 2     & 50.0\% & 4 \\
		North\_Ame & GL    & 1.0   & 1     & 1     & 50.0\% & 2 \\
		North\_Ame & GM    & 1.5   & 24    & 16    & 40.0\% & 40 \\
		North\_Ame & LO    & 4.0   & 4     & 1     & 20.0\% & 5 \\
		North\_Ame & SR    & 1.0   & 1     & 1     & 50.0\% & 2 \\
		North\_Ame & VN    & 1.0   & 12    & 12    & 50.0\% & 24 \\
	\end{tabular}%
	\label{tab:addlabel}%
\end{table}%
Zusammenfassend lässt sich aussagen, dass ungefähr 60\% der Fail-Projekte in den Geschäftsbereichen VN und GM respektive in den Regionen EU und NAM vorkommen. Dieses Ergebnis ist aufgrund der Tatsache, dass GM der grösste Geschäftsbereich und EU die grösste Region ist nicht. Die Kombination EU-GM, EU-GL, EU-VN machen knapp 30\% aller Fail-Projekte aus und haben eine sehr niedrige Erfolgsquoten. Die Region NAM hat relativ zu seiner Anzahl abgewickelter Projekte am meisten Fail-Projekte und auch die tiefste Erfolgsquote. Die Kombination NAM-VN scheint noch vor NAM-GM mehr Risiken für einen Projektabschluss unter Budget zu haben.
\newline\newline\textbf{Cost:} Das Umsatzvolumen soll Aufschluss über die Grösse und Wichtigkeit eines Projekts geben. Die zugrundeliegende Prämisse postuliert, dass Fail-Projekte höher Umsatzvolumen haben als Success-Projekte. Denn es wird davon ausgegangen, dass grössere Projekte in Bezug auf Ressourcenplanung bei den einzelnen Projektphase oder mehr involvierter Parteien schwieriger zu managen sind. Die Gegenhypothese unterstellt, dass höhere Umsatzvolumen den grösseren Effekt auf die DB1-Marge sowohl des Geschäftsbereich und der Region haben. Demzufolge erhalten  solche Projekte mehr Aufmerksamkeit und werden mit besonderer Vorsicht gemanaget. 
\newline\textbf{Histogram TO Bud}
Die Verteilung des Umsatzvolumen ist linksschief und zeigt dass der Grossteil der Projekte ein Umsatzvolumen von weniger als 10 Mio. CHF haben. Um etwas mehr Aufschluss über die Verteilungen und die Lokalisation der Fail-Projekte zu erhalten, wurde eine zusätzliche Variable generierte TOBud\_Cat. Sie teilt das Umsatzvolumen in 15 Kategorien ein. Alle Projekte unter 500 TCHF Umsatzbudget bilden die erste Kategorie. Projekte mit einem Umsatzbudget zwischen 500 TCHF und 5 Mio. CHF wurden in Klassen mit der Breite 500 TCHF unterteilt. Anschliessend folgt die Klasse mit Projekten, deren Umsatzvolumen zwischen 5 Mio. CHF und 10 Mio. liegt. Die letzte Klasse enthält alle Projekte mit einem Umsatzbudget von mehr als 10 Mio. CHF.  
\newline\textbf{Einfügung Histogram TOBud\_cat - SWEAVE}
Das Histogramm für die TOBud\_Cat zeigt, dass ca. zwei Drittel aller untersuchten Projekte ein Umsatzvolumen von bis und mit 2 Mio. CHF hat. Die Anzahl Fail-Projekte konzentriert sich folglich in diesen vier untersten Kategorien. Die Auswertung der Erfolgsquote pro Klasse ergab, dass Projekte mit einem Umsatzbudget im Bereich von 2 bis 5 Mio. relativ erfolgreich abgeschlossen wurden. Demgegenüber ist die Erfolgsquote von Projekten mit einem Umsatzbudget zwischen 5 und 10 Mio. tiefer. Basierend auf diesen Erkenntnissen lässt sich die folgende Hypothese formulieren: Das Umsatzvolumen begünstigt bis zu einem gewissen Schwellenwert, die Gegenhypothese und ab diesem Schwellenwert die ursprüngliche These.
\newline\newline Die absoluten und relativen Abweichungen zwischen den aktuellen und den budgetierten Kosten war bereits Bestandteil der finanziellen Analyse.
\newline\newline Die Zusammensetzung der Projektkosten soll Hinweise zur Natur der Projekte liefern, beispielsweise, ob Unterschiede zwischen den untersuchten Gruppen festzustellen sind. Die nachfolgende Tabelle zeigt die Mittelwerte pro relativem Kostenanteil. Es lassen sich keine auffallende Unterschiede feststellen.
\begin{table}[htbp]
	\centering
	\caption{Arithmetisches Mittel der relativen Anteile am Gesamtkostenbudget je Kostenart [\%]}
	\begin{tabular}{lrrrr}
		\textbf{Success} & \multicolumn{1}{l}{\textbf{BudMSTot}} & \multicolumn{1}{l}{\textbf{BudMETot}} & \multicolumn{1}{l}{\textbf{BudPATot}} & \multicolumn{1}{l}{\textbf{BudISTot}} \\
		FALSE & 67.1  & 6.2   & 5.9   & 7.7 \\
		TRUE  & 67.9  & 5.4   & 5.1   & 6.8 \\
	\end{tabular}%
	\label{tab:addlabel}%
\end{table}%
Der durchschnittliche budgetierte MS-Anteil am Kostenbudget des Projekts beträgt 67\%. Der durchschnittliche ME-Anteil und PA-Anteil ist für Failprojekten um etwa 80 Prozentpunkt höher. Der IS-Anteil von Fail-Projekten ist um 0.9\% höher als bei Success-Projekten. 
\newline\newline Nachlieferungen können einerseits ein Indiz für die Nicht-Einhaltung der vorgegeben Lieferzeit und anderseits für Fehlkonstruktionen sein. Allfällige Mehrkosten werden bei Verschulden der Bühler AG von der Bühler AG übernommen. Vermutungsweise ist der Anteil der Kosten aus Nachlieferungen bei Fail-Projekten höher als bei Success-Projekten. Die Auswertung des arithmetischen Mittels der prozentualen SU Kosten am Umsatz bestätigt die erwartete Vermutung.
\begin{table}[htbp]
	\centering
	\caption{Arithmetisches Mittel der SUCostTO [\%]}
	\begin{tabular}{lr}
		\textbf{Success} & \multicolumn{1}{l}{\textbf{SUCostTO}} \\
		FALSE & -0.81 \\
		TRUE  & -0.36 \\
	\end{tabular}%
	\label{tab:addlabel}%
\end{table}%

\begin{table}[htbp]
	\centering
	\caption{Arithmetisches Mittel der SUCostTO [\%] pro TO-Kategorie}
	\begin{tabular}{llr}
		\textbf{Success} & \textbf{TOBud\_Cat} & \multicolumn{1}{l}{\textbf{SUCostTO}} \\
		FALSE & [4.5e+03,5e+03) & -6.88 \\
		FALSE & [13.2,500) & -0.93 \\
		FALSE & [2.5e+03,3e+03) & -0.82 \\
		FALSE & [5e+03,1e+04) & -0.82 \\
		FALSE & [1.5e+03,2e+03) & -0.72 \\
		FALSE & [500,1e+03) & -0.69 \\
		FALSE & [3.5e+03,4e+03) & -0.69 \\
		FALSE & [2e+03,2.5e+03) & -0.57 \\
		FALSE & [4e+03,4.5e+03) & -0.52 \\
		FALSE & [1e+03,1.5e+03) & -0.47 \\
		FALSE & [1e+04,3.42e+04) & -0.36 \\
		FALSE & [3e+03,3.5e+03) & -0.31 \\
	\end{tabular}%
	\label{tab:addlabel}%
\end{table}%
Die Analyse der SUCostTO pro TO-Kategorie zeigt, dass für Projekte mit einem Umsatzvolumen zwischen 13.2 TCHF und 500 TCHF die Nachlieferungskosten in Relation zum Umsatz am höchsten war. Der Wert 6.9\% kann als Anomalie betrachtet werden, ein Projekt mit einem SUCostTO-Wert von ca. 40\% ein Einzelfall darstellt.
\newline\newline Tendenziell wird die Anpassung des Forecast für die Projektkosten bei erwarteten Mehrkosten möglichst lange hinausgezögert. Einerseits kann mit diesem Vorgehen, die Erklärungsdirektive umgangen werden und anderseits besteht wahrscheinlich, dass die Projektkosten sich wieder normalisieren. Deshalb wird erwartet, dass die Differenz zwischen der letzten FC-Anpassung und den tatsächlichen Kosten bei Fail-Projekten höher ist. Die tatsächlichen Kosten waren durchschnittlich höher als beim letzten Kostenforecast. Die durchschnittliche Differenz bei den IS-Kosten war für Fail-Projekte doppelt so hoch. Dies könnte ein Indiz sein, dass bei Fail-Projekten die Installation kostenintensiver verlief. Mögliche Gründe könnte die unzureichende Vorbereitung durch den Kunden oder mangelnde personelle Ressourcen, die zu Mehrkosten in der letzten Projektphase führen. 
\begin{table}[htbp]
	\centering
	\caption{Arithmetisches Mittel der Abweichung der effektiven Kosten vom FC [TCHF]}
	\begin{tabular}{lrrr}
		\textbf{Success} & \multicolumn{1}{l}{\textbf{DeltaLastFCAct}} & \multicolumn{1}{l}{\textbf{DeltaLastFCActMS}} & \multicolumn{1}{l}{\textbf{DeltaLastFCActME}} \\
		FALSE & -490.54 & -445.53 & 7.48 \\
		TRUE  & -436.41 & -454.24 & 7.93 \\
	\end{tabular}%
	\label{tab:addlabel}%
\end{table}%
\begin{table}[htbp]
	\centering
	\caption{Arithmetisches Mittel der Abweichung der effektiven Kosten vom FC [TCHF]}
	\begin{tabular}{lrr}
		\textbf{Success} & \multicolumn{1}{l}{\textbf{DeltaLastFCActPA}} & \multicolumn{1}{l}{\textbf{DeltaLastFCActIS}} \\
		FALSE & -12.87 & -14.52 \\
		TRUE  & -13.41 & -6.49 \\
	\end{tabular}%
	\label{tab:addlabel}%
\end{table}%
\textbf{FF-Variablen:} Der bedeutenste Einflussfaktor im Projektmanagement ist der Projektmanager selbst. Die Evaluation der realisierten Projekte pro Projektmanager inklusive der Erfolgsquote hat ergeben, dass die 966 Projekte von 301 unterschiedlichen Projektmanager abgewickelt wurde. 145 Projektmanager haben ihre Projekte aussschliesslich erfolgreich beendet, wohingegen gerade einmal 45 PM nur unzureichend Projekte abgewickelt hat. Die detaillierte Liste ist im Anhang zu finden.
\newline\newline Der Wechsel des Projektmanagers kann ein Indiz für konfligierende Verhältnisse zwischen den Vertragsparteien sein, weshalb hypothetisch vermutet wird, dass Fail-Projekte eher mit einem PMChange einhergehen. 
\begin{table}[htbp]
	\centering
	\caption{Häufigkeit PMChange}
	\begin{tabular}{lrrrr}
		\textbf{PMChange} & \multicolumn{1}{l}{\textbf{Success}} & \multicolumn{1}{l}{\textbf{Fail}} & \multicolumn{1}{l}{\textbf{Fail [\%]}} & \multicolumn{1}{l}{\textbf{Total}} \\
		no    & 628   & 295   & 31.96\% & 923 \\
		yes   & 26    & 17    & 39.53\% & 43 \\
		\textbf{Total} & \textbf{654} & \textbf{312} &       & \textbf{966} \\
	\end{tabular}%
	\label{tab:addlabel}%
\end{table}%
Insgesamt wurden 43 Projekte mit einem Wechsel des Projektmanagers über die letzten drei Jahre abgewickelt. Davon sind 17 gescheitert und 26 wurden erfolgreich abgeschlossen.
\newline Die Anzahl Projektmanager ist direkt mit der Variable PMChange verbunden, das ein PMChange zwei Projektmanager währen des Projektverlaufs impliziert. Die nachfolgende Tabelle zur Häufigkeitsübersicht reflektiert diese Relation. Es gab eine kleine Anzahl Projekte (insgesamt 6 Projekte) bei denen zweimal ein Wechsel des Projektmanager erfolgte, wovon 5 nicht erfolgreich abgeschlossen werden konnten.
\newline Das Alter des Projektmanagers ist eine Proxyvarialbe für die Lebens- und Berufserfahrung generell. Hierbei wird unterstellt, dass je erfahrener der Projektmanager ist, desto eher können die Projekte erfolgreich abgeschlossen werden. Die Betriebszugehörigkeit des Projektmanagers (PMTen) ist eine Proxyvariable für die Kenntnisse der Bühlerwelt. Die Varialbe postuliert einen Zusammenhang zwischen der Erfolgschance und den Kenntnissen über die Bühlerwelt, folglich müssten jüngere Mitglieder der Bühler-Familie weniger Erfolg im Projektmanagement haben. Das durchschnittliche Alter der untersuchten Stichprobe beträgt 39 Jahre. Diesem Durchschnitt kann unterstellt werden, dass relativ erfahrene Projektmanager während der Zeit von 2013 bis 2015 bei der Bühler AG gearbeitet haben. Die durchschnittliche Betriebszugehörigkeit beträgt 10 Jahre, womit sich postulieren lässt, dass die PM der betrachteten Stichprobe relativ gute Kenntnisse von den Bühler-Praktiken hatten.
\begin{table}[htbp]
	\centering
	\caption{Durchschnittswerte PMAge und PMTen}
	\begin{tabular}{lrr}
		\textbf{Success} & \multicolumn{1}{l}{\textbf{Age}} & \multicolumn{1}{l}{\textbf{Ten}} \\
		FALSE & 41.1 & 12.4 \\
		TRUE  & 39.5 & 11.7 \\
	\end{tabular}%
	\label{tab:addlabel}%
\end{table}%
 Die Lead SAS des Projekts trägt die Gesamtverantwortung. Einige Gesellschaften sind bessere Projektmanager als anderen. Der Vergleich Erfolgsquoten pro SAS zeigt, dass die europäischen Gesellschaften n den letzten drei Jahren eine unterdurchschnittlich Erfolgsrate hatten. Es lässt sich eine Übereinstimmung mit den Befunden aus der Regionen-Analyse feststellen.
\newline Die LeadSASFF ist verantwortlich für die Projektabwicklung, wobei sie sich von der LeadSASPr unterscheiden kann. Da bei geteilter Verantwortlichkeiten die Anforderungen an die Kommunikation zwischen den Schnittstellen steigt, wird vermutet, dass bei getrennter Verantwortlichkeiten ein Merkmal von Fail-Projekten sind. Die nachfolgende Informationen der Tabelle deuten an, dass das Gegenteil wahr ist. 
\begin{table}[htbp]
	\centering
	\caption{Häufigkeit geteilter Verantwortlicheit [yes]}
	\begin{tabular}{lrrrr}
		\textbf{LeadSAS.PrFF} & \multicolumn{1}{l}{\textbf{Success}} & \multicolumn{1}{l}{\textbf{Fail}} & \multicolumn{1}{l}{\textbf{Fail [\%]}} & \multicolumn{1}{l}{\textbf{Total}} \\
		No    & 569   & 296   & 34.2\% & 865 \\
		Yes   & 85    & 16    & 15.8\% & 101 \\
		\textbf{Total} & \textbf{654} & \textbf{312} &       & \textbf{966} \\
	\end{tabular}%
	\label{tab:addlabel}%
\end{table}%
In der Stichprobe war die Projektverantwortung für ungefähr 90\% zentralisiert, wovon 34.2\% nicht erfolgreich waren. Bei den restlichen 101 Projekten mit geteilter Projektverantwortung wurden lediglich 16\% mit einem DB1Act unter Budget abgeschlossen. Die Anzahl der LeadSASFF steht in direkter Verbindung zum Faktor LeadSAS.PrFF, da die Ausprägung 'No' impliziert, dass nur eine SAS die Projektverantwortung inne hat. Deshalb liefert diese Determinante keine zusätzlichen Informationen. 
\newline\newline\textbf{Zeit:} Die Beurteilung des Zeitmanagement hängt von der Einhaltung des vereinbarten Liefertermins ab. Mehrkosten und Zeitverzug gehen oftmals einher, weshalb unterstellt wird, dass Fail-Projekte den vereinbarten Projektabschluss nicht einhalten konnten. Ferner soll ergründet werden, ab welchem Zeitpunkt respektive bei Milestone der Zeitverzug üblicherweise eintritt. 
\begin{table}[htbp]
	\centering
	\caption{Projektlaufzeiten und Zeitverzug [in Monaten]}
	\begin{tabular}{lrrrrrrr}
		\textbf{Success} & \multicolumn{1}{l}{\textbf{Base}} & \multicolumn{1}{l}{\textbf{Act}} & \multicolumn{1}{l}{\textbf{Delay}} & \multicolumn{1}{l}{\textbf{MS2}} & \multicolumn{1}{l}{\textbf{MS8}} & \multicolumn{1}{l}{\textbf{MS10}} & \multicolumn{1}{l}{\textbf{MS11}} \\
		TRUE  & 11.9  & 17.3  & -5.4  & -0.1  & -2.0  & -5.0  & -5.5 \\
		FALSE & 11.4  & 18.7  & -7.2  & -0.2  & -1.7  & -5.7  & -7.3 \\
	\end{tabular}%
	\label{tab:addlabel}%
\end{table}%
Die durchschnittliche budgetierte Projektlaufzeit unterscheidet sich zwischen erfolgreichen und  nicht-erfolgreichen Projekten kaum wohingegen die effektive Projektlaufzeit der Fail-Projekte einen Monat mehr betrug. Gemäss der Tabelle sind Success-Projekte ca. 2 Monate weniger zeitverzögert. Die Termineinhaltung beim MS 2 Concept approved bewegt sich im vernachlässigbaren Bereich. Demgegenüber steigt der durchschnittliche Zeitverzug nach MS8 Documented auf zwei und nach MS10 Takeover auf 5-6 Monate an. Bei den Fail-Projekten stieg die durchschnittliche Zeitverzögerung auf 7.3 Monate an. Ein möglicher Erklärungsansatz wäre, dass bei der Übergabe Mängel beanstandet wurden und nachgebessert werden musste.
\begin{table}[htbp]
	\centering
	\caption{Add caption}
	\begin{tabular}{lrrrrrrrrrr}
		\textbf{Success} & \multicolumn{1}{l}{\textbf{Delay}} & \multicolumn{1}{l}{\textbf{Total}} & \multicolumn{1}{l}{\textbf{DelayMS2}} & \multicolumn{1}{l}{\textbf{onTimeMS2}} & \multicolumn{1}{l}{\textbf{DelayMS8}} & \multicolumn{1}{l}{\textbf{onTimeMS8}} & \multicolumn{1}{l}{\textbf{DelayMS10}} & \multicolumn{1}{l}{\textbf{onTimeMS10}} & \multicolumn{1}{l}{\textbf{DelayMS11}} & \multicolumn{1}{l}{\textbf{onTimeMS11}} \\
		FALSE & TRUE  & 268   & 39    & 229   & 187   & 81    & 249   & 19    & 267   & 1 \\
		FALSE & FALSE & 44    & 4     & 40    & 22    & 22    & 26    & 18    & 3     & 41 \\
		\textbf{Total FALSE} &       & \textbf{312} & \textbf{43} & \textbf{269} & \textbf{209} & \textbf{103} & \textbf{275} & \textbf{37} & \textbf{270} & \textbf{42} \\
		TRUE  & TRUE  & 515   & 63    & 452   & 353   & 162   & 476   & 39    & 513   & 2 \\
		TRUE  & FALSE & 139   & 12    & 127   & 55    & 84    & 56    & 83    & 3     & 136 \\
		\textbf{Total TRUE} &       & \textbf{654} & \textbf{75} & \textbf{579} & \textbf{408} & \textbf{246} & \textbf{532} & \textbf{122} & \textbf{516} & \textbf{138} \\
		\textbf{Grand Total} &       & \textbf{966} & \textbf{118} & \textbf{848} & \textbf{617} & \textbf{349} & \textbf{807} & \textbf{159} & \textbf{786} & \textbf{180} \\
	\end{tabular}%
	\label{tab:addlabel}%
\end{table}%
Die Mehrheit der untersuchten Projekte konnte die Zeitvereinbarungen im MS2 einhalten. Dieses Verhältnis ändert sich bei Erreichung des MS8 und steigt bei MS10 so an, dass letztendlich der Grossteil der Projekte zeitverzögert abgeschlossen wird (783 Projekte respektive 86\%). Mittels Dummyvariablen pro Milestone wurde ausgewertet, ob sich ein anfängliche Verspätung sich durch die Projektlaufzeit durchzieht. auch in einer Verspätung beim letzten MS gemündet hat. Die meisten Projekte hatten die Eigenschaft, bei einer Verspätung im MS8 ebenso auch im MS11 verspätet zu sein. Die zweithäufigste Gruppe war diejenige, die erst bei MS10 den Liefertermin nicht mehr einhalten konnten, allerdings dann auch den finalen Liefertermin nicht mehr einhalten konnte. Auf Basis der Tabellen lassen sich jedoch keine Aussagen machen, ob bei einer Nichteinhaltung des vereinbarten Termin, das Projekt zwangsläufig unter Budget abschliessen wird. 
\newline\newline\textbf{SQ:} Die Incentivierung der Area Manager erfolgt über das Auftragsvolumen. Ausserdem muss jeder Geschäftsbereich die Budgetvorgaben für das Auftragsvolumen pro Monat erfüllen.  Es konnte einzig ausgewertet werden, ob die Geschäftsbereiche bei Projektabschluss einen finanziellen Druck hatte, da sie hinter dem Auftragsvolumenbudget lagen. Hierbei wird unterstellt, dass Projekte unvorsichtiger geprüft werden, da der Geschäftsbereich respektive die Region auf das Projektvolumen angewiesen ist, um die Budgetvorgaben zu erreichen. Folglich wird erwartet, dass die durchschnittlichen werte für die Differenz zwischen dem Act und Bud absolut und relativ bei Fail-Projekten höher liegt als bei Succes-Projekten. Die Untersuchung der beiden Stichproben bestätigt die Vermutung, dass Fail-Projekte tendenziell einem grösseren Budgetdruck ausgesetzt sind als Success-Projekte sowohl in relativen als auch in absoluten Grössen.
\newline\newline\textbf{Komplexität:} Da es kein Komplexitätsfaktor gibt, wurden sogenannte Proxyvariablen erhoben, um die Komplexität abzubilden. Es wurde die Anzahl involvierter Parteien bei der Herstellung der Maschine als Komplexitätsfaktor identifiziert sowie die Anzahl Aufträge. Es wird unterstellt, dass das Management einer hohen Anzahl Schnittstellen und Aufträge komplexer wird, da der Aufwand alle Parteien uf einander abzustimmen höher ist. Gewisse Projekte werden in einem Konsortium abgewickelt, was bedeutete, dass das Projekt zusammen mit einem Dritt-Unternehmen (keine Bühler-Gesellschaft) durchgeführt wird. Die Zusammenarbeit mit einem anderen Unternehmen wird ebenso als Komplexitätsindikator gewertet. Ca. 79 Projekte der Stichprobe wurden in einem Konsortium abgewickelt, davon waren 48 (61\%) erfolgreich und 31 (39\%) gescheitert. Das sind verhältnismässig wenig Projekte.\newline
Für die Anzahl Aufträge zeigt sich, dass die Mehrheit aller Projekte in der Stichprobe genau einen Vertrag hat. 90\% aller Projekte hat entweder 1 oder 2 Verträge. Vergleicht man die Erfolgsquote pro Anzahl Aufträge, ist diejenige mit zwei Aufträgen höher als diejenigen mit einem oder drei Aufträge. Es könnte damit zusammenhängen, dass ein grosser Auftrag weniger Übersichtlich ist als zwei kleinere separate. Zu viele Aufträge könnten dann wieder für Verwirrungen sorgen.\newline
Die Anzahl involvierte SAS hat einen Datenbereich von 0 bis 10. 0 bedeutet, dass die Zulieferung für die Maschine aus Eigenproduktion oder von Drittlieferanten stammt.  Allerdings zeigen sämtliche Histogramme keine Auffäligkeiten, welche darauf schliessen lassen, dass Fail-Projekte andere Eigenschaften aufweisen als Success-Projekte. Die Erfolgsquote pro Ausprägung zeigt, dass wenn die Zulieferung durch eine andere Gesellschaft erfolgt, höher ist als wenn eine Eigenproduktion oder ein Drittlieferung stattfindet. Bei der Anzahl Supplying SAS MS zeigt sich dasselbe Bild. Hingegen zeigt sich bei den Supplying ME, PA und IS, dass wenn keine weitere Bühler GEsellschaft involviert war, die Erfolgsquote höher ist.

\subsection{Kritische Würdigung der Ergebnisse}
\newpage	
	


