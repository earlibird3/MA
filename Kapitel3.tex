% !TEX root = MA.tex
\chapter{Methodolgie}\label{sec:methode}
In diesem Kapitel wird zuerst das analytische Vorgehen erläutert und anschliessend die Ergebnisse präsentiert sowie kritisch gewürdigt. In vergangenen Studien wurden die Erfolgsfaktoren von Projekten mittels der statistischen Auswertung von Einschätzungen ausgewählter Attribute zu deren Relevanz für den Projekterfolg erhoben. Die nachfolgende Analyse unterscheidet sich insofern, da für die Ergründung der Charakteristiken nicht-erfolgreicher Projekte unternehmensspezifische Projektinformationen verwendet wurden.
%%Datengrundlage: Erläutere Stichprobe: Art, Grösse, Erhebung der Stichprobe
%%Operationalisierung der Variablen: Welche Variablen: abhängige (Erfolgskriterium, Erfolgsquote), unabhängige(Einflussfaktoren), Erklärung, Verweis auf Prozess, Erklärung Erhebung und Prämsisen, Ausklammerung der Variablen, weshalb wieso.
\section{Datengrundlage}\label{sec:datagr}
\begin{table}[htbp]
	\centering
	\caption{Übersicht der Faktoren}
	\begin{tabular}{llll}
		\textbf{Erfolgskriterium} &       &       &  \\\hline
		DB1BudDev &   Dummy\_Success *    &       &  \\
		Success * &     Dummy\_Fail * &       &  \\
		\textbf{Rahmenbedingungen} & \multicolumn{1}{l}{\textbf{Zeitmanagement}} & \multicolumn{1}{l}{\textbf{Sales \& Quoatation}} & \multicolumn{1}{l}{\textbf{Komplexität}} \\\hline
		CuNo  & \multicolumn{1}{l}{PrTimeBase} & \multicolumn{1}{l}{BUORBudGapAbs} & \multicolumn{1}{l}{ConPart} \\
		EquLoc & \multicolumn{1}{l}{PrTimeAct} & \multicolumn{1}{l}{BUORBudGapRel} & \multicolumn{1}{l}{NoSupplSAS} \\
		BA    & \multicolumn{1}{l}{PrTimeDelay} & \multicolumn{1}{l}{RegiORBudGapAbs} & \multicolumn{1}{l}{NoSupplSASMS} \\
		BU    & \multicolumn{1}{l}{PrTimeDelayMS2} & \multicolumn{1}{l}{RegiORBudGapRel} & \multicolumn{1}{l}{NoSupplSASME} \\
		MS    & \multicolumn{1}{l}{PrTimeDelayMS8} &       & \multicolumn{1}{l}{NoSupplSASPA} \\
		Region & \multicolumn{1}{l}{PrTimeDelayMS10} &       & \multicolumn{1}{l}{NoSupplSASIS} \\
		& \multicolumn{1}{l}{PrTimeDelayMS11} &       & \multicolumn{1}{l}{NoContr} \\
		& \multicolumn{1}{l}{Delay * } &       &  \\
		\textbf{Kostenmanagement} &       & \multicolumn{1}{l}{\textbf{Fulfillment}} &  \\\hline
		TOBud & \multicolumn{1}{l}{CostActBudISRel} & \multicolumn{1}{l}{PMNo} & \multicolumn{1}{l}{CostFCadjPA} \\
		BudMSTot & \multicolumn{1}{l}{DeltaLastFCAct} & \multicolumn{1}{l}{PMAge2} & \multicolumn{1}{l}{CostFCadjIS} \\
		BudMETot & \multicolumn{1}{l}{DeltaLastFCActMS} & \multicolumn{1}{l}{PMTen2} & \multicolumn{1}{l}{HOMYellCost} \\
		BudPATot & \multicolumn{1}{l}{DeltaLastFCActME} & \multicolumn{1}{l}{PMChange} & \multicolumn{1}{l}{HOMYellQual} \\
		BudISTot & \multicolumn{1}{l}{DeltaLastFCActPA} & \multicolumn{1}{l}{NoPM} & \multicolumn{1}{l}{HOMYellTime} \\
		DB1Bud & \multicolumn{1}{l}{DeltaLastFCActIS} & \multicolumn{1}{l}{LeadSASPr} & \multicolumn{1}{l}{HOMRedCost} \\
		DB1Act & \multicolumn{1}{l}{TOAct} & \multicolumn{1}{l}{LeadSAS.PrFF} & \multicolumn{1}{l}{HOMRedQual} \\
		CostActBudMSabs & \multicolumn{1}{l}{TOBudDevabs *} & \multicolumn{1}{l}{NoLeadSASFF} & \multicolumn{1}{l}{HOMRedTime} \\
		CostActBudMEabs & \multicolumn{1}{l}{CostBud *} & \multicolumn{1}{l}{CostFCadj} & \multicolumn{1}{l}{PrStartDate} \\
		CostActBudPAabs & \multicolumn{1}{l}{CostAct *} & \multicolumn{1}{l}{CostFCadjMS} & \multicolumn{1}{l}{Cat\_age * } \\
		CostActBudISabs & \multicolumn{1}{l}{CostBudDevabs *} & \multicolumn{1}{l}{CostFCadjME} &  \\
		SUCostTO & \multicolumn{1}{l}{DB1Budabs} &       &  \\
		CostActBudRel & \multicolumn{1}{l}{DB1Actabs} &       &  \\
		CostActBudMSRel & \multicolumn{1}{l}{DB1BudDevabs *} &       &  \\
		CostActBudMERel & \multicolumn{1}{l}{TOBudCat *} &       &  \\
		CostActBudPARel &       &       &  \\
	\end{tabular}%
	\label{tab:ovvar}%
\end{table}%
\newpage

%%Datenanalyse: Finanzielle Analyse
%%Datenanalyse: Häufigkeitsverteilung etc.
%%Datenanalyse: Hpyothesen
\section{Datenanalyse}\label{sec:dataana}
Die untersuchte Stichprobe enthält alle Projekte, die im Zeitraum zwischen 2013 und 2105 abgeschlossen wurden. Die eindeutigen Abgrenzungskriterien bilden der Projektstatus und das Datum des Project Closure (MS11). Zuerst wurden alle Projekte mit einem MS11-Datum zwischen dem 1.1.2013 und dem 31.12.2015 eingegrenzt. Anschliessend wurde mittles dem Projektstatus sichergestellt, dass das Projekt auch aus finanzieller Sicht abgeschlossen war. Denn gewisse Projekte sind zwar operativ bereits beendet, gelten aber aufgrund ausstehender Rechnungen aus finanzieller Sicht als 'nicht abgeschlossen'.
\newline Nach der ersten Datenexploration und Prüfung der Annahmen für lineare statistische Modelle, wurde festgestellt, dass die ursprünglich geplante Methodenwahl nicht angewendet werden konnte. Denn die unabhängigen Variablen hatten geringe bis keine Korrelation mit der abhängigen Variable sprich dem Erfolgskriterium. Die lineare Variablentransformationen und andere Methoden, um eine Verteilungskurve zu simulieren, führten nur zu kleineren Verbesserung. Dieser Umstand und die Tatsache, dass Erfolgsfaktoren bereits sehr gut erforscht wurden, hat die Entscheidung auf Inferenzstatistik zu verzichten bestärkt. Die nachfolgende Analyse ist deshalb deskriptiver Natur und hat ausserdem das Ziel, die finanziellen Einbussen von sogenannten nicht-erfolgreichen Projekten zu untersuchen. Die Aussagekraft der Ergebnisse wird dadurch so eingeschränkt, dass da keine Rückschlüsse auf die Grundgesamtheit (sämtliche Projekte der Bühler AG) gemacht werden können.  Die Ergebnisse haben nur in Bezug auf die die untersuchte Stichprobe Gültigkeit. Es ist jedoch denkbar, auf Basis der Ergebnisse neue Hypothesen zu formulieren, welche mittels anderer, geeigneter statistischer Methoden geprüft werden könnten. Die erstmalige Auswertung der Projektdaten kann zudem Erkenntnisse zu möglichen Charakteristiken nicht-erfolgreicher Projekte liefern.
\newline Das Erfolgskriterium (DB1BudDev) wurde in Zusammenarbeit mit der Bühler AG festgelegt. Aus finanzieller und interne Perspektive ist die Abweichung der relativen Projektmarge (DB1Act) vom Budget (DB1Bud) von zentraler Bedeutung. Denn sowohl die Finanzziele wie auch die Incentivierung der Projekt- und Verkaufsmanager sowie der Geschäftsbereichsleitung basieren auf dem DB1 und den entsprechenden Budgetvorgaben. Die relative Projektmarge errechnet sich aus Umsatz minus Kosten in Relation zum Umsatz. Anhand der Differenz zwischen Act und Bud wird der Erfolg ($Differenz \geq 0$) respektive Nicht-Erfolg ($Differenz < 0$) von Projekten gemessen. Der DB1BudDev wurde zu Analysezwecken in eine binäre Variable (Success) transformiert. Daraus folgt, dass alle positiven (negativen) Differenzen als erfolgreiche (nicht-erfolgreiche) Projekte betrachtet werden. Im Folgenden werden erfolgreiche Projekte und Success-Projekte sowie nicht-erfolgreiche Projekte und Fail-Projekte als Synonyme verwendet. Obwohl retrospektiv Erkenntnisse und Erfahrungen aufgrund der Durchführung eines Projekts einen Gewinn für das Unternehmen darstellen können, wird diesem Aspekt in dieser Analyse nicht Rechnung getragen.
\newline\newline\textbf{Datenaufbereitung:} Der Rohdatensatz enthält sämtliche Daten zu den Faktoren der untersuchten Projekte (Stichprobe). Er setzt sich aus drei Datensätzen zusammen, die separat aus dem Bühler-System extrahiert wurden. Der Stichprobenumfang beträgt $N = 1471$ und die Anzahl Faktoren $i = 93$. Das Alter (PMAge und AMAge) und die Betriebszugehörigkeit (PMTen und AMTen) der Projekt- und Areamanager mussten korrigiert werden, da der ursprüngliche Datensatz die Unterscheidung zwischen fehlenden Werten und Nullwerten nicht zu liess.
\newline
Es wurden alle unplausiblen Faktoren und Berechnungsspalten vom Datensatz entfernt. Anschliessend wurde die Anzahl fehlender Daten pro Faktor ausgewertet und sämtliche Determinanten mit mehr als 300 fehlender Datensätze von der weiteren Analyse ausgeschlossen (vgl. Tabelle \ref{tab:na}). Zudem verbleibt nebst PrTimeDelayMS5, AMAge2 und AMTen2 die Variable AMNo unberücksichtigt, da durch den Ausschluss der verbundenen Faktoren wenig zusätzlicher Informationsgewinn erwartet wird. Ausserdem mussten alle Variablen, welche die Zeitdifferenz zwischen dem letzten Kostenforecast und dem Projektende messen (CostMostnegFCadj für \gls{abk:MS}, \gls{abk:ME}, \gls{abk:PA} und \gls{abk:IS}) sowie Indikatoren für die relative Wichtigkeit eines Projekts (BAImportPr, BUImportPr, MSImportPr) aufgrund fragwürdiger Plausibilität und Korrektheit der Daten von der Analyse ausgeschlossen werden. Mittels diesem Vorgehen kann der Datenverlust infolge fehlender Daten in Grenzen gehalten werden. Der neue Stichprobenumfang beträgt $N = 1076$ und die Anzahl Faktoren $i = 71$.
\newline
Im Anschluss wurden die Datensätze auf ihre Plausibilität getestet und Ausreisser entfernt. Die Plausibilitätsüberlegungen basieren auf der logischen Interpretation und Herleitung der Indikatoren. Die Tabellen mit den Begründungen der unplausiblen Werte und Ausreisser befinden sich im Anhang. Die Outliers wurden mit Hilfe von Boxplots, Histogramme und der 'Interquartile Ranges' (IQR) der numerischen Variablen identifiziert. Zur quantitativen Bestimmung der Ausreisser wurde folgendes Entscheidungskalkül angewendet:
\newline\newline
Ausreisse sind Werte, die $< Q1 - 1.5 * IQR$ respektive $ > Q3 + 1-5 * IQR$ sind.
\newline
Extreme Ausreisser sind Werte, die $< Q1 - 3 * IQR$ respektive $> Q3 + 3 * IQR$ sind.
\newline\newline
Je nach Zweck der Analyse und untersuchten Objekten sind Ausreisser unterschiedlich einzustufen. Die Geschäftsbereiche der Bühler AG verkaufen unterschiedliche Anlangen, weshalb die Datenbereiche der Faktoren stark variieren können. Die realisierte Projektmarge (DB1Act), wurde auf die Werte des doppelten IQR berichtigt, da extrem negative Margen auf sogenannte Crash-Projects schliessen lassen, welche bereits mittels internem Audit untersucht wurden und die Stichprobenergebnisse unnötige verzerren können. Extreme positive DB1Act sind bei einer durchschnittlichen Projektmarge von ca. 30\% relativ unwahrscheinlich und lassen Zweifel zur Richtigkeit der Kostenverbuchung zu. Bei den relativen Kostenabweichungen für PA und IS wurden jeweils einzelne Extremalwerte nur dann entfernt, wenn kein entsprechendes Budget geplant wurde. Denn es wurde davon ausgegangen, dass die Budgetierung der Projektkosten nicht korrekt verlaufen ist, was letztendlich zu extremalen relativen Kostenabweichung geführt hat. Es wurden keine weiteren Ausreisser eliminiert, selbst wenn einige Werte ausserhalb des Entscheidungskalküls lagen. Nach der Datenbereinigung umfasst die zu untersuchende Stichprobe $N = 966$ Projekte und $ i = 71$ Faktoren.
\newline\newline
%%
%%Zusätliche Variablen
%%
\textbf{Zusätzliche Variablen:} Nach dem Datenbereinigungsprozess wurde zu analytischen Zwecken vor allem kategoriale Variablen auf Basis der vorhandene Daten erhoben (mit * gekennzeichnet). Die nachfolgende Tabelle zeigt sämtliche alle in der Auswertung berücksichtigten Faktoren nach der Variablenkategorie strukturiert. Sämtliche Berechnungsformeln sowie die Interpretationen der Faktoren sind im Anhang enthalten.



