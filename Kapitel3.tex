% !TEX root = MA.tex
%Wahl der Methodik???? verzicht auf Befragungen der Bühler AG...externe Perspektive im Methodikteil
\chapter{Methodolgie}\label{sec:methode}
In diesem Kapitel wird zuerst die Datengrundlage einschliesslich der zahlreichen Variablen und anschliessend das analytische Vorgehen erläutert.  
%%Datengrundlage: Erläutere Stichprobe: Art, Grösse, Erhebung der Stichprobe
%%Operationalisierung der Variablen: Welche Variablen: abhängige (Erfolgskriterium, Erfolgsquote), unabhängige(Einflussfaktoren), Erklärung, Verweis auf Prozess, Erklärung Erhebung und Prämsisen, Ausklammerung der Variablen, weshalb wieso.
\section{Datengrundlage}\label{sec:datagr}
Der untersuchte Datensatz enthält insgesamt $N = 1497$ Projekte, die im Zeitraum zwischen 2013 und 2105 abgeschlossen wurden. Die eindeutigen Abgrenzungskriterien bilden der Projektstatus und das Datum des Project Closure (MS11). Zuerst wurden alle Projekte mit einem MS11-Datum zwischen dem 1.1.2013 und dem 31.12.2015 eingegrenzt. Anschliessend wurde mittles dem Projektstatus sichergestellt, dass das Projekt auch aus finanzieller Sicht abgeschlossen war. Denn gewisse Projekte sind zwar operativ bereits beendet, gelten aber aufgrund ausstehender Rechnungen aus finanzieller Sicht als \glqq nicht abgeschlossen \grqq{}. Die Datenerhebung erfolgt mittels der dem Projektmanagementtool der Bühler AG, das sämtliche Informationen der Projekte enthält. Das BPM-Cockpit dient zudem dem monatlichen Projektreporting.
\newline\newline
Insgesamt wurden $i = 92$ kategoriale und metrische Variablen erhoben und in sechs Kategorien, die sich am Projektmanagementprozess der Bühler AG orientieren, unterteilt: Rahmenbedingung, SQ-Variablen, FF-Variablen, Komplexität, Kosten und Zeit. Die nachfolgende Tabelle zeigt die Verteilung der kategorischen und metrischen Variablen pro Kategorie.
% Verteilung kategorialer und metrischer Variablen'
\begin{table}[htbp]
	\centering
	\caption{Anzahl kategorialer und metrischer Variablen pro Kategorie}
	\begin{tabular}{lrr|r}
		\toprule
		Kategorie & \multicolumn{1}{l}{kategoriale Variablen} & \multicolumn{1}{l}{metrisch Variablen} & \multicolumn{1}{l}{\textbf{Total}} \\
		\midrule
		Rahmenbedingung & 8     & 3     & \textbf{10} \\
		SQ-Variablen & 4     & 6     & \textbf{9} \\
		FF-Variablen & 9    & 22    & \textbf{33} \\
		Komplexität & 1     & 6     & \textbf{7} \\
		Kosten  & 0     & 24    & \textbf{24} \\
		Zeit  & 0     & 9     & \textbf{9} \\
		\bottomrule
		\textbf{Total} & \textbf{22} & \textbf{70} & \textbf{92} \\
	\end{tabular}%
	\label{tab:katmet}%
\end{table}%
Nachfolgend werden die Variablen tabellarisch dargestellt und erläutert. Sämtliche Berechnungsformeln sind im Anhang zu finden. 
%%
\paragraph{Rahmenbedingungen:} Die Variablen dieser Kategorie definieren den Rahmen des Projekts in lokaler, technologischer und organisatorischer Hinsicht. Die Lead \gls{abk:sas} Organisation trägt die Gesamtverantwortung des Projekts. Der Stellenwert am Beispiel für den Geschäftsbereich, drückt aus, auf welche Projekte die Ressourcen konzentriert werden.
% Variablen Rahmenbedingungen
\begin{table}[htbp]
	\centering
	\caption{Variablen der Kategorie Rahmenbedingungen}
	\begin{tabular}{ll}
		\toprule
		\textbf{Code} & \textbf{Name der Variable} \\
		\midrule
		CuNo  & Kundenummer \\
		CuName & Kundenname \\
		EquLoc & Land, in dem die Anlage gebaut wird \\
		Region & Region \\
		BA    & Geschäftsbereich \\
		BU    & Geschäftseinheit \\
		MS    & Marktsegement \\
		BAImportPr & Stellenwert des Projekts im Geschäftsbereich \\
		BUImportPr & Stellenwert des Projekts in der Geschäftseinheit \\
		MSImportPr & Stellenwert des Projekts im Marktsegment \\
		LeadSASPr & Lead SAS Projekt\\
		\bottomrule
	\end{tabular}%
	\label{tab:rahm}%
\end{table}%
\paragraph{\gls{abk:sq}-Variablen:} Die Tabelle \ref{tab:sqvar} zeigt alle Variablen, die mit dem Verkaufsprozess zusammenhängen. Der Output des \gls{abk:sq}-Prozess ist zugleich der Input des Projektabwicklungsprozess (s. Kapitel \ref{sec:pmbueh}), weshalb der Projekterfolg auch von den Entscheidungen der involvierten Parteien im \gls{abk:sq}-Prozess beeinflusst wird. Das Alter und die Betriebszugehörigkeit sind stellvertretende Variablen der Erfahrung und der Kenntnisse der Bühler Prozesse. Die Differenz zwischen dem realisierten und budgetierten Auftragsvolumen der Region und Geschäftseinheit sind ein Indiz für  Verkaufsdruck des Area Manager bei Projektabschluss. Dieser kann den Verkauf risikoreicher Projekte begünstigen, da die Beurteilung des Verkaufsmanager an den \gls{abk:OR} gekoppelt ist. Die Zeitverzögerung zwischen der Auftragsfreigabe und dem Projektbeginn kann als Hinweis für Unklarheiten und Unsicherheiten bezüglich des Projekts zwischen dem Verkaufs- und Projektmanager interpretiert werden kann.
% Tabelle SQ Variabln
\begin{table}[htbp]
	\centering
	\caption{Variablen der Kategorie Sales \& Quotation}
	\begin{tabular}{ll}
		\toprule
		\textbf{Code} & \textbf{Name der Variable} \\
		\midrule
		AM    & Name des Area Manager \\
		AMNo  & Personalidentifikationsnummer des Area Manager  \\
		AMAge & Alter des Area Manager  \\
		AMTen & Betriebeszugehörigkeit Area Manager  \\
		ORDate & Datum der Auftragsfreigabe \\
		PrStartDate & Projektstartdatum \\
		BUORBudGapAbs & Budgetabweichung des Auftragsvolumen des Geschäftsbereich absolut \\
		BUORBudGapRel & Budgetabweichung des Auftragsvolumen des Geschäftsbereich relativ \\
		RegiORBudGapAbs & Budgetabweichung des Auftragsvolumen der Region absolut \\
		RegiORBudGapRel & Budgetabweichung des Auftragsvolumen der Region relativ \\
		\bottomrule
	\end{tabular}%
	\label{tab:sqvar}%
\end{table}%

\paragraph{FF-Variablen:}  Die Tabelle \ref{tab:ffvar} beinhaltet sämtliche  Variablen in Bezug auf den Projektmanager und das Forecast Management. Das Alter und die Betriebszugehörigkeit des Projektmanagers sind wie bereits beim Area Manager stellvertretende Variablen für die Berufserfahrung und das unternehmensspezifische Wissen. Der PMChange misst, ob der Projektmanager während der Projektlaufzeit gewechselt werden musste.
\newline Im Rahmen des Projektcontrolling macht der Projektmanager monatlich eine Prognose (Forecast) in Bezug auf die Kosten- und Umsatzentwicklung. Da Verschlechterung der Kostenperformance relativ spät kommuniziert werde, misst CostMostnegFCadj zu welchem Zeitpunkt während der Projektlaufzeit die negativste Anpassung des Kosten FC gemacht wurde. Negativ impliziert dabei die Generierung von Mehrkosten. CostFCajd beurteilt, ob ein Forecast gemacht wurde unterteilt sie in drei Gruppen: Kosten und Umsatz angepasst, weniger Kosten und mehr Kosten. Das Ampelsystem der Bühler AG wurde in der Einleitung kurz erläutert. Es zeigt auf einer dreifarbigen Skala an, welcher Projektstatus künftig erwartet wird. HOMYellCost mit deshalb, wann das Projekt zum ersten Mal seit Projektbeginn den Status Gelb erhalten hat. Die organisatorische Verantwortung für das ganze Projekt und den Abwicklungsprozess, kann bei einer Gesellschaft oder zwei verschiedenen Gesellschaften angesiedelt sein. Die zusätzliche Schnittstelle und die Trennung der Verantwortlichkeiten kann den Projekterfolg wirken.
% F-Variablen
\begin{longtable}[ht]{p{0.25\textwidth}p{0.7\textwidth}}
 	\caption{Variablen der Kategorie Fulfillment}\\
 	\toprule
		\textbf{Code} & \textbf{Name der Variable} \\ 
		\midrule
		\endfirsthead 
		PM    & Name des Projektmanager \\
		PMAge2 & Alter des Projektmanager \\
		PMTen2 & Betriebszugehörigkeit des Projektmanager  \\
		PMNo  & Identifikationsnummer des Projektmanager  \\
		PMChange & Wechsel des Projektmanager \\
		NoPM  & Anzahl Projektmanager während der Laufzeit \\
		LeadSAS.PrFF & Leas SAS Projekt unterscheidet sich von Lead SAS Projektabwicklung \\
		NoLeadSASFF & Anzahl involvierter SAS bei der Projektabwicklung\\
		CostFCadj & Anpassung des letzten Kosten FC \\
		CostFCadjMS & Anpassung des letzten Kosten FC von MeS \\
		CostFCadjME & Anpassung des letzten Kosten FC von ME \\
		CostFCadjPA & Anpassung des letzten Kosten FC von PA \\
		CostFCadjIS & Anpassung des letzten Kosten FC von IS \\
		CostFirstadj & Anzahl Monate zwischen der ersten  negativen Anpasung des Kosten Forecast (FC) und dem Projektende \\
		CostMostnegFCadj & Anzahl Monate zwischen der negativsten Anpasung des Kosten FC und dem Projektende \\
		CostMostnegFCadjMS & Anzahl Monate zwischen der negativsten Anpasung des Kosten FC  von MeS und dem Projektende  \\
		CostMostnegFCadjME & Anzahl Monate zwischen der negativsten Anpasung des Kosten FC  von ME und dem Projektende  \\
		CostMostnegFCadjPA & Anzahl Monate zwischen der negativsten Anpasung des Kosten FC  von PA und dem Projektende  \\
		CostMostnegFCadjIS & Anzahl Monate zwischen der negativsten Anpasung des Kosten FC  von IS und dem Projektende  \\
		HOMYellCost & Anzahl Monate zwishen HOM und dem ersten gelben Kostenstatus in Relation zur erreichten Projektlaufzeit \\
		HOMYellQual & Anzahl Monate zwishen HOM und dem ersten gelben Qualitätsstatus in Relation zur erreichten Projektlaufzeit \\
		HOMYellTime & Anzahl Monate zwishen HOM und dem ersten gelben Zeitstatus in Relation zur erreichten Projektlaufzeit \\
		HOMRedCost & Anzahl Monate zwishen HOM und dem ersten roten Kostenstatus in Relation zur erreichten Projektlaufzeit \\
		HOMRedQual & Anzahl Monate zwishen HOM und dem ersten roten Qualitätsstatus in Relation zur erreichten Projektlaufzeit \\
		HOMRedTime & Anzahl Monate zwishen HOM und dem ersten roten Zeitstatus in Relation zur erreichten Projektlaufzeit \\
		\bottomrule
	\label{tab:ffvar}%
\end{longtable}%
%%
\paragraph{Komplexität:} Die Komplexität eines Projekts kann unterschiedliche Dimensionen betreffen, so zum Beispiel können die technische Anforderung an die Anlage, die Anzahl involvierter Parteien, die Zusammenarbeit mit externen Partnern sowie neuartige Prozesse den Komplexitätsgrad eines Projekts erhöhen. Die Tabelle \ref{tab:covar} führt die Variablen zur Abbildung der Komplexität auf. Die Anzahl Aufträge soll die Überisichtlichkeit bei der Projektabwicklung erfassen, allerdings wurde keine interne Bestimmungen, wann der Auftrag gesplittet werden muss. Die Anzahl involvierte Parteien, wurde mit Anzahl Zulieferer während der Projektphase approximiert. Der Zusammenschluss mit einem externen Unternehmen (Konsortium) kann die Komplexität erhöhen.
% Tabelle Komplexität
\begin{table}[htbp]
	\centering
	\caption{Variablen der Kategorie Komplexität}
	\begin{tabular}{ll}
		\toprule
		\textbf{Code} & \textbf{Name der Variable} \\
		\midrule
		ConPart & Konsortium \\
		NoSupplSAS & Anzahl zuliefernder Sales and Service Unternehmen (SAS) \\
		NoSupplSASMS & Anzahl zuliefernder SAS Mechnical Supply (MeS) \\
		NoSupplSASME & Anzahl zuliefernder SAS Mechnical Engineering (ME) \\
		NoSupplSASPA & Anzahl zuliefernder SAS Plant and Automation (PA) \\
		NoSupplSASIS & Anzahl zuliefernder SAS Installation (IS) \\
		NoContr & Anzahl Aufträge \\
		\bottomrule
	\end{tabular}%
	\label{tab:covar}%
\end{table}%
\paragraph{Kosten:} Die monetären Aspekte eines Projekts in der Tabelle \ref{costvat} umfassen die budgetierten und realisierten Zahlen von Umsatz, Kosten, Marge, in absoluten und relativen Grössen. Die Abweichung der realisierten Kosten vom letzten Kostenforecast vor dem Projektende kann als Indiz, des Zeitmanagement der Prognosenpassung interpretiert werden. 
% Tabelle Kosten
\begin{table}[htbp]
	\centering
	\caption{Variablen der Kategorie Kosten:}
	\begin{tabular}{ll}
		\toprule
		\textbf{Code} & \textbf{Name der Variable} \\
		\midrule
		TOBud & Umsatzbudget \\
		TOAct & Turnover Act \\
		BudMSTot & Anteil der MeS Kosten am Gesamtkostenbduget \\
		BudMETot & Anteil der ME Kosten am Gesamtkostenbduget \\
		BudPATot & Anteil der PA Kosten am Gesamtkostenbduget \\
		BudISTot & Anteil der IS Kosten am Gesamtkostenbduget \\
		DB1Bud & budgetierte  Projektmarge (DB1-Marge) \\
		DB1Act & realisierte DB1-Marge \\
		DB1Budabs & absolutes DB1 Budget (Bud) \\
		DB1Actabs & absolute DB1 Actual (Act) \\
		SUCostTO & Kosten aus Nachlieferung im Verhältnis zum Umsatz \\
		CostActBudMSabs & absolute Kostenabweichung vom Bud der MeS Kosten \\
		CostActBudMEabs & absolute Kostenabweichung vom Bud der ME Kosten \\
		CostActBudPAabs & absolute Kostenabweichung vom Bud der PA Kosten \\
		CostActBudISabs & absolute Kostenabweichung vom Bud der IS Kosten \\
		CostActBudRel & relative Kostenabweichung  der Projektkosten \\
		CostActBudMSRel & relative Kostenabweichung der MeS Kosten \\
		CostActBudMERel & relative Kostenabweichung der ME Kosten \\
		CostActBudPARel & relative Kostenabweichung der PA Kosten \\
		CostActBudISRel & relative Kostenabweichung der IS Kosten \\
		DeltaLastFCAct & Kostenabweichung zwischen dem letzten FC und Act des Projekts \\
		DeltaLastFCActMS & Kostenabweichung zwischen dem letzten FC und Act von MeS \\
		DeltaLastFCActME & Kostenabweichung zwischen dem letzten FC und Act von ME \\
		DeltaLastFCActPA & Kostenabweichung zwischen dem letzten FC und Act von PA \\
		DeltaLastFCActIS & Kostenabweichung zwischen dem letzten FC und Act von IS \\
		\bottomrule
	\end{tabular}%
	\label{tab:addlabel}%
\end{table}%
%%
\paragraph{Zeit:} Die Variablen der Tabelle \ref{zeitvar} messen sämtliche Zeitverzögerungen in Bezug auf die gesamte Projektlaufzeit und einzelner Meilensteine: MS2 \glqq Concept approved \grqq{ }, MS5 \glqq Point of no return \grqq{ }, MS8 \glqq Documented \grqq{ }, MS10 \glqq Takeover\grqq{ } und MS11 \glqq Project Closure \grqq{ }.
% Table generated by Excel2LaTeX from sheet 'Ch3'
\begin{table}[htbp]
	\centering
	\caption{Variablen der Kategorie Zeit}
	\begin{tabular}{ll}
		\toprule
		\textbf{Code} & \textbf{Name der Variable} \\
		\midrule
		PrTimeBase & geplante Projektlaufzeit \\
		PrTimeAct & erreichte Projektlaufzeit \\
		PrTimeDelay & Zeitverzögerung bei Projektabschluss \\
		PrTimeDelayMS2 & Zeitverzögerung bei MS2 \\
		PrTimeDelayMS5 & Zeitverzögerung bei MS5 \\
		PrTimeDelayMS8 & Zeitverzögerung bei MS8 \\
		PrTimeDelayMS10 & Zeitverzögerung bei MS10 \\
		PrTimeDelayMS11 & Zeitverzögerung bei MS11 \\
		\bottomrule
	\end{tabular}%
	\label{tab:zeitvar}%
\end{table}%

%%Datenanalyse: Finanzielle Analyse
%%Datenanalyse: Häufigkeitsverteilung etc.
%%Datenanalyse: Hpyothesen
\section{Datenanalyse}\label{sec:dataana}
Obwohl die Performance eines Bühler-Projekts mittels des magischen Dreiecks - Zeit, Kosten und Qualität - beurteilt wird, hat letztendlich der Kostenaspekt aus finanzieller Perspektive die relativ gewichtigere Bedeutung als die anderen zwei Dimensionen. Deshalb wurde in Zusammenarbeit mit der Bühler AG, das folgende Erfolgskriterium festgelegt:
\begin{equation}
\text{Deviation DB1 \%} = \text{DB1 Act \%} - \text{DB1 Bud \%}
\end{equation}
Der KPI rechnet sich realisierte Marge (DB1 Act) in \% minus budgetierte Marge (DB1 Bud) in \% und wird jeweils am Projektende respektive nach Erreichung des MS11 (Kapitel \ref{zweieins}) kalkuliert. Die Kosten der Garantieperiode werden nicht direkt dem Projekt belastet sondern summarisch in einem anderen buchhalterischen Konto. Bei jedem Projekt wird präventiv ein Kostenpuffer im Bereich von 4\% bis 9\% einkalkuliert, der nach dem Projektende (MS11) bei einer Nullbeanspruchung im Projektergebnis realisiert wird.
\newline Die nachfolgende Abbildung \ref{Einflussfaktoren} zeigt die ursprünglich 93 Determinanten eingeteilt in sechs Kategorien. Gewisse Faktoren könnten ihrer Natur nach auch in eine andere Gruppe gegliedert werden. In der Folge wird die Bedeutung und Relevanz der Variablen erklärt.

\newpage


Nach der ersten Datenexploration und Prüfung der Annahmen für lineare statistische Modelle, wurde festgestellt, dass die ursprünglich geplante Methodenwahl nicht angewendet werden konnte. Denn die unabhängigen Variablen hatten geringe bis keine Korrelation mit der abhängigen Variable sprich dem Erfolgskriterium. Die lineare Variablentransformationen und andere Methoden, um eine Verteilungskurve zu simulieren, führten nur zu kleineren Verbesserung. Dieser Umstand und die Tatsache, dass Erfolgsfaktoren bereits sehr gut erforscht wurden, hat die Entscheidung auf Inferenzstatistik zu verzichten bestärkt. Die nachfolgende Analyse ist deshalb deskriptiver Natur und hat ausserdem das Ziel, die finanziellen Einbussen von sogenannten nicht-erfolgreichen Projekten zu untersuchen. Die Aussagekraft der Ergebnisse wird dadurch so eingeschränkt, dass da keine Rückschlüsse auf die Grundgesamtheit (sämtliche Projekte der Bühler AG) gemacht werden können.  Die Ergebnisse haben nur in Bezug auf die die untersuchte Stichprobe Gültigkeit. Es ist jedoch denkbar, auf Basis der Ergebnisse neue Hypothesen zu formulieren, welche mittels anderer, geeigneter statistischer Methoden geprüft werden könnten. Die erstmalige Auswertung der Projektdaten kann zudem Erkenntnisse zu möglichen Charakteristiken nicht-erfolgreicher Projekte liefern.
\newline Das Erfolgskriterium (DB1BudDev) wurde in Zusammenarbeit mit der Bühler AG festgelegt. Aus finanzieller und interne Perspektive ist die Abweichung der relativen Projektmarge (DB1Act) vom Budget (DB1Bud) von zentraler Bedeutung. Denn sowohl die Finanzziele wie auch die Incentivierung der Projekt- und Verkaufsmanager sowie der Geschäftsbereichsleitung basieren auf dem DB1 und den entsprechenden Budgetvorgaben. Die relative Projektmarge errechnet sich aus Umsatz minus Kosten in Relation zum Umsatz. Anhand der Differenz zwischen Act und Bud wird der Erfolg ($Differenz \geq 0$) respektive Nicht-Erfolg ($Differenz < 0$) von Projekten gemessen. Der DB1BudDev wurde zu Analysezwecken in eine binäre Variable (Success) transformiert. Daraus folgt, dass alle positiven (negativen) Differenzen als erfolgreiche (nicht-erfolgreiche) Projekte betrachtet werden. Im Folgenden werden erfolgreiche Projekte und Success-Projekte sowie nicht-erfolgreiche Projekte und Fail-Projekte als Synonyme verwendet. Obwohl retrospektiv Erkenntnisse und Erfahrungen aufgrund der Durchführung eines Projekts einen Gewinn für das Unternehmen darstellen können, wird diesem Aspekt in dieser Analyse nicht Rechnung getragen.
\newline\newline\textbf{Datenaufbereitung:} Der Rohdatensatz enthält sämtliche Daten zu den Faktoren der untersuchten Projekte (Stichprobe). Er setzt sich aus drei Datensätzen zusammen, die separat aus dem Bühler-System extrahiert wurden. Der Stichprobenumfang beträgt $N = 1471$ und die Anzahl Faktoren $i = 93$. Das Alter (PMAge und AMAge) und die Betriebszugehörigkeit (PMTen und AMTen) der Projekt- und Areamanager mussten korrigiert werden, da der ursprüngliche Datensatz die Unterscheidung zwischen fehlenden Werten und Nullwerten nicht zu liess.
\newline
Es wurden alle unplausiblen Faktoren und Berechnungsspalten vom Datensatz entfernt. Anschliessend wurde die Anzahl fehlender Daten pro Faktor ausgewertet und sämtliche Determinanten mit mehr als 300 fehlender Datensätze von der weiteren Analyse ausgeschlossen (vgl. Tabelle \ref{tab:na}). Zudem verbleibt nebst PrTimeDelayMS5, AMAge2 und AMTen2 die Variable AMNo unberücksichtigt, da durch den Ausschluss der verbundenen Faktoren wenig zusätzlicher Informationsgewinn erwartet wird. Ausserdem mussten alle Variablen, welche die Zeitdifferenz zwischen dem letzten Kostenforecast und dem Projektende messen (CostMostnegFCadj für \gls{abk:MS}, \gls{abk:ME}, \gls{abk:PA} und \gls{abk:IS}) sowie Indikatoren für die relative Wichtigkeit eines Projekts (BAImportPr, BUImportPr, MSImportPr) aufgrund fragwürdiger Plausibilität und Korrektheit der Daten von der Analyse ausgeschlossen werden. Mittels diesem Vorgehen kann der Datenverlust infolge fehlender Daten in Grenzen gehalten werden. Der neue Stichprobenumfang beträgt $N = 1076$ und die Anzahl Faktoren $i = 71$.
\newline
Im Anschluss wurden die Datensätze auf ihre Plausibilität getestet und Ausreisser entfernt. Die Plausibilitätsüberlegungen basieren auf der logischen Interpretation und Herleitung der Indikatoren. Die Tabellen mit den Begründungen der unplausiblen Werte und Ausreisser befinden sich im Anhang. Die Outliers wurden mit Hilfe von Boxplots, Histogramme und der 'Interquartile Ranges' (IQR) der numerischen Variablen identifiziert. Zur quantitativen Bestimmung der Ausreisser wurde folgendes Entscheidungskalkül angewendet:
\newline\newline
Ausreisse sind Werte, die $< Q1 - 1.5 * IQR$ respektive $ > Q3 + 1-5 * IQR$ sind.
\newline
Extreme Ausreisser sind Werte, die $< Q1 - 3 * IQR$ respektive $> Q3 + 3 * IQR$ sind.
\newline\newline
Je nach Zweck der Analyse und untersuchten Objekten sind Ausreisser unterschiedlich einzustufen. Die Geschäftsbereiche der Bühler AG verkaufen unterschiedliche Anlangen, weshalb die Datenbereiche der Faktoren stark variieren können. Die realisierte Projektmarge (DB1Act), wurde auf die Werte des doppelten IQR berichtigt, da extrem negative Margen auf sogenannte Crash-Projects schliessen lassen, welche bereits mittels internem Audit untersucht wurden und die Stichprobenergebnisse unnötige verzerren können. Extreme positive DB1Act sind bei einer durchschnittlichen Projektmarge von ca. 30\% relativ unwahrscheinlich und lassen Zweifel zur Richtigkeit der Kostenverbuchung zu. Bei den relativen Kostenabweichungen für PA und IS wurden jeweils einzelne Extremalwerte nur dann entfernt, wenn kein entsprechendes Budget geplant wurde. Denn es wurde davon ausgegangen, dass die Budgetierung der Projektkosten nicht korrekt verlaufen ist, was letztendlich zu extremalen relativen Kostenabweichung geführt hat. Es wurden keine weiteren Ausreisser eliminiert, selbst wenn einige Werte ausserhalb des Entscheidungskalküls lagen. Nach der Datenbereinigung umfasst die zu untersuchende Stichprobe $N = 966$ Projekte und $ i = 71$ Faktoren.
\newline\newline
%%
%%Zusätliche Variablen
%%
\textbf{Zusätzliche Variablen:} Nach dem Datenbereinigungsprozess wurde zu analytischen Zwecken vor allem kategoriale Variablen auf Basis der vorhandene Daten erhoben (mit * gekennzeichnet). Die nachfolgende Tabelle zeigt sämtliche alle in der Auswertung berücksichtigten Faktoren nach der Variablenkategorie strukturiert. Sämtliche Berechnungsformeln sowie die Interpretationen der Faktoren sind im Anhang enthalten.



