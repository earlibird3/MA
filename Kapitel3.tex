% !TEX root = MA.tex
\section{Analye der Erfolgsfaktoren des Bühler Projektmanagements}\label{drei}
Basierend auf der vorangehende Literaturrecherche kann mit Leichtigkeit der Eindruck gewonnen werden, dass eine weitere Analyse der Faktoren, welche den Erfolg von Projekten beeinflussen, keine neuen Erkenntnisse liefern würde. Für eine Vielzahl der berücksichtigen Studien gründen weitergehende statistische Analysen auf einer anfänglichen Wertung von Erfolgsattributen durch Personen, die in der entsprechenden Industrie oder Projektmanagement tätig sind. Die erforschten Faktoren wurden vorgängig jeweils aus früheren Studien extrahiert. Der Zusammenhang zwischen dem Erfolg von Projekten, der entweder als binäre Ausprägung oder als indexiertes Kriterium repräsentiert war, und den unabhängigen Erfolgsattributen wurde mittels der entsprechender Regressionsanalysen erforscht. Die gewählte Methode der Likert-Skala führte jeweils dazu, dass die erforderlichen Annahmen für eine Regression oder Faktorenanalyse gegeben waren. Da sich die Ergebnisse zu einem Teil überschneiden, kann postuliert werden, dass unabhängig vom gewählten Performancekriterium ein gewisser Konsens bezüglich der Erfolgsfaktoren existiert. Diese Aussage ist mit Vorsicht zu geniessen, da die Studien nicht eins zu eins miteinander verglichen werden können, weshalb sie als Annahme formuliert wurde. Ausserdem lässt sich aus den betrachteten Forschungsberichten schliessen, dass keine unternehmensspezifische Daten respektive unternehmensbezogene Daten zu den Erfolgsattributen erhoben wurden beziehungsweise für die Analyse herangezogen wurde.\newline
Die nachfolgende Analyse wird sich aufgrund der Daten und Fixierung eines bestimmten Performancekriteriums sowie dem Fokus dieser Arbeit von bisherigen Analysen unterscheiden. 
\subsection{Daten und statistische Methoden}
In Kapitel \ref{zweizwei} wurde beschrieben, dass die Bühler AG die zu evaluierenden Faktoren auf Basis bisheriger Erfahrungen aus einer finanziellen Perspektive identifiziert und auch neue Indikatoren geschaffen hat. Der Rahmen für die Datenerhebung bildete die Datenverfügbarkeit des BPM-Cockpits und des SAP. In der Folge wurde für die Datenextraktion ein eigene Query geschaffen, die sämtliche Faktoren pro Projekt abbildet. Trotz mehrfacher Validierung der Daten, konnte nach Erreichung des Fertigstellungstermin keine vollständige Korrektheit vor allem einiger berechneter Indikatoren gewährleistet werden. Die Stichprobe enthält alle abgeschlossenen Projekte im Zeitraum zwischen 2013 und 2015. Das eindeutige Abgrenzungskriterium bilden hierbei der Projektstatus und das Datum des MS11 Start und Übergabe. Zuerst wurden alle Projekte mit einem MS11-Datum zwischen dem 1.1.2013 und dem 31.12.2015 eingegrenzt. Der Projektstatus stellte sicher, dass das Projekt auch aus finanzieller Sicht als abgeschlossen betrachtet werden kann, da gewisse Projekte MS11 bereits erreicht haben können, aber fehlende Rechnungen noch zu verbuchen sind.
\newline Das ursprüngliche beabsichtigte Analysemodell orientierte sich an den bisherigen Studien und hätte sich aus einer Faktorenanalyse zur Reduktion der Anzahl Faktoren mit anschliessender Regressionsanalyse zur Bestimmung der Abhängigkeiten, zusammengesetzt. Allerdings konnte bei der Prüfung der Modellvoraussetzungen die zwingende Linearitätsannahme zwischen der abhängigen und den unabhängigen Variablen nicht zufriedenstellend erfüllt werden. Selbst eine entsprechende lineare Variablentransformationen hätte die Linearitätsannahme nicht besser erfüllt. Dementsprechend mussten lineare statistische Modelle von den möglichen Analysemethoden ausgeschlossen werden. Aus diesem Grund und der Tatsache, dass Erfolgsfaktoren bereits sehr gut erforscht wurden, hat dazu beigetragen, dass sich die Analyse im Bereich der deskriptiven Statistik bewegt. Darüber hinaus sollen soll mittels explorativer Analysen, ein Teilgebiet der Datenanalyse, Strukturen respektive neue Hypothesen zu möglichen Erfolgsfaktoren formulieren werden. Die Aussagekraft der Ergebnisse wird mit dem Verzicht auf die Anwendung der Inferenzstatistik insofern eingeschränkt, da keine Rückschlüsse auf die Grundgesamtheit (sämtliche Projekte der Bühler AG) gemacht werden können. Allerdings können Aussagen und Vermutungen bezüglich der Stichprobe gemacht werden. Demzufolge kann die nachfolgenden Analyse auch als expost-Analyse betrachtet werden. Ziel dieser expost-Betrachtung aus finanzieller Perspektive ist einerseits einen genauerer Untersuchung des monetären Margenverlusts. Hierbei ist die Ausprägung der Faktoren der nicht erfolgreichen Projekte von zentralem Interesse. Das zweite Ziele ist die Faktoren zu beschreiben, um anschliessend Hypothesen für mögliche Erfolgsfaktoren der Projekte der Bühler AG zu formulieren. Als Basis dienen die Vermutungen und Einschätzungen pro Faktor der Bühler AG, welche zusammen mit den Faktoren ergründet wurden. Diese Vorgehensweise ermöglicht das bisherige Datenmodell zu prüfen und ergänzende Faktoren zu finden. Da diese Analyse die erste ihrer Art für die Bühlerprojekte ist, kann sie zudem wertvolle Hinweise zu Projekten und allenfalls möglichen Problemfeldern liefern.
\newline Wie bereits in Kapitel \ref{zweizwei} erwähnt wurde, bildet das Erfolgskriterium die Abweichung des realisierten vom budgetierten Deckungsbeitrag eines Projekts. Sie stellt die finanzielle Perspektive eines Projekts dar und hat direkten Einfluss auf das Ergebnis eines Geschäftsbereich. Ausserdem hängt die variable Vergütung der Projektmanager und Verkaufsmanager vom realisierten DB1 ab. In erster Linie wurde die Logik Erfolg (über Budget) Fail (unter Budget) angewandt. Allerdings wurde für einige Analysen und Darstellungen die Ampellogik der Kosten des BPM-Cockpits angewandt (s. Kapitel \ref{zweizwei}).
\newline
\newline\textbf{Rohdaten}
\newline Die Rohdaten setzen sich aus drei verschiedenen Datensätzen zusammen. Im grössten Datensatz sind sämtliche Faktoren enthalten, welche mittels BPM Cockpit erhoben wurden. Dieser Datensatz wurde um Umsatz- und Margenwerten und die Regionenzuordnung eines Projektes ergänzt. Des weiteren wurden Daten zum Alter und der Betriebszugehörigkeit der Projekt- und Verkaufsmanager manuell auf Basis der intern ermittelten Daten hinzugefügt, da dieser Datensatz die Unterscheidung zwischen fehlenden Werten und Nullwerten zuliess. Von diesem Datensatz wurden einerseits Berechnungsvariabeln und anderseits doppelte Variablen sowie fehlerhafte Variabeln entfernt (siehe Anhang). Der Rohdatensatz ist folglich bereits ein korrigierter Datensatz, der einen Stichgrössenumfang von $N = 1471$ und $i = 93$ Indikatoren enthält. Vor der deskriptiven Analyse wurde auf Basis dieses Rohdatensatzes, fehlende Werte, die Datenplausibilität sowie Ausreisser untersucht. Da bei der grossen Anzahl von Variablen, die Komplettheit eines Datensatzes, das heisst, dass alle Faktoren konnten für ein Projekt erhoben werden konnte, schwierig zu erreichen ist, wurde der Fokus darauf gelegt, möglichst viele Datensätze in der Stichprobe zu erhalten. In der nachfolgenden Tabelle sind sämtliche Variablen mit fehlenden Werten inklusiver der Anzahl fehlender Daten aufgelistet. Grundsätzliche wurde sämtliche Indikatoren, welche mehr als $200$ NA's aufweisen von der Analyse ausgeschlossen. AMNo wurde zudem ausgeschlossen, da ausser einem direkten Personenbezug keine weitere Informationen bezüglich der Verkaufsmanager zu erwarten ist, da bereits AMAge2 und AMTen2 ausgeschlossen werden mussten. Einzig könnte eine separate Analyse der Verkaufsmanager gemacht werden, um sie auf ihre Performance und Auslastung hin untersuchen zu können. CostMostnegFCadj wurde ebenso von der Analyse ausgeschlossen, da eine Einzelauswertung auf Kostenträgerebene sinnvoller erscheint. Die Eliminierung der fehlenden Daten führt dazu, dass die Stichprobe von $N = 1471$ auf $N = $ sank und die Variablenanzahl sich von $i = 93$ auf $i = $ einschränkte.
\newline
\begin{table}[h]
	\centering
	\caption{Anzahl NA's per variable}
	\begin{tabular} {| l| l | p{6cm} |}
		\textbf{Variable Code} & \textbf{Anz. NA} & \textbf{Handhabung} \\\hline
		CostMostnegFCadjPA & 749   & von der Analyse ausgeschlossen \\
		PrTimeDelayMS5 & 538   & von der Analyse ausgeschlossen \\
		CostMostnegFCadjIS & 537   & von der Analyse ausgeschlossen \\
		AMAge2 & 444   & von der Analyse ausgeschlossen \\
		AMTen2 & 444   & von der Analyse ausgeschlossen \\
		PrTimeDelay & 254   &  \\
		PrTimeDelayMS11 & 227   &  \\
		PrTimeDelayMS10 & 214   &  \\
		PrTimeAct & 192   &  \\
		CostMostnegFCadj & 177   & von der Analyse ausgeschlossen \\
		PrTimeDelayMS2 & 156   &  \\
		PrTimeDelayMS8 & 139   &  \\
		AMNo  & 132   & von der Analyse ausgeschlossen \\
		PrTimeBase & 118   &  \\
		PMAge2 & 98    &  \\
		PMTen2 & 98    &  \\
		CostFirstadj & 61    &  \\
		PrStartDate & 13    &  \\
		PMNo  & 6     &  \\
		BA    & 6     &  \\
		BU    & 6     &  \\
		TOAct & 6     &  \\
		DB1Budabs & 6     &  \\
		DB1Actabs & 6     &  \\
		EquLoc & 2     &  \\	
	\end{tabular}
\end{table}
\newline
\newline\textbf{Plausibilität:} Die Plausibilitätsüberlegungen basieren auf der logischen Interpretation und Herleitung der Indikatoren. Die Tabelle im Anhang zeigt, welche Werte für die Indikatoren möglich sind. Die Garantierung der Plausibilität hat direkten Einfluss auf die Stichprobengrössen. Diese schrumpfte infolge der Anpassungen auf $N = $  (Tabelle im Anhang EINFÜGEN).
\newline Im Anschluss wurden die Daten auf mögliche Aussreisser hin untersucht. Bei dieser Untersuchung wurde das Augenmerk auch auf den Erhalt möglichst vieler Datensätze gelegt. Allerdings ist die Entfernung der Ausreisser auch in der deskreptiven Statistik von grosser Bedeutung, da die Aussagekraft der Ergebnisse beeinträchtgt werden kann. Die Ausreisser wurden anhand Boxplots, Histogramme und der 'Interquartile Ranges' der numerischen Variablen identifiziert.  Vorab muss erläutert werden, dass der Wert $1'111'111$ bei den CostMostnegFCajd-Variablen keinen Ausreisser darstellt sondern angibt, dass das Projekt nur positive FC-Anpassungen gehabt hat. Dies impliziert, dass der Forecast für die Kosten gesunken sind und somit weniger Kosten erwartet wurden, wobei der Umsatz konstant geblieben oder gestiegen ist. Bei den HOM-YellowStatus und die HOM-RedStatus drückt der Wert $1'111'111$ aus, dass der entsprechende Status nicht als erstes oder gar nicht aufgetreten ist. Die Interpretation wäre somit, dass bei den HOMYellow-Status Variablen der Status immer grün war oder zuerst respektive direkt den roten Status hatte. Eine ähnliche Interpretation gilt für HOMRedStatus-Variablen, somit hätte dieses Projekt, den roten Status gar nicht erst erreicht. Da diese Interpretationen valide sind und somit keine fehlende Werte darstellen, verbleiben sie im Datensatz. 
\newline
Die Bestimmung von extremen Werten ist stichprobenabhängig und je nach Zweck der Analyse und untersuchten Objekten sind Ausreisser unterschiedlich einzustufen. Ziel der nachfolgenden Analyse ist es, sämtliche Projekte der Bühler AG auf ihre Erfolgsfaktoren und finanzielle Performance untersuchen. Die Geschäftsbereiche der Bühler AG verkaufen unterschiedliche Anlangen, weshalb die Datenbereiche der Faktoren stark variieren können. Da aufgrund der unterschiedlichen Anlagen, kein direkter Vergleich zwischen den Daten gemacht werden kann, sind Ausreisser schwierig zu bestimmen. Methodisch wurde der einfache und doppelte Interquartile Range für alle numerischen Variablen berechnet und das folgende Entscheidungskalkül zur Bestimmung von Extremalwerten herangezogen. Anschliessend wurde pro Variable ausgewertet, ob Werte ausserhalb der Grenzwerte des doppelten IQR liegen. In einem solchen Fall wurde je Variable entschieden, ob eine Korrektur um die Ausreisser sinnvoll erscheint. Die argumentative Erklärung pro Variable befindet sich im Anhang. Einzig die realisierte Marge, DB1Act, wurde auf die Werte des doppelten IQR berichtigt, da extreme negative Margen auf sogenannte Crash-Projects schliessen lassen, welche bereits mittels internem Audit untersucht wurden und die Stichprobenergebnisse unnötige verzerren. Extrem positive DB1 sind bei einer durchschnittlichen Projektmarge von ca. 30\% relativ unwahrscheinlich und lassen Zweifel zur Richtigkeit der Kostenverbuchung zu. Bei den  relativen Kostenabweichungen für PA und IS wurden jeweils diejenigen Daten entfernt, welche extrem hoch waren und kein entsprechendes Budget geplant wurde. Hierbei wurde davon ausgegangen, dass die Budgetierung der Projektkosten nicht korrekt verlaufen ist, was letztendlich zu extremalen relativen Kostenabweichung geführt hat. Es wurden keine weiteren Ausreisser eliminiert, selbst wenn einige Werte ausserhalb des Entscheidungskalküls lagen. Die Erklärung pro Faktor sowie die Tabelle mit den IQR befindet sich im Anhang. Die Stichprobengrösse hat sich auf $N = $ veringert. Die Anzahl Faktoren entspricht $ i = $. Im Anschluss folgt die finanzielle Analyse und Untersuchung der Erfolgsfaktoren. 
\newline\newline
\begin{centering}
		$ all avlues \leq Q1 - 1.5 * IQR$
		\newline
		$ all values \geq Q3 + 1-5 * IQR$
\end{centering}
\newline
\newline
\begin{centering}
		$ all avlues \leq Q1 - 3 * IQR$
		\newline
		$ all values \geq Q3 + 3 * IQR$
\end{centering}
\newline\newline
 $N = 883 $ und die Anzahl Variablen $ x = 70$.
\subsection{Ergebnisse und Interpretation}
Im ersten Schritt wurde versucht, der finanzielle Verlust nicht erfolgreicher Projekte zu evaluieren und sukzessive weitere Kriterien in die Analyse einfliessen zu lassen um dadurch Hinweise zu möglichen Einflussfaktoren zu erhalten. Bei der nachfolgenden Analyse handelt es sich um eine post-Analyse der Projekt, deren Kostperformance in Abhängigkeit der identifizieren Erfolgsfaktoren. Es werden einige statistische Methoden zur Analyse hinzugezogen. Es sind Rückschlüsse auf die zugrundeliegende Stichprobe mögliche, jedoch keine Generalisierungen auf zukünftige oder laufende Projekte. Allerdings können Hinweise auf mögliche Erfolgsfaktoren gewonnen werden, deren Relevanz jedoch in einem weiteren Schritt untersucht werden müsste. Die Analyse erfolgt anhand der in Kapitel 2.2 definierten Kategorien. Wobei sich die mögliche Anzahl zu evaluierenden Faktoren aufgrund Plausibilitätkriterien und fehlender Daten verkürzt hat.
\newline Modell mit Faktoren einfügen\newline
\subsubsection{Finanzielle Analyse}
Die Finanzielle Analyse basiert auf dem binären Erfolgskriterium, die prozentuale Abweichung der realisierten von der budgetierten Projektmarge. 
\newline\newline
$Success = (DB1Act-DB1Bud)/DB1Bud \geq 0$\newline\newline
$Fail = (DB1Act-DB1Bud)/DB1Bud < 0$
\newline\newline
Als Vergleich wurde der Erfolgsschlüssel des BPM-Cockpits der Bühler AG hinzugezogen. Das interne Ampelsystem  legt für die Kosten folgenden Schlüssel fest:\newline\newline
$ green = (DB1Act-DB1Bud)/DB1Bud > -4\%$\newline\newline
$ yellow = -4\% \leq (DB1Act-DB1Bud)/DB1Bud > -10\%$\newline\newline
$ red = (DB1Act-DB1Bud)/DB1Bud  \leq -10\%$\newline\newline
Das Ampelsystem wird während des Projektverlaufs angewandt, um die Abweichung zwischen dem FC und dem Budget der Projektmarge zu evaluieren. Der gleiche Schlüssel wurde für die Abweichung zwischen dem DB1 Act und Bud angewandt, um zu evaluieren wie sich die Performance in monetärer Hinsicht und in Bezug auf die Häufigkeit von erfolgreichen und nicht erfolgreichen Projekten verändert. Erfolgreiche Projekte haben demzufolge den Status grün wohingegen gescheiterte Projektee die Status yellow und red haben. Die Anzahl erfolgreicher Projekte wird zwangsläufig ansteigen sowie auch die finanzielle Performance beeinflussen. Insbesondere ist die Veränderung des Margenverlust zwischen erfolgreichen und gescheiterten Projekten ein Interessenspunkt.\newline
Die Stichprobe setzt sich wie folgt zusammen:
\begin{table}[htbp]
	\centering
	\caption{Übersicht Stichprobe}
	\begin{tabular} {l|r|r}
		\textbf{Stichprobe} & \textbf{absolut} & \textbf{relativ} \\
		\textbf{Total} & 928 & 100\% \\
		\textbf{Success} & 624 & 67\% \\
		\textbf{Fail} & 304 & 33\% \\
	\end{tabular}
\end{table}
\newline
Die Erfolgsquote $Anzahl Projekte (Success)/Anzahl Projekte(Fail)$ beträgt 2.05. Daraus folgt, dass jedes dritte Projekte ein Erfolg wird. Gemäss subjektiver Einschätzung ist diese Erfolgsquote relativ gut. Nachfolgend wurde die finanzielle Performance für die Stichprobe berechnet.
2 Table einfügen mit TO Bud, Cost Bud DB1 Bud, DB1Bud, DB1 Act, Cost Act, TOBudabs, DB1Budabs, CostBudDevabs.
\begin{table}[htbp]
	\centering
	\caption{Finanzieller Verlust: Binäres Erfolgskriterium }
	\begin{tabular}{lrrrrrrrrrrrr}
		\textbf{Erfolgskriterium} & \multicolumn{1}{l}{\textbf{TO Bud}} & \multicolumn{1}{l}{\textbf{Cost Bud}} & \multicolumn{1}{l}{\textbf{DB1 Bud}} & \multicolumn{1}{l}{\textbf{DB1 Bud [\%]}} & \multicolumn{1}{l}{\textbf{TO Act}} & \multicolumn{1}{l}{\textbf{Cost Act}} & \multicolumn{1}{l}{\textbf{DB1 Act}} & \multicolumn{1}{l}{\textbf{DB1 Act [\%]}} & \multicolumn{1}{l}{\textbf{TO Dev}} & \multicolumn{1}{l}{\textbf{Cost Dev.}} & \multicolumn{1}{l}{\textbf{DB1 Dev.}} & \multicolumn{1}{l}{\textbf{DB1 Dev. [\%]}} \\
		SUCCESS & 1'515'433 & -1'129'938 & 385'494 & 25.4\% & 1'522'924 & -1'018'031 & 504'893 & 33.2\% & 7'491 & 111'908 & 119'399 & 7.7\% \\
		FAIL  & 615'509 & -463'513 & 151'996 & 24.7\% & 629'118 & -525'333 & 103'785 & 16.5\% & 13'609 & -61'820 & -48'211 & -8.2\% \\
		Grand Total & 2'130'942 & -1'593'451 & 537'490 & 25.2\% & 2'152'042 & -1'543'364 & 608'679 & 28.3\% & 21'101 & 50'088 & 71'188 & 13.2\% \\
	\end{tabular}%
	\label{tab:addlabel}%
\end{table}%
\newline Das budgetierte Umsatzvolumen der nicht erfolgreichen Projekte beträgt ca. einen Drittel. Die Abweichung des realisierten vom budgetierten Umsatz ist positiv, wobei die Fail-Projekte ca. zwei Drittel der Abweichung ausmachen. Ein möglicher Grund hierfür könnten Projektanpassungen sein, die dem Kunden verrechnet wurden und gleichzeitig mehr Kosten verursachten. Die Kostenabweichung beträgt ca. 62 Mio. CHF (ca. 10\%). Die Success-Projekte weisen eine Gewinn über den gesamten Projektkosten aus von ca. 112 Mio. CHF. Dieser hohe Kostengewinn könnte damit erklärt werden, dass er Kostenpuffer nicht gebraucht wurde und somit das Projektergebnis positiv beeinflusst. Die Fail-Projekte haben folglich die Kostenreserve ausgeschöpft und die Kosten zusätzlich überzogen, der effektive Kostenverlust müsste also höher sein als 62 Mio. CHF. (mit Mary besprechen). Die realsierte Marge über alle Success-Projekte beträgt 33\% und liegt 7.7\% über der budgetierten Marge von 25.4\%. Demgegenüber beträgt der DB1Act der Fail-Projekte 16.5\% und liegt 8.2\% unter dem DB1 Bud von 24.7\%. Auffallend an diesem Ergebnis ist, dass die summierte DB1 in Prozent über beide Erfolgskategorien sowie auch die Under- respektive Overperformance faktisch identisch sind. Der Margenverlust der Stichprobe beträgt 48 Mio. CHF und senkt den DB1 über alle Projekte auf 72 Mio. CHF. Die Stichprobe beträgt 928 Datensätze von ursprünglich 1471, was ca. zwei Drittel beträgt. Eine grobe Aufrechnung des Margenverlust bedeutet, dass über die letzten drei Jahre ein Margenverlust von ca. 72. Mio. CHF realisiert wurde. 
\newline Wird der Erfolgsschlüssel des Ampelsystems der Bühler AG angewandt, so beträgt das budgetierte Umsatzvolumen der Fail-Projekte gerade noch 335 TCHF (16\%). Die Kostenperformance verbessert sich um ca. 9TCHF und die realisierte Marge über alle Fail-Projekte verschlechter sich um 400 Prozentpunkt. Ca. die Hälfte der Fail-Projekte gemäss dem binären Erfolgskriterium fällt unter die Success-Projekte gemäss dem Ampelsystem. 
\begin{table}[htbp]
	\centering
	\caption{Finanzieller Verlust: Erfolgskriterium ist Erfolg gemäss Ampel}
	\begin{tabular}{lrrrrrrrrrrrr}
		\textbf{Erfolgskrit} & \multicolumn{1}{l}{\textbf{TO Bud}} & \multicolumn{1}{l}{\textbf{Cost Bud}} & \multicolumn{1}{l}{\textbf{DB1 Bud}} & \multicolumn{1}{l}{\textbf{DB1 Bud [\%]}} & \multicolumn{1}{l}{\textbf{TO Act}} & \multicolumn{1}{l}{\textbf{Cost Act}} & \multicolumn{1}{l}{\textbf{DB1 Act}} & \multicolumn{1}{l}{\textbf{DB1 Act [\%]}} & \multicolumn{1}{l}{\textbf{TO Dev}} & \multicolumn{1}{l}{\textbf{Cost Dev.}} & \multicolumn{1}{l}{\textbf{DB1 Dev.}} & \multicolumn{1}{l}{\textbf{DB1 Dev. [\%]}} \\
		green & 1'796'310 & -1'344'946 & 451'364 & 25.1\% & 1'808'320 & -1'241'767 & 566'553 & 31.3\% & 12'010 & 103'179 & 115'189 & 6.2\% \\
		red   & 175'002 & -131'684 & 43'318 & 24.8\% & 183'152 & -172'966 & 10'185 & 5.6\% & 8'149.862 & -41'283 & -33'133 & -19.2\% \\
		yellow & 159'630 & -116'821 & 42'809 & 26.8\% & 160'571 & -128'630 & 31'941 & 19.9\% & 941   & -11'809 & -10'868 & -6.9\% \\
		Grand Total & 2'130'942 & -1'593'451 & 537'490 & 25.2\% & 2'152'042 & -1'543'364 & 608'679 & 28.3\% & 21'101 & 50'088 & 71'188 & 3.1\% \\
	\end{tabular}%
	\label{tab:addlabel}%
\end{table}%
\newline\newline
Zussamenfassend kann ausgesagt werden, dass für die Stichprobe auf einen budgetierten DB1 Bud von 537 Mio. CHF ein Margenverlust von ungefähr 10\% resultiert. Im Anschluss wird im Bereich der Kostenkategorie ergründet, auf welche Kostenarten die Kostenüberschreitungen der Fail-Projects zurückzuführen sind. Die nachfolgenden Ausführungen werden gemäss folgendem Schema strukturiert: "Frame-Faktoren", "Kostenfaktoren", "FF-Faktoren", "SQ-Faktoren" und  "Komplexitäts"-faktoren. Das Augenmerk liegt auf der Lokalisierung derjenigen Faktoren, die möglicherweise den Erfolg der Bühler-Projekte beeinflussen.
\subsubsection{Erfolgsfaktoren}
Als erstes wurden die sogenannten Rahmenbedingungen eines Projektes untersucht. Da die Bühler-AG eine Matrix-Organisation hat, werden bei der Analyse der finanziellen Ergebnisse die zwei Dimensionen, Geschäftsbereich und Region, herangezogen. Die Erfolgsquote dient hierbei als Kriterium, ob eine Region respektive ein Geschäftsbereich relativ gut abgeschlossen hat. Der Standard setzt die Erfolgsquote der gesamten Stichprobe und beträgt. 2.05. Auf die Analyse pro Kunde und pro Equipment Location wurde verzichte. Denn bei ca. 800 Kunden und xx Länder auf 929 Projekte sind verhältnismässig viele Einzelwerte vorhanden, die wenige Aufschluss über ein bestimmtes Muster geben können. Es könnte jedoch sinnvoll sein, eine Liste zu erstellen, bei welcher Success und Fail Projekte ausgewertet werden, die dann Aufschluss über "gute" respektive "schlechte" Kunden oder Länder geben kann.
\textbf{Rahmenbedingungen:}
\newline \textbf{einfügen pro Region, pro BA und BA / Region split für Success/Fail, dann Verlust gegenüber stellen und sehen, ob kompatibel: Lokalisation der Region und BA}
\newline Die Evaluation der Regionen zeigt, dass ca. 70\% aller untersuchten Projekte in drei Regionen gebaut wurden. In Europa, der grössete Absatzmarkt der Bühler AG, werden ein Drittel der Projekte abgewickelt und weist folglich absolut die grösste Anzahl Fail-Projekte aus. Relativ zur gesamt Anzahl betrachteter Projekte   Nord Amerika die meisten Fail-Projekte. Gleichzeitig ist die Erfolgsquote von Europa und Nordamerika am niedrigsten un unter der Erfolgsquote aller untersuchten Projekte. Absolute High-Performer in dieser Betrachten ist die Region Ostasien und Südasien. Die Performance des mittleren Osten inklusive Afrika, China und Südamerika liegen im Mittelfeld. Eine Gegenüberstellung der Projektkosten und -marge der Fail-Projekte zeigt allerdings dass die höchste Einbusse in der DB1-Marge der Projekte in MEA liegt. Obwohl Nordamerika die tiefste Erfolgsquote hat, liegt ihre Kostenperformance im Mittelfeld. Relativ zeigt Europa die zweit schlechteste Kostenperformance wohingegen in absoluten Grössen diejenige Region mit der höchsten Anzahl Fail-Projekte auch den höchsten Verlust ausweist. Mögliche Erklärungen für diese Divergenz könnten der realisierten Umsatz und die Abweichung zum Umsatzbudget liefern. 
Der Geschäftsbereich und die Geschäftseinheiten zeigen, dass VN, GL und GM die niedrigsten Erfolgsquoten ausweisen, wobei GM absolut betrachtet, die grösste Anzahl an Fail-Projekten aufweist. Die Analyse der Geschäftseinheiten für GM, VN und GL zeigt, dass das Hauptgeschäft der Bühler AG, Industriall Milling, mit den meisten Projekten, die beste Erfolgsquote ausweist. Die übrigen Geschäftseinheiten von GM schneiden eher weniger gut ab. Bei der Matrixbetrachtung Region-BA zeigt sich, dass die zuvor identifizierten Geschäftsbereiche in denjenigen Regionen Projekte realisiert haben, die relativ gesehen eine niedrige Erfolgsquote aufweisen. Die Betrachtung des finanziellen Verlust der FAIL-Projekte ist mit der Matrixanalyse übereinstimmend, da dort am meisten Verlust generiert wird, wo die meisten FAIL-Projekte auftauchen. Die Betrachtung der Kostenperformance und DB1-Performance für alle Projekte zeigt, dass für Verluste im Normalfall kompensiert werden, allerdings gilt dies nicht für GL und VN in Europa, für GL in MEA und NAM sowie für VN in NAM. Obwohl MEA mit 15 Mio. CHF am zweit meisten Margeneinbusse hat, wird zweimal kompensiert. Dieser Effekt ist in EU mit den meisten DB1-Einbussen gerade einmal die Hälfte. Die Analyse der Geschäftsart könnten noch eine Stufe tiefer, beim Marktsegement angesetzt werden. Allerdings wurde aus Platzgründen hierrauf verzichtet.
\newline\textbf{Kosten:} Die Analyse der direkten basiert vor allem auf dem Umsatzvolumen, das als zugleich Proxyvariable für grosse und kleine Projekte dient, sowie die Kostenabweichungen pro Kostenart. Ausserdem kann die Zusammensetzung des Kostenbudgets Aufschluss darüber geben, ob beispielsweise Projekte mit relativ hohe Engineeringanteil eher scheitern als Projekte mit niedrigem Engineering-Anteil. 
Das Verteilung der Umsatzvolumen ist linksschief und zeigt dass der Grossteil der Projekte ein Umsatzvolumen von weniger als 10 Mio. CHF haben. Um etwas mehr Aufschluss über die Verteilungen und die Lokalisation der Fail-Projekte zu erhalten, wurde eine zusätzliche Variable generierte TOBud\_Cat. Sie teilt das Umsatzvolumen in 15 Kategorien ein. Alle Projekte unter 500 TCHF Umsatzbudget bilden die erste Kategorie, alle Projekte mit Umsatzvolumen zwischen 5 Mio. CHF und 10 Mio. CHF bilden eine klasse und alle Projekte über 10 Mio.CHF bilden eine Klasse. Die restlichen Projekte mit Umsatzvolumen zwischen 500 und 5000 TCHF wurden in Klassen mit dem Intervall 500 TCHF eingeteilt. 
\newline\textbf{Einfügung Histogram TOBud\_cat}
Das Histogramm für die TOBud\_Cat zeigt, dass ca zwei Drittel aller untersuchten Projekte ein Umsatzvolumen von bis und mit 2 Mio. CHF hat. In diesen vier untersten Kategorien liegen auch die meisten Fail-Projekte.  Die Erfolgsquote zeigt zudem, dass Projekte im Bereich von 2 bis 5 Mio. Umsatzbudget relativ erfolgreich abschneiden. Projekte mit einem Umsatzbudget zwischen 5 und 10 Mio zeigen wieder eine schlechtere Erfolgsquote, wobei das Klassenintervall mit 5 Mio. CHF grösser ist. Intuitiv könnte dies damit zusammenhängen, dass grosse Projekte als relativ wichtig wahrgenommen werden und deshalb mehr Fokus auf die Performance gelegt wird, da das Ergebniss des Geschäftsbereichs und der Region relativ stark durch diese vereinzelt grossen Projekte beeinflusst wird. In Bezug auf die Kategorisierung der Projekte gemäss ihrem budgetierten Umsatzvolumen, erscheint es sinnvoll zu evaluieren, ob wenige grosse Projekte oder viele kleinere Projekten für den finanziellen Verlust gemacht werden können. Gleichzeit wird analysiert, wie sich die absoluten Kostenabweichungen voneinander unterscheiden. 
\newline\textbf{Einfügung Kostenperformance pro TOBud\_Cat + Kommentierung}
\newline Die gesamte Kostenabweichung für Fail-Projekte beträgt ca. -61 Mio. CHF und für die Success-Projekte 111 Mio. CHF. Die Kostenabweichung von Fail-Projekten ist zu je einem Drittel auf Installtationskosten und die Mechanical Supply-Kosten zurückuführen. Das Restliche Drittel ist auf die ME-Kosten und PA-Kosten zurückzuführen, wobei PA die geringste Kostenabweichung aufweist. Bei den Success-Projekten ist die IS-Kostenüberschreitung mit ca. 7 MCHF erwähnenswert, wird jedoch vollständig durch die positive Abweichung der MS-Kosten kompensiert. Bei den Fail-Projekten ist die Kostendifferenz zwischen Act und Bud über alle Fail-Projekte negativ, was impliziert, dass die Kostenreserve vollständig aufgebraucht wurde und die Kosten tatsächlich über dem Budget liegen. Sie wird zudem fast vollständig durch die vier Kostenarten erklärt. Bei den Success-Projekten fehlt ein Teil, welcher die 111 Mio. CHF erklärt. Die Realisierung der Kostenreserve bei der Kostenart WA (Warranty) fehlt im Datensatz, erklärt aber höchst wahrscheinlich die Differenz zwischen der Summe der Kostenabweichungen der Kostenarten und der gesamten Kostenabweichung. 
\newline\textbf{Einfügen, Tabelle zu relativen BudMSTot}
Die Zusammensetzung des Kostenbudgets liefert relativ wenig Hinweise über die Natur der erfolgreichen respektive nicht erfolgreichen Projekte. Gesamthaft betrachtet beträgt ca. 70\% der ME-Anteil ca. 5\%, der PA Anteil ca. 2 \% und der IS Anteil ca. 5\%.
\newline\textbf{Einfügung Histogramme (evlt), mean evaluation per Success and per TOBudCat Success/Fail}
In der Stichprobe ist der prozentuale Anteil der Kosten von Nachlieferungen von Fail-Projekten am Umsatz höher als bei Success-Projekten. Dieses Bild ist intuitiv logisch, da Fail-Projekte höhere Kosten aufweisen müssen als Successs-Projekte. Allerdings kann kein spezifisches Muster erkannt werden zwischen den beiden Gruppen. Einzig bei Fail-Projekten ist der SU-Anteil bei der TOBud-Kategorie 4.5-5MCHF mit ca. 6\% relativ hoch. Diese Annomalie ist auf ein Projekt, welches einen SUCostTO von ca. 40\% gehabt hab und somit als Einzelfall zu betrachten ist.
\newline\textbf{Einfügung Summe der Delta Last FC Act:}
\newline Grundsätzlich besteht die Vermutung, dass wenn die Projektkosten drohen zu explodieren, der Kostenforecast relativ spät angepasst wird, um einerseits Erklärungsdirektiven auszuweichen und da wahrscheinlich die Hoffnung besteht, dass die Projektkosten sich wieder normalisieren. Aus diesem Grund wird vermutet, dass die Anpassung des FC für Fail-Projekte relativ höher sein muss als für Success-Projekte. Sowohl die Histogramme und als auch die Durchschnittswerte lassen keine Bestätigung respektive Verneinung der Vermutung zu, da die Werte und die Verteilung nahezu identisch sind. 
\newline\textbf{FF-Variablen:} Die Hypothesen für die Variablen, welche den Projektmanager untersuchen ist, dass der Projektmanager (PMNo) direkten Einfluss auf den Projekterfolg hat. Der Wechsel des Projektmanagers kann ein Indiz für konfliktierende Verhältnisse zwischen den Vertragsparteien sein, weshalb hypothetisch vermutet wird, dass Projekte mit einem PMChange eher scheitern als solche ohne. Unter den evaluierten Projekten sind allerdings gerade einmal 39 Projekte, wovon 16 gescheitert sind und 28 erfolgreich waren, die einen Projektmanagerwechsel gehabt haben. Demgegenüber stehen 889 Projekte ohne PMChange, von denen 601 erfolgreich waren und 288 gescheitert sind. Die berechneten Wahrscheinlichkeiten P(change, fail) von 41\% und P(no change/fail) von 33\% sind aufgrund der geringen Stichprobenanzahl der Projekte mit Change volatil. Ausserdem müsste die statistische Signifikanz erhoben werden, um eine Aussage treffen zu können. Der Fall PMChange tritt jedoch relativ wenig auf. Die Anzahl Projektmanager ist direkt mit der Variable PMChange verbunden, das ein PMChange zwei Projektmanager währen des Projektverlaufs impliziert. Das zeigt sich daran, dass die Betrachtung der Anzahl PM ähnliche Resultat liefert, wie bereits PM Change. Der Unterschied besteht darin, dass der Fall auftritte, bei dem der Projektmanager zweimal gewechselt wurde. Von diesen 5 Projekten wurde 1 erfolgreich abgeschlossen. Es könnte folglich eine neue Hypothese formuliert werden, dass bei steigender Anzahl PM währen des Projektverlaufs das Projekt eher scheitert. Das Alter des Projektmanagers ist eine Proxyvarialbe für die Erfahrung und die Berufserfahrung generell. Hierbei wird unterstellt, dass je erfahrener der Projektmanager ist, desto eher können die Projekte erfolgreich abgeschlossen werden. Die Betriebszugehörigkeit des Projektmanagers (PMTen) ist eine Proxyvarialbe für die Kenntnisse der Bühlerwelt. Die Varialbe postuliert einen Zusammenhang zwischen der Erfolgschange und der Kenntnisse über die Bühlerwelt, folglich sind jüngere Mitglieder der Bühler-Familie weniger erfolgreich. Das durchschnittliche Alter der untersuchten Stichprobe beträgt 39 Jahre. Diesem Durchschnitt kann unterstellt werden, dass relativ erfahrene Projektmanager während der Zeit von 2013 bis 2015 bei der Bühler AG gearbeitet haben. Die durchschnittliche Betriebszugehörigkeit beträgt 10 Jahre, womit sich postulieren lässt, dass die PM der betrachteten Stichprobe relativ gute Kenntniss von den Bühler-Praktiken hatten.
\newline Die Lead SAS des Projekts trägt die Gesamtverantwortung. Es wird postuliert, dass gewisse Gesellschaften erfolgreicher sind als andere. Der Fokus liegt einerseits, darauf zu identifizieren welche Gesellschaften es sind. Die Erfolgsquote zeigt, dass die europäischen Gesellschaften eher eine schlechte Erfolgsquote aufweisen, was kompatible mit den Regionenresultat ist. Die LeadSASFF ist verantwortlich für die Projektabwicklung. Bei 828 Projekten ist diese Gesellschaft identisch mit derjenigen, welche die gesamte Projektverantwortung hat. Somit wird unterstellt, dass Projekte, bei welchen die Projektverantwortung quasi geteilt ist, eher scheitern, als solche mit zentralisierte Projektverantworten. Es zeigt sich jedoch, dass von den 100 Projekten mit geteilte Verantwortung 84\% erfolgreich waren. Somit liefern die Daten eher einen Hinweis für das Gegenteil. Ausserdem kann die Anzahl der LeadSASFF variieren, wobei unterstellt wird, dass bei mehr am Projekt involvierte Parteien, das Projekt eher scheitert als bei weniger. Die Stichprobe zeigt, dass bei 100 Projekten mehr als eine SAS die Verantwortung der Projektabwicklung hatte. Es zeigt sich ein ähnliches Bild wie zuvor bei der Evaluieren, der geteilten Verantwortung. Folglich liegt ein Hinweise vor, dass geteilte Verantwortung den Projekterfolg begünstigen kann. 
\newline\newline\textbf{Zeit:} Im Zeitmanagement geht es vor allem darum, zu evaluieren ob Fail-Projekte die Zeitvorgabe eingehalten haben. Die unterstellte Hypothese besagt, dass Zeit und Kosten oftmals zusammenhangen und folglich Fail-Projekte mehr Zeit benötigten um das Projekt abzuschliessen. Ferne bestand das Ziel, darin zu erheben, ab welchem Zeitpunkt die Projekte die Zeitverzögerung einfahren. Da im BPM-Cockpit die Zeitvorgabe pro Milestone gemessen wird, wurde die Zeitdifferenz für ursprünglich fünf Milestones erhoben. Da die Datenpflege in der Obhut der Projektmanager liegt, musst der MS5 aus der Analyse ausgeschlossen werden, da die Daten nicht gut gepflegt waren. Die durchschnittliche budgetierte Projektlaufzeit beträgt ungefähr 11 Monate, und die realisierte Projektzweit 18 Monate. Der durchschnittliche Zeitverzug für Fail-Projekte beträgt 7.5 Monate unf für Success-Projekte 5.5 Monate. Beim Milestone 2 Concept approved ist grundsätzlich tief und bewegt sich um Tage. Er ist allerdings bei den Fail-Projekten ein bisschen höhen. Daraus könnte eine Hypothese formuliert werden, dass Fail-Projekte mehr Ausarbeitung benötigen, da es sich um komplexe Projekte handeln kann, oder die Vorarbeit durch den SQ-Manager unzureichend war und deshalb mehr nachträglich besprochen werden muss. Interessanterweise ist im Milestone 8 Documentation ready die Zeitverzögerung der Fail-Projekte wieder tiefer als bei den Success-Projekten. Demgegenüber beträgt die Zeitverzögerung nach MS10 Takeover für die Stichprobe ca. 5 Monate und ist für Fail-Projekte nur leicht höher. Die Verspätung in Monaten steigt nach dem MS 11 Project Closed für Success-Projekte nur gering an, wohingegen bei Fail-Projekten die durchschnittliche Verzögerung auf 7.5 Monate ansteigt. Anzahlsmässig war  die Mehrheit der Projekte, das heisst 813 Projekte nach dem MS2 nicht verspätet. Diese ändert sich nach Betrachtung des MS8, nach dem ca zwei Drittel aller Projekte verspätet waren. Der relative Anteil der Fail-Projekte bewegt sich über die Projektlaufzeit um die 30\% wohingegen der relative Anteil der Success-Projekte von 53\% auf 65\% ansteigt. Es wurde zudem ausgewertet, ob eine anfängliche Verspätung auch in einer Verspätung beim letzten MS gemündet hat. Die meisten Projekte hatten die Eigenschaft, bei einer Verspätung im MS8 ebenso auch im MS11 verspätet zu sein. Die zweithäufigste Gruppe war diejenige dass bei einer Verspätung im MS10 auch eine Verspätung im MS11 vorlag. Eindeutige Aussagen lassen sich nicht machen, vor allem da sowohl schlecht als auch gute Projekte eine Verspätung im MS11 aufweisen. 766 Projekte waren unabhängig von ihrem Erfolg waren verspätete. Folglich sind Rückschlüsse, ob Fail-Projekte tendenziell eher zeitverzögert sind schwierig zu treffen.
\newline\textbf{SQ:} Die SQ-Faktoren mussten aufgrund fehlender Daten von der Analyse ausgeschlossen werden. Es konnte einzig ausgewertet werden, ob die Geschäftsbereiche bei Projektabschluss einen finanziellen Druck hatte, da sie hinter dem Auftragsvolumenbudget lagen. Hierbei wird unterstellt, dass Projekte unvorsichtiger geprüft werden, da der Geschäftsbereich respektive die Region auf das Projektvolumen angewiesen ist, um die Budgetvorgaben zu erreichen. Folglich wird erwartet, dass die durchschnittlichen werte für die Differenz zwischen dem Act und Bud absolut und relativ bei Fail-Projekten höher liegt als bei Succes-Projekten. Die Untersuchung der beiden Stichproben bestätigt die Vermutung, dass Fail-Projekte tendenziell einem grösseren Budgetdruck ausgesetzt sind als Success-Projekte sowohl in relativen als auch in absoluten Grössen.
\newline\textbf{Komplexität:} Da es kein Komplexitätsfaktor gibt, wurden sogenannte Proxyvariablen erhoben, um die Komplexität abzubilden. Es wurde die Anzahl involvierter Parteien bei der Herstellung der Maschine als Komplexitätsfaktor identifiziert sowie die Anzahl Aufträge. Es wird unterstellt, dass das Management einer hohen Anzahl Schnittstellen und Aufträge komplexer wird, da der Aufwand alle Parteien uf einander abzustimmen höher ist. Gewisse Projekte werden in einem Konsortium abgewickelt, was bedeutete, dass das Projekt zusammen mit einem Dritt-Unternehmen (keine Bühler-Gesellschaft) durchgeführt wird. Die Zusammenarbeit mit einem anderen Unternehmen wird ebenso als Komplexitätsindikator gewertet. Ca. 79 Projekte der Stichprobe wurden in einem Konsortium abgewickelt, davon waren 48 (61\%) erfolgreich und 31 (39\%) gescheitert. Das sind verhältnismässig wenig Projekte.\newline
Für die Anzahl Aufträge zeigt sich, dass die Mehrheit aller Projekte in der Stichprobe genau einen Vertrag hat. 90\% aller Projekte hat entweder 1 oder 2 Verträge. Vergleicht man die Erfolgsquote pro Anzahl Aufträge, ist diejenige mit zwei Aufträgen höher als diejenigen mit einem oder drei Aufträge. Es könnte damit zusammenhängen, dass ein grosser Auftrag weniger Übersichtlich ist als zwei kleinere separate. Zu viele Aufträge könnten dann wieder für Verwirrungen sorgen.\newline
Die Anzahl involvierte SAS hat einen Datenbereich von 0 bis 10. 0 bedeutet, dass die Zulieferung für die Maschine aus Eigenproduktion oder von Drittlieferanten stammt.  Allerdings zeigen sämtliche Histogramme keine Auffäligkeiten, welche darauf schliessen lassen, dass Fail-Projekte andere Eigenschaften aufweisen als Success-Projekte. Die Erfolgsquote pro Ausprägung zeigt, dass wenn die Zulieferung durch eine andere Gesellschaft erfolgt, höher ist als wenn eine Eigenproduktion oder ein Drittlieferung stattfindet. Bei der Anzahl Supplying SAS MS zeigt sich dasselbe Bild. Hingegen zeigt sich bei den Supplying ME, PA und IS, dass wenn keine weitere Bühler GEsellschaft involviert war, die Erfolgsquote höher ist.

\subsection{Kritische Würdigung der Ergebnisse}
\newpage	
	


