% !TEX root = MA.tex
\section{Projektmanagement und Erfolgsfaktoren}
	
Der Erfolg von Projekten und deren Management ist ein in der Forschung viel diskutiertes Thema, weshalb hier in den nachfolgenden Kapiteln zunächst im Allgemeinen auf die Erfolgsdefinition von Projekten respektive Projektmanagement und bereits erforschte Projekterfolgsdeterminanten eingegangen wird. Im Anschluss folgt eine Erläuterung dieser Termini im Kontext mit der Bühler-Welt.
	
\subsection{Projet, Projektmanagement und Projekterfolg}
	
Gemäss dem Deutschen Institut für Normung (DIN) ist ein Projekt: " ein Vorhaben, das im Wesentlichen durch Einmaligkeit der Bedingungen in ihrer Gesamtheit gekennzeichnet ist, z.B. Zielvorgabe, zeitliche, finanzielle, personelle und andere Begrenzungen, Abgrenzung gegenüber anderen Vorhaben, projektspezifische Organisation". Daraus folgt, dass Projekte sich bezüglich einzelner Faktoren unterscheiden können, allerdings die Gesamtheit der Faktoren ihre Einzigartigkeit definiert. Beispielsweise begründet bei der Bühler AG die internationale Tätigkeit, das diverse Anlageportfolio und die breite Kundenbasis ein Indiz für einmalige Projekte. Obwohl es unterschiedliche Projekte gibt, beispielsweise im Tiefbau, Hochbau und Ingenieurbau und sich deren Management sowohl durch Differenzen als auch Gemeinsamkeiten charakterisiert, weist ein Projekt gemäss Projektmanagement-Handbuch (ohne Datum) folgende Eigenschaften auf: "komplexe, neuartige Aufgabenstellung, messbare Ziele und Ergebnisse, zeitliche Befristung (Anfang und Ende), begrenzte Ressourcen und die Notwendigkeit von Teamarbeit". Meyer und Rehrer (2012, S.2) sehen die progressive Elaboration, die eine kontinuierliche Konkretiesierung des Projekts während dessen Verlauf als weiteres Merkmal von Projekten. Der exakte Projektbegriff der vorliegenden Arbeit orientiert sich anschliessend an den geschäftsinternen Definitionen der Bühler AG
\newline
Der Managementbegriff wird vom Projektmanagementhandbuch(ohne Datum) als "systematischer Prozess zur Führung komplexer Variablen definiert. Er beinhaltet die Organisation, Planung, Steuerung und Überwachung aller Aufgaben und Ressourcen, die notwendig sind, um die Projektziele zu erreichen". Das Projekt Management Institute (PMI) (PMBOK, 2004) beschreibt Projektmanagement als eine Anwendung von Wissen, Fähigkeiten, Instrumente und Techniken bei Projektaktivitäten, um Projektanforderungen zu erfüllen. Nach Alama und Gühl (2016) wird Projektmanagement als "die Koordination von Menschen und der optimale Einsatz von Ressourcen zum Erreichen Projektzielen dargelegt. Pierce (S.2015, S.2) führt eine generelle Definition aus, gemäss derer Ziele, Prozesse, Planung und Kontrolle den Managementterminus beschreiben. Die Literatur zeigt keine einheitliche Definition, dennoch kann zusammengefasst konstatiert werden, dass Projektmanagement die zielgerichtete Planung, Steuerung und Überwachung von  Ressourcen in den Prozessschritten umfasst. Das Projektmanagementprozess kann trotz inhaltlich je nach Industrie, in folgende Schritte unterteilt werden; Projektinitiierung, Projektplanung, Projektdurchführung und -kontrolle, und Projektabschluss (PMHandbook, ohne Datum). An dieser Stelle wird nicht weiter auf die inhaltlichen Aspekte des Projektmanagements im Analagebau eingegangen, da der Datenanalyse und der Entwicklung von Frühwarnindikatoren der Bühler Projektmanagementprozess zu Grunde liegt, welcher im nächsten Unterkapitel erläutert wird.
\newline
Der Erfolg von Projekten beschäftigt die Forschung seit längerer Zeit. Das vorherrschende Paradigma zur Beurteilung des Projekterfolgs ist das magischen Dreieck, Zeit, Kosten und Qualität. Allerdings ist in der Praxis letztendlich der Kostenaspekt von zentraler Bedeutung, da er mit Geldverlust korreliert ist. Die akutelle Forschung der Erfolgsfaktoren zeigt, dass er Erfolg von Projekten nicht auf ein Faktor reduziert werden kann. Deshalb unterscheidet der Besteiero, Pinto \& Novaski (2015) zwischen kritischen Erfolgsfaktoren und Erfolgskriterien. Die Faktoren erhöhen ihrer Auffassung nach die Wahrscheinlichkeit des Erfolgs wohingegen die Kriterien darüber bestimmen, ob ein Projekt erfolgreich war (Besteiro, Pinto, Novaski, 2015). Hieraus entsteht eine Differenz in der zeitlichen Betrachtung, Faktoren sind während der Projektabwicklung relevant, um das Projekt erfolgreich abzuschliessen wohingegen Kriterien erst nach des Projektabschluss hinzugezogen werden, um über den Erfolg zu bestimmen. Folglich wäre der monetäre Aspekt ein Erfolgskriterium. Es muss jedoch hinterfragt werden, ob sich die Erfolgsbeurteilung von Projekten angesichts ihrer Eigenartigkeit auf ein Kriterium reduzieren lässt. Folglich wird die Vergleichbarkeit deren Erfolg durch die Projektdefinition in Frage gestellt. Damit Projekte verglichen werden können schlägt Autor (Datum) ein gewichtetes Erfolgskriterium vor, welches unterschiedliche Bestandteile der unternehmensinternen Erfolgsdefinition wiederspiegelt. Ebenso werden bei einer reinen Kostenbetrachtung die Einhaltung von Zeitvorgaben und die Qualitätsanforderung ausser Acht gelassen. Kerzner (2014) schlägt vor anstatt der traditionellen Erfolgsbetrachtung vor, den Projekterfolg als die Erreichung des gewünschten Geschäftswertes innerhalb der sich konfligierenden Zielvorgaben zu verstehen. Dabei  unterstellt er eine neue Projektdefintion: Ansammlung nachhaltiger Geschäftswerte, deren Realisierung terminiert ist. Davon ausgehend erkennt Kerzner (2014), dass bei der Erfolgsrealisierung immer ein trade-off erfolgen muss, da beispielsweise bei einer Zeit- und Kostenüberschreitung ein Projekt nicht zwingend ein unerfolgreiches Projekt darstellen muss, da Wissen generiert werden konnte, welches in anderen Projekten oder in anderen Bereichen von Nutzen sein kann. Somit schlägt er einen vierstufige Erfolgsdefinition vor, bei der Projekte in vier Kategorien, gesamtheitlicher Erfolg, Teilerfolg, Teilscheiter und gesamtheitliches Scheitern. Die der Analyse und dieser Arbeit zugrunde liegenden Erfolgsdefinition respektive Erfolgskriterium wird durch die Bühler AG festgelegt. Allerdings lässt sich aus den vorherigen Ausführungen schliessen, dass die Erfolgsdefinition kritisch betrachtet werden muss und zunehmend entwurzelt wird, da die nicht monetären Wertegenerierung eines Projekts bisher kaum Bestandteil des Erfolges war. Zudem ist die Einhaltung der Kosten- und Zeitvorgaben in einem Projekt eine herausfordernde Aufgaben, da sich während des Projekts Änderungen ergeben, die zu einem Kostenanstieg und Zeitverzug führen können.
\newline
Wie bereits eingangs erwähnt, wurden einige Erfolgsfaktoren mit unterschiedlichen statischen Methoden erforscht. Grundsätzlich liegt dem Konzept der Erfolgsfaktoren die Prämisse, dass Erfolg wiederholbar ist und an bestimmte Faktoren geknüpft ist, zugrunde. Folglich müsste es auch Determinanten geben, welche den Projekterfolg negativ begünstigen. Iyer \& Jha (2006) fanden mittels Expertenbewertungen und anschliessender Faktoranalyse heraus, dass das Engangement der Projektmitarbeiter und die Fähigkeiten des Projekteigners sich positiv auf die Zeitperformance auswirken. Konflikte zwischen dem Projektmanager und Top-Management, dem Projekteigner oder anderen externen Parteien sowie ein Missgunstkultur können zur Überschreitung der vorgegebenen Projektzeit führen. Chan, Ho \& Tam (2001) haben bereits in einer früheren Studie aus 31 möglichen Erfolgsfaktoren mittles der Faktorenanalyse auf fünf relevante Erfolgsfaktoren geschlossen. Basierend auf einer fünfstufigen Erfolgsskala hat ihre Analyse ergeben, dass das Engagement des Projektteams, welches Kooperation, Konfliktlösung, Vertrauenskultur, Verständnis der Projektziele sowie Kommunikation miteinschliesst, als kritischer Erfolgsfaktor zu qualifizieren ist (Chan, Ho \& Tam, 2001). Dieses Ergebnis ist mit den Befunden früheren Studien (Ashley, et al. 1987, Pinto und Slevin, 1988) kompatibel, bei denen das Engagement der Projekteilnehmer ebenso als erfolgskritisch identifiziert wurde. Die Analyse von Chan, Ho \&Tam (2001) hat zudem ergeben, dass sowohl die Fähigkeiten des Auftragsnehmers wie zum Beispiel, die Qualität des Projektmanagement oder die Anwendung innovativer Technologien, als auch die Kompetenz des Kunden, Konstruktionsprojekte abzuwickeln, eine entscheidende Rolle für den Projekterfolg haben. Die Unterstützung der des Managements, Kommunikation oder die Projektmission bei Change- oder IT-Projekten von grosser Bedeutung (Hyvräri, 2006 in Besteiro, Pinto, Novaski, 2015). Varajão, Dominguez \& Ribeiro et al. (2014) untersuchten, ob zwischen Software- und Konstruktionsprojekten Differenzen bezüglich der Erfolgsfaktoren existieren. Mittels einer Likert-Skala und bereits in früheren Studien entdeckten Erfolgsdeterminanten konnte festgestellt werden, dass die Projektplanung und das Verständnis der Projektziele sowie -anforderungen bei beiden Projektarten kritische Erfolgsaspekte sind. Allerdings wird beispielsweise die Effizienz des Projektmanagements und der Miteinbezug aller Projektteilnehmer bei Konstruktionsprojekten als kritischer für den Erfolg erachtet als bei Software-Projekten. Lam, Chan \& Chan (2008) haben ein anderes Erfolgskriterium, welches Kosten, Zeit, Qualität und Funktionalität in einem gewichteten KPI zusammenfasst, zur Bestimmung der kritischen Erfolgsdeterminanten herangezogen. Ihre Analyse hat ergeben, dass die Projektnatur, die Effizienz des Projektmanagement und die Anwendung innovativer Managementtechnologien einen erfolgreichen Projektabschluss begünstigen (Lam, Chan \&Chang, 2008). Gemäss ihrer Aussage würde die Einbringung des Auftragseigners, die Attraktivität und Komplexität des Projekts, kurz die Projektnatur), die Projektmanager dazu veranlassen, mehr Effort für das Projekt zu leisten, da solche Projekte mit Prestige und Selbstverwirklichung verbunden sind (Lam, Chan \& Chan, 2008). Mittels logistischer Regressionsanalyse erforschten Lu, Hua \& Zhang (2017) erforschten die Erfolgsfaktoren aus einer Kostenperspektive. Hierbei wurde festgestellt, die Fähigkeiten des Auftragsnehmers, welche vergleichbare Erfahrungen, Teamfähigkeit und Kostenaffinität, inkludieren, relevant für den Erfolg des Projektes sind. Die obigen Ausführungen fassen die Ergebnisse zahlreicher Studien zu Erfolgsfaktoren zusammen. Trotz unterschiedlicher Erhebungsmethoden und Projekterfolgsdefinitionen wurden sich überschneidende Determinanten identifiziert. Abweichungen können aufgrund unterschiedlicher Projektarten auftreten. Zusammenfassend kann postuliert werden, dass der Erfolg stark mit den jeweils im Projekt involvierten Personen und der Projektumgebung zusammenhängt. Attribute wie, Fehlerkultur, Teamfähigkeit, Konfliktlösen, Vertrauen, gemeinsame Mission oder Wertschätzung stellen nur eine Auswahl dar, um die Projektteilnehmer und das Arbeitsklima zu beschreiben, welche den Projekterfolg begünstigen. Alam \& Gühl (2016) sprechen in diesem Zusammenhang auch von Anforderungen, die während jeder Projektphase gegeben sein müssen, damit Projekte erfolgreich bearbeitet werden können. Ob und wie erfolgreich Projekte letztendlich waren, ist zudem immer von der Erfolgsdefinition abhängig. Erfolgskriterien, die nur auf einer Variablen gründen, erschweren die Vergleichbarkeit von Projekten wohingegen zu komplexe Erfolgskriterien zu Missverständnisen und Unübersichtlichekeit führen können. Im Moment ist ein Trend spürbar, der die Erfolgsperzeption vom magischen Dreieck aufweicht und andere Werte mitberücksichtigt. Dies würde wiederum verschiedene Aspekte wie eine Fehlerkultur fördern. Dieser Abschnitt diente lediglich dazu, bisherige Forschungsergebnisse aufzuzeigen. Im folgenden Kapitel wird näher auf das Untersuchungsobjekt, das Projektmanagement der Bühler AG, eingegangen und die der Analyse zugrunde liegenden Prämissen verdeutlicht.
%%
%% 
%%%%%%%
%%%%%%%%%%%%%%%%%%%%%%%%%%%%%%%%%%%%%%%%%%%%%%%%%%%%%%%%%%%%%%%%%%%%%%%%%%%%%%%%%%%%%%%%%%%%%%%%%%%%%%%%%%
%Bühler Porjektmgmt
%%%%%%%%%%%%%%%%%%%%%%%%%%%%%%%%%%%%%%%%%%%%%%%%%%%%%%%%%%%%%%%%%%%%%%%%%%%%%%%%%%%%%%%%%%%%%%%%%%%%%%%%%%
%%%%%%%%%%%%%%%%%%%%%%%%%%%%%%%%%%%%%%%%%%%%%%%%%%%%%%%%%%%%%%%%%%%%%%%%%%%%%%%%%%%%%%%%%%%%%%%%%%%%%%%%%%
%%
%%	
\subsection{Bühler Projektmanagement}
In diesem Kapitel wird der Bühler Projektmanagement-Prozess, der die Basis für die Ergründung der möglichen Einflussfaktoren bildete, erläutert. Die nachfolgenden Ausführungen basieren auf den intern dokumentierten Prozessbeschreibungen des C2C-Prozesses, Customer Project (CP). Er gliedert sich in zwei Kernprozesse, den Sales \& Quotation- und den Fulfilment-Prozess gliedert. Anschliessend hat das zweite Unterkapitel zum Ziel, die Faktoren, welche möglicherweise den Projekterfolg beeinflussen oder Charakteristiken von nicht-erfolgreichen Projekten bilden sowie ihre Bedeutung erklärt.

\subsubsection{Der Projektmanagementprozess}\label{zweieins}
Die nachfolgende Abbildung \ref{fig: processcp} zeigt die zwei Kernprozesse, wobei beide Prozesse durch das Hand-over-Meeting (HOM) direkt ineinander übergehen. 
\begin{figure}[H]
	\centering
	\includegraphics[width=5cm]{processcp.png}
	\caption{Prozess Customer Project der Bühler AG}
	\label{fig: processcp}
\end{figure}
Linkerhand ist der SQ-Prozess dargestellt, der durch die Übergabe des Projekts in den FF-Prozess mündet. In der Folge sollen beide Prozesse zusammengefasst beschrieben werden, wobei der Fokus auf denjenigen Bestandteilen liegt, die im Zusammenhang mit den Erfolgsdeterminanten steht. Der Verkaufsprozess umfasst vier Phasen: Identify Potential, Set Priorities, Quote and Evaluate Risk und Close Order. Die Verantwortlichkeit für den Prozess liegt hauptsächlich beim Area Manager(AM). Die Business Unit (BU) und entsprechende Centers of Competences tragen eine Mitverantwortung. Das Fundament des CP-Prozess bilden die Milestones (MS), welche die Erreichung oder die Lieferung vordefinierter Ziele einfordern, bevor mit dem Prozess weitergefahren werden darf. Die Phase I und II konzentrieren sich darauf, Geschäftspotenziale und Kundenbedürfnisse zu identifizieren, Kontakte mit den Kunden aufzunehmen und letztendlich auf Basis von diversen Checks zu entscheiden,  welche Projekte fokussiert, das heisst, offeriert werden sollen. In der Phase III und IV gilt es die möglichen Projekte einer Detailprüfung in technischer, kommerzieller und finanzieller Hinsicht zu unterziehen sowie ein Basiskonzept auszuarbeiten. Nach den anschliessenden Vertragsverhandlungen zwischen dem Kunden und der BU respektive dem AM endet der Prozess mit dem MS Orders Released (OR.) Der Auftrag wird freigegeben und das Projekt wird nach kurzer Zeit beim HOM an die Projektabwicklung übergeben. Im Fulfillment werden fünf Phasen unterschieden, wie der Abbildung \ref{fig: processff} zu entnehmen ist.
\begin{figure}[H]
	\centering
	\includegraphics[width=5cm]{processff.png}
	\caption{Prozess Customer Project der Bühler AG}
	\label{fig: processff}
\end{figure}
\textbf{Phase I: Planning and Basic Engineering}
\newline
Bei der Übergabe des Projekts vom Verkauf an die Abwicklung sind typischerweise der AM, der PM und der Teamleiter involviert. Diese wichtige Schnittstelle dient dazu alle relevanten Informationen zu übergeben und offene Punkte zu klären. Die Phase I beinhaltet die Projektanalyse, die Ausarbeitung respektive Überarbeitung des Konzepts, die Projektplanung und das Kick-off-Meeting (KOM). Das Ziel der Projektanalyse ist die Realisierbarkeit mittels der Identifizierung von technischen als auch kommerziellen Risiken und Chancen sowie entsprechenden Massnahmen zu prüfen. Die anschliessende Konzeptphase beinhaltet die Ausarbeitung oder Nachbearbeitung des Maschinen- oder Anlagekonzept und sowohl die interne als auch die externe Genehmigung einzuholen. In der Projektplanung werden überwiegend organisatorische und administrative Aufgaben wie zum Beispiel die Planung der Deadline oder die Definition von Arbeitspaketen gemacht. Der letzte MS dieser Phase bildet MS5, das Kick-off-Meeting, welches der Schaffung eines gemeinsamen, identischen Verständnis unter sämtlichen Teilprozessverantwortlichen bildet. Im KOM werden in Abstimmung mit den vertraglichen Bedingungen, verbindliche Vereinbarungen bezüglich, Termine, Kosten, Qualität und Zuständigkeiten getroffen. Ausserdem bietet das KOM Raum zur Diskussion unklarer Punkte. In der Regel findet das KOM nach der kommerziellen respektive Gesamtfreigabe statt.
\newline
\textbf{Phase II: Engineering and Specification}
\newline
Charakteristisch für diese Phase ist die Ausarbeitung verbindlicher Pläne zur Anlagen- oder Maschinendisposition. Optimierungen am Maschinen- respektive Anlagenkonzept und der interne Review sind während dieser Zeit von zentraler Bedeutung. Nach der schriftlichen Einverständniserklärung des Kunden zum Layout der Prozessanlage sind Änderungen dem Risiko von Mehrkosten, Zeitverzögerung und anderen Anpassungen in der Projektstruktur ausgesetzt. Der MS 5 "Point of now return" ist ein interner Meilenstein, deren Erreichung die Verbindlichkeit der Liefertermine gegenüber dem Kunden sowie die Sicherstellung der Kundenzahlung, erfordert. 
\newline
\textbf{Phase III: Manufacturing, Procurement and Logisic Out}
\newline
Diese Phase beginnt mit der Fabrikation und endet mit der Lieferung der Maschine an den Sitz des Kunden respektive an den vereinbarten Lieferort. Die Einhaltung des Lieferversprechen sowie die vertragskonforme Ablieferungen der Anlage oder Maschine(n) ist hierbei von besonderer Wichtigkeit. Der nachfolgende Prozessschritt 9 Project Documentation umfasst die Erstellung der Projektdokumentation für den Monteur und den Kunden, welche die Nachvollziehbarkeit der Änderungen garantiert. Das Ende dieser Phase wird durch den MS8 festgehalten, der erreicht wird, wenn die Dokumentation offiziell an den Kunden und den Monteur überreicht wurde. 
\newline
\textbf{Phase IV: Installation and Start up}
\newline
Die Installation, Inbetriebsetzung und Übergabe sind die elementaren Prozesschritte dieser Phase. Der Zusammenbau einer Anlage und die Inbetriebsetzung einer Maschine erfordert eine Instruktion des Montageteams, die zugleich eine unabdingbare Voraussetzung für ein gewisses Qualitätsniveau gewährt. Die Montageverantwortlichen werden durch die Projekt- und Verkaufsleiter ständig unterstützt, wobei gleichzeitig die Überwachung des Prozesses gewährleistet wird. Der Meilenstein 10 wir mit der kompletten Übergabe der Anlage an den Kunden nach dem Abschluss der Inbetriebsetzung erreicht. Hierbei ist darauf zu achten, dass möglichst alle vertraglich vereinbarten Anforderungen, wie zum Beispiel Tests und Inhalt, Umfang und Darstellung der Übergabedokumente erfüllt werden, da oftmals die letzten Kundenzahlungen an die Leistungserfüllung gekoppelt sind. 
\newline
\textbf{Phase V: Evaluation and Transfer}
\newline
Als letztes folgt das Debriefing, bei welchem im Sinne des kontinuierlichen Verbesserungsprozess, Rückmeldungen zur Optimierung der Projektabwicklung für künftige Projekt festgehalten werden, so dass die gleichen Fehler nicht wiederholt  werden. Der elfte Meilenstein legt intern den Projektabschlussfest. Danach beginnt die zweijährige Garantieperiode. 
\nomenclature{SQ}{Sales and Quotation}
\nomenclature{MS}{Mile Stone}
\nomenclature{FF}{Fullfillment}
\nomenclature{HOM}{Hand-over-Meeting}
\nomenclature{KOM}{Kick-off-Meeting}
\subsubsection{Einflussfaktoren des Projektmanagementprozesses}\label{zweizwei}
Nachfolgend werden sämtliche Einflussdeterminaten und das Projektmanagement-Tool, das sämtlich Projektinformationen enthält, erläutert.
\newline
Das BPM-Cockpit enthält Informationen zu Kosten, Zeit, Verantwortlichkeiten, Risiken und Aktionen und zum Engineering inklusive verschiedener Projektstatus. Die Beurteilung der Projektperformance erfolgt anhand eines dreifarbigen Ampelsystems für Kosten, Zeit und Qualität. Die erste beiden werden vom System automatisch gerechnet wohingegen die Qualität auf der subjektiven Einschätzung der Projektmanager beruht, die sie selbst berichten können. Basierend auf der internen Guideline für das BPM-Cockpit ist der Kostenstatus grün, wenn die konsolidierte Abweichung der Projektmarge zwischen dem Forecast und dem Budget $> -400$ Prozentpunkte beträgt. Dieser Status ändert von grün auf gelb, sobald die Abweichung mehr als -4\% beträgt und von gelb auf rot, wenn die Schwelle von -10\% überschritten wird. Die Zeitampel basiert einerseits auf der Differenz zwischen dem realisierten und dem geplanten Termin und anderseits auf der Eintragung im System. Sofern Angaben zum realisierten Termin im System enthalten sind, ist der Status grün, wenn er vor oder auf geplanten Datum liegt und gelb, wenn das Zeitversprechen nicht eingehalten wurde. Die rote Farbe impliziert, dass das BPM-Cockpit keine Informationen zur Erfüllung des Termins enthält und der geplante Termin vor dem heutigen Datum liegt. Die Farbe der Ampeln hängt davon ab, ob der Forecast für eine Reportingperiode angepasst wurde oder nicht. Je nach Grösse der Abweichung zwischen dem Budget und dem Forecast wird eine Erklärung vom entsprechenden Geschäftsbereich erwartet. Dies kann dazu führen, dass die Anpassung der Prognose hinausgezögert wird. Diese Tatsache spielt bei der Erklärung einiger Faktoren eine zentrale Rolle.
\newline
Obwohl die Performance eines Bühler-Projekts mittels des magischen Dreiecks - Time, Cost und Quality - beurteilt wird, hat letztendlich der Kostenaspekt aus finanzieller Perspektive die relativ gewichtigere Bedeutung als die anderen zwei Dimensionen. Deshalb wurde in Zusammenarbeit mit der Bühler AG, das folgende Erfolgskriterium festgelegt:
\begin{equation}
	\text{Deviation DB1 \%} = \text{DB1 Act \%} - \text{DB1 Bud \%}
\end{equation}
Der KPI rechnet sich realisierte Marge (DB1 Act) in \% - budgetierte Marge (DB1 Bud) in \% und wird jeweils am Projektende respektive nach Erreichung des MS11 (Kapitel \ref{zweieins}) kalkuliert. Die ursprünglich 93 Determinanten wurden in sechs Kategorien unterteilt, um eine übersichtliche Darstellung vornehmen zu können. Gewisse Faktoren könnten ihrer Natur nach auch in eine andere Gruppe gegliedert werden. 
\begin{figure}[H]
	\centering
	\includegraphics[width=90mm]{Model.jpg}
	\caption{Einflussfaktoren
	\label{Einflussfaktoren}}
\end{figure}
Jedes Projekt hat eine Identifikationsnummer, die BPM-ID, die es ermöglicht ein Projekt eindeutig zu identifizieren. Sie zählt jedoch nicht zu den Erfolgsdeterminanten.
\paragraph{Rahmenbedingungen:} In dieser Kategorie sind kategoriale Variablen zusammengefasst, die den eindeutigen Rahmen eines Projekts festlegen. Dazu gehören die Region (Region) respektive das Land (EquLoc), in welchem die Anlage gebaut wird, der Kunde (CuNo) und Geschäftsbereich (BA, BU und MS). Ausserdem zählt die relative Wichtigkeit eines Projekts (BAImportPr, BUImportPr und MSImportPr) für den entsprechenden Geschäftsbereich ebenso zu den Rahmenbedingungen. Die zugrundeliegende Hypothese unterstellt, dass gewisse Charakteristiken, das heisst, bestimmte Kombinationen den Projekterfolg begünstigen. Denn Kunden beispielsweise lassen sich bezügliche der individuellen Anlagespezifikationen, ihrer Bonität oder Kultur unterscheiden. Die Region in welcher die Anlage gebaut werden soll, birgt differenzierbare Risiken im Bereich der Politik, Wirtschaftsentwicklung oder länderspezifischer Handelsregelungen. Der Geschäftsbereich kann als eindeutiges Diversifikationskriterium der Anlage gewertet werden. Obwohl der Projektmanagementleitfaden intern universelle Gültigkeit hat, können während der Projektlaufzeit verschiedene Herausforderungen in Abhängigkeit der jeweiligen Anlage auftreten. Zudem kann davon ausgegangen werden, dass die Teamarbeit und Teamkultur pro Geschäftsbereich und -einheit verschieden sind und den Projekterfolg unterschiedlich beeinflussen. Die Wichtigkeit eines Projekts, das Umsatzbudget des Projekts im Verhältnis zum Median des Umsatzbudgets aller laufenden Projekte, kann als Indikator zur Konzentration von Ressourcen bei der Projektabwicklung interpretiert werden. Demzufolge müssten bedeutendere Projekte, die auch einen erheblichen Einfluss auf das Geschäftsbereichsergebnis haben, mehr Aufmerksamkeit in Bezug auf Risikominimierung erhalten. Da die Bühler AG Budgetvorgaben im Bezug auf das Auftragsvolumen (Orders released) kann die relative Wichtigkeit gleichzeitig ein Indiz für risikoreichere Projekte sein. Denn das Interesse am Vertragsabschluss müsste verhältnismässig grösser sein und die Verkaufsmanager nicht mittels dem Projekterfolg incentiviert sind.
\paragraph{Sales \& Quotation (SQ):} Der Verkaufsprozess geht unmittelbar in die Projektabwicklung über, weshalb die vorgelagerten Entscheidungen direkt oder indirekt den Projekterfolg beeinflussen können. Zum Beispiel beeinflussen die Qualität der Offerte sowie die vertraglichen Vereinbarungen die Rahmenbedingungen für die Projektabwicklung. Die Offertstellung und vorgängige Risikoanalysen des Projekts liegen im Aufgaben- und Verantwortungsbereich des Area Managers (AM und AMNo). Es wird davon ausgegangen, dass erfahrenere (AMAge) und langjährige (AMTen) Verkaufsmanager über mehr Kenntnisse zu den Projekten allgemein, deren Risiken und der internen Prozess verfügen und deshalb 'erfolgreichere' Projekte verkaufen. Die Incentivierung und Performancemessung der Verkaufsmanager erfolgt über das Auftragsvolumen des Geschäftsbereich und der Region. Die Abweichung von den Budgetvorgaben auf regionaler Ebene und der Geschäftsbereichsstufe (BUORBudGap und RegioORBudGap) zum Zeitpunkt des Projektabschlusses kann den Zielerreichungsdruck allem zum Jahresende erhöhen. Die Schnittstelle zwischen Sales \& Quotation und dem Fulfillment ist für den Projekterfolg von zentraler Bedeutung, weshalb eine Zeitverzögerung zwischen der Auftragsfreigabe (ORDate) und dem Projektbeginn (PrStartDate) als Indiz für Komplikationen, Unklarheiten und Unsicherheiten interpretiert werden kann. 
\paragraph{Fulfillment (FF): } In dieser Kategorie werden sämtliche Faktoren im Zusammenhang mit dem Projektmanager (PM und PMNo), dem Forecastmanagement (FC-Management) und der Unternehmensverantwortung subsumiert, da sie den Projektabwicklungsprozess tangieren. Der Betriebszugehörigkeit der Projektmanager (PMTen) sowie dessen Erfahrungsschatz (PMAg) sind stellvertretende Variablen für die Kenntnisse der Internen Prozess und das vorhanden Wissen in Bezug auf den Beruf. Bei Unstimmigkeiten zwischen dem Kunden und dem Projektmanager, kann er ersetzt werden (PMChange). Je nach Status des Projekts und Zeitpunkt des Wechsels können nicht alle Differenzen durch den neuen Projektmanager kompensiert werden, weshalb ein Austausch als Indiz für nicht-erfolgreiche Projekte betrachtet wird. In sehr seltenen Fällen muss die Funktion des Projektmanager sogar mehrmals neu besetzt werden (NoPM), was den positiven Ausgang eines Projekts beeinträchtigen kann.
\newline Die organisatorische Verantwortung für das ganze Projekt (LeadSASPr) und den Abwicklungsprozesse (LeadSASFF), kann bei einer Gesellschaft oder zwei verschiedenen Gesellschaften (LeadSAS.PrFF) angesiedelt sein. Die zusätzliche Schnittstelle erhöht den Komplexitätsgrad eines Projekts und kann deshalb nachteilig für den Projekterfolg sein. Die Zusammenarbeit sowohl zwischen den Gesellschaften als auch innerhalb der Unternehmen kann sich voneinander unterscheiden, weshalb einige Gesellschaften wahrscheinlich mehr Erfolg im Projektmanagement aufweisen.
\newline Das  Forecastmanagement liegt im Verantwortungsbereich des Projektmanager und bezieht sich auf die Prognose des Projektumsatzes, der -kosten sowie der -marge, welche monatlich geprüft und entsprechend angepasst werden muss. Im Bezug auf den Projekterfolg kann die frühzeitige Kenntnisse allfälliger Mehrkosten eine matchentscheidene Bedeutung zu deren Vermeidung oder Verminderung haben. Deshalb wurde pro Projektphase, Mechnical Supply (MS), Mechnical Engineering (ME), Plant \&Automation (PA) und Installation pro Projekt erhoben, ob der Forecast angepasst wurde, wobei zwischen 'nur Mehrkosten' und 'Mehrkosten inklusive Umsatzerhöhung' unterschieden wurde, (CostFCadj) und wie viele Monate vor Projektabschluss (MS11) die negativste FC-Anpassung (CostmostnegFCadj) gemacht wurde. Die Abweichungen von den Vorgaben in Bezug auf Zeit und Kosten wird direkt und automatisch durch das dreistufige Ampelsystem des Bühler Projektmanagement-Cockpit (BPM-Cockpit) reflektiert. Es wurde für die drei Ampeln ausgewertet, wie viele Monate zwischen HOM und dem Datum des ersten Wechsels von grün auf gelb oder rot liegen. Diese Differenz wird ins Verhältnis zur budgetierten Projektlaufzeit gesetzt, so dass der Indikator eine Aussage darüber macht, nach wieviel Prozent der budgetierten Projektzeit der Status gewechselt hat. Hiermit lässt sich feststellen, ob bereits in einem frühen Stadium die erwarte Performance hätte abgeschätzt werden können. Je früher, beispielsweise der Kostenstatus auf gelb oder rot gewechselt hat, desto früher wurden Forecast-Anpassung gemacht und desto früher hätten Gegenmassenahmen ergriffen werden könne. Ähnliche Überlegungen können für die Zeitbetrachtung gemacht werden, wobei hier eventuell bestimmte Muster auftreten können, beispielsweise sämtliche Zeitverzögerungen erfolgen relative spät. 
\newline\newline\textbf{Monetäre:} Die monetären Aspekte beziehen sich auf Umsatz und Kosten. Das Umsatzvolumen gilt in diesem Modell als Indikator für die Grösse des Projekts, wobei höhere Budgets mit komplexeren und umfangreicheren Projekten assoziert werden. Darüber hinaus besteht die Vermutung das der Budgetmix einen Hinweis auf die Eigenschaften nicht erfolgreicher Projekte könnte.
\newline Nachgelagerte Faktoren sind vor allem auf Abweichungen von den Kostenbudget von Interesse. Wie bereits erwähnt, hat die finanzielle Komponente relativ mehr Fokus für Bühler. Aus diesem Grund wurden als mögliche Einflussfaktoren sämtliche Kostenabweichungen in absoluter und relativer Höhe in das Model miteinbezogen. Diese Grössen sind direkt mit dem finanziellen Erfolg korreliert, so dass es hierbei darum ginge, eine statistische Signifikanz zu ergründen. Die Kosten werden zudem durch sogenannte Subsequent Deliveries (SU), die nach dem offiziellen Liefertermin nachgeliefert werden müssen, beeinflusst. Die Höhe der SU wird ins Verhältnis zum Umsatz gesetzt. Je höher der SU-Anteil desto eher könnte das Projekt weniger erfolgreich abschliessen.
\newline Bühler schrieb während des Betrachtungshorizontes einen Mindestdeckungsbeitrag von 23\% vor. Projekte die ein Budget unter diesem Werte haben, bedurften der Zustimmung der nächst höheren Managementstufe. Es wird davon ausgegangen, dass Projekte mit einem DB1 Bud nahe diesem Grenzwert tendenziell schlechter abschliessen, da sie überbewertet wurden, um dem Genehmigungsprozess zu entgehen. Deshalb ist der DB1 Bud ein Element der Kostenkategorie.
\newline Die letzte monetäre Komponente bildet die Abweichungen der realisierten Kosten vom letzten Kostenforecast. Kleinere Abweichungen können als Indiz gewertet werden, dass 
\paragraph{Zeit: } Diese Kategorie fasst sämtliche Indikatoren zur Abweichung der geplanten zur realisierten Projektlaufzeit zusammen. Das Interesse gilt vor allem die Zeitdivergenz zu lokalisieren. Hierfür wurde für fünf ausgewählte Milestones die Zeitdifferenz zwischen der geplanten und aktuellen Laufzeit gemessen. Die Milestones MS2 Concept approved, MS5 Point of no return, MS8 Documented, MS10 Takeover und MS11 Project sowie die  Closed sowie die gesamte Zeitdifferenz wurden hierbei berücksichtigt. 	\newline\newline\textbf{Komplexität:} Unter der Annahme, dass komplexere Projekte eher die Tendenz aufweisen zu scheitern, wurden Proxyvariablen ergründet, um die Komplexität eines Projektes abzubilden. Die Anzahl involvierter Zulieferer sowie der Verträge lässt den Schluss zu, dass das Projekt so komplex war, dass die Aufteilung der Zulieferung respektive die Einteilung in verschiedene Aufträge (Verträge) sinnvoll erschien.
\newline\newline\textbf{Erfolgskriterium:} Das Erfolgskriterium Abweichung von der budgetierten Marge errechnet sich aus realisierter und geplanten Deckungsbeitrag. Nach Abschluss des Projekts (nach MS 11) beginnt die zweijährige Garantieperiode. Die angefallen Kosten dieser Phase werden jedoch nicht direkt dem Projekt belastet, sondern laufen über ein anderes buchhalterisches Konto. Als Risikoprävention wird bei der Budgetierungsphase eines Projekts eine Marge von ca. 8\% zusätzlich einkalkuliert, so dass allfällige Mehrkosten während der Projektbearbeitungszeit abgefedert werden. Daraus folgt, dass sämtliche Projekte, welche diesen Kostenpuffer nicht benötigen beim Projektabschluss eine um diesen Betrag höhere Marge ausweisen. Die in der Analyse verwendeten Projektmargen sind bereits um diesen Faktor bereinigt und sind somit realitätsgetreu.
\nomenclature{SAS}{Sales and Service}
\nomenclature{FF}{Fulfillment}
\nomenclature{TO}{Turnover}
\nomenclature{MS}{Mechanical Supplies}
\nomenclature{MC1}{Mechanical supplies}
\nomenclature{ME}{Mechnical Engineering}
\nomenclature{PA}{Plant Automation}
\nomenclature{IS}{Installation}


