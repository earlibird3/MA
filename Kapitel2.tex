% !TEX root = MA.tex
\chapter{Theoretischer Rahmen}\label{sec:theor}
Nachfolgend werden das Projektmanagement und die Tendenzen zu neuen Projekterfolgsdefinition erläutert. Im Anschluss werden bisher erforschte Erfolgsfaktoren unter der Berücksichtigung unterschiedlicher Projektarten und Brachen aufgezeigt. Im nächsten Unterkapitel wird der Bühler Projektmanagementprozess beschrieben. Zuletzt erfolgt eine Einführung in das Thema der Früherkennung die den theoretischen Rahmen für die Diskussion des \ref{sec:diskfru} bildet.
%%
%%part Erfolgsfaktoren und Projektmanagement
% Erläuterung Projekt: UT Software/Konstruktion, Fokus der Arbeit (FdA): Maschinen/Anlageproj der Bühler AG
% Projektmgmt: Definition, Komplexität, Interdisziplinarität, Grundprozess nach Din
% Projektmgmt: Methoden: Agile und Traditionelle Methode, FdA: traditionller Ansatz der Bühler AG
% Projektmgmt: Einflussfaktoren im PM, die Erfolg begünstigen können in Abh. Projektarten, Projektmgmtmethoden
% Erfolg: Definition, Unterscheidung Erfolg und Kriterien, Ansätze: traditionelle vs. andere: FdA Bühler AG
% Erfolgsfaktoren: Forschungsstand: Relevante Faktoren und weitere Einflussfaktoren, Überleitung zu Bühler Prozes, Bühler Einflussfaktoren unter der Berücksichtigung der Bühler Erfolgsdefinition
\section{Erfolgsfaktoren im Projektmanagement} \label{sec:erfprj}	
Gemäss dem Deutschen Institut für Normung (DIN)  ist ein Projekt: \glqq ein Vorhaben, das im Wesentlichen durch Einmaligkeit der Bedingungen in ihrer Gesamtheit gekennzeichnet ist, z.B. Zielvorgabe, zeitliche, finanzielle, personelle und andere Begrenzungen, Abgrenzung gegenüber anderen Vorhaben, projektspezifische Organisation\grqq{ } (DIN 69901-5, zit. in \citealp*{alamg16}). Daraus folgt, dass Projekte an verschieden Vorgaben gebunden sind, die sich im Einzelnen voneinander unterscheiden können. Erst die Gesamtheit dieser Vorgaben begründet gemäss der obigen Definition die Einmaligkeitein Projekt. Infolge der internationalen Tätigkeit und des breiten Angebot an Maschinen zur Herstellung unterschiedliche Nahrungsmittel sind lediglich zwei Aspekte, welche die Einmaligkeit der Bühler-Projekte ausmachen können. Das Projektmanagement Handbuch \citeyear{pmhod} fügt als weitere Abgrenzungskriterien von Projekten die Ressourcenknappheit sowie die Notwendigkeit zur Teamarbeit an. Die Aufgabenstellung einiger Projekts kann sich zudem während der Laufzeit konkretisieren, das neue Informationen verarbeitet und die Vorstellung des Endprodukts klarer wird \citet[S.~1]{meyreh16}. Obwohl die Anlage und deren Spezifikation bei der Bühler AG im Vorfeld in Abstimmung mit dem Kunden festgelegt werden, können sich  sich Änderungen auf Wunsch des Kunden im Verlauf des Projekts ergeben. Basierend auf diesen Definition wird deutlich, dass die Abwicklung von Projekte eine geeignete Methode erfordert, sodass die zahlreichen Bedingungen eingehalten werden können. Das Projektmanagement ist ein \glqq generischer Managementprozess \grqq{ } der auf unterschiedliche Projekte angewendet werden kann (DIN 69 904 zit. in \citealp*{pmhod}. Er bezeichnet die \glqq Gesamtheit von Führungsaufgaben, -organisation, -techniken und -mitteln für die Initiierung, Definition, Planung, Steuerung und den Abschluss von Projekten\grqq{ } (DIN 69901-5, 2009 zit. in \citealp*[S.~3]{meyreh16}; \citet{pmhod}). Das Management von Projekten ist eine interdisziplinäre Aufgabe, die nach \citet[S.~2]{alamg16} auch "die Koordination von Menschen und der optimale Einsatz von Ressourcen zum Erreichen der Projektziele \grqq { }. Diese Definitionen verdeutlichen die umfassende Aufgaben des Projektmanagements und implizieren die Existenz unterschiedlicher Ansätze. An dieser Stelle wird nicht weiter auf die Methodiken eingegangen, da im Rahmen dieser Arbeit der Bühler Projektmanagementprozess des Kapitels \ref{sec:pmbueh} von zentraler Bedeutung ist, der sich grob an der Definition von DIN orientiert.
%%
%%Erfol und Erfolgsdef.
\newline\newline
Einleitend wurde das vorherrschende Paradigma zur Beurteilung des Projekterfolgs, das eiserne Dreieck Zeit, Kosten und Qualität erläutert. Aus einer finanziellen Perspektive liegt der Fokus auf dem Kostenaspekt, wobei dies die Zielerreichung der anderen primären Ziele beeinträchtigen kann. \citealp[S.~40]{kerz14} weist darauf hin, dass die Projekte selten innerhalb der Zielvorgaben abgeschlossen wurden, weshalb die historischen Definition um den Aspekt der Kundenakzeptanz erweitert wurde. Dadurch wurde die strikte Einhaltung der Dreierbedingung gelockert, da der Kunde trotz Mehrkosten oder Zeitverzug das Projekt akzeptieren konnte. Die Weiterentwicklung der Projektmanagementtechniken führte zur Erkenntnis, dass die Projekte an mehr als drei Bedingungen geknüpft sind,, die zusätzlich bei der Ermittlung des Erfolgs mittels Nebenzielen berücksichtigt werden sollen \citet[S.~41]{kerz14}. \citep{Atk1999} bemängelte zudem, dass die traditionelle Definition langfristige Nutzen nicht berücksichtigt. Zudem argumentierte er, dass die vorherrschende Erfolgsdefinition vermutungsweise der Anwendung neuer Instrumente, Fähigkeiten oder Projektmanagementansätze nicht Rechnung trage \citet{Atk1999}. Aus diesen Gründen und der bedingten Vergleichbarkeit von Projekten aufgrund ihrer Einmaligkeit sind neue Ansätze zur Bestimmungen des Projekterfolgs gefragt. \citep{lchch08} schlagen einen gewichteten Erfolgsindex auf der Basis der Kosten-Zeit-Qualität-Bedingung zur Beurteilung des Erfolgs vor, um die Projekte trotz ihrer charakteristischen Eigenschaften vergleichen zu können. \citep{kerz14} entwickelt gänzlich einen neue Projekterfolgsdefinition, die sich auf die Erreichung des gewünschten Geschäftswertes innerhalb der sich konfligierenden Zielvorgaben konzentriert. Diese Geschäftswerten können unternehmensbezogen ausgearbeitet werden, wobei angesichts der Evaluierung von mehreren Bedingungen eine vierteilige Erfolgskategoriesierung wie sie \citep[S.~48]{kerz14} vorschlägt, sinnvoll erscheint. Diese Ausführungen zeigen, dass letztendliche die Erfolgsdefinition vom Unternehmen abhängig ist und keinen Einschränkungen unterliegt. Der Analyse dieser Arbeit liegt einen finanzielle Erfolgsbeteiligung auf der Basis des Kostenaspekts der traditionellen Erfolgsdefinition zugrunde.
%%
%%Erfolgsfaktoren
\newline\newline
Nachfolgend werden bisher erforschte Erfolgsfaktoren dargelegt. Dabei kann sich die Beurteilung des Projekterfolges, der an mehrere Kriterien gebunden sein kann, zwischen den in den Studien untersuchten Unternehmen unterscheiden. Gemäss \citep{iyerjha06} hat das Engagement der Projektmitarbeiter und die Fähigkeiten des Projekteigners einen positiven Zusammenhang mit der Zeitperformance, wohingegen sich Konflikte zwischen dem involvierten Projektteilnehmer (bspw. Projektmanager, Top-Management, Projekteigner, externe Parteien) die Einhaltung der Zeitvorgabe beeinträchtigen können. Die Projektperformance in Bezug auf Zeit und Kosten hat mit den Einsatz des Projektteams, der unter anderem mit Vertrauenskultur, Konfliktlösung und Verständnis der Projektziele assoziiert wird, und den Kompetenzen des Kunden eine positive Korrelation\citet{chahota01}. Diese Erfolgsfaktoren können zusammenfassend in Fähigkeiten der involvierten Projektteilnehmer und Aspekte der Projektkultur eingeordnet werden. Weitere Erfolgsfaktoren von Konstruktionsprojekten sind zudem die Kompetenzen des Auftragsnehmer und die Anwendung innovativer Technologien \citep{chahota01}. Der Erfolg von Projekten, der anhand eines gewichteten Erfolgskriterium der Kosten, Zeit, Qualität und Funktionalität, kann zudem von der Effizienz des Projektmanagements und der Projektnatur abhängen, wie \citet{lchch08} bewiesen. Letztere gründet auf der Annahme, dass attraktive und komplexe Projekte (sprich die Projektnatur) mehr Aufmerksamkeit von den Projektmanagern erhalten, weil sie mit Prestige und Selbstverwirklichung verknüpft sind \citep{lchch08}. Die Erfolgsfaktoren, die positiv mit der Kostenperformance von Industrieprojekten korrelieren, sind gemäss \citet{luhuazha17} die Fähigkeiten des Auftragsnehmers. Denn sein Aufgabenbereich umfasst, das Design, die Beschaffung und die Konstruktion, welche für die Fertigstellung des Projekts \citep{luhuazha17}. Ling et al. (2004 in \citealp*{luhuazha17}) bemerkte zudem, dass fehlende finanzielle Kompetenzen des Auftragsnehmers die Kostenkontrolle erschweren.
\newline \citet{BeDeNov2015} haben vier Kategorien von Erfolgstreiber gebildet, Managementfähigkeiten, kritische Erfolgsfaktoren, Projektcontrolling und \glqq Lessons Learned\grqq{ } und festgestellt, dass Kommunikation in allen Bereichen des Projektmanagements von zentraler Bedeutung war. Der Erfolgsfaktor Kommunikation wurde bereits in früheren Studien ermittelt (Hyvräri, 2006 in \citealp*{BeDeNov2015}). Die Analyse der Unterschiede zwischen Software- und Konstruktionsprojekten führte zur Schlussfolgerung, dass das Verständnis der Projektziele und die Projektplanung von beiden Projektarten kritische Erfolgsfaktoren sind \citep{VarDom14}. Demgegenüber war der Zusammenhang des Faktors \glqq Miteinbezug aller Projektteilnehmer \grqq{ } bei Konstruktionsprojekten mit dem Projekterfolg höher als bei Software-Projekten. Folglich kann postuliert werden, trotz unterschiedlicher Projekttypen ähnliche Faktoren positiv mit dem Projekterfolg korrelieren.
\newline\newline Aus den vorangehenden Ausführungen geht hervor, dass trotz unterschiedlicher Beurteilung des Projekterfolgs, ähnliche Erfolgsfaktoren identifiziert wurden. Zusammenfassend kann deshalb postuliert werden, dass die im Projekt involvierten Personen und die Projektkultur zentrale Erfolgsfaktoren sind. Dabei bilden nachfolgende Attribute wie, Fehlerkultur, Teamfähigkeit, Konfliktfähigkeit, Vertrauen, gemeinsame Mission nur einen Teil derer Gesamtmenge ab, um die Art der Projektteilnehmer und das Arbeitsklima zu erfassen. Gemäss \citet{alamg16} sind dies jene Anforderungen, die während jeder Projektphase gegeben sein müssen, damit Projekte erfolgreich bearbeitet werden können. Die bedingte Vergleichbarkeit von Projekten aufgrund ihrer Natur scheint demzufolge hinsichtlich der Erfolgsfaktoren von zweitrangiger Bedeutung zu sein. Der Erfolg von Projekten kann durch unterschiedliche Kriterien gemessen respektive beurteilt werden. Angesicht der Einzigartigkeit und Lerneffekten von Projekten, erscheint die Anwendung von mehreren Erfolgskategorien sinnvoll. 
%%
%%part PM der Bühler AG 
%%subpart Prozess: Customer Project Prozess
%%subpart Einflussfaktoren: Summarische Erläuterung, Kategorien & Begründung, Erfolgskriterium nochmals erwähnen?, Verweis auf Kapitel 3, Hypothese: Dass Variablen Attribute von erfolgreichen respektive nicht erfolgreichen sein können.
\section{Projektmanagementprozess der Bühler AG}\label{sec:pmbueh}
In der Folge wird der Projektmanagementprozess der Bühler AG erläutert, welcher die Grundlage der Datenerhebung bildete. Der Fokus liegt dabei auf denjenigen Bestandteilen, die dem Verständnis der Daten dienen. Der Kundenprojektprozess der Bühler AG gliedert sich in zwei Subprozesse, wie die Abbildung \ref{fig:processcp} illustriert: den \gls{abk:sq} und den \gls{abk:ff}, die nachfolgend auch als Projektverkaufsprozess und -abwicklungsprozess bezeichnet werden. Das Bindeglied bildet das \gls{abk:HOM}, bei welchem der Verkaufsmanager auch Area Manager genannt das Projekt an den Projektmanager übergibt. Die einzelnen Prozessphasen sind durch \gls{abk:mst}, bei denen gewisse Anforderungen erfüllt werden müssen, beispielsweise die Genehmigung des Projektzeitplans oder gewisse Risikochecks, getrennt.
\begin{figure}[H]
	\centering
	\includegraphics[width=8cm]{processcp.png}
	\caption{Kundenprojektprozess der Bühler AG}
	\label{fig:processcp}
\end{figure}
Der als Kreis dargestellte \gls{abk:sq}-Prozess (s. Abbildung \ref{fig:processcp}) umfasst vier Phasen: Potenzial identifizieren (I), Prioritäten setzen (II), Anbieten und Risiko (III) evaluieren und Auftrag abschliessen (IV). Die involvierten Parteien im Prozess sind der Area Manager und das Backoffice, wobei erstere in direktem Kundenkontakt steht und letztere für die Angebotsstellung verantwortlich ist. Die Phase I und II konzentrieren sich darauf, Geschäftspotenziale und Kundenbedürfnisse zu identifizieren, Kontakte mit den Kunden aufzunehmen und letztendlich auf Basis von diversen Checks zu entscheiden, welche Projekte fokussiert. In der Phase III werden die Projektmöglichkeiten detailliert in technischer, kommerzieller und finanzieller Hinsicht geprüft, so dass nach der Ausarbeitung des Basiskonzepts dem Kunden ein Angebot unterbreitet werden kann. In der vierten Phase erfolgt nach der Risikoprüfung und allfälligen Verhandlungen der vertragliche Abschluss des Auftrages mit dem Kunden. Der Output dieses Prozess wird als \gls{abk:OR} bezeichnet, was die Freigabe es Auftrages bedeutet. Dieser Auftrag wird beim \gls{abk:HOM} an den Projektmanager übergeben, der die Verantwortung für den Projektabwicklungsprozess hat. Der Fulfillment-Prozess der Abbildung \ref{fig: processff} ist in fünf Phasen unterteilt. 
\begin{figure}[H]
	\centering
	\includegraphics[width=8cm]{processff.png}
	\caption{Projektabwicklungsprozess der Bühler AG}
	\label{fig: processff}
\end{figure}
\textbf{Phase I: Planning and Basic Engineering}
\newline
Bei der Übergabe des Projekts vom Verkauf an die Abwicklung sind typischerweise der Kunde, der Verkaufs- und Projektmanager sowie der Teamleiter involviert. Diese wichtige Schnittstelle dient dazu alle relevanten Informationen zu übergeben und offene Punkte zu klären. In dieser Phase erfolgt Projektanalyse durch den Projektmanager, die Ausarbeitung respektive Überarbeitung des Konzepts, die Projektplanung und das Kick-off-Meeting. Das Ziel der Projektanalyse ist die Realisierbarkeit mittels der Identifizierung von technischen als auch kommerziellen Risiken und Chancen sowie entsprechenden Massnahmen zu prüfen. Die anschliessende Konzeptphase beinhaltet die Ausarbeitung oder Nachbearbeitung des Maschinen- oder Anlagekonzept, das die interne und  externe Genehmigung voraussetzt. In der Projektplanung werden überwiegend organisatorische und administrative Aufgaben wie zum Beispiel die Planung der Liefertermine pro Meilenstein oder die Definition von Arbeitspaketen. Der letzte Schritt dieser Phase bildet das Kick-off-Meeting, welches der Schaffung eines gemeinsamen und einheitlichen Verständnis unter sämtlichen Teilprozessverantwortlichen dient. Unter der Berücksichtigung der vertraglichen Bedingungen werden verbindliche Vereinbarungen bezüglich der Termine, Kosten, Qualität und Zuständigkeiten getroffen. Zeitlich findet dieses Treffen in der Regel vor Freigabe durch die kaufmännischen Berater statt und bietet einen Diskussionsraum für ungeklärte Aspekte.
\newline\newline
\textbf{Phase II: Engineering and Specifications}
\newline
Die Ausarbeitung verbindlicher Pläne zur Anlagen- oder Maschinendisposition, Optimierungen am Maschinen- respektive Anlagenkonzept bilden die zentralen Aufgaben dieser Phase. Während der Design Meetings wird mit dem Kunden das Einverständnis hinsichtlich der Spezifikationen und Pläne schriftlich protokolliert, so dass Änderungen dem Risiko von Mehrkosten und Zeitverzögerung ausgesetzt sind. Der \gls{abk:mst}5 \glqq Point of now return\grqq{ } ist ein interner Meilenstein, bei dem die Liefertermine gegenüber dem Kunden verbindlich werden und eine Finanzierungslösung durch die Bühler AG sichergestellt sein muss.
\newline\newline
\textbf{Phase III: Manufacturing, Procurement \& Shipping}
\newline
Diese Prozessphase beginnt mit der Fabrikation und endet mit der Lieferung der Maschine an den vereinbarten Ort einschliesslich der der Beschaffung und der Dokumentation. Die Einhaltung des Liefertermins sowie die vertragskonforme Übergabe der Anlage ist hierbei von besonderer Wichtigkeit. Die Dokumentation wird bei der Installation der Maschine oder des Maschinenparkes benötigt und hat zudem die Gewährleistung der Nachvollziehbarkeit der Änderungen zum Zweck. Die Übergabe dieser Dokumentation an den Kunden und den Monteur begründet das Ende dieser Phase bei \gls{abk:mst}8 \glqq Documented \grqq.
\newline\newline
\textbf{Phase IV: Installation and Start up}
\newline
Die Installation, Inbetriebsetzung und Übergabe sind die elementaren Prozesschritte dieser Phase. Der Zusammenbau einer Anlage und die Inbetriebsetzung einer Maschine erfordert eine Instruktion des Montageteams, die zugleich eine unabdingbare Voraussetzung für ein gewisses Qualitätsniveau gewährt. Die Montageverantwortlichen werden durch die Projekt- und Verkaufsleiter laufend unterstützt, was die gleichzeitig die Überwachung des Installationsprozesses ermöglicht. Am Ende bei \gls{abk:mst}10 \glqq Take-over fulfilled\grqq{ } dieser Phase folgt nach abgeschlossener Inbetriebsetzung die Übergabe der Anlage an den Kunden. Dabei ist darauf zu achten, dass möglichst alle vertraglich vereinbarten Anforderungen, wie zum Beispiel Tests, Umfang und Darstellung der Übergabedokumente erfüllt werden, denn die letzten Kundenzahlungen sind oftmals an die Leistungserfüllung gekoppelt sind. 
\newline\newline
\textbf{Phase V: Evaluation and Transfer}
\newline
Zuletzt findet das Debriefing statt, bei welchem Rückmeldungen zur Optimierung der Projektabwicklung für künftige Projekt festgehalten, so dass die gleichen Fehler nicht wiederholt werden. Das Projekt wird danach beim \gls{abk:mst} intern abgeschlossen, wobei gleichzeitig die Projektabschlussfest zweijährige Garantieperiode beginnt.
\newline\newline
Die Daten, welche auf Basis dieses Prozess erhoben wurden, werden im Kapitel \ref{sec:methode} erläutert.
%%
%%part Frühwarnsystem im PM: theoretischer Rahmen für Diskussion
%Frühwarnsystem: Definition, Anwendung hauptsälich, Grund: Anwendung im PM, Anpassung Begrifflichkeiten
%Früherkennung: Definition (Watch Redundanzen), Methoden: Generationen, Fokus der Forschung 3. Generation
%Früherkennung: Einführung und Verwendung Begriff Frühwarnsignale, Methodische Ansätze summarisch,
%Früherkennung: Getestet und Bewährte Methoden
%
\section{Frühwarnsystem im Projektmanagement}
Frühwarnsysteme haben die Funktion zukünftige Risiken zu identifizieren, sodass proaktiv Massnahmen ergriffen werden können, um die akute Gefahr abzuwenden. Angesichts des überraschenden Projektscheiterns ist die frühzeitige Kenntnis neuer Herausforderungen von zentraler Bedeutung, damit potenzielle Mehrkosten vorgebeugt werden können. Im Projektmanagementtool der Bühler AG sind \glqq gefährdete\grqq{ }Projekte erst bei fortgeschrittener Projektlaufzeit ersichtlich. Zu diesem Zeitpunkt kann es bereits zu spät sein, um mittels reaktiver Handlungen entgegenzuwirken. Aus diesem Grund wird die Anwendung eines Frühwarnsystems evaluiert, um diese Situation zu optimieren und entscheiden Handlungsvorteile zu generieren. Aus finanzieller und unternehmerischer Sicht kann sich die Implementierung eines Frühwarnsystems auszahlen, da einerseits Kosteneinsparungen realisiert und anderseits die Erfolgswahrscheinlichkeit der Projekte beeinflusst werden kann. Im Anschluss sollen einige Arten von Frühwarnsysteme im Projektmanagement erläutert werden.  
\newline\newline
Der Informationsfluss in Frühwarnsystemen erfolgt über die Frühwarnindikatoren, die zukünftige Ereignisse zeitnah erfassen können \citep[S.~24]{jacrieg12}. Die ersten beiden Generationen von Frühwarnsystemen, die Orientierung an Kennzahlen, Hochrechnungen und Indikatoren konnten diesem Anspruch nicht gerecht werden \citep[S.~25-30]{jacrieg12}. Es wurde die unzureichende Aussagekraft bezüglich der zukünftigen Ereignisse, die Schwierigkeit Strukturbrüche zu erkennen sowie den Ausschluss weitere Indikatoren bemängelt \citep [S.~26-28]{jacrieg12}. Die dritte Generation basiert auf der Theorie von Ansoff (1997, zit. in \citealp{haan13}), gemäss derer sich Diskontinuitäten nicht plötzlich sonder relativ früh durch sogenannte schwache Signale, die als frühe Hinweise zu bevorstehenden einflussreichen Ereignisse zu verstehen sind, ankündigen. Da Unternehmen in der Regel die Ursachen des Projektscheiterns kennen und auch genügend Signale das mögliche Scheitern andeuteten, ist es relativ unverständlich, dass diese offensichtlichen Hinweise nicht berücksichtigt wurden \citep{haan13}. Die nachfolgende Tabelle \ref{tab:Ans} zeigt die methodische Ansätze zur Identifikation von Frühwarnindikatoren und Frühwarnsignalen. Der Begriff Frühwarnsignal leitet sich aus Theorie der schwachen Signale ab und ist zugleich ein Frühwarnindikator. Frühwarnindikatoren müssen nicht zwingendermassen ein Frühwarnsignale sein.
%Tabelle mit möglichen Ansätzen zur identifikation von Frühwarnsignalen
\begin{table}[H]
	\centering
	\begin{threeparttable}
		\caption{Ansätze zur Identifikation von Frühwarnsignalen und Frühwarnindikatoren}
		\begin{tabular}{l|l}
			\toprule
			Risikomanagement & Past Project Consultation\\
			Earned Value Management & Cause-Effect-Analyse\\
			Projekt Assessment Ansätze & Gut feelings \\
			Performance Management & Interface Analysis\\
			Stakeholder Analyse & Project Analysis \\
			Maturity Assessment & Project Surrounding Analysis\\
			\bottomrule
		\end{tabular}
		\begin{tablenotes}
			\small
			\item in Anlehnung an: \citealp{haan13}, Tabelle 1
		\end{tablenotes}
	\label{tab:Ans}
	\end{threeparttable}
\end{table}
\citealp[S.~59]{haan13} haben mittels mehrere Fallstudien eruiert, dass sich das Projekt Assessment sowie \glqq Gut feelings \grqq{ } in der Praxis bewährt haben, wobei Experten der Überzeugung waren, dass Frühwarnsignale qualitativer Natur sind und eher durch Intuition und Erfahrungswissen entdeckt werden. Klakegg, et.al (2010, zit. in \citealp{haankra13}) haben mittels Projektassessment unter anderem die Anzahl fehlender Informationen, fehlende oder verspätete Beurteilungen und Dokumentationen sowie unklare Anforderung als mögliche Frühwarnsignale ergründet. Missverständnisse bezüglich der Bedürfnisse, mangelnde Offenheit der Unternehmenskultur und Kommunikationsbereitschaft zwischen den Projektteilnehmer, sowie angespannte Projektatomsphäre wurden in Fallstudien mittels der \glqq Gut feelings\grqq{ }-Ansätzen als wichtige Früherkennungsindikatoren erkannt (Klakegg, et al., 2010 zit. in \citealp{haankra13}). Ein Nachteil von Projektassessments während der Projektlaufzeit ist, dass die gewonnen Erkenntnisse erst für künftige Projekte von Nutzen sein können. Die bedingte Messbarkeit qualitativer Frühwarnsignale (s. Tabelle \ref{tab:Ans}), kann einerseits als Nachteil interpretiert werden, da sich die Erfassung und Strukturierung der Daten mit hohem Aufwand verbunden sein kann. Anderseits ist die Erkennung von zusätzlichen Variablen, die bei konsequenter Überwachung ausgewählter Indikatoren unentdeckt bleiben können, als Vorteil einzustufen. Performance Management hat sich als quantitatives Frühwarnsystem in der Praxis bewährt \citealp{haan13}. Die kontinuierlichen Kontrolle der Bearbeitung von Schnittstellen-Problemen, der Mitarbeiterzufriedenheit und der Risiken wurden in einer Fallstudie als effiziente Frühwarnindikatoren identifiziert \citealp{haan13}. Dabei sind vor allem \glqq leading \grqq{ } Indikatoren von besonderem Nutzen, da sie bereits früh in der Ursachen-Wirkungs-Kette Gefährdungspotenzial aufzeigen. \glqq Lagging \grqq Indikatoren, wie zum Beispiel Kostenabweichungen können die potenziellen Risiken bereits zu spät erfassen, da sie erst zeitverzögert erhoben werden können. Die grösste Herausforderung bei der Anwendung von Projektmanagement als Frühwarnsystem ist die Selektion der Indikatoren, die den Monitoringbereich einschränkt, sodass andere Frühwarnsignale nicht erkannt werden.
% Faktoren von Klakegg und Kommentieren
% Faktoren von Performance Ansatz, 
% Leading Lagging Faktor,  
% wan ist früh, leading not lagging indicators, IdentifyAc says that performance only measures lagging 38
% possible early warning signs culture, lack of an outsiders perspective on the project, anchoring in the permanent organization, lack of consistency between stakeholders ambition and certain organizations. gut felt signs: detection of unrealism, lack of clarity, misalignment btw qualitative and quantitaive risk analysis 42
%no early signs in later stages, change...not used 43
%problems: difficult to stop projects despite EWS
%need for formalized proces for finding ealry warning signs, outside of the box thinking


