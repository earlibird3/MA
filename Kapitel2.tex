% !TEX root = MA.tex
\chapter{Theoretischer Rahmen}\label{sec:theor}
	
Der Erfolg von Projekten und deren Management ist ein in der Forschung viel diskutiertes Thema, weshalb hier in den nachfolgenden Kapiteln zunächst im Allgemeinen auf die Erfolgsdefinition von Projekten respektive Projektmanagement und bereits erforschte Projekterfolgsdeterminanten eingegangen wird. Im Anschluss folgt eine Erläuterung dieser Termini im Kontext mit der Bühler-Welt.
%%
%%part Erfolgsfaktoren und Projektmanagement
% Erläuterung Projekt: UT Software/Konstruktion, Fokus der Arbeit (FdA): Maschinen/Anlageproj der Bühler AG
% Projektmgmt: Definition, Komplexität, Interdisziplinarität, Grundprozess nach Din
% Projektmgmt: Methoden: Agile und Traditionelle Methode, FdA: traditionller Ansatz der Bühler AG
% Projektmgmt: Einflussfaktoren im PM, die Erfolg begünstigen können in Abh. Projektarten, Projektmgmtmethoden
% Erfolg: Definition, Unterscheidung Erfolg und Kriterien, Ansätze: traditionelle vs. andere: FdA Bühler AG
% Erfolgsfaktoren: Forschungsstand: Relevante Faktoren und weitere Einflussfaktoren, Überleitung zu Bühler Prozes, Bühler Einflussfaktoren unter der Berücksichtigung der Bühler Erfolgsdefinition
\section{Erfolgsfaktoren im Projektmanagement} \label{sec:erfprj}	
Gemäss dem Deutschen Institut für Normung (DIN) ist ein Projekt: " ein Vorhaben, das im Wesentlichen durch Einmaligkeit der Bedingungen in ihrer Gesamtheit gekennzeichnet ist, z.B. Zielvorgabe, zeitliche, finanzielle, personelle und andere Begrenzungen, Abgrenzung gegenüber anderen Vorhaben, projektspezifische Organisation" (Quelleangabe). Daraus folgt, dass Projekte sich bezüglich einzelner Faktoren unterscheiden können, allerdings die Gesamtheit der Faktoren ihre Einzigartigkeit definiert. Beispielsweise begründet bei der Bühler AG die internationale Tätigkeit, das diverse Anlageportfolio und die breite Kundenbasis ein Indiz für einmalige Projekte. Obwohl es unterschiedliche Projekte gibt, beispielsweise im Tiefbau, Hochbau und Ingenieurbau und sich deren Management sowohl durch Differenzen als auch Gemeinsamkeiten charakterisiert, weist ein Projekt gemäss Projektmanagement-Handbuch (ohne Datum) folgende Eigenschaften auf: "komplexe, neuartige Aufgabenstellung, messbare Ziele und Ergebnisse, zeitliche Befristung (Anfang und Ende), begrenzte Ressourcen und die Notwendigkeit von Teamarbeit" (Quellenangabe). Meyer und Rehrer (2012, S.2) sehen die progressive Elaboration, die eine kontinuierliche Konkretiesierung des Projekts während dessen Verlauf als weiteres Merkmal von Projekten. Der exakte Projektbegriff der vorliegenden Arbeit orientiert sich anschliessend an den geschäftsinternen Definitionen der Bühler AG
\newline
Der Managementbegriff wird vom Projektmanagementhandbuch (ohne Datum, Jahr) als systematischer Prozess zur Führung komplexer Variablen definiert. Er beinhaltet die Organisation, Planung, Steuerung und Überwachung aller Aufgaben und Ressourcen, die notwendig sind, um die Projektziele zu erreichen". Das Projekt Management Institute (PMI) (PMBOK, 2004) beschreibt Projektmanagement als eine Anwendung von Wissen, Fähigkeiten, Instrumente und Techniken bei Projektaktivitäten, um Projektanforderungen zu erfüllen. Nach Alama und Gühl (2016) wird Projektmanagement als "die Koordination von Menschen und der optimale Einsatz von Ressourcen zum Erreichen von Projektzielen dargelegt. Pierce (S.2015, S.2) führt eine generelle Definition aus, gemäss derer Ziele, Prozesse, Planung und Kontrolle den Managementterminus beschreiben. Die Literatur zeigt keine einheitliche Definition, dennoch kann zusammengefasst konstatiert werden, dass Projektmanagement die zielgerichtete Planung, Steuerung und Überwachung von  Ressourcen in den Prozessschritten umfasst. Der Projektmanagementprozess kann generell in folgende Schritte unterteilt werden; Projektinitiierung, Projektplanung, Projektdurchführung und -kontrolle, und Projektabschluss (PMHandbook, ohne Datum). Der der Analyse zugrundeliegende Prozess wird im Kapitel \ref{zweizwei} erläutert, weshalb an dieser Stelle nicht weiter auf den Prozess als solches eingegangen wird. 
\newline
Der Erfolg von Projekten beschäftigt die Forschung seit längerer Zeit. Das vorherrschende Paradigma zur Beurteilung des Projekterfolgs ist das magischen Dreieck, Zeit, Kosten und Qualität. Allerdings ist in der Praxis letztendlich der Kostenaspekt von zentraler Bedeutung, da er mit Geldverlust korreliert ist. Die aktuelle Forschung der Erfolgsfaktoren zeigt, dass der Erfolg von Projekten nicht auf ein Faktor reduziert werden kann. Deshalb unterscheidet Besteiero, Pinto \& Novaski (2015) zwischen kritischen Erfolgsfaktoren und Erfolgskriterien. Die Faktoren erhöhen ihrer Auffassung nach die Erfolgswahrscheinlichkeit wohingegen die Kriterien darüber bestimmen, ob ein Projekt erfolgreich war (Besteiro, Pinto, Novaski, 2015). Hieraus entsteht eine Differenz in der zeitlichen Betrachtung, Faktoren sind während der Projektabwicklung relevant, um das Projekt erfolgreich abzuschliessen wohingegen Kriterien erst nach dem Projektabschluss hinzugezogen werden, um den Erfolg zu bestimmen. Folglich wäre der monetäre Aspekt ein Erfolgskriterium. Es muss jedoch hinterfragt werden, ob sich die Erfolgsbeurteilung von Projekten angesichts ihrer Eigenartigkeit auf ein Kriterium reduzieren lässt.  Denn bei einer reinen Kostenbetrachtung werden die Einhaltung von Zeitvorgaben und die Qualitätsanforderung ausser Acht gelassen. Kerzner (2014) schlägt anstatt der traditionellen Erfolgsbetrachtung vor, den Projekterfolg als die Erreichung des gewünschten Geschäftswertes innerhalb der sich konfligierenden Zielvorgaben zu verstehen. Daraus folgt, dass ein gewisser trade-off resultiert, da beispielsweise der Liefertermin nicht eingehalten werden kann, aber neues Wissen generiert werden konnte, welches in anderen Projekten von Nutzen sein kann. 
\newline
Nachfolgend werden einige Ergebnisse bisheriger Studien zu den Erfolgsfaktoren dargelegt, wobei sich sowohl die Branche (IT und Schwerindustrie) als auch die Erfolgsdefinitionen unterscheiden. Grundsätzlich liegt dem Konzept der Erfolgsfaktoren die Prämisse, dass Erfolg wiederholbar ist und an bestimmte Faktoren geknüpft ist, zugrunde. Folglich müsste es auch Determinanten geben, welche den Projekterfolg negativ begünstigen. Iyer \& Jha (2006) fanden mittels Expertenbewertungen und anschliessender Faktoranalyse heraus, dass das sich Engagement der Projektmitarbeiter und die Fähigkeiten des Projekteigners positiv auf die Zeitperformance auswirken. Konflikte zwischen dem Projektmanager und Top-Management, dem Projekteigner oder anderen externen Parteien sowie eine Missgunstkultur können zur Überschreitung der vorgegebenen Projektzeit führen (Iyer \& Jha, (2006). Ein ähnliches Ergebnis ergab die Studie von Chan, Ho \& Tam (2001), wonach das Engagement des Projektteams, das im weiteren Sinne Kooperation, Konfliktlösung, Vertrauenskultur, Verständnis der Projektziele sowie Kommunikation bedeutet, als kritische Erfolgsfaktoren zu qualifizieren sind. Dieses Ergebnis ist mit den Befunden früherer Studien (Ashley, et al. 1987, Pinto und Slevin, 1988) kompatibel, bei denen das Engagement der Projekteilnehmer ebenso als erfolgskritisch identifiziert wurde. Chan, Ho \&Tam (2001) machten zudem die Konklusion, dass die Fähigkeiten des Auftragsnehmers hinsichtlich eines qualitativen Projektmanagements und der Anwendung innovativer Technologien sowie die Kundenkompetenz in Bezug auf die Abwicklung von Konstruktionsprojekten eine entscheidende Rolle für den Projekterfolg haben. Demgegenüber sind bei Softwareprojekten vor allem die Unterstützung des Managements, Kommunikation oder das gemeinsame Verständnis Projektmission von grosser Bedeutung (Hyvräri, 2006 in Besteiro, Pinto, Novaski, 2015). Der Einbezug von Studien, die unterschiedliche Projekte analysierten, hatte zum Ziel, Unterschiede in Bezug auf die Erfolgsfaktoren feststellen zu können. Varajão, Dominguez \& Ribeiro et al. (2014) stellten fest, dass die Projektplanung und das Verständnis der Projektziele sowie -anforderungen bei beiden Projektarten kritische Erfolgsaspekte sind. Allerdings wird beispielsweise die Effizienz des Projektmanagements und der Miteinbezug aller Projektteilnehmer bei Konstruktionsprojekten als kritischer für den Erfolg erachtet als bei Software-Projekten. Unter der Anwendung eines gewichteten Erfolgskriterium aus Kosten, Zeit, Qualität und Funktionalität haben Lam, Chan \& Chan (2008) herausgefunden, dass die Projektnatur, die Effizienz des Projektmanagement und die Anwendung innovativer Managementtechnologien einen erfolgreichen Projektabschluss begünstigen. Attraktive und komplexe Projekte würden die Projektmanager dazu veranlassen, mehr Effort für das Projekt zu leisten, da solche Projekte mit Prestige und Selbstverwirklichung verbunden sind (Lam, Chan \& Chan, 2008). Mittels logistischer Regressionsanalyse fanden Lu, Hua \& Zhang (2017) auf Basis einer Kostenperspektive heraus, dass die Fähigkeiten des Auftragsnehmers aufgrund von Erfahrungen und bezüglich der Teamfähigkeit sowie Kostenaffinität relevant für den Erfolg des Projekts sind. 
\newline Aus den vorangehenden Ausführungen geht hervor, dass trotz unterschiedlicher Erhebungsmethoden, Erfolgskriterien und Projektarten ähnliche Erfolgsfaktoren identifiziert wurden. Zusammenfassend kann deshalb postuliert werden, dass der Erfolg mit den Projektteilnehmern und der Projektkultur korreliert. Dabei stellen Attribute wie, Fehlerkultur, Teamfähigkeit, Konfliktlösen, Vertrauen, gemeinsame Mission oder Wertschätzung nur eine Auswahl dar, um die Projektteilnehmer und das Arbeitsklima zu beschreiben. Alam \& Gühl (2016) sprechen in diesem Zusammenhang auch von denjenigen Anforderungen, die während jeder Projektphase gegeben sein müssen, damit Projekte erfolgreich bearbeitet werden können. Der Erfolg von Projekten bestimmt sich auf Basis der gewählten Erfolgsdefinition, wobei die Vergleichbarkeit zwischen den Projekten aufgrund ihrer Einzigartigkeit nur bedingt möglich ist. Das vorherrschende Erfolgsparadigma, das magische Dreieck, wird jedoch kritisch gewürdigt, da bisher nicht-monetäre Wertgenerierung aussser Acht gelassen wurde. Denn ausschliessliche Fokussierung der Kosten- und Zeitvorgaben berücksichtigt die Änderungen des Projektinhalts aufgrund neuer Ereignisse während der Projektlaufzeit  nicht.
%%
%%part PM der Bühler AG 
%%subpart Prozess: Customer Project Prozess
%%subpart Einflussfaktoren: Summarische Erläuterung, Kategorien & Begründung, Erfolgskriterium nochmals erwähnen?, Verweis auf Kapitel 3, Hypothese: Dass Variablen Attribute von erfolgreichen respektive nicht erfolgreichen sein können.
\section{Projektmanagement der Bühler AG}
In diesem Kapitel wird der Bühler Projektmanagement-Prozess, der die Basis für die Ergründung der möglichen Einflussfaktoren bildete, erläutert. Die nachfolgenden Ausführungen basieren auf den intern dokumentierten Prozessbeschreibungen des C2C-Prozesses, Customer Project (CP). Er gliedert sich in zwei Kernprozesse, den Sales \& Quotation- und den Fulfilment-Prozess. Anschliessend hat das zweite Unterkapitel zum Ziel, die Faktoren, welche möglicherweise den Projekterfolg beeinflussen oder Charakteristiken von nicht-erfolgreichen Projekten bilden sowie ihre Bedeutung erklärt.
\subsection{Projektmanagementprozess der Bühler AG}\label{zweieins}
Die nachfolgende Abbildung \ref{fig: processcp} zeigt die zwei Kernprozesse, wobei beide Prozesse durch das Hand-over-Meeting (HOM) direkt ineinander übergehen. 
\begin{figure}[H]
	\centering
	\includegraphics[width=5cm]{processcp.png}
	\caption{Prozess Customer Project der Bühler AG}
	\label{fig:processcp}
\end{figure}
Links in der Abbildung \ref{fig:processcp} ist der SQ-Prozess dargestellt, der durch die Übergabe des Projekts in den FF-Prozess mündet. In der Folge sollen beide Prozesse zusammengefasst beschrieben werden, wobei der Fokus auf denjenigen Bestandteilen liegt, die im Zusammenhang mit den Erfolgsdeterminanten stehen. Der Verkaufsprozess umfasst vier Phasen: Identify Potential, Set Priorities, Quote and Evaluate Risk und Close Order. Die Verantwortlichkeit für den Prozess liegt hauptsächlich beim Area Manager(AM). Die Business Unit (BU) und entsprechende Centers of Competences tragen eine Mitverantwortung. Das Fundament des CP-Prozess bilden die Milestones (MS), welche die Erreichung oder die Lieferung vordefinierter Ziele einfordern, bevor mit dem Prozess weitergefahren werden darf. Die Phase I und II konzentrieren sich darauf, Geschäftspotenziale und Kundenbedürfnisse zu identifizieren, Kontakte mit den Kunden aufzunehmen und letztendlich auf Basis von diversen Checks zu entscheiden,  welche Projekte fokussiert, das heisst, offeriert werden sollen. In der Phase III und IV gilt es die möglichen Projekte einer Detailprüfung in technischer, kommerzieller und finanzieller Hinsicht zu unterziehen sowie ein Basiskonzept auszuarbeiten. Nach den anschliessenden Vertragsverhandlungen zwischen dem Kunden und der BU respektive dem AM endet der Prozess mit dem MS Orders Released (OR.) Der Auftrag wird freigegeben und das Projekt wird nach kurzer Zeit beim HOM an die Projektabwicklung übergeben. Im Fulfillment werden fünf Phasen unterschieden, wie der Abbildung \ref{fig: processff} zu entnehmen ist.
\begin{figure}[H]
	\centering
	\includegraphics[width=5cm]{processff.png}
	\caption{Prozess Customer Project der Bühler AG}
	\label{fig: processff}
\end{figure}
\textbf{Phase I: Planning and Basic Engineering}
\newline
Bei der Übergabe des Projekts vom Verkauf an die Abwicklung sind typischerweise der AM, der PM und der Teamleiter involviert. Diese wichtige Schnittstelle dient dazu alle relevanten Informationen zu übergeben und offene Punkte zu klären. Die Phase I beinhaltet die Projektanalyse, die Ausarbeitung respektive Überarbeitung des Konzepts, die Projektplanung und das Kick-off-Meeting (KOM). Das Ziel der Projektanalyse ist die Realisierbarkeit mittels der Identifizierung von technischen als auch kommerziellen Risiken und Chancen sowie entsprechenden Massnahmen zu prüfen. Die anschliessende Konzeptphase beinhaltet die Ausarbeitung oder Nachbearbeitung des Maschinen- oder Anlagekonzept und sowohl die interne als auch die externe Genehmigung einzuholen. In der Projektplanung werden überwiegend organisatorische und administrative Aufgaben wie zum Beispiel die Planung der Deadline oder die Definition von Arbeitspaketen gemacht. Der letzte MS dieser Phase bildet MS5, das Kick-off-Meeting, welches der Schaffung eines gemeinsamen, identischen Verständnis unter sämtlichen Teilprozessverantwortlichen bildet. Im KOM werden in Abstimmung mit den vertraglichen Bedingungen, verbindliche Vereinbarungen bezüglich, Termine, Kosten, Qualität und Zuständigkeiten getroffen. Ausserdem bietet das KOM Raum zur Diskussion unklarer Punkte. In der Regel findet das KOM nach der kommerziellen respektive Gesamtfreigabe statt.
\newline
\textbf{Phase II: Engineering and Specification}
\newline
Charakteristisch für diese Phase ist die Ausarbeitung verbindlicher Pläne zur Anlagen- oder Maschinendisposition. Optimierungen am Maschinen- respektive Anlagenkonzept und der interne Review sind während dieser Zeit von zentraler Bedeutung. Nach der schriftlichen Einverständniserklärung des Kunden zum Layout der Prozessanlage sind Änderungen dem Risiko von Mehrkosten, Zeitverzögerung und anderen Anpassungen in der Projektstruktur ausgesetzt. Der MS 5 "Point of now return" ist ein interner Meilenstein, deren Erreichung die Verbindlichkeit der Liefertermine gegenüber dem Kunden sowie die Sicherstellung der Kundenzahlung, erfordert. 
\newline
\textbf{Phase III: Manufacturing, Procurement and Logisic Out}
\newline
Diese Phase beginnt mit der Fabrikation und endet mit der Lieferung der Maschine an den Sitz des Kunden respektive an den vereinbarten Lieferort. Die Einhaltung des Lieferversprechen sowie die vertragskonforme Ablieferung der Anlage oder Maschine(n) ist hierbei von besonderer Wichtigkeit. Der nachfolgende Prozessschritt 9 Project Documentation umfasst die Erstellung der Projektdokumentation für den Monteur und den Kunden, welche die Nachvollziehbarkeit der Änderungen garantiert. Das Ende dieser Phase wird durch den MS8 festgehalten, der erreicht wird, wenn die Dokumentation offiziell an den Kunden und den Monteur überreicht wurde. 
\newline
\textbf{Phase IV: Installation and Start up}
\newline
Die Installation, Inbetriebsetzung und Übergabe sind die elementaren Prozesschritte dieser Phase. Der Zusammenbau einer Anlage und die Inbetriebsetzung einer Maschine erfordert eine Instruktion des Montageteams, die zugleich eine unabdingbare Voraussetzung für ein gewisses Qualitätsniveau gewährt. Die Montageverantwortlichen werden durch die Projekt- und Verkaufsleiter ständig unterstützt, wobei gleichzeitig die Überwachung des Prozesses gewährleistet wird. Der Meilenstein 10 wir mit der kompletten Übergabe der Anlage an den Kunden nach dem Abschluss der Inbetriebsetzung erreicht. Hierbei ist darauf zu achten, dass möglichst alle vertraglich vereinbarten Anforderungen, wie zum Beispiel Tests und Inhalt, Umfang und Darstellung der Übergabedokumente erfüllt werden, da oftmals die letzten Kundenzahlungen an die Leistungserfüllung gekoppelt sind. 
\newline
\textbf{Phase V: Evaluation and Transfer}
\newline
Als letztes folgt das Debriefing, bei welchem im Sinne des kontinuierlichen Verbesserungsprozess, Rückmeldungen zur Optimierung der Projektabwicklung für künftige Projekt festgehalten werden, so dass die gleichen Fehler nicht wiederholt  werden. Der elfte Meilenstein legt intern den Projektabschlussfest. Danach beginnt die zweijährige Garantieperiode. 
\subsection{Einflussfaktoren im Projektmanagement der Bühler AG}\label{zweizwei}
Nachfolgend werden sämtliche Einflussdeterminaten und das Projektmanagement-Tool, das sämtlich Projektinformationen enthält, erläutert.
\newline
Das BPM-Cockpit enthält Informationen zu Kosten, Zeit, Verantwortlichkeiten, Risiken und Aktionen als auch zum Engineering inklusive verschiedener Projektstatus. Die Beurteilung der Projektperformance erfolgt anhand eines dreifarbigen Ampelsystems für Kosten, Zeit und Qualität. Die ersten beiden werden vom System automatisch gerechnet wohingegen die Qualität auf der subjektiven Einschätzung der Projektmanager beruht, die sie selbst berichten können. Basierend auf der internen Richtlinie für das BPM-Cockpit ist der Kostenstatus grün, wenn die konsolidierte Abweichung der Projektmarge zwischen dem Forecast und dem Budget mehr als -400 Prozentpunkte beträgt. Dieser Status ändert von grün auf gelb, sobald die Abweichung weniger als -4\% beträgt und von gelb auf rot, wenn die Schwelle von -10\% unterschritten wird. Die Zeitampel basiert einerseits auf der Differenz zwischen dem realisierten und dem geplanten Termin und anderseits auf der Eintragung im System. Sofern Angaben zum realisierten Termin im System enthalten sind, ist der Status grün, wenn er vor oder auf dem geplanten Datum liegt und gelb, wenn das Zeitversprechen nicht eingehalten wurde. Die rote Farbe impliziert, dass das BPM-Cockpit keine Informationen zur Erfüllung des Termins enthält und der geplante Termin vor dem heutigen Datum liegt. Die Farbe der Ampeln hängt davon ab, ob der Forecast für eine Reportingperiode angepasst wurde oder nicht. Je nach Grösse der Abweichung zwischen dem Budget und dem Forecast wird eine Erklärung vom entsprechenden Geschäftsbereich erwartet. Dies kann dazu führen, dass die Anpassung der ProgMse hinausgezögert wird.
\newline
Obwohl die Performance eines Bühler-Projekts mittels des magischen Dreiecks - Zeit, Kosten und Qualität - beurteilt wird, hat letztendlich der Kostenaspekt aus finanzieller Perspektive die relativ gewichtigere Bedeutung als die anderen zwei Dimensionen. Deshalb wurde in Zusammenarbeit mit der Bühler AG, das folgende Erfolgskriterium festgelegt:
\begin{equation}
	\text{Deviation DB1 \%} = \text{DB1 Act \%} - \text{DB1 Bud \%}
\end{equation}
Der KPI rechnet sich realisierte Marge (DB1 Act) in \% minus budgetierte Marge (DB1 Bud) in \% und wird jeweils am Projektende respektive nach Erreichung des MS11 (Kapitel \ref{zweieins}) kalkuliert. Die Kosten der Garantieperiode werden nicht direkt dem Projekt belastet sondern summarisch in einem anderen buchhalterischen Konto. Bei jedem Projekt wird präventiv ein Kostenpuffer im Bereich von 4\% bis 9\% einkalkuliert, der nach dem Projektende (MS11) bei einer Nullbeanspruchung im Projektergebnis realisiert wird.
\newline Die nachfolgende Abbildung \ref{Einflussfaktoren} zeigt die ursprünglich 93 Determinanten eingeteilt in sechs Kategorien. Gewisse Faktoren könnten ihrer Natur nach auch in eine andere Gruppe gegliedert werden. In der Folge wird die Bedeutung und Relevanz der Variablen erklärt.
%%%%%%%%%%%%%%%%%%%%%%%% WAraanty shit!
\begin{figure}[H]
	\centering
	\includegraphics[width=90mm]{Model.jpg}
	\caption{Einflussfaktoren
	\label{Einflussfaktoren}}
\end{figure}
\paragraph{Rahmenbedingungen:} In dieser Kategorie sind kategoriale Variablen zusammengefasst, die den eindeutigen Rahmen eines Projekts festlegen. Dazu gehören die Region (Region) respektive das Land (EquLoc), in welchem die Anlage gebaut wird, der Kunde (CuNo) und Geschäftsbereich (BA, BU und MS). Ausserdem zählt die relative Wichtigkeit eines Projekts (BAImportPr, BUImportPr und MSImportPr) für den entsprechenden Geschäftsbereich ebenso zu den Rahmenbedingungen. Die zugrundeliegende Hypothese unterstellt, dass bestimmte Kombinationen der Charakteristiken den Projekterfolg begünstigen. Kunden beispielsweise lassen sich bezügliche der individuellen Anlagespezifikationen, ihrer Bonität oder Kultur unterscheiden. Die Region in welcher die Anlage gebaut werden soll, birgt differenzierbare Risiken im Bereich der Politik, Wirtschaftsentwicklung oder länderspezifischer Handelsregelungen. Der Geschäftsbereich kann als eindeutiges Diversifikationskriterium der Anlage gewertet werden. Obwohl der Projektmanagementleitfaden intern universelle Gültigkeit hat, können während der Projektlaufzeit verschiedene Herausforderungen in Abhängigkeit der jeweiligen Anlage auftreten. Zudem kann davon ausgegangen werden, dass die Teamarbeit und Teamkultur pro Geschäftsbereich und -einheit verschieden sind und den Projekterfolg unterschiedlich beeinflussen. Die Wichtigkeit eines Projekts, das Umsatzbudget des Projekts im Verhältnis zum Median des Umsatzbudgets aller laufenden Projekte, kann als Indikator zur Konzentration von Ressourcen bei der Projektabwicklung interpretiert werden. Demzufolge müssten bedeutendere Projekte, die auch einen erheblichen Einfluss auf das Geschäftsbereichsergebnis haben, mehr Aufmerksamkeit in Bezug auf Risikominimierung erhalten.
\paragraph{Sales \& Quotation (SQ):} Der Verkaufsprozess geht unmittelbar in die Projektabwicklung über, weshalb die vorgelagerten Entscheidungen direkt oder indirekt den Projekterfolg beeinflussen können. Zum Beispiel beeinflussen die Qualität der Offerte sowie die vertraglichen Vereinbarungen die Rahmenbedingungen für die Projektabwicklung. Die Offertstellung und vorgängige Risikoanalysen des Projekts liegen im Aufgaben- und Verantwortungsbereich des Area Managers (AM und AMNo). Es wird davon ausgegangen, dass erfahrenere (AMAge) und langjährige (AMTen) Verkaufsmanager über mehr Kenntnisse zu den Projekten allgemein, als auch deren Risiken und dem internen Prozess verfügen und deshalb 'erfolgreichere' Projekte verkaufen. Die Incentivierung und Performancemessung der Verkaufsmanager erfolgt über das Auftragsvolumen des Geschäftsbereichs und der Region. Die Abweichung von den Budgetvorgaben auf regionaler Ebene und der Geschäftsbereichsstufe (BUORBudGap und RegioORBudGap) zum Zeitpunkt des Projektabschlusses kann den Zielerreichungsdruck vor allem zum Jahresende erhöhen. Die Schnittstelle zwischen Sales \& Quotation und dem Fulfillment ist für den Projekterfolg von zentraler Bedeutung, weshalb eine Zeitverzögerung zwischen der Auftragsfreigabe (ORDate) und dem Projektbeginn (PrStartDate) als Indiz für Komplikationen, Unklarheiten und Unsicherheiten interpretiert werden kann. 
\paragraph{Fulfillment (FF): } In dieser Kategorie werden sämtliche Faktoren im Zusammenhang mit dem Projektmanager (PM und PMNo), dem Forecastmanagement (FC-Management) und der Unternehmensverantwortung subsumiert, da sie den Projektabwicklungsprozess tangieren. Der Betriebszugehörigkeit der Projektmanager (PMTen) sowie dessen Erfahrungsschatz (PMAge) sind stellvertretende Variablen für die Kenntnisse der internen Prozesse und das vorhandene Wissen in Bezug auf den Beruf. Bei Unstimmigkeiten zwischen dem Kunden und dem Projektmanager, kann letzterer ersetzt werden (PMChange). Je nach Status des Projekts und Zeitpunkt des Wechsels können nicht alle Differenzen durch den neuen Projektmanager kompensiert werden, weshalb ein Austausch als Indiz für nicht-erfolgreiche Projekte betrachtet wird. In sehr seltenen Fällen muss die Funktion des Projektmanagers sogar mehrmals neu besetzt werden (NoPM), was den positiven Ausgang eines Projekts beeinträchtigen kann.
\newline Die organisatorische Verantwortung für das ganze Projekt (LeadSASPr) und den Abwicklungsprozess (LeadSASFF), kann bei einer Gesellschaft oder zwei verschiedenen Gesellschaften (LeadSAS.PrFF) angesiedelt sein. Die zusätzliche Schnittstelle erhöht den Komplexitätsgrad eines Projekts und kann deshalb nachteilig auf den Projekterfolg wirken. Die Zusammenarbeit sowohl zwischen den Gesellschaften als auch innerhalb der Unternehmen kann sich voneinander unterscheiden, weshalb einige Gesellschaften wahrscheinlich mehr Erfolg im Projektmanagement aufweisen.
\newline Das  Forecastmanagement liegt im Verantwortungsbereich des Projektmanagers und bezieht sich auf die Prognose des Projektumsatzes, der -kosten sowie der -marge, welche monatlich geprüft und entsprechend angepasst werden muss. Das frühzeitige Erkennen von drohenden Mehrkosten kann deren Verminderung oder Vermeidung begünstigen. Deshalb wurde pro Projektphase, Mechnical Supply \gls{abk:MS}, Mechnical Engineering \gls{abk:ME}, Plant \&Automation \gls{abk:PA} und Installation \gls{abk:IS} erhoben, ob der Forecast angepasst wurde. Dabei wurde zwischen 'nur Mehrkosten' und 'Mehrkosten inklusive Umsatzerhöhung' unterschieden, (CostFCadj). Zudem wurde die Anzahl Monate zwischen dem Projektabschluss (MS11) und der negativsten FC-Anpassung (CostmostnegFCadj) gemessen. Die Abweichungen von den Vorgaben in Bezug auf Zeit und Kosten wird systemisch automatisch berechnet und  durch das dreistufige Ampelsystem des Bühler Projektmanagement-Cockpit (BPM-Cockpit) reflektiert. Obwohl intern vorgeschrieben wird, dass jede mögliche Veränderung in der monatlichen Prognose unverzüglich einfliessen muss, wird aufgrund des Begründungszwangs bei hohen Abweichungen versucht, die Kommunikation der negativen Veränderung so lange wie möglich hinauszuzögern.  % Erklärung wieso HOM
Deshalb wurde die Periode zwischen dem erstmaligen gelben respektive roten Status und dem Projektbeginn HOM gemessen und ins Verhältnis zur vereinbarten Projektzeit (HOMRed/YellowCost/Time/Quality) gesetzt.  %Auf diese Weise kann herausgefunden werden..
\paragraph{Kosten:} Die monetären Aspekte eines Projekts umfassen Umsatz (TO), Kosten (Cost), Marge (DB1), Budget (Bud) und realisierte Zahlen (Act), deren Vergleich und die monetären Abweichungen zwischen dem letzten Forecast und den Istzahlen. Ein höheres Umsatzbudget (TOBud) wird mit komplexeren und umfangreicheren Projekten, die ein höheres Mass an Planung, Ressourcen sowie Betreuung erfordern, assoziiert. Ausserdem ist ihr finanzieller Einfluss auf das Geschäftsbereich- bzw. Regionenergebnis von besonderer Wichtigkeit. Die Kostenabweichungen (CostActBud) pro Projektphase in absoluten und relativen Grössen sollen Aufschluss über die Verlustbereiche geben. Die Zusammensetzung der budgetierten Kosten pro Projektphase in Relation zum Umsatz (BudMS/ME/PA/ISTot) kann zudem Aufschluss über die Projektart geben, da beispielsweise ein hoher Engineering-Anteil erwartungsgemäss eher mit Mehrkosten einhergeht als ein hoher MS-Anteil. Es wurden zusätzlich die Kosten aus Nachlieferungen infolge Nichteinhaltung des vereinbarten Liefertermins in das Modell mit einbezogen, da hypothetisch vermutet wird, dass dieser Kostenanteil in Bezug zum Umsatzbudget bei nicht-erfolgreichen Projekten höher sein muss. Im Zusammenhang mit der Projektmarge liegt der Fokus vor allem auf Projekten mit einem Budget nahe des intern festgelegten Grenzwertes von 23\%. Denn sämtliche Projekte, deren budgetierte Projektmarge unter diesem Wert liegt, bedarf einer Zustimmung zur Eingehung dieses Risikos sprich der Projektdurchführung durch die nächst höhere Managementstufe. Davon ausgehend, dass aufgrund des Budgetdrucks versucht wird diesen Genehmigungsprozess zu umgehen, wird vermutete, dass risikoreichere Projekte verkauft werden, die letztendlich eher mit negativer Performance einhergehen. Die Abweichung der realisierten Kosten vom letzten Kostenforecast (DeltaLastFCAct) pro Projektphase soll zudem erfassen, wie hoch die Mehrkosten kurz vor dem Ende der Projektlaufzeit waren, um Rückschlüsse auf den Zeitpunkt der Herausforderungen machen zu können. %%%%%%%%%%%%%%%%%%%%%%%%%%%%%%%%%%%Prüfen und Bühler besprechen
\paragraph{Zeit: } In dieser Gruppe sind alle Variablen, die in Verbindung mit der Projektlaufzeit und den Milestones stehen zusammengefasst. Die Einhaltung des Liefertermins sowie die Lokalisierung von Zeitverzögerungen anhand ausgewählter Milestones sind hierbei von grossem Interesse. Dazu wurde die Zeitdifferenz (PrTimeDelay) zwischen der vereinbarten (PrTimeBaseline) und der erreichten Projektlaufzeit(PrTimeAct) für das ganze Projekte und die folgenden Milestones gemessen: MS2 Concept approved, MS5 Point of no return, MS8 Documented, MS10 Takeover und MS11 Project.
\paragraph{Komplexität:} Die Komplexität eines Projekts kann unterschiedliche Dimensionen betreffen, so zum Beispiel können die technische Anforderung an die Anlage, die Anzahl involvierter Parteien, die Zusammenarbeit mit externen Partnern sowie neuartige Prozesse den Komplexitätsgrad eines Projekts erhöhen. Zur Abbildung der Komplexität wurden als sogenannte Proxyvariablen die Anzahl involvierter Zulieferer respektive Partner pro Projektphase (NoSupplSAS) und Verträge (NoContr) pro Projekt erhoben. Gewisse Projekte werden im Konsortium (ConPart), das heisst mit einem externen Partner abgewickelt, da dieser beispielsweise mehr oder ergänzende Expertise in Bezug auf die Anlage hat.
%%
%%part Frühwarnsystem im PM: theoretischer Rahmen für Diskussion
%Frühwarnsystem: Definition, Anwendung hauptsälich, Grund: Anwendung im PM, Anpassung Begrifflichkeiten
%Früherkennung: Definition (Watch Redundanzen), Methoden: Generationen, Fokus der Forschung 3. Generation
%Früherkennung: Einführung und Verwendung Begriff Frühwarnsignale, Methodische Ansätze summarisch,
%Früherkennung: Getestet und Bewährte Methoden
%
\section{Frühwarnsystem im Projektmanagement}
Die Notwendigkeit von Früherkennung im Projektmanagement lässt mit dem Auftauchen neuer Herausforderungen während der Projektlaufzeit sowie der sich kontinuierlich verändernden Projektumwelt begründen. Die Anforderungen an die Flexibilität im Projektmanagement und die Fähigkeit zukünftige Ereignisse "vorauszusehen" sind gestiegen. Früherkennung hat zum Ziel aufkommende Gefahren und Chancen frühzeitig zu identifizieren, so dass rechtzeitig entsprechende Massnahmen eingeleitet werden können. In gegenwärtigen Managementsystem der Bühler AG wird ein sich verschlechternder Projektstatus erst bei fortgeschrittener Projektlaufzeit ersichtlich, weshalb beabsichtigt wird mittels der Implementierung eines Frühwarnsystems, dieser Situation entgegenzuwirken. Dadurch sollen mehr Projekte erfolgreich abgeschlossen und das die Marge des Anlagegeschäfts verbessert werden. Anschliessend folgt die Erläuterung der drei Generationen von Frühwarnsystem und sowie deren die Anforderungen. Danach werden Ideen und Ansätze für Früherkennung bei der Bühler AG diskutiert.
\subsection{Frühwarnsysteme und Frühwarnindikatoren}\label{viereins}
Die Voraussetzungen zur Anwendung eines Frühwarnsystems sind gemäss Jacobs, Riegler \& Matter (2012, S. 23), die Möglichkeit ambivalenter Ausgänge, die Gefährdung dominanter Ziele sowie der prozessuale Ablauf einer drohenden "Krise". Obwohl diese Kriterien im Zusammenhang mit Unternehmenskrisen ausgearbeitet wurden, können sie für das Projektmanagement ebenso angewendet werden. Grundsätzlich werden folgende drei Generationen unterschieden.
\begin{description}
	\item[1. Generation:] Die erste Generation orientiert sich an traditionellen Kennzahlen- und Hochrechnungen. Vergleiche der Prognosen mit Sollwerten werden dann als Frühwarnsignale verwendet.
	\item[2. Generation:] Die zweite Generation basiert auf der Anwendung von Indikatoren und Prognosen auf Basis von Faktorenmodelle, die statistisch signifikante Zusammenhänge aufweisen mit dem gewählten Indikator haben.
	\item[3. Generation:] Die dritte Generation stützt sich auf die Theorie der schwachen Signale, die intuitiver und unstrukturierter Natur sind. 
\end{description}
Die erste und zweite Generation haben den Nachteil, dass die Aussagekraft der vergangenheits- und gegenwartsorientierten Datengrundlage relativ beschränkt ist und anderseits durch die Selektion relevante Faktoren unbeachtet bleiben. Aus diesen Gründen sind Diskontinuitäten aus Basis der Hochrechnungen nur schwer erkennbar und an die zugrunde liegenden mathematischen respektive statistischen Modelle gebunden   (Jacobs, Riniger \& Matter, 2012, S. 26- 28). Die dritte Generation versucht die Schwächen der vorangehenden Frühwarnsystem zu kompensieren. Gemäss Ansoff (1967, S.129ff), kündigen sich Diskontinuitäten nicht plötzlich sonder relativ früh durch sogenannte schwache Signale, die als frühe Hinweise zu bevorstehenden einflussreichen Ereignisse zu verstehen sind (Ansoff in Haji-Kazemi \& Anderson, 2013). Diesem Ansatz liegt die Prämisse zugrunde, dass Unternehmen die wahrscheinlichsten Faktoren, welche das Scheitern des Projekts begünstigen sowie die Anzeichen eines bevorstehenden Scheiterns, bereits kennen. Allerdings wird erst in nachgelagerten Projekt Assessments dieses Bewusstsein gefördert. Unter diesem Aspekt erscheint es relativ unverständlich, weshalb diese ignoriert wurden.
\newline
Die Ansätze, wie Projektmanager solche Signale erkennen und zu ihren Gunsten interpretieren können sind vielfältig, wie die Tabelle \ref{tab:Ans} aufzeigt.
%Tabelle mit möglichen Ansätzen
\begin{table}[H]
	\centering
	\caption{Ansätze zur Identifikation von Frühwarnsignalen
		\newline in Anlehnung an Haji-Kazemi, Andersen \& Krane (2013)}\label{tab:Ans}	
	\begin{tabular}{l|l}
		Risikomanagement & Past Project Consultation\\
		Earned Value Management & Cause-Effect-Analyse\\
		Projekt Assessment Ansätze & Gut feelings \\
		Performance Management & Interface Analysis\\
		Stakeholder Analyse & Project Analysis \\
		Maturity Assessment & Project Surrounding Analysis\\
	\end{tabular}		
\end{table}
Die Wahl der Methode ist abhängig von der Projektart und des Unternehmens. Haji-Kazemi, Andersen \& Krane (2013, S. 59) haben mittels mehrere Fallstudien eruiert, dass das Projekt Assessment sowie 'Gut feelings' in der Praxis die bewährt haben, wobei Experten die Überzeugung haben, dass Frühwarnsignale qualitativer Natur sind und eher durch Intuition Erfahrungswissen entdeckt werden. Klakegg, et.al, (2010) haben mittels formalen Assessments die Anzahl fehlender Informationen, fehlende Beurteilungen und Dokumentationen sowie unklare Anforderung der Meilenstein und verspätete Bericht als mögliche Frühwarnsignale ergründet. Missverständnisse bezüglich der Bedürfnisse, sowie mangelnde Offenheit der Unternehmenskultur und Kommunikationsbereitschaft zwischen den Projektteilnehmer, sowie angespannte Projektatomsphäre wurden in Fallstudien mittels der "Gut feelings"-Ansätzen als wichtige Früherkennungshinweise erkannt. Diese Erkenntnisse bestätigen unter anderem die Ansichten der befragten Experten aus anderen Studien. Denn bei der Evaluation von Assessments während der Projektlaufzeit können zwar wichtige Hinweise für nachfolgende Projekte ausgearbeitet, die aber im aktuellen Projekt nicht mehr berücksichtigt werden können. Während diese Indikatoren qualitativen Charakter haben und eher schwierig zu messen sind, konnten Haji-Kazem \& Anderson (2013) im Rahmen des Performance Management die Überwachung der Schnittstellenmassnahmen, die Mitarbeiterzufriedenheit und Risikoüberwachung als effiziente Quellen von Frühwarnsignalen erheben. Ihre Gemeinsamkeiten sind die quantitative und kontinuierliche Messbarkeit sowie die Funktion als sogenannte "leading" Indikatoren, die es ermöglichen in der Ursachen-Wirkungs-kette möglichst früh Hinweise zu möglichen Gefahren zu erhalten. "Lagging" Faktoren liefern dementsprechend eher spät oder zu spät Signale zu möglichen Risiken. Die Herausforderung bei der Identifikation von Frühwarnsignalen mit dem Performancemanagement-Ansatz ist die Selektion der zu überwachenden Faktoren. Ausserdem kann der Einfluss von Drittvariablen unentdeckt respektive unterschätzt werden.
% Faktoren von Klakegg und Kommentieren
% Faktoren von Performance Ansatz, 
% Leading Lagging Faktor,  
% wan ist früh, leading not lagging indicators, IdentifyAc says that performance only measures lagging 38
% possible early warning signs culture, lack of an outsiders perspective on the project, anchoring in the permanent organization, lack of consistency between stakeholders ambition and certain organizations. gut felt signs: detection of unrealism, lack of clarity, misalignment btw qualitative and quantitaive risk analysis 42
%no early signs in later stages, change...not used 43
%problems: difficult to stop projects despite EWS
%need for formalized proces for finding ealry warning signs, outside of the box thinking


