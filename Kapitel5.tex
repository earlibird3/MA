% !TEX root = MA.tex
\chapter{Diskussion}\label{sec:disk}
In der nachfolgenden Diskussion werden zunächst die Ergebnisse der Analyse kritisch betrachtet. Das Ziel war, die charakteristischen Unterschiede zwischen erfolgreichen und nicht erfolgreichen Projekten der Bühler AG herauszufinden. Die nachfolgenden Ausführen sind gemäss den Variablenkategorien strukturiert. Zudem werden sowohl die Datengrundlage als auch die angewendeten Methoden in Bezug auf das Ziel dieser Arbeit diskutiert. Danach folgt im zweiten Unterkapitel die Erläuterung des Ansatz der zur Früherkennung im Bühler Projektmanagementprozess angewendet werden könnte. Der Begriff Projekterfolg bezieht sich im Bezug auf die Ergebnisse immer auf das finanzielle Erfolgskriterium des Kapitel \ref{sec:dataana}.
\newline
Die Rahmenbedingungen legen den eindeutigen Rahmen eines Projekts von Beginn weg fest. Die Häufigkeitsverteilung der kategorialen Variablen in Abhängigkeit des Erfolgskriterien und der Erfolgsquote, hat ergeben das die Regionen Europa und Nordamerika im Vergleich schlechter abschneiden. Das heisst ihre Erfolgsquote liegt unter dem der aller ausgewerteten Projekte. Zugleich muss berücksichtigt werden, dass in Europa am meisten Projekte abgewickelt werden. Daraus folgt, dass das finanzielle Ergebnis der Projekte in Europa einen relativ gewichtigeren Einfluss auf die Marge des Anlagegeschäfts der Bühler AG hat. Um diese Aussage eindeutig zu bestätigen, müsste die Kostenperformance der Projekte pro Region ausgewertet werden. Die tiefe Erfolgsquote in Europa könnte mit dem hohen Volumen zusammenhängen, so dass beispielsweise die Projektmanager dieser Region mehr Projekte gleichzeitig abwickeln müssen. Eine weitere Auswertung, die die Auslastung der Projektmanager pro Region und die Anzahl Projektmanager in Relation zu Projektvolumen stellt, könnten mehr Aufschluss zur Ressourcenverteilung und Kapazitäten geben. Die Regionen Mittlere Osten und Afrika sowie China weisen eine deutlich bessere Erfolgsquote als Europa und Nordamerika aus, wickeln aber verglichen mit Europa nur halb so viele Projekte ab. Die deutlich höheren Erfolgsquoten in Südasien und Ostasien mit gleichzeitig noch tieferem Projektvolumen, scheinen die vorherige Vermutung bestätigen. in der Gesamtbetrachtung aller Regionen können die Quoten von Nordamerika und Südamerika nicht erklärt werden. 
\newline Die Auswertung der Geschäftsbereiche zeigt, dass \gls{abk:lo}, \gls{abk:gl} und \gls{abk:vn} im Vergleich deutlich tiefere Erfolgsquoten ausweisen. Die Erfolgsbilanz von \gls{abk:gm}, des grössten Geschäftsbereichs mit absolut am meisten gescheiterten Projekten, liegt nur geringfügig unter derjenigen über alle Projekte. Consumer Food und Value Nutritione haben eine ähnliches Projektvolumen, wobei erstere Erfolgsquote doppelt so hoch ist. Aufgrund der unterschiedlichen Maschinen oder Absatzmärkte ist es schwierig, neben dem Einfluss des Projektmanagements mögliche Erklärungsansätze zu finden.  %% Besser schreiben!! 
\newline
Die kombinierte Betrachtung der Regionen und des Geschäftsbereich bestätigt die Ergebnisse der Einzelauswertungen, denn in Europa und Nordamerika sind vor allem die Erfolgsquoten von \gls{abk:gl}, \gls{abk:vn} aber auch \gls{abk:gm} auffallend niedrig.
\newline Die Lead SAS des Projekts legt den organisatorischen Rahmen des Projekts fest, weshalb es naheliegend ist, dass sich deren Erfolgsquoten unterscheiden können. Vor alle die europäischen und amerikanischen Gesellschaften haben niedrige Verhältnisse zwischen erfolgreichen und gescheiterten Projekten. Dieses Ergebnis stimmt mit den vorangehenden Resultaten überein.
\newline
Der Kunde und auch der Projektmanager der Kategorie Fulfillment wurden als zentrale Einflussgrössen des Projekterfolgs identifiziert (vgl. Kapitel \ref{sec:erfprj}). Jedoch hätten die Häufigkeitsverteilungen aufgrund der hohen Einzelwert  wenig generalisierende Aufschlüsse über die Charakteristiken der erfolgreichen und nicht erfolgreichen Projekte gegeben.
%%
%%SQ
\newline\newline
Aufgrund der Datenplausibilität und der Datenverfügbarkeit konnte lediglich die Budgetabweichung zum Zeitpunkt des Projektabschluss für den Geschäftsbereich und die Region ausgewertet werden. Das arithmetische Mittel für Fail-Projekte war deutlich höher. Dieses Ergebnis kann als Hinweis für den Verkaufsdruck eines Area Manager bei Projektverkauf interpretiert werden. Anderseits kann dieses Ergebnis auch irreführend sein, da die zeitliche Komponente nicht berücksichtigt wurden. Die Zahlen zum Auftragsvolumen werden auf monatlicher Basis zur Verfügung gestellt, so dass die Area Manager dieselben Voraussetzung für Projektakquirierung haben. Demzufolge können auch trotz des Verkaufsdruck Projekte verkauft werden, die dann erfolgreich abschliessen. Eine Analyse der Projekte die im gleichen Monat und der darauffolgenden Monaten abgeschlossen wurden könnten mehr Aufschluss geben. Zudem muss angemerkt werden, dass die Berechnung der Budgetabweichung mit dem Projektstartdatum approximiert wurde. Korrekterweise hätte das Auftragsfreigabedatum als Bezugsgrösse hinzugezogen werden müssen, das aber aus systemischen Gründen nur für einen geringen Teil der Projekte möglich gewesen wäre. Da beide Daten normalerweise sehr eng aufeinander liegen, das heisst im selben Monat, und die Budgetabweichung für das Auftragsvolumen monatlich erhoben wurden, ist die Approximation relativ genau.
%%
%%Fulfillment
\newline\newline
Die Fulfillment-Variablen betreffen zum einen den Projektmanager und das Forecastmanagement sowie die verantwortliche Organisation für die Projektabwicklung. Der Wechsel des Projektmanager führte nur in 17 von insgesamt 43 Fällen dazu, dass das Projekt scheiterte. Der Austausch des Projektmanagers ist meistens eine ungeplante Konsequenz und ein Indiz für eine inadäquate Situation, beispielsweise Missverständnisse zwischen dem Kunden und dem Projektmanager oder mangelhafte Leistungen des Projektmanagers. Das Alter des Projektmanagers ist eine stellvertretende Variable für die Arbeitserfahrung. Die Auswertung der Alterskategorien zeigt, dass ältere und somit erfahrener Projektmanager tiefer Erfolgsquoten haben als ihre jüngeren Kollegen, wobei die Aspekte und Zuweisung der Projekte und die Erfahrung im Projektmanagement berücksichtigt werden müssen. Erstere unterstellt, dass ältere Projektmanager komplexere Projekte abwickeln. Zudem approximiert das Alter des Projektmanagers die gesamte Arbeitserfahrung, sodass kein Bezug zur Erfahrung im Projektmanagement, welche jedoch für den Erfolg eine zentrale Bedeutung hat, besteht. 
\newline Die Verantwortungsbereiche für die Projektabwicklung und des gesamten Projekts können von zwei unterschiedlichen Bühler Gesellschaften wahrgenommen werden. Insgesamt wurden die Verantwortungsbereiche bei 101 Projekten getrennt, wovon letztendlich 85 erfolgreich zu Ende geführt wurden. Es kann postuliert werden, dass bei getrennten Verantwortlichkeiten, die entsprechenden Aufgaben besser fokussiert werden können. 
\newline Der Effekt auf den Erfolg eines Projekts sowohl des PMChange als auch der getrennten Verantwortungsbereiche kann nicht abschliessend beurteilt werden, da zunächst der Zusammenhang zwischen dem Projekterfolg und den Variablen statistisch ergründet werden müsste. Die Indikatoren die die Zeitkomponente, des Kostenforecast erfassen konnten aufgrund unzureichender Plausibilität nicht ausgewertet werden. 
\newline\newline
%%
%%Komplexität
Die Komplexität der Projekte wurde mittels der Anzahl zuliefernden Parteien pro Projektphase und dem Konsortium ausgedrückt. Die kombinatorische Auswertung zeigte, dass entweder Eigenbeschaffung oder die Arbeit mit Drittlieferanten sowohl bei erfolgreich und nicht erfolgreichen Projekten das häufigsten Merkmal war, gefolgt von der Zulieferung durch eine oder zwei andere Bühler-Gesellschaft beim Mechnical Supply. Im Vergleich zu Kombinationen, bei denen mindestens eine weitere Bühler-Gesellschaft im Mechanical Engineering involviert war, weisen die oben beschriebenen drei Kombinationen die höhere Erfolgsquote aus. Dies könnte mit der Art der Tätigkeit zusammenhängen, da beispielsweise Maschinenbestandteile selbstständig hergestellt werden können, während der Koordinationsaufwand bei der Entwicklung der Maschinenpläne grösser ist.
\newline Von den 78 Konsortium-Projekten in konnte wurden 47 erfolgreich abgeschlossen. Die Aufschlüsselung nach der Region und des Geschäftsbereichs ergab, dass sich die Anzahl Konsortium-Projekte in Europa und China in den Geschäftsbereichen \gls{abk:gm}, \gls{abk:vn} und \gls{abk:gl} konzentriert und die Erfolgsquoten entsprechend niedrig waren. Dieses Resultat könnte einen kleinen Anteil der gescheiterten Projekte in Europa in den entsprechenden Geschäftsbereichen erklären. 
\newline Zur abschliessenden Beurteilung müsste wie bereits zuvor ein Zusammenhang zwischen dem Projekterfolg und den obigen Variable ermittelt werden.
\newline\newline
%%
%%Kosten
Die Auswertungen zu den Kosten zeigten, dass die gescheiterten gegenüber den erfolgreichen Projekten deutlich höhere Mehrkosten realisierten, was eine logische Implikation des gewählten Erfolgskriterium ist. Zu je einem Drittel sind sie auf erhebliche Kostenüberschüsse während der Beschaffung und der Installation zurückzuführen. 
%. W und der damit einhergehender Aufmerksamkeit des Projekts    Ausserdem zählt die relative Wichtigkeit eines Projekts (BAImportPr, BUImportPr und MSImportPr) für den entsprechenden Geschäftsbereich ebenso zu den Rahmenbedingungen. Die zugrundeliegende Hypothese unterstellt, dass bestimmte Kombinationen der Charakteristiken den Projekterfolg begünstigen. Kunden beispielsweise lassen sich bezügliche der individuellen Anlagespezifikationen, ihrer Bonität oder Kultur unterscheiden. Die Region in welcher die Anlage gebaut werden soll, birgt differenzierbare Risiken im Bereich der Politik, Wirtschaftsentwicklung oder länderspezifischer Handelsregelungen. Der Geschäftsbereich kann als eindeutiges Diversifikationskriterium der Anlage gewertet werden. Obwohl der Projektmanagementleitfaden intern universelle Gültigkeit hat, können während der Projektlaufzeit verschiedene Herausforderungen in Abhängigkeit der jeweiligen Anlage auftreten. Zudem kann davon ausgegangen werden, dass die Teamarbeit und Teamkultur pro Geschäftsbereich und -einheit verschieden sind und den Projekterfolg unterschiedlich beeinflussen. Die Wichtigkeit eines Projekts, das Umsatzbudget des Projekts im Verhältnis zum Median des Umsatzbudgets aller laufenden Projekte, kann als Indikator zur Konzentration von Ressourcen bei der Projektabwicklung interpretiert werden. Demzufolge müssten bedeutendere Projekte, die auch einen erheblichen Einfluss auf das Geschäftsbereichsergebnis haben, mehr Aufmerksamkeit in Bezug auf Risikominimierung erhalten.
%%SQ Zum Beispiel beeinflussen die Qualität der Offerte sowie die vertraglichen Vereinbarungen die Rahmenbedingungen für die Projektabwicklung. Die Offertstellung und vorgängige Risikoanalysen des Projekts liegen im Aufgaben- und Verantwortungsbereich des Area Managers (AM und AMNo). Es wird davon ausgegangen, dass erfahrenere (AMAge) und langjährige (AMTen) Verkaufsmanager über mehr Kenntnisse zu den Projekten allgemein, als auch deren Risiken und dem internen Prozess verfügen und deshalb 'erfolgreichere' Projekte verkaufen. Die Incentivierung und Performancemessung der Verkaufsmanager erfolgt über das Auftragsvolumen des Geschäftsbereichs und der Region.
%%FF
%Die In dieser Kategorie werden sämtliche Faktoren im Zusammenhang mit dem Projektmanager (PM und PMNo), dem Forecastmanagement (FC-Management) und der Unternehmensverantwortung subsumiert, da sie den Projektabwicklungsprozess tangieren. Der Betriebszugehörigkeit der Projektmanager (PMTen) sowie dessen Erfahrungsschatz (PMAge) sind stellvertretende Variablen für die Kenntnisse der internen Prozesse und das vorhandene Wissen in Bezug auf den Beruf. Bei Unstimmigkeiten zwischen dem Kunden und dem Projektmanager, kann letzterer ersetzt werden (PMChange). Je nach Status des Projekts und Zeitpunkt des Wechsels können nicht alle Differenzen durch den neuen Projektmanager kompensiert werden, weshalb ein Austausch als Indiz für nicht-erfolgreiche Projekte betrachtet wird. In sehr seltenen Fällen muss die Funktion des Projektmanagers sogar mehrmals neu besetzt werden (NoPM), was den positiven Ausgang eines Projekts beeinträchtigen kann.

%%Cost
%%Budget (Bud) und realisierte Zahlen (Act), deren Vergleich und die monetären Abweichungen zwischen dem letzten Forecast und den Istzahlen. Ein höheres Umsatzbudget (TOBud) wird mit komplexeren und umfangreicheren Projekten, die ein höheres Mass an Planung, Ressourcen sowie Betreuung erfordern, assoziiert. Ausserdem ist ihr finanzieller Einfluss auf das Geschäftsbereich- bzw. Regionenergebnis von besonderer Wichtigkeit. Die Kostenabweichungen (CostActBud) pro Projektphase in absoluten und relativen Grössen sollen Aufschluss über die Verlustbereiche geben. Die Zusammensetzung der budgetierten Kosten pro Projektphase in Relation zum Umsatz (BudMS/ME/PA/ISTot) kann zudem Aufschluss über die Projektart geben, da beispielsweise ein hoher Engineering-Anteil erwartungsgemäss eher mit Mehrkosten einhergeht als ein hoher MS-Anteil. Es wurden zusätzlich die Kosten aus Nachlieferungen infolge Nichteinhaltung des vereinbarten Liefertermins in das Modell mit einbezogen, da hypothetisch vermutet wird, dass dieser Kostenanteil in Bezug zum Umsatzbudget bei nicht-erfolgreichen Projekten höher sein muss. Im Zusammenhang mit der Projektmarge liegt der Fokus vor allem auf Projekten mit einem Budget nahe des intern festgelegten Grenzwertes von 23\%. Denn sämtliche Projekte, deren budgetierte Projektmarge unter diesem Wert liegt, bedarf einer Zustimmung zur Eingehung dieses Risikos sprich der Projektdurchführung durch die nächst höhere Managementstufe. Davon ausgehend, dass aufgrund des Budgetdrucks versucht wird diesen Genehmigungsprozess zu umgehen, wird vermutete, dass risikoreichere Projekte verkauft werden, die letztendlich eher mit negativer Performance einhergehen.
%%Liefertermins sowie die Lokalisierung von Zeitverzögerungen anhand ausgewählter Milestones sind hierbei von grossem Interesse. Dazu wurde die Zeitdifferenz (PrTimeDelay) zwischen der vereinbarten (PrTimeBaseline) und der erreichten Projektlaufzeit(PrTimeAct) für das ganze Projekte und die folgenden Milestones gemessen: Die organisatorische Verantwortung für das ganze Projekt (LeadSASPr) und den Abwicklungsprozess (LeadSASFF), kann bei einer Gesellschaft oder zwei verschiedenen Gesellschaften (LeadSAS.PrFF) angesiedelt sein. Die zusätzliche Schnittstelle erhöht den Komplexitätsgrad eines Projekts und kann deshalb nachteilig auf den Projekterfolg wirken. Die Zusammenarbeit sowohl zwischen den Gesellschaften als auch innerhalb der Unternehmen kann sich voneinander unterscheiden, weshalb einige Gesellschaften wahrscheinlich mehr Erfolg im Projektmanagement aufweisen. Das  Forecastmanagement liegt im Verantwortungsbereich des Projektmanagers und bezieht sich auf die Prognose des Projektumsatzes, der -kosten sowie der -marge, welche monatlich geprüft und entsprechend angepasst werden muss. Das frühzeitige Erkennen von drohenden Mehrkosten kann deren Verminderung oder Vermeidung begünstigen. Deshalb wurde pro Projektphase, Mechnical Supply \gls{abk:MS}, Mechnical Engineering \gls{abk:ME}, Plant \&Automation \gls{abk:PA} und Installation \gls{abk:IS} erhoben, ob der Forecast angepasst wurde. Dabei wurde zwischen 'nur Mehrkosten' und 'Mehrkosten inklusive Umsatzerhöhung' unterschieden, (CostFCadj). Zudem wurde die Anzahl Monate zwischen dem Projektabschluss (MS11) und der negativsten FC-Anpassung (CostmostnegFCadj) gemessen. Die Abweichungen von den Vorgaben in Bezug auf Zeit und Kosten wird systemisch automatisch berechnet und  durch das dreistufige Ampelsystem des Bühler Projektmanagement-Cockpit (BPM-Cockpit) reflektiert. Obwohl intern vorgeschrieben wird, dass jede mögliche Veränderung in der monatlichen Prognose unverzüglich einfliessen muss, wird aufgrund des Begründungszwangs bei hohen Abweichungen versucht, die Kommunikation der negativen Veränderung so lange wie möglich hinauszuzögern.  % Erklärung wieso HOM Deshalb wurde die Periode zwischen dem erstmaligen gelben respektive roten Status und dem Projektbeginn HOM gemessen und ins Verhältnis zur vereinbarten Projektzeit (HOMRed/YellowCost/Time/Quality) gesetzt.  %Auf diese Weise kann herausgefunden werden..

%%Ziel: eigene Ergebnisse interpretieren und praktische Relevanz erläutern
%%Einleitung: Erläutern kurz was gemacht wurde, Zusammenfassung der wichtigsten Ergebnissen, 
%%Erfolgsfaktoren: pro Kategorie, Hypothesen beantworten
%%Erfolgsfaktoren: in Bezug zu Hypothese
%%Erfolgsfaktoren: in Bezug zu Theorie
%%Erfolgsfaktoren: Scheiter und Scheitern lassen, Erfolgsdefiniion = Schwarz/Weiss Denken
%%Erfolgsfaktoren: Theorie sagt was anderes, Vorschläge zur Ergründung, aber Einflussdeterminaten könnten sein.....
%%Schlussfolgerung: Haupterkenntnisse
%%Schlussfolgerung: Relevanz für weitere Forschung und praktische Anwendung, was muss zukünftig geleistet werden, in Planform!
%%Fazit:
\section{Erfolgsfaktoren}\label{sec:diskerf}
%%Ausgangslage: Faktoren können nicht zur Früherkennung dienen
%%Konezptionelle Ansätze: Kritische Stelle im Prozess beleuchten, Hypothetische Vermutungen formulieren, wo Problem liegen könnte
%%Implementierung, Incentivierung
%%Andere Ansätze: Fokus anders legen, 
Nachfolgend werden die Ergebnisse aus Kapitel \ref{sec:ergebnisse} kritisch beurteilt, um einerseits Hinweise für mögliche Erfolgsfaktoren und anderseits Ansätze für weiterführende Analysen zu ergründen.
%% alle vars...vergleich mit theori^^
%% erfolgskriterium
%% pro Kategorie
\newline\newline
Die Analyse von Umsatz, Kosten und DB1 hat ergeben, dass Fail-Projekte in den letzten drei Jahren einen Margenverlust von 48.2 MCHF der insgesamt auf Mehrkosten von 61.5 MCHF im Vergleich zum Budget zurückzuführen ist. Die Kostenabweichungen der Projektphasen Installation und Beschaffung sind dabei am höchsten, womit ein Hinweis vorliegt, welche Prozesse zur Ergründung der Ursachen der Mehrkosten untersucht werden müssen. Die Auswertung der durchschnittlichen Kostenabweichungen zeigt, dass relativ die ME-Kosten am meiste von ihrem Budget abweichen. Daraus folgt, dass pro geplanter Franken an ME-Kosten doppelt soviel Kosten benötigt werden. Das kann entweder daran liegen, dass im Budgetprozess die Kosten zu tief angesetzt wurden, oder aufgrund von Komplikationen erhebliche Mehrkosten entstanden sind. Diese Frage und ob beispielsweise vorgelagerte Fehler in der Prozesskette zu garantierten überhöhten Kosten führen, müssten geklärt werden, um Optimierungspotenziale ausschöpfen zu können.
\newline Der Margenverlust stammt zu einem grossen Teil von verhältnismässig wenig Projekten mit hohem Umsatzvolumen. Da grössere Projekte mit mehr Risiken einhergehen und das Gesamtergebnis eines Geschäftsbereichs erheblich beeinflussen kann, werden solche Projekte oftmals prioritär behandelt. Allerdings darf der kumulierte Verlust gegenüber dem Budget in Bezug auf den Deckungsbeitrag von Projekte mit geringeren Umsatzvolumen nicht vernachlässigt werden. Denn hinsichtlich der Anzahl sind Projekte bis und mit 1 Mio. CHF in der Überzahl. Somit könnte eine Änderung der Prioritätsregel die Verlustverteilung verschieben, wobei allerdings der trade off durchdacht werden sollte. Zur Begründung könnte statistisch evaluiert werden, welche Projektgrösse mehr Einfluss auf die Erfolgswahrscheinlichkeit hat. Obwohl aus finanzieller Sicht eindeutig ist, dass sowohl viele kleine als auch wenig grosse Projekte die negative DB1-Performance bestimmen, kann auf Basis der Ergebnisse kein eindeutiger Rückschluss, ob das Umsatzvolumen eines Projekts als Erfolgsfaktor in Frage käme. Da grosse Projekte tendenziell eher einem Projektassessment nach Ende der Projektlaufzeit unterzogen werden, könnte es lohnend sein die Ursachen des Scheiterns der kleineren Projekte zu eruieren. Daraus könnten sich dann die Erfolgsfaktoren herauskristallisieren, die nicht zwingend mit denjenigen der grossen Projekte übereinstimmen müssen. Zusammenfassend ist darauf hinzuweisen, dass die Kostenabweichungen trotz ihres direktem Zusammenhang mit dem der Auswertung zugrundeliegenden Erfolgskriterium eher der Beurteilung der Projekte und weniger der Einflussnahme auf die Erfolgswahrscheinlichkeit eines Projekts dient. Demzufolge sind sie gemäss der Unterscheidung von (Besteiero, Pinto und Novaski, 2015) als Kriterium und weniger als Erfolgsfaktor zu betrachten (vgl. Kapitel \ref{sec:proj}). 
\newline\newline
Die Analyse der Rahmenbedingungen hat aufgezeigt in welchen Regionen und Geschäftsbereichen, das Volumen abgewickelt wird und wie sich die erfolgreichen zu den nicht-erfolgreichen Projekte verhalten. Typischerweise ist ein nicht-erfolgreiches Projekt in der Region EU, NAM oder SAM in den Geschäftsbereichen GL, VN und GM zu finden. Da die Anzahl der Projekte in EU und GM relativ am höchsten ist, könnte die zusätzliche Auswertung der Ressourcenverteilung aufzeigen, in welchem Masse die Projektmanager ausgelastet sind. Demzufolge könnten beispielsweise die Projektanzahl pro Manager die Priorisierung bestimmter Projekte den Erfolg anderer beeinträchtigen, so dass ein gewisser Kanibalisierungseffekt auftritt. Die Eignung als Erfolgsfaktor, im Sinne einer Erhöhung der Erfolgswahrscheinlichkeit, ist jedoch aufgrund der Konsequenzen entsprechender Massnahmen fragwürdig. Denn die Attribute Region und Geschäftsbereich können über die Steuerung des Verkaufs beeinflusst werden, was zur Folge hätte, dass mehr Projekte in anderen Geschäftsbereichen respektive Regionen verkauft würden. Der Effekt hinsichtlich des Projekterfolgs bleibt dabei ungeklärt. Die Rahmenbedingungen können jedoch ein Argument dafür liefern, welche Projekte beobachtet werden sollen, sodass auftauchende Probleme früher erkannt und entsprechenden Handlungen vorgenommen werden können.
\newline\newline
Da die Abweichungen der Kosten bereits diskutiert wurden, werden nachfolgend die anderen Variablen der Kategorie der Kosten evaluiert. Die durchschnittlichen Kosten aus Nachlieferung für Fail-Projekte waren höher, was mit Mängel im Engineering, zeitverzögerter Beschaffung oder fehlerhaften Konstruktionen zusammenhängen kann. Die Kosten Nachlieferung verursachen Mehrkosten und beurteilen das Projekt in Bezug auf die Einhaltung der Kostenbudgetvorgaben, weshalb sie eher als Erfolgskriterium zu quantifizieren sind (Besteiero, Pinto und Novaski, 2015). Die Zusammensetzung des Budgets in dieser Form hat relativ wenig Aufschluss über den möglichen Einfluss auf den Projekterfolg gegeben. Es könnte beispielsweise eine kategoriale Variable erhoben werden, die in Abhängigkeit eines Schwellenwertes die Projekte in MS-, ME-, PA- und IS-Projekte unterteilt. Dadurch könnte lokalisiert werden, ob sich Fail-Projekte in einem gewissen Projekttyp konzentrieren. Unter der Berücksichtigung der Kundenwünsche ist die Einflussnahmen auf die Zusammensetzung des Budgets eher beschränkt. Obwohl diese Faktoren den Projekterfolg begünstigen können, liegt aufgrund des begrenzten Handlungsspielraum weniger ein Erfolgsfaktor vor. Wie bereits zuvor, kann sich diese Determinante zur Eingrenzung eines Monitoringbereichs eignen. Die Durchschnittswerte der Differenz zwischen dem letzten Forecast und den realisierten Kosten war bei Fail-Projekten für alle Projektphasen höher, insbesondere bei der Installationsphase. Dies kann als Hinweis für erhebliche Mehrkosten kurz vor dem Projektende oder eine zeitverzögerte Kommunikation einer sich verschlechternder Kostenprognose interpretiert werden. 
\newline\newline
Die evaluierten Einflussdeterminanten des Fulfillment-Prozess sind der Projektmanager, das Unternehmen und das Forecastmanagement. Die Verteilung der Anzahl nicht-erfolgreichen Projekte pro Alterskategorie ist nahezu uniform. Ältere im Vergleich zu jüngeren Projektmanager weisen eine tiefere Erfolgsquote aus. Das könnte damit zusammen hängen, dass erfahrenere Projektmanager tendenziell die risikoreicheren Projekte abwickeln. Diese Vermutung müsste zuerst mit Daten belegt werden, um eine abschliessende Beurteilung vornehmen zu können. Der Wechsel des Projektmanagers erfolgte in lediglich 43 Fällen, wovon 17 letztendlich scheiterten. Der überwiegende Teil nicht-erfolgreicher Projekte hatte während der gesamten Laufzeit genau einen Projektmanager. Aufgrund des fehlenden Signifikanztest des Zusammenhangs zwischen Erfolg und Wechsel des Projektmanager kann nicht abschliessend beurteilt werden, ob dieser Indikator als Erfolgsfaktor zu qualifizieren ist. Nichtsdestotrotz kann eingewendet werden, dass der Wechsel des Projektmanagers eher eine ungeplante Konsequenz aus vorgelagertem Handeln ist. Die Definition eines Erfolgsfaktor impliziert jedoch, dass während der Projektlaufzeit proaktiv die Aspekte des Erfolges mittels Entscheidungen für oder gegen eine Handlungsoption gesteuert werden können.
\newline Die Unternehmenskultur der für das Projekt verantwortliche Organisation kann Einfluss auf die Projektumgebung nehmen. Obwohl alle Produktionsstätten und Verkaufsorganisationen demselben Konzern angehören können personelle Unterschiede das Projektmanagement massgeblich beeinflussen. Die Auswertung der Lead SAS des Projekts zeigt, dass BUZ Verantwortungsträgerin für die meisten Projekte ist. Der Erfolgsquote liefert einen Hinweis, dass beispielsweise bei BBS im Vergleich zu BBAN mit ähnlichem Projektvolumen deutlich weniger gut abschneidet und BUZ trotz der häufigsten Fail-Projekte eine relativ gute Erfolgsbilanz ausweist. Die Unterschiede zwischen BBAN und BBS können beispielsweise Projektmanagementprozess induziert, durch Schwierigkeitsgrad der Projekte bedingt sein oder mit dem Personalressourcenmanagement in Verbindung gebracht werden. Ein interessantes Ergebnis liefert die Aufteilung der Gesamtprojektverantwortung und der Projektabwicklungsverantwortung. Obwohl insgesamt nur 101 Projekte mit geteilten Verantwortungsbereiche aufweisen, konnten 88 (84\%) erfolgreich beendet werden. Der Einfluss auf den Projekterfolg der unternehmensspezifischen Charakteristiken scheint intuitiv offensichtlich, wobei der Handlungsspielraum wie bereits bei den Regionen und den Geschäftsbereichen relativ begrenzt ist. Eine weiterführende Analyse beispielsweise der BBS-Projekte oder des Verhältnis bei geteilter Verantwortung könnte mehr Aufschluss über die Erfolgsattribute liefern. 
\newline\newline
Die Evaluation des Zeitmanagement ergab, dass die Mehrheit aller Projekte den Liefertermin nicht einhalten konnte. Das Verhältnis der Anzahl verspäteter Projekte bei MS8 im Vergleich zu MS2 ist genau spiegelverkehrt. Während bei MS2 die durchschnittliche Zeitverzögerung weniger Projekte im Tagesbereich lag, betrug sie für die Mehrheit der Projekte ab MS8 bis zu zwei respektive zwischen fünf und sieben Monate bei Projektende. Dabei lag der gemessene Rückstand bei Fail-Projekten nach dem Meilenstein 11 gegenüber 10 im Mittel um zwei Monate höher. Die Gründe für die Verspätung sind nicht bekannt, könnten aber entscheidende Hinweise für Optimierungspotenziale des Zeitmanagement liefern. Aufgrund der Wechselbeziehung zwischen Zeit und Kosten kann der Effekt auf den Projekterfolg nicht eindeutig bestimmt werden. Zudem wäre gemäss der Definition von Besteiero, Pinto \& Novaski (2015) die Zeitperformance eher als Erfolgskriterium einzuordnen. 
\newline\newline
Die Auswertung der Budgetperformance des Auftragsvolumen zum Zeitpunkt des Projektabschlusses hat ergeben, dass sowohl auf Regions- und Geschäftsbereichsebene das Mittel der nicht-erfolgreichen Projekte relativ höher war. Das Ergebnis sollte jedoch mit Vorsicht beurteilt werden, da erfahrungsgemäss die Zahlen des Auftragsvolumen der ersten gegenüber der zweiten Hälfte des Jahres eher unter den Budgetvorgaben liegen, sodass die Mittelwerte verzerrt sein können. Eine weitere Analyse von zwei Vergleichsgruppen, "Projekte Anfang des Jahres" und "Projekte Ende des Jahres" könnte mehr Erkenntnisse liefern. Dennoch kann der Unterschied zwischen nicht-erfolgreichen und erfolgreichen Projekten hinsichtlich der Erreichung der Budgetvorgaben als Indiz für die Priorisierung des Verkaufsabschluss gegenüber dem Risikopotenzial interpretiert werden. Dabei wird unterstellt, dass im SQ-Prozess beispielsweise Risiken und Kosten unterbewertet werden, sodass verhältnismässig günstiger verkauft werden kann. Zur abschliessenden Beurteilung der Auswirkungen des Budgetdrucks müssten die Offerten und die daraus resultierenden Effekte für den Abwicklungsprozess genauer untersucht werden. Die Eignung als Erfolgsfaktor kann aufgrund des fehlenden Signfikanztests nicht abgeschätzt werden.
\newline\newline 
Die Analyse Komplexitätsfaktoren hat gezeigt, dass die Anzahl involvierter Parteien relativ gering war, der Projektabwicklung mehrheitlich ein Aufträge zu Grunde lag und nur wenige Projekte mit Konsortium abgewickelt wurden. Da die Aufteilung eines Projekts auf mehrere Aufträge an kein Kriterium gebunden, ist die Aussagekraft der Anzahl Aufträge in Bezug auf die Komplexität relativ gering. Projekte mit Konsortium gibt es insgesamt 78, wovon 31 gescheitert sind. Davon wurden 14 in Europa, 7 in MEA\_Afr und 7 in China mehrheitlich von den Geschäftsbereichen GL, GM und VN abgewickelt. Das Merkmal Konsortium erklärt somit einen Teil der nicht-erfolgreichen Projekte der Regionen und Geschäftsbereich, die eine relative niedrige Erfolgsquote aufweisen. 
Die Auswertung der Supplying SAS über alle Projektphasen zeigte dass, Eigenproduktion und die Zusammenarbeit mit Drittlieferanten das häufigste Charakteristika von allen Projekten war. Gewisse Kombinationen hatten eine absolute Häufigkeit von nur einem Projekt. Die Aussagekraft der Unterschiede zwischen erfolgreichen und nicht-erfolgreichen Projekten ist hinsichtlich der Komplexität sehr beschränkt.
\newline\newline
Zusammenfassend kann ausgesagt werde, dass nicht-erfolgreiche Projekte folgenden Attribute aufweisen können:
% Table generated by Excel2LaTeX from sheet 'Sheet1'
\begin{table}[H]
	\centering
	\caption{Mögliche}
	\begin{tabular}{rlrlr}
		\multicolumn{1}{l}{\textbf{Region}} & \textbf{BA} & \multicolumn{1}{l}{\textbf{PMChange}} & \textbf{LeadSASPr} & \multicolumn{1}{l}{\textbf{Konsortium}} \\
		\multicolumn{1}{l}{EU} & GM    & \multicolumn{1}{l}{Yes} & BJHB  & \multicolumn{1}{l}{Yes} \\
		\multicolumn{1}{l}{NM} & GL    &       & BBS   &  \\
		& VN    &       & BMIL  &  \\
	\end{tabular}%
	
	\label{tab:addlabel}%
\end{table}%
Einschränkend muss ausgesagt werden, dass der Einfluss auf den Erfolg eines Projekts mittels einem statistischen Modell geschätzt werden muss.  
\newpage

\section{Frühwarnsystem}\label{sec:diskfru}
Die Erkenntnisse aus der Theorie des vorangehenden Kapitel setzen den Rahmen für die nachfolgende Ausführungen. Die zentralen Anforderungen an Frühwarnsignale sind zum einen, dass sie bereits zu einem frühen Zeitpunkt gemessen werden können. Die Implementierung eines ganzen Frühwarnsystems erfordert allerdings eine strategische Verankerung, da ein konstantes Monitoring und Screening, sowie eine anschliessende Auswertung und Interpretation der Daten notwendig ist. Ohne die Unterstützung des Managements wird die Durchsetzung eines solchen Vorhabens kaum durchsetzbar sein.
%DAten
Die Auswertungen der Einflussfaktoren des Kapitel \ref{drei} geben zwar Hinweise, was mögliche Attribute nicht-erfolgreicher Projekte sein können, allerdings fehlt es an einer statistisch begründeten Signifikanz des Zusammenhangs mit dem Erfolgskriterium. Nichtsdestotrotz können einige Faktoren bereits aufgrund ihrer Natur und des möglichen Erhebungszeitpunktes als mögliche Frühwarnindikatoren ausgeschlossen werden. Denn vorzugsweise sollen sogenannte "leading factors" fokussiert werden, zu denen sämtliche Performanceindikatoren, beispielsweise Kostenabweichungen, Kosten aus Nachlieferung oder Zeitverzögerungen gemäss (Zitat) nicht zählen. Die Rahmenbedingungen sowie auch der Projektmanager oder die Verantwortungsgesellschaften sind zwar zu Beginn des Projektes bekannt, verändern sich jedoch über die Projektlaufzeit kaum. Allerdings könnte nach einer entsprechenden Analyse ihres Einflusses auf die Erfolgswahrscheinlichkeit Projekte mit entsprechenden Attributen eher überwacht werden als andere. Ein solches Vorgehen kann dazuführen, dass andere Signale ausser Acht gelassen werden. Zudem wurden wie bereits gesagt, gewisse Faktoren in der Analyse nicht berücksichtigt, die möglicherweise auch als Frühwarnindikatoren funktionieren könnten. Aus diesen Gründen erscheint es erforderlich, dass neue Faktoren bezüglich ihrer Fähigkeit als Frühwarnsignal zu fungieren, ergründet werden. Es gibt jedoch keinen formalisierten Prozess, zu deren Identifikation. 
\newline Obwohl Projekt Assessments und seine Formen sowie Risikoanalysen bereits viele Hinweise zu möglichen Risiken und Chancen liefern, 
Bühler hat selbst für sogenannte Crash-Projekte Projekt-Assessments durchgeführt, die zu wichtigen Erkenntnissen für zukünftige Projekte geführt hat. Ausserdem wird am Ende jedes Projekts (Milestoen Debriefen) ein sogenanntes Debriefing abgehalten, welches implizierte, das Stärken und Schwächen eines jeden Projekts diskutiert wurden. Der Customer-Project Prozess hat neben anderen eine zentrale Schnittstelle vom Verkaufsprozess zum Fullfillment-Prozess. Basierend auf den Erkenntnissen der Literatur hat sich die Interface Analyse respektive die Beobachtung der
\paragraph{Sales \& Quotation} Schnittstellenthemen als relativ guter Frühwarnindikator erwiesen. Es liegt auf der Hand, dass der Output aus dem SQ-Prozess direkter Input im FF-Prozess bildet. Deshalb entstand die Idee, sozusagen die Frühwarnung für den Projekterfolg ab diesem Zeitpunk zu implementieren. Aus den Prozessabläufen der Bühler AG geht hervor, dass sowohl der SQ-Prozess und der FF-Prozess für Projekte grösser als eine Million umfangreiche Risikoanalysen gemacht werden. Vermutungsweise wird bereits zu diesem Zeitpunkt mögliche Erkenntnisse über zukünftige Herausforderung gewonnen, die wenig Beachtung erhalten. Fehlende Informationen, Assessments oder Dokumentation können als mögliche Frühwarnsignale interpretiert werden (siehe Klaggeg). Deshalb kann es von Nutzen sein eine Kennzahl zu entwickeln, welche auf automatisierte Basis die erforderlichen Dokumente gemäss den Anforderungen des MS1 beobachtet werden, sodass sichergestellt werden kann, dass keine Informationslücken entstehen. Somit wäre bereits früh klar, bei welchen Projekten alle relevanten Informationen zu Verfügung standen und der Übergabeprozess geglückt war. Auf Basis der in Kapitel \ref{sec:drei} erhobenen Daten hat sich gezeigt, dass Volumenmässig der Anteil an Projekte unter oder gleich einer Million fast die Hälfe aller Projekte ausmacht. Es wäre denkbar, dass bei diesen Projekten umfangreiche Risikoprüfungen ausgeblieben sind, da sie nicht priorisiert werden. Diese Annahmen und auch Probleme dieser Projekte müssten genauer untersucht werden, um andere Frühwarnindikatoren zu entdecken. Nichtsdestotrotz ist die Schnittstelle von SQ zu FF auch bei kleinere Projekte wichtig, damit das Projekt erfolgreich abgewickelt werden kann. In diesem Fall würde könnte es sich anbieten, eine Art Interface Analysis, die die Anzahl Interface-Themen und deren Bearbeitung in Bezug auf nur diese Schnittstelle misst, so dass es einerseits eine Plattform gibt, die Interface-Themen erfasst und offene/ungelöste Themen ersichtlich sind. Diese Idee liefert allerdings nur dem Projektabwicklungsprozess nähere Informationen, ob ein Projekte auf die schiefe Bahn gerät. Deshalb müssen auch für den weiteren Projekt-Management Prozess mögliche Ansätze diskutiert werden.
\paragraph{Projektabwicklung:} Im Propjektabwicklungsprozess sind weitere Schnittstellen vorhanden, welche genauer berücksichtigt werden müssen. Gemäss der Datenanalyse ist sowohl für Fail-Projekte und Success-Projekte der Mehrkostenanteil der IS-Kosten am höchsten gewesen. Unabhängig von der Ursache dieses Erscheinungsbild, ist die Installation die letzte Projektphase, so dass es von grossem Nutzen frühzeitig über mögliche Komplikationen Bescheid zu wissen, damit vorbereitende Massnahmen getroffen werden können. Zur Steuerung mittels Frühwarnsystem könnte ein Kombination aus Interfacemanagement und 'Gut Feelings' angewendet werden. Mittels Interfacemanagement soll sichergestellt werden, dass die auftauchenden Themen bearbeitet in nützlicher Frist bearbeitet werden. Der Ansatz der Gut Feelings hat zum Zweck, dass eine breiter Fokus für Variablen existiert, die einerseits während der Projektphase als mögliche Bedrohungen identifiziert werden und anderseits weder in den Risikochecks des SQ noch des FF inkludiert waren. Interne Dokumente belegen, dass der Projektmanager Erkenntnise aus der Risikoanalyse im Projektmanagement in detaillierter Form pflegen muss. Da erfahrungsgemäss ein gewisser Widerwille gegenüber umfangreichen Datenpflege festzustellen ist, sollte es im Tool eine Inputstelle geben für auf Intuition basierende Frühwarnsignale geben. Diese Anlaufstelle soll möglichst wertneutral, frei von Rechtfertigungsanforderung von übergeordneten Parteien, mit effizienter Handhabung und Zugang für sämtliche Projektteilnehmer ausgestattet sein. Dies ermöglicht dem Projektmanager eine Art Radar für zukünftige Herausforderung zu haben. Es sollte möglich sein, ein konstantes Monitoring pro Projekt ohne dabei detaillierte Angaben bereits erfassen zu müssen, sicherzustellen. Aspekte die berücksichtigt werden müssten sind die Strukturierung der Datenmenge sowie die Nutzung und Auswertung der Daten durch die Projektmanager.
\paragraph{Verschuldungsfrage:} Die vorangehende Analyse der Projekte wendet sozusagen ein schwarz-weiss Denken in Bezug auf den Erfolg ab. Allerdings konnte während der Analysephase festgestellt werden, dass es schwierig ist zu unterscheiden, welcher Projekttyp vorliegt. Beispielsweise werden Projekte gemacht, um neue Kunden zu gewinnen oder eine neue Technologie zu fördern, was zur Folge haben kann, dass die Kostenvorgaben relative zum Umsatz ambitiös ausfallen. Es würde jedoch Sinn machen, solche Projekte vom Standardgeschäft abgrenzen zu können, um mögliche Kernkompetenzen respektive Faktoren, die den Projekterfolg beeinflussen zu identifizieren. Zudem ist es wahrscheinlich, dass der Grund für die Mehrkosten nicht immer beim der Bühler AG liegen muss, unter dem Ausschluss, dass die Auswahl der Kunden von ihr beeinflusst wird. In Zusammenhang mit der Identifikation der Erfolgsfaktoren der Bühler AG könnte es folglich von Nutzen sein, Faktoren zu herben, die Aufschluss über kundenseitig induzierte Ursachen geben und wie anschliessend die Mehrkosten gehandhabt wurden. Zudem könnte die Befragung der Projektmanager und Verkaufsmanager mehr Aufschluss über zu berücksichtigenden Erfolgsfaktoren des Projektmanagements der Bühler AG geben. Hinzu kommt, dass aufgrund der Datenqualität wichtige Faktoren, wie zum Beispiel der Zeitpunkt der Forecast-Anpassung nicht ausgewertet werden konnten.
\paragraph{Projektprioritäten:} Aus einer finanziellen Perspektive die Fokussierung der Projekte mit grossem Umsatzvolumen, da ihr Einfluss auf das Ergebnis im Anlagengeschäft relativ gewichtig ist. Dennoch sollte der DB1 Verlust von Projekten mit einem Umsatzvolumen bis maximal 1.0 Mio. CHF nicht vernachläassigt werden, da sie am Projektvolumen eine relativ hohen Anteil haben. Wie bereits erwähnt wurde, werden Projekte unter 1 Mio. CHF von die internen Richtlinien zur vertieften Risikoanalyse der Bühler AG nicht erfasst.
%%% Text fehlt 
\paragraph{Incentivierung:} Die Anwendung von Frühwarnindikatoren bedingt, dass die Unternehmenskultur sowie auch die Projektmanagementstrategie entsprechend verändert respektive ausgerichtet wird. Die Implementierung von Frühwarnsignalen kann keine einmalige Übung darstellen, da es ein laufender Prozess ähnlich dem monatlichen Reporting ist. Die Abstimmung und Ausrichtung der Prozess und involvierten Personen auf ein gemeinsames Ziel "erfolgreiche Projekte" abzuschliessen ist dabei von grosser Wichtigkeit. Der Fokus sollte auf die Ergreifung von Massnahmen zur entsprechenden Gegensteuerung bei Komplikationen gerichtet sein anstatt auf die interne Schuldfrage. Eine sogenannte Fehlerkultur, die den Umgang mit Fehlern, Fehlerfolgen und Fehlerrisiken inkludiert, kann ein konstruktives Lernen aus Fehlern oder die Entdeckung effektiver Massnahmen bei Fehlerrisiken begünstigen. Alam Gühl (2016, S.20-21) plädiert im Rahmen der Projektkultur den positiven Umgang mit Fehlern und  den umfangreichen Austausch von entsprechenden Informationen, das von der Unternehmenskultur begünstigen werden kann. Ein anderes zentraler Einflussfaktor im Zusammenhang mit Frühwarnsignalen, ist auch die Fähigkeit des Projektmanagers zu Eingeständnissen, dass sein Projekt zu scheitern droht. Denn werden drohende Risiken und deren mögliche Realisation verkannt oder verhältnismässig spät kommuniziert, können der Handlungsspielraum eingegrenzt werden. Allerdings muss diese Denkweise aktiv im Unternehmen gefördert werden, damit das Bewusstsein der Fehlerakzeptanz, d.h. die positive Assoziation zu Fehlern und Scheitern gefestigt wird. 


%%Personen abhängig alli müsssen an einem Strang ziehen, Ausrichtung der Menschen an Zielen es Unternehmesn
%% Probleme Installation
%% Projektkategorisierung
%% Verschulden Bühler etc.
 