% !TEX root = MA.tex
\chapter{Diskussion}\label{sec:disk}
In der nachfolgenden Diskussion werden zunächst die Ergebnisse der Analyse kritisch betrachtet. Das Ziel war, die charakteristischen Unterschiede zwischen erfolgreichen und nicht erfolgreichen Projekten der Bühler AG herauszufinden. Die nachfolgenden Ausführen sind gemäss den Variablenkategorien strukturiert. Zudem werden sowohl die Datengrundlage als auch die angewendeten Methoden in Bezug auf das Ziel dieser Arbeit diskutiert. Danach folgt im zweiten Unterkapitel die Erläuterung des Ansatz der zur Früherkennung im Bühler Projektmanagementprozess angewendet werden könnte. Der Begriff Projekterfolg bezieht sich im Bezug auf die Ergebnisse immer auf das finanzielle Erfolgskriterium des Kapitel \ref{sec:dataana}.
\newline
%%
%%Diskussion Erfolgsfaktoren
\section{Erfolgsfaktoren}\label{sec:diskerf}
Die Rahmenbedingungen legen den eindeutigen Rahmen eines Projekts von Beginn weg fest. Die Häufigkeitsverteilung der kategorialen Variablen in Abhängigkeit des Erfolgskriterien und der Erfolgsquote, hat ergeben das die Regionen Europa und Nordamerika im Vergleich schlechter abschneiden. Das heisst ihre Erfolgsquote liegt unter dem der aller ausgewerteten Projekte. Zugleich muss berücksichtigt werden, dass in Europa am meisten Projekte abgewickelt werden. Daraus folgt, dass das finanzielle Ergebnis der Projekte in Europa einen relativ gewichtigeren Einfluss auf die Marge des Anlagegeschäfts der Bühler AG hat. Um diese Aussage eindeutig zu bestätigen, müsste die Kostenperformance der Projekte pro Region ausgewertet werden. Die tiefe Erfolgsquote in Europa könnte mit dem hohen Volumen zusammenhängen, so dass beispielsweise die Projektmanager dieser Region mehr Projekte gleichzeitig abwickeln müssen. Eine weitere Auswertung, die die Auslastung der Projektmanager pro Region und die Anzahl Projektmanager in Relation zu Projektvolumen stellt, könnten mehr Aufschluss zur Ressourcenverteilung und Kapazitäten geben. Die Regionen Mittlere Osten und Afrika sowie China weisen eine deutlich bessere Erfolgsquote als Europa und Nordamerika aus, wickeln aber verglichen mit Europa nur halb so viele Projekte ab. Die deutlich höheren Erfolgsquoten in Südasien und Ostasien mit gleichzeitig noch tieferem Projektvolumen, scheinen die vorherige Vermutung bestätigen. in der Gesamtbetrachtung aller Regionen können die Quoten von Nordamerika und Südamerika nicht erklärt werden. 
\newline Die Auswertung der Geschäftsbereiche zeigt, dass \gls{abk:lo}, \gls{abk:gl} und \gls{abk:vn} im Vergleich deutlich tiefere Erfolgsquoten ausweisen. Die Erfolgsbilanz von \gls{abk:gm}, des grössten Geschäftsbereichs mit absolut am meisten gescheiterten Projekten, liegt nur geringfügig unter derjenigen über alle Projekte. Consumer Food und Value Nutritione haben eine ähnliches Projektvolumen, wobei erstere Erfolgsquote doppelt so hoch ist. Aufgrund der unterschiedlichen Maschinen oder Absatzmärkte ist es schwierig, neben dem Einfluss des Projektmanagements mögliche Erklärungsansätze zu finden.  %% Besser schreiben!! 
\newline
Die kombinierte Betrachtung der Regionen und des Geschäftsbereich bestätigt die Ergebnisse der Einzelauswertungen, denn in Europa und Nordamerika sind vor allem die Erfolgsquoten von \gls{abk:gl}, \gls{abk:vn} aber auch \gls{abk:gm} auffallend niedrig.
\newline Die Lead SAS des Projekts legt den organisatorischen Rahmen des Projekts fest, weshalb es naheliegend ist, dass sich deren Erfolgsquoten unterscheiden können. Vor alle die europäischen und amerikanischen Gesellschaften haben niedrige Verhältnisse zwischen erfolgreichen und gescheiterten Projekten. Dieses Ergebnis stimmt mit den vorangehenden Resultaten überein.
\newline
Der Kunde und auch der Projektmanager der Kategorie Fulfillment wurden als zentrale Einflussgrössen des Projekterfolgs identifiziert (vgl. Kapitel \ref{sec:erfprj}). Jedoch hätten die Häufigkeitsverteilungen aufgrund der hohen Einzelwert  wenig generalisierende Aufschlüsse über die Charakteristiken der erfolgreichen und nicht erfolgreichen Projekte gegeben.
%%
%%SQ
\newline\newline
Aufgrund der Datenplausibilität und der Datenverfügbarkeit konnte lediglich die Budgetabweichung zum Zeitpunkt des Projektabschluss für den Geschäftsbereich und die Region ausgewertet werden. Das arithmetische Mittel für Fail-Projekte war deutlich höher. Dieses Ergebnis kann als Hinweis für den Verkaufsdruck eines Area Manager bei Projektverkauf interpretiert werden. Anderseits kann dieses Ergebnis auch irreführend sein, da die zeitliche Komponente nicht berücksichtigt wurden. Die Zahlen zum Auftragsvolumen werden auf monatlicher Basis zur Verfügung gestellt, so dass die Area Manager dieselben Voraussetzung für Projektakquirierung haben. Demzufolge können auch trotz des Verkaufsdruck Projekte verkauft werden, die dann erfolgreich abschliessen. Eine Analyse der Projekte die im gleichen Monat und der darauffolgenden Monaten abgeschlossen wurden könnten mehr Aufschluss geben. Zudem muss angemerkt werden, dass die Berechnung der Budgetabweichung mit dem Projektstartdatum approximiert wurde. Korrekterweise hätte das Auftragsfreigabedatum als Bezugsgrösse hinzugezogen werden müssen, das aber aus systemischen Gründen nur für einen geringen Teil der Projekte möglich gewesen wäre. Da beide Daten normalerweise sehr eng aufeinander liegen, das heisst im selben Monat, und die Budgetabweichung für das Auftragsvolumen monatlich erhoben wurden, ist die Approximation relativ genau.
%%
%%Fulfillment
\newline\newline
Die Fulfillment-Variablen betreffen zum einen den Projektmanager und das Forecastmanagement sowie die verantwortliche Organisation für die Projektabwicklung. Der Wechsel des Projektmanager führte nur in 17 von insgesamt 43 Fällen dazu, dass das Projekt scheiterte. Der Austausch des Projektmanagers ist meistens eine ungeplante Konsequenz und ein Indiz für eine inadäquate Situation, beispielsweise Missverständnisse zwischen dem Kunden und dem Projektmanager oder mangelhafte Leistungen des Projektmanagers. Das Alter des Projektmanagers ist eine stellvertretende Variable für die Arbeitserfahrung. Die Auswertung der Alterskategorien zeigt, dass ältere und somit erfahrener Projektmanager tiefer Erfolgsquoten haben als ihre jüngeren Kollegen, wobei die Aspekte und Zuweisung der Projekte und die Erfahrung im Projektmanagement berücksichtigt werden müssen. Erstere unterstellt, dass ältere Projektmanager komplexere Projekte abwickeln. Zudem approximiert das Alter des Projektmanagers die gesamte Arbeitserfahrung, sodass kein Bezug zur Erfahrung im Projektmanagement, welche jedoch für den Erfolg eine zentrale Bedeutung hat, besteht. 
\newline Die Verantwortungsbereiche für die Projektabwicklung und des gesamten Projekts können von zwei unterschiedlichen Bühler Gesellschaften wahrgenommen werden. Insgesamt wurden die Verantwortungsbereiche bei 101 Projekten getrennt, wovon letztendlich 85 erfolgreich zu Ende geführt wurden. Es kann postuliert werden, dass bei getrennten Verantwortlichkeiten, die entsprechenden Aufgaben besser fokussiert werden können. 
\newline Der Effekt auf den Erfolg eines Projekts sowohl des PMChange als auch der getrennten Verantwortungsbereiche kann nicht abschliessend beurteilt werden, da zunächst der Zusammenhang zwischen dem Projekterfolg und den Variablen statistisch ergründet werden müsste. Die Indikatoren die die Zeitkomponente, des Kostenforecast erfassen konnten aufgrund unzureichender Plausibilität nicht ausgewertet werden. 
\newline\newline
%%
%%Komplexität
Die Komplexität der Projekte wurde mittels der Anzahl zuliefernden Parteien pro Projektphase und dem Konsortium ausgedrückt. Die kombinatorische Auswertung zeigte, dass entweder Eigenbeschaffung oder die Arbeit mit Drittlieferanten sowohl bei erfolgreich und nicht erfolgreichen Projekten das häufigsten Merkmal war, gefolgt von der Zulieferung durch eine oder zwei andere Bühler-Gesellschaft beim Mechnical Supply. Im Vergleich zu Kombinationen, bei denen mindestens eine weitere Bühler-Gesellschaft im Mechanical Engineering involviert war, weisen die oben beschriebenen drei Kombinationen die höhere Erfolgsquote aus. Dies könnte mit der Art der Tätigkeit zusammenhängen, da beispielsweise Maschinenbestandteile selbstständig hergestellt werden können, während der Koordinationsaufwand bei der Entwicklung der Maschinenpläne grösser ist.
\newline Von den 78 Konsortium-Projekten in konnte wurden 47 erfolgreich abgeschlossen. Die Aufschlüsselung nach der Region und des Geschäftsbereichs ergab, dass sich die Anzahl Konsortium-Projekte in Europa und China in den Geschäftsbereichen \gls{abk:gm}, \gls{abk:vn} und \gls{abk:gl} konzentriert und die Erfolgsquoten entsprechend niedrig waren. Dieses Resultat könnte einen kleinen Anteil der gescheiterten Projekte in Europa in den entsprechenden Geschäftsbereichen erklären. 
\newline Zur abschliessenden Beurteilung müsste wie bereits zuvor ein Zusammenhang zwischen dem Projekterfolg und den obigen Variable ermittelt werden.
\newline\newline
%%
%%Kosten
Die Auswertungen zu den Kosten zeigten, dass die gescheiterten gegenüber den erfolgreichen Projekten deutlich höhere Mehrkosten realisierten, was eine logische Implikation des gewählten Erfolgskriterium ist. Zu je einem Drittel sind sie auf erhebliche Kostenüberschüsse während der Beschaffung und der Installation zurückzuführen. Selbst bei den erfolgreichen Projekten sind während der Installationsphase erhebliche Mehrkosten entstanden. Diese Auswertung geben zwar Hinweise, bei welchen Phasen ein Kostenproblem vorliegt, allerdings müssten deren Ursachen untersucht werden. Insbesondere sind aus Bühler-Sicht selbstverursachte Kosten von denjenigen, die durch den Kunden entstanden sind, zu unterscheiden. Dadurch können die eigenen Stärken und Schwächen in den jeweiligen Phasen besser eingeschätzt oder optimiert werden. Die Auswertung des Projektportfolios auf der Basis des Umsatzvolumen zeigte, dass die Mehrheit der gescheiterten Projekte ein Umsatzbudget kleiner als 1.5 Mio. CHF hatte. Dennoch stammt der absolute Verlust auf der Projektmarge von wenigen grossen Projekten. Die höhere Erfolgsquote der grösseren Projekten, kann als Indiz für den Prestige-Effekt interpretiert werden. Dieser Effekt postuliert, dass komplexere und grössere Projekte mehr Beachtung von Projektmanagern erhalten, da sie mit Prestige assoziiert werden (Quelle). Die Priorisierung der grösseren Projekte kann dazu führen, dass sich die geringere Beachtung der kleineren Projekte in Verlusten manifestieren. Trotz der kleinen Verluste pro Projekt, ist der Effekt der Aufsummierung dieser Mehrheit nicht zu unterschätzen. Die Entscheidung, welche Art von Projekt prioritär behandelt werden soll, ist zwangsläufig mit einem Tradeoff verbunden. Die Auswertung der Zusammensetzung des Budgets in Bezug auf Unterschiede zwischen erfolgreichen und nicht erfolgreichen Projekten wenig Informationsgehalt.
\newline Mehrkosten haben direkten Einfluss auf die Projektmarge, weshalb die Ergründung der Ursachen von besonderem Interesse ist. Dadurch können zentrale Probleme entdeckt und entsprechend Lösungsansätze entwickelt werden. 
%%
%% Zeit
\newline\newline
Sämtliche Projekte hatten eine durchschnittlichen Zeitverzögerung zwischen fünf und acht Monaten. Dabei betrug die Verspätung bei gescheiterten Projekten im Mittel zwei Monate mehr. In Bezug auf alle Projekt kann ausgesagt werden, dass die Zeitverzögerung während der Fabrikationsphase eintritt, wobei Fail-Projekte nach der Installationsphase durchschnittliche zwei Monate mehr zeitverzögert waren. Die Erfassung der Gründe und die Kategorisierung nach Verursacher, können das Optimierungspotenzial im Zeitmanagement der Bühler AG aufweisen. Die Häufigkeitsverteilung zeigt, dass nach dem zweiten Meilenstein insgesamt fast 90\% aller Projekte noch in den Zeitvorgaben lag, während beim MS8 die Verteilung gespiegelt wird. Inwiefern die Zeitverzögerungen zu Kostenüberschreitungen und somit die Projektmarge beeinflussen müssten statistisch ergründet werden. 
%%
%% Zusammenfassung
\newline\newline
Zusammenfassend hat der Vergleich der Erfolgsquoten für kategoriale Variablen ergeben, dass nicht erfolgreiche Projekte 
%%
%%
Zur Ergründung der Erfolgsfaktoren müsste ein Zusammenhang zwischen den Parametern und dem Erfolg untersucht werden und die Signifikanz ausgewertet werden. Eine möglicher Ansatz wäre ein nicht lineares Modell zu schätzen, bei dem ein Zusammenhang zwischen den Parametern und dem binären Erfolgskriterium untersucht wird. Die gewählte Methode der obigen Analyse liefert demzufolge keine Hinweise zu möglichen Faktoren. Allerdings konnten unterschiedliche Attribute für erfolgreiche und nicht-erfolgreiche Projekte herausgearbeitet werden. Insbesondere geben die Auswertungen für die Region, den Geschäftsbereich und der Kosten Hinweise, bei welchem Prozessschritt und bei welchen Prozessorganisationen sich weitere Untersuchungen auszahlen können.
\newline\newline
Die Datengrundlage dieser Arbeit und der untersuchten Daten der Empirie unterscheiden sich dadurch, dass sie lediglich Parameter der Projekte enthalten und keine Faktoren. Die kategorialen Variablen beschreiben vor allem die Natur des Projekts, wohingegen die numerischen Variablen vor allem zu Kosten und Zeit als Kriterien anstatt als Faktoren zu bezeichnen sind. Ein qualitatives Design, bei dem mittels strukturierte Interviews mit Experten und den Projektteilnehmern zunächst Einflussgrössen im Projektmanagement erhoben werden und anschliessend bezüglich ihres Effekts auf den Erfolg eingeschätzt, könnten mehr Erkenntnisse zu Optimierungspotenzialen im Prozess einerseits und Erfolgsfaktoren anderseits liefern. 
\newline\newline
Die Erkenntnisse der Erfolgsfaktorenforschung zeigen, dass die Kompetenzen der involvierten Personen im Projekt, Kommunikation und die kulturellen Einflüsse des Unternehmens einen entscheidenden Einfluss auf den Projekterfolg haben. Zu dieser Erkenntnis gelangten bereits mehrere empirische Studien, die unter anderem auch identische Ergebnisse wie frühere Untersuchungen erhielten. Davon ausgehend, dass auch bei der Bühler AG die Mitarbeitenden und der Klient den Projekterfolg entscheidend beeinflussen, stellt sich die Frage, aus welchen Personen die Projektteams bestehen. Auf der Basis des Prozess und der Datengrundlage, können der Projektmanager, der Kunde, die Verkaufsmanager, der Geschäftsbereichsleiter und allenfalls der Regionenleiter als Projektteam bezeichnet werden. Die zuletzt genannten Personen haben dabei eine unterstützende Funktion. Die aktiven Rollen im Team haben der Area Manager beim Verkauf des Projekts und der Projektmanager bei der Abwicklung und der Kunde bei der Vorbereitung und Kommunikation der eigenen Bedürfnissen. Zusätzlich sind die Verantwortungspersonen während den Projektphasen Mechanical Supply, Plant and Automation sowie Installation als Bestandteil des Projektteams zu bewerten. Der Projektmanager bildet dabei der zentralen Kern, bei dem alle Fäden zusammenlaufen. Demzufolge ist es naheliegend, dass einen Teams besser untereinander kommunizieren und effizienter Zusammenarbeiten als andere. 

%%Ziel: eigene Ergebnisse interpretieren und praktische Relevanz erläutern
%%Einleitung: Erläutern kurz was gemacht wurde, Zusammenfassung der wichtigsten Ergebnissen, 
%%Erfolgsfaktoren: pro Kategorie, Hypothesen beantworten
%%Erfolgsfaktoren: in Bezug zu Hypothese
%%Erfolgsfaktoren: in Bezug zu Theorie
%%Erfolgsfaktoren: Scheiter und Scheitern lassen, Erfolgsdefiniion = Schwarz/Weiss Denken
%%Erfolgsfaktoren: Theorie sagt was anderes, Vorschläge zur Ergründung, aber Einflussdeterminaten könnten sein.....
%%Schlussfolgerung: Haupterkenntnisse
%%Schlussfolgerung: Relevanz für weitere Forschung und praktische Anwendung, was muss zukünftig geleistet werden, in Planform!
%%Fazit:

\section{Frühwarnsystem}\label{sec:diskfru}
Die Erkenntnisse aus der Theorie des vorangehenden Kapitel setzen den Rahmen für die nachfolgende Ausführungen. Die zentralen Anforderungen an Frühwarnsignale sind zum einen, dass sie bereits zu einem frühen Zeitpunkt gemessen werden können. Die Implementierung eines ganzen Frühwarnsystems erfordert allerdings eine strategische Verankerung, da ein konstantes Monitoring und Screening, sowie eine anschliessende Auswertung und Interpretation der Daten notwendig ist. Ohne die Unterstützung des Managements wird die Durchsetzung eines solchen Vorhabens kaum durchsetzbar sein.
%DAten
Die Auswertungen der Einflussfaktoren des Kapitel \ref{drei} geben zwar Hinweise, was mögliche Attribute nicht-erfolgreicher Projekte sein können, allerdings fehlt es an einer statistisch begründeten Signifikanz des Zusammenhangs mit dem Erfolgskriterium. Nichtsdestotrotz können einige Faktoren bereits aufgrund ihrer Natur und des möglichen Erhebungszeitpunktes als mögliche Frühwarnindikatoren ausgeschlossen werden. Denn vorzugsweise sollen sogenannte "leading factors" fokussiert werden, zu denen sämtliche Performanceindikatoren, beispielsweise Kostenabweichungen, Kosten aus Nachlieferung oder Zeitverzögerungen gemäss (Zitat) nicht zählen. Die Rahmenbedingungen sowie auch der Projektmanager oder die Verantwortungsgesellschaften sind zwar zu Beginn des Projektes bekannt, verändern sich jedoch über die Projektlaufzeit kaum. Allerdings könnte nach einer entsprechenden Analyse ihres Einflusses auf die Erfolgswahrscheinlichkeit Projekte mit entsprechenden Attributen eher überwacht werden als andere. Ein solches Vorgehen kann dazuführen, dass andere Signale ausser Acht gelassen werden. Zudem wurden wie bereits gesagt, gewisse Faktoren in der Analyse nicht berücksichtigt, die möglicherweise auch als Frühwarnindikatoren funktionieren könnten. Aus diesen Gründen erscheint es erforderlich, dass neue Faktoren bezüglich ihrer Fähigkeit als Frühwarnsignal zu fungieren, ergründet werden. Es gibt jedoch keinen formalisierten Prozess, zu deren Identifikation. 
\newline
Obwohl Projekt Assessments und seine Formen sowie Risikoanalysen bereits viele Hinweise zu möglichen Risiken und Chancen liefern, Bühler hat selbst für sogenannte Crash-Projekte Projekt-Assessments durchgeführt, die zu wichtigen Erkenntnissen für zukünftige Projekte geführt hat. Ausserdem wird am Ende jedes Projekts (Milestoen Debriefen) ein sogenanntes Debriefing abgehalten, welches implizierte, das Stärken und Schwächen eines jeden Projekts diskutiert wurden. Der Customer-Project Prozess hat neben anderen eine zentrale Schnittstelle vom Verkaufsprozess zum Fullfillment-Prozess. Basierend auf den Erkenntnissen der Literatur hat sich die Interface Analyse respektive die Beobachtung der
\paragraph{Sales \& Quotation} Schnittstellenthemen als relativ guter Frühwarnindikator erwiesen. Es liegt auf der Hand, dass der Output aus dem SQ-Prozess direkter Input im FF-Prozess bildet. Deshalb entstand die Idee, sozusagen die Frühwarnung für den Projekterfolg ab diesem Zeitpunk zu implementieren. Aus den Prozessabläufen der Bühler AG geht hervor, dass sowohl der SQ-Prozess und der FF-Prozess für Projekte grösser als eine Million umfangreiche Risikoanalysen gemacht werden. Vermutungsweise wird bereits zu diesem Zeitpunkt mögliche Erkenntnisse über zukünftige Herausforderung gewonnen, die wenig Beachtung erhalten. Fehlende Informationen, Assessments oder Dokumentation können als mögliche Frühwarnsignale interpretiert werden (siehe Klaggeg). Deshalb kann es von Nutzen sein eine Kennzahl zu entwickeln, welche auf automatisierte Basis die erforderlichen Dokumente gemäss den Anforderungen des MS1 beobachtet werden, sodass sichergestellt werden kann, dass keine Informationslücken entstehen. Somit wäre bereits früh klar, bei welchen Projekten alle relevanten Informationen zu Verfügung standen und der Übergabeprozess geglückt war. Auf Basis der in Kapitel \ref{sec:drei} erhobenen Daten hat sich gezeigt, dass Volumenmässig der Anteil an Projekte unter oder gleich einer Million fast die Hälfe aller Projekte ausmacht. Es wäre denkbar, dass bei diesen Projekten umfangreiche Risikoprüfungen ausgeblieben sind, da sie nicht priorisiert werden. Diese Annahmen und auch Probleme dieser Projekte müssten genauer untersucht werden, um andere Frühwarnindikatoren zu entdecken. Nichtsdestotrotz ist die Schnittstelle von SQ zu FF auch bei kleinere Projekte wichtig, damit das Projekt erfolgreich abgewickelt werden kann. In diesem Fall würde könnte es sich anbieten, eine Art Interface Analysis, die die Anzahl Interface-Themen und deren Bearbeitung in Bezug auf nur diese Schnittstelle misst, so dass es einerseits eine Plattform gibt, die Interface-Themen erfasst und offene/ungelöste Themen ersichtlich sind. Diese Idee liefert allerdings nur dem Projektabwicklungsprozess nähere Informationen, ob ein Projekte auf die schiefe Bahn gerät. Deshalb müssen auch für den weiteren Projekt-Management Prozess mögliche Ansätze diskutiert werden.
\paragraph{Projektabwicklung:} Im Propjektabwicklungsprozess sind weitere Schnittstellen vorhanden, welche genauer berücksichtigt werden müssen. Gemäss der Datenanalyse ist sowohl für Fail-Projekte und Success-Projekte der Mehrkostenanteil der IS-Kosten am höchsten gewesen. Unabhängig von der Ursache dieses Erscheinungsbild, ist die Installation die letzte Projektphase, so dass es von grossem Nutzen frühzeitig über mögliche Komplikationen Bescheid zu wissen, damit vorbereitende Massnahmen getroffen werden können. Zur Steuerung mittels Frühwarnsystem könnte ein Kombination aus Interfacemanagement und 'Gut Feelings' angewendet werden. Mittels Interfacemanagement soll sichergestellt werden, dass die auftauchenden Themen bearbeitet in nützlicher Frist bearbeitet werden. Der Ansatz der Gut Feelings hat zum Zweck, dass eine breiter Fokus für Variablen existiert, die einerseits während der Projektphase als mögliche Bedrohungen identifiziert werden und anderseits weder in den Risikochecks des SQ noch des FF inkludiert waren. Interne Dokumente belegen, dass der Projektmanager Erkenntnise aus der Risikoanalyse im Projektmanagement in detaillierter Form pflegen muss. Da erfahrungsgemäss ein gewisser Widerwille gegenüber umfangreichen Datenpflege festzustellen ist, sollte es im Tool eine Inputstelle geben für auf Intuition basierende Frühwarnsignale geben. Diese Anlaufstelle soll möglichst wertneutral, frei von Rechtfertigungsanforderung von übergeordneten Parteien, mit effizienter Handhabung und Zugang für sämtliche Projektteilnehmer ausgestattet sein. Dies ermöglicht dem Projektmanager eine Art Radar für zukünftige Herausforderung zu haben. Es sollte möglich sein, ein konstantes Monitoring pro Projekt ohne dabei detaillierte Angaben bereits erfassen zu müssen, sicherzustellen. Aspekte die berücksichtigt werden müssten sind die Strukturierung der Datenmenge sowie die Nutzung und Auswertung der Daten durch die Projektmanager.
\paragraph{Verschuldungsfrage:} Die vorangehende Analyse der Projekte wendet sozusagen ein schwarz-weiss Denken in Bezug auf den Erfolg ab. Allerdings konnte während der Analysephase festgestellt werden, dass es schwierig ist zu unterscheiden, welcher Projekttyp vorliegt. Beispielsweise werden Projekte gemacht, um neue Kunden zu gewinnen oder eine neue Technologie zu fördern, was zur Folge haben kann, dass die Kostenvorgaben relative zum Umsatz ambitiös ausfallen. Es würde jedoch Sinn machen, solche Projekte vom Standardgeschäft abgrenzen zu können, um mögliche Kernkompetenzen respektive Faktoren, die den Projekterfolg beeinflussen zu identifizieren. Zudem ist es wahrscheinlich, dass der Grund für die Mehrkosten nicht immer beim der Bühler AG liegen muss, unter dem Ausschluss, dass die Auswahl der Kunden von ihr beeinflusst wird. In Zusammenhang mit der Identifikation der Erfolgsfaktoren der Bühler AG könnte es folglich von Nutzen sein, Faktoren zu herben, die Aufschluss über kundenseitig induzierte Ursachen geben und wie anschliessend die Mehrkosten gehandhabt wurden. Zudem könnte die Befragung der Projektmanager und Verkaufsmanager mehr Aufschluss über zu berücksichtigenden Erfolgsfaktoren des Projektmanagements der Bühler AG geben. Hinzu kommt, dass aufgrund der Datenqualität wichtige Faktoren, wie zum Beispiel der Zeitpunkt der Forecast-Anpassung nicht ausgewertet werden konnten.
\paragraph{Projektprioritäten:} Aus einer finanziellen Perspektive die Fokussierung der Projekte mit grossem Umsatzvolumen, da ihr Einfluss auf das Ergebnis im Anlagengeschäft relativ gewichtig ist. Dennoch sollte der DB1 Verlust von Projekten mit einem Umsatzvolumen bis maximal 1.0 Mio. CHF nicht vernachläassigt werden, da sie am Projektvolumen eine relativ hohen Anteil haben. Wie bereits erwähnt wurde, werden Projekte unter 1 Mio. CHF von die internen Richtlinien zur vertieften Risikoanalyse der Bühler AG nicht erfasst.
%%% Text fehlt 
\paragraph{Incentivierung:} Die Anwendung von Frühwarnindikatoren bedingt, dass die Unternehmenskultur sowie auch die Projektmanagementstrategie entsprechend verändert respektive ausgerichtet wird. Die Implementierung von Frühwarnsignalen kann keine einmalige Übung darstellen, da es ein laufender Prozess ähnlich dem monatlichen Reporting ist. Die Abstimmung und Ausrichtung der Prozess und involvierten Personen auf ein gemeinsames Ziel "erfolgreiche Projekte" abzuschliessen ist dabei von grosser Wichtigkeit. Der Fokus sollte auf die Ergreifung von Massnahmen zur entsprechenden Gegensteuerung bei Komplikationen gerichtet sein anstatt auf die interne Schuldfrage. Eine sogenannte Fehlerkultur, die den Umgang mit Fehlern, Fehlerfolgen und Fehlerrisiken inkludiert, kann ein konstruktives Lernen aus Fehlern oder die Entdeckung effektiver Massnahmen bei Fehlerrisiken begünstigen. Alam Gühl (2016, S.20-21) plädiert im Rahmen der Projektkultur den positiven Umgang mit Fehlern und  den umfangreichen Austausch von entsprechenden Informationen, das von der Unternehmenskultur begünstigen werden kann. Ein anderes zentraler Einflussfaktor im Zusammenhang mit Frühwarnsignalen, ist auch die Fähigkeit des Projektmanagers zu Eingeständnissen, dass sein Projekt zu scheitern droht. Denn werden drohende Risiken und deren mögliche Realisation verkannt oder verhältnismässig spät kommuniziert, können der Handlungsspielraum eingegrenzt werden. Allerdings muss diese Denkweise aktiv im Unternehmen gefördert werden, damit das Bewusstsein der Fehlerakzeptanz, d.h. die positive Assoziation zu Fehlern und Scheitern gefestigt wird. 


%%Personen abhängig alli müsssen an einem Strang ziehen, Ausrichtung der Menschen an Zielen es Unternehmesn
%% Probleme Installation
%% Projektkategorisierung
%% Verschulden Bühler etc.
 