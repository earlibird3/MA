%%Allgemeine Formatierung
\documentclass[11pt]{report} %Einstellung Design des Dokuments und Schriftgrösse 11
\usepackage[T1]{fontenc} %Ausgabe der Umlaute erlauben
\usepackage[utf8]{inputenc} %Eingabe der Umlaute erlauben
\usepackage[german]{babel} %Spracheinstellung Deutsch
\usepackage{hyphenat} %Ermögliche Trennung der Wörter im Blocksatz


%%Seiteneinstellung
\usepackage{lscape} %Erlaube Landschaft-Seiteneinstellung
\usepackage{geometry} %Erlaube Veränderung der Seiteneinstellungen
\geometry{a4paper,left=2.5cm,right=2.5cm,top=2.5cm,bottom=2cm} %Seiten einrichten
\usepackage{textcomp} %Blocksatz einrichten
\usepackage[onehalfspacing]{setspace} %Zeilenabstand setzen

%%Grafiken
\usepackage{graphicx,%Ermögliche Einbindung von Grafiken
	wrapfig} %Grafiken mit umschliessendem Text

%%Tabellen
\usepackage{longtable,%Tabellen über mehrere Seiten ermöglichen
float}  %Forcierung der Tabelle an der Stelle ermöglichen
\usepackage{array} %Erweiterung der Array und Tabellenumgebung

%%Mathematik
\usepackage{amsmath} %Mathematische Eingaben ermöglichen

%Inhaltsverzeichnis - inkludiere Abkürzungs-, Abbildungs- und Tabellenverzeichnis 
\usepackage[nottoc]{tocbibind}
\usepackage{tocloft} %Ermögliche Punkte 
\renewcommand{\cftchapfont}{\normalfont\bfseries}% titles in bold
\renewcommand{\cftchappagefont}{\normalfont\bfseries}% page numbers in bold
\renewcommand{\cftdotsep}{1}
\renewcommand{\cftchapleader}{\bfseries\cftdotfill{\cftsecdotsep}}% dot leaders in bold
\usepackage{titlesec}
\titleformat{\chapter}{\normalfont\LARGE\bfseries}{\thechapter}{1em}{}

%%Abkürzungsverzeichnis / Erklärung Abkürzungsverzeichnis nomencl http://strobelstefan.org/?p=153
\usepackage[acronym,nomain,nopostdot,nonumberlist,toc]{glossaries} 
\makeglossaries % Erstelle Abkürzungsverzeichnis, d.h. .gls Datei
%Einträge des Abkürzungsverzeichnis
\newacronym{abk:db1}{DB1}{Deckungsbeitrag 1}
\newacronym{abk:ME}{ME}{Mechnical Engineering}
\newacronym{abk:PA}{PA}{Plant Automation}
\newacronym{abk:IS}{IS}{Installation}
\newacronym{abk:MS}{MS}{Mechanical Supply}

%%Literaturverzeichnis
\usepackage[natbibapa]{apacite} %Setze APA Zietierungsstandard
\usepackage{natbib}
\usepackage{hyperref} %Ermögliche Einstellungen für Hyperlinks
\hypersetup{hidelinks = true}
\usepackage{url}

%%Anhang
\usepackage[titletoc,toc,title]{appendix} %Inkludiere Anhang

%Titel
\title{Einflussfaktoren und Frühwarnsystem im Projektmanagement der Bühler AG}
%Autor
%\author{Michèle Schoch}
%\date{21. August 2016}

%%Beginn des Dockuments
\begin{document}\selectlanguage{german}
	
%%Titelseite
\pagenumbering{roman}%römische Nummerierung für Seiten vor Einleitung
\begin{titlepage}
\vspace*{4cm}
\begin{center}
%Titel
{\LARGE Einflussfaktoren und Frühwarnsystem im Projektmanagement der Bühler AG\par}
\vspace{4cm}
%Autor
{\Large Universität St. Gallen \par}
{\Large Michèle Schoch (10-607-448)\par}
\vspace{2cm}
%Degree
{\Large Masterarbeit\par}
{\Large Institut für Accounting, Controlling und Auditing\par}
\vspace{1cm}
%Referent
{\Large Referent Dr. Simon Pfister\par}
\vspace{1cm}
%Date
{\Large 20. November 2017\par}
\end{center}
\clearpage
\end{titlepage}


%%Inhaltsverzeichnis
\setlength{\parindent}{0pt}
\tableofcontents\newpage

%Abkürzungsverzeichnis
\printglossary[style = super, title=Abkürzungsverzeichnis, toctitle = Abkürzungsverzeichnis]
\newpage

%Abbildungsverzeichnis
\listoffigures\newpage
%Tabellenverzeichnis
\listoftables\newpage
%Einleitung
\pagenumbering{arabic}
% !TEX root = MA.tex
\section{Einleitung}
Der Erfolg, eine erfolgreiche Einleitung schreiben, ist nur eine Perzeption von unzähligen. Erfolg spielt in der gegenwärtigen Gesellschaft eine wichtige Rolle, Erfolg bei der Partnersuche, bei der Arbeit, in der Schule, im Sport oder bei der Suche nach Lösungen. Jedes Individuuzm wird während seiner Lebenszeit zwangsläufig mit Erfolg in Berührung kommen, je nachdem wie Erfolg definiert wird. Wenn Erfolg bedeutet, "Laufen können", sind all jene Menschen, die ohne körperliche Beeinträchtigung in Bezug auf die Gehfähigkeit gelernt haben zu laufen, in ihrem Leben bereits einmal erfolgreich gewesen. Die 'neutrale' Definition des Duden für den Erfolg lautet "positives Ergebnis einer Bemühung; Eintreten einer beabsichtigten, erstrebten Wirkung". Daraus folgt, dass für den Erfolg ein positives Ergebnis, ein vorgängig definiertes Ziel und geleisteter Effort vorliegen muss. Diese breite Definition lässt sich auf beliebige Situationen im Leben anwenden. Ausserdem kann das vorausgesetzte positive Ergebnis mittels unterschiedlicher Attributen individualisiert respektive verändert werden. Die Perzeption und Bedeutung von Erfolg ist somit bezüglich seiner Vielfältigkeit unbegrenzt.
\newline Die vorliegende Arbeit befasst sich mit dem Erfolg von Projekten und diesbezüglich mit dem erfolgreichen Projekt Management. Der Erfolg von Projekten beschäftigt die Forschungsgemeinschaft seit den 60iger Jahren, weshalb eine dichte Anzahl von Studien existiert, welche die Faktoren, die den Projekterfolg beeinflussten. Die Studien untersuchen unterschiedliche Projektarten, beispielsweise Softwareentwicklungsprojekte aber auch Industrieprojekte. Die Unterscheidung ist notwendig, da sich die angewandten Methoden zur Abwicklung der Projekte unterscheidet sowie die Rahmenbedingungen divergieren können. So wird bei Softwareprojekten zwischen agilem Vorgehen und der Wasserfall-Methode unterschieden. Die Wasserfallmethode wäre das Pendant zum traditionellen Managementansatz bei Industrieprojekten. Typischerweise erfolgt das Projektmanagement mit Hilfe sogenannten Milestones, das heisst ein Zwischenziel muss erreicht werden, bevor mit dem nächsten Schritt begonnen werden kann. Ebenso unterscheiden sich die Erfolgsdefinitionen, welche in den Studien unterstellt wurden. Im traditionellen Projektmanagementansatz bewegt sich die Erfolgsdefinition von Projekten im magischen Dreieck; Kosten, Zeit und Qualität. Allerdings liegt letztendlich der Fokus auf dem finanzielle Verlust aus der Projektmanagementtätigkeit, da dieser direkten Einfluss auf das Unternehmensergebnis hat.
\newline In dieser Arbeit beschränkt sich die Untersuchung der Projekte auf die Anlageprojekt der Bühler AG, ein Maschinentechnologiekonzern im Familienbesitz. Die Bühler AG produziert Maschinen für die Nahrungsmittelindustrie. Das Hauptgeschäft der Bühler AG war ursprünglich der Bau von Mühlen, wobei die Herstellung von Mahlwerken noch heute der grösste Geschäftsbereich der Bühler AG ist. Seit der Gründung wurde das Maschinenportfolio laufend ergänzt und beinhaltet heute Maschinen zur Herstellung von Pasta, Schokolade, Reis, Anlagen für die Linsenbeschichtung oder Kugelmühlen für den Mischprozess bei Farben. Gemäss den Angaben der Bühler AG decken sie dem gesamten Wertschöpfungsprozess für die Herstellung der entsprechenden Nahrungsmittel ab, das heisst vom Korn bis zum Endprodukt oder von der Kakaofrucht bis zur Schokolade.
\newline




Das Problem zu erkennen ist wichtiger als die Lösung zu erkennen, denn die genaue Darstellung des Problems führt zur Lösung. 
Albert Einstein, Physiker
	\subsection{Ziel der Arbeit}
	\subsection{Methodik und Vorgehen}
	

	




\newpage
%Theorieteil
% !TEX root = MA.tex
\chapter{Theoretischer Rahmen}\label{sec:theor}
	
Der Erfolg von Projekten und deren Management ist ein in der Forschung viel diskutiertes Thema, weshalb hier in den nachfolgenden Kapiteln zunächst im Allgemeinen auf die Erfolgsdefinition von Projekten respektive Projektmanagement und bereits erforschte Projekterfolgsdeterminanten eingegangen wird. Im Anschluss folgt eine Erläuterung dieser Termini im Kontext mit der Bühler-Welt.
%%
%%part Erfolgsfaktoren und Projektmanagement
% Erläuterung Projekt: UT Software/Konstruktion, Fokus der Arbeit (FdA): Maschinen/Anlageproj der Bühler AG
% Projektmgmt: Definition, Komplexität, Interdisziplinarität, Grundprozess nach Din
% Projektmgmt: Methoden: Agile und Traditionelle Methode, FdA: traditionller Ansatz der Bühler AG
% Projektmgmt: Einflussfaktoren im PM, die Erfolg begünstigen können in Abh. Projektarten, Projektmgmtmethoden
% Erfolg: Definition, Unterscheidung Erfolg und Kriterien, Ansätze: traditionelle vs. andere: FdA Bühler AG
% Erfolgsfaktoren: Forschungsstand: Relevante Faktoren und weitere Einflussfaktoren, Überleitung zu Bühler Prozes, Bühler Einflussfaktoren unter der Berücksichtigung der Bühler Erfolgsdefinition
\section{Erfolgsfaktoren im Projektmanagement} \label{sec:erfprj}	
Gemäss dem Deutschen Institut für Normung (DIN) ist ein Projekt: " ein Vorhaben, das im Wesentlichen durch Einmaligkeit der Bedingungen in ihrer Gesamtheit gekennzeichnet ist, z.B. Zielvorgabe, zeitliche, finanzielle, personelle und andere Begrenzungen, Abgrenzung gegenüber anderen Vorhaben, projektspezifische Organisation" (Quelleangabe). Daraus folgt, dass Projekte sich bezüglich einzelner Faktoren unterscheiden können, allerdings die Gesamtheit der Faktoren ihre Einzigartigkeit definiert. Beispielsweise begründet bei der Bühler AG die internationale Tätigkeit, das diverse Anlageportfolio und die breite Kundenbasis ein Indiz für einmalige Projekte. Obwohl es unterschiedliche Projekte gibt, beispielsweise im Tiefbau, Hochbau und Ingenieurbau und sich deren Management sowohl durch Differenzen als auch Gemeinsamkeiten charakterisiert, weist ein Projekt gemäss Projektmanagement-Handbuch (ohne Datum) folgende Eigenschaften auf: "komplexe, neuartige Aufgabenstellung, messbare Ziele und Ergebnisse, zeitliche Befristung (Anfang und Ende), begrenzte Ressourcen und die Notwendigkeit von Teamarbeit" (Quellenangabe). Meyer und Rehrer (2012, S.2) sehen die progressive Elaboration, die eine kontinuierliche Konkretiesierung des Projekts während dessen Verlauf als weiteres Merkmal von Projekten. Der exakte Projektbegriff der vorliegenden Arbeit orientiert sich anschliessend an den geschäftsinternen Definitionen der Bühler AG
\newline
Der Managementbegriff wird vom Projektmanagementhandbuch (ohne Datum, Jahr) als systematischer Prozess zur Führung komplexer Variablen definiert. Er beinhaltet die Organisation, Planung, Steuerung und Überwachung aller Aufgaben und Ressourcen, die notwendig sind, um die Projektziele zu erreichen". Das Projekt Management Institute (PMI) (PMBOK, 2004) beschreibt Projektmanagement als eine Anwendung von Wissen, Fähigkeiten, Instrumente und Techniken bei Projektaktivitäten, um Projektanforderungen zu erfüllen. Nach Alama und Gühl (2016) wird Projektmanagement als "die Koordination von Menschen und der optimale Einsatz von Ressourcen zum Erreichen von Projektzielen dargelegt. Pierce (S.2015, S.2) führt eine generelle Definition aus, gemäss derer Ziele, Prozesse, Planung und Kontrolle den Managementterminus beschreiben. Die Literatur zeigt keine einheitliche Definition, dennoch kann zusammengefasst konstatiert werden, dass Projektmanagement die zielgerichtete Planung, Steuerung und Überwachung von  Ressourcen in den Prozessschritten umfasst. Der Projektmanagementprozess kann generell in folgende Schritte unterteilt werden; Projektinitiierung, Projektplanung, Projektdurchführung und -kontrolle, und Projektabschluss (PMHandbook, ohne Datum). Der der Analyse zugrundeliegende Prozess wird im Kapitel \ref{zweizwei} erläutert, weshalb an dieser Stelle nicht weiter auf den Prozess als solches eingegangen wird. 
\newline
Der Erfolg von Projekten beschäftigt die Forschung seit längerer Zeit. Das vorherrschende Paradigma zur Beurteilung des Projekterfolgs ist das magischen Dreieck, Zeit, Kosten und Qualität. Allerdings ist in der Praxis letztendlich der Kostenaspekt von zentraler Bedeutung, da er mit Geldverlust korreliert ist. Die aktuelle Forschung der Erfolgsfaktoren zeigt, dass der Erfolg von Projekten nicht auf ein Faktor reduziert werden kann. Deshalb unterscheidet Besteiero, Pinto \& Novaski (2015) zwischen kritischen Erfolgsfaktoren und Erfolgskriterien. Die Faktoren erhöhen ihrer Auffassung nach die Erfolgswahrscheinlichkeit wohingegen die Kriterien darüber bestimmen, ob ein Projekt erfolgreich war (Besteiro, Pinto, Novaski, 2015). Hieraus entsteht eine Differenz in der zeitlichen Betrachtung, Faktoren sind während der Projektabwicklung relevant, um das Projekt erfolgreich abzuschliessen wohingegen Kriterien erst nach dem Projektabschluss hinzugezogen werden, um den Erfolg zu bestimmen. Folglich wäre der monetäre Aspekt ein Erfolgskriterium. Es muss jedoch hinterfragt werden, ob sich die Erfolgsbeurteilung von Projekten angesichts ihrer Eigenartigkeit auf ein Kriterium reduzieren lässt.  Denn bei einer reinen Kostenbetrachtung werden die Einhaltung von Zeitvorgaben und die Qualitätsanforderung ausser Acht gelassen. Kerzner (2014) schlägt anstatt der traditionellen Erfolgsbetrachtung vor, den Projekterfolg als die Erreichung des gewünschten Geschäftswertes innerhalb der sich konfligierenden Zielvorgaben zu verstehen. Daraus folgt, dass ein gewisser trade-off resultiert, da beispielsweise der Liefertermin nicht eingehalten werden kann, aber neues Wissen generiert werden konnte, welches in anderen Projekten von Nutzen sein kann. 
\newline
Nachfolgend werden einige Ergebnisse bisheriger Studien zu den Erfolgsfaktoren dargelegt, wobei sich sowohl die Branche (IT und Schwerindustrie) als auch die Erfolgsdefinitionen unterscheiden. Grundsätzlich liegt dem Konzept der Erfolgsfaktoren die Prämisse, dass Erfolg wiederholbar ist und an bestimmte Faktoren geknüpft ist, zugrunde. Folglich müsste es auch Determinanten geben, welche den Projekterfolg negativ begünstigen. Iyer \& Jha (2006) fanden mittels Expertenbewertungen und anschliessender Faktoranalyse heraus, dass das sich Engagement der Projektmitarbeiter und die Fähigkeiten des Projekteigners positiv auf die Zeitperformance auswirken. Konflikte zwischen dem Projektmanager und Top-Management, dem Projekteigner oder anderen externen Parteien sowie eine Missgunstkultur können zur Überschreitung der vorgegebenen Projektzeit führen (Iyer \& Jha, (2006). Ein ähnliches Ergebnis ergab die Studie von Chan, Ho \& Tam (2001), wonach das Engagement des Projektteams, das im weiteren Sinne Kooperation, Konfliktlösung, Vertrauenskultur, Verständnis der Projektziele sowie Kommunikation bedeutet, als kritische Erfolgsfaktoren zu qualifizieren sind. Dieses Ergebnis ist mit den Befunden früherer Studien (Ashley, et al. 1987, Pinto und Slevin, 1988) kompatibel, bei denen das Engagement der Projekteilnehmer ebenso als erfolgskritisch identifiziert wurde. Chan, Ho \&Tam (2001) machten zudem die Konklusion, dass die Fähigkeiten des Auftragsnehmers hinsichtlich eines qualitativen Projektmanagements und der Anwendung innovativer Technologien sowie die Kundenkompetenz in Bezug auf die Abwicklung von Konstruktionsprojekten eine entscheidende Rolle für den Projekterfolg haben. Demgegenüber sind bei Softwareprojekten vor allem die Unterstützung des Managements, Kommunikation oder das gemeinsame Verständnis Projektmission von grosser Bedeutung (Hyvräri, 2006 in Besteiro, Pinto, Novaski, 2015). Der Einbezug von Studien, die unterschiedliche Projekte analysierten, hatte zum Ziel, Unterschiede in Bezug auf die Erfolgsfaktoren feststellen zu können. Varajão, Dominguez \& Ribeiro et al. (2014) stellten fest, dass die Projektplanung und das Verständnis der Projektziele sowie -anforderungen bei beiden Projektarten kritische Erfolgsaspekte sind. Allerdings wird beispielsweise die Effizienz des Projektmanagements und der Miteinbezug aller Projektteilnehmer bei Konstruktionsprojekten als kritischer für den Erfolg erachtet als bei Software-Projekten. Unter der Anwendung eines gewichteten Erfolgskriterium aus Kosten, Zeit, Qualität und Funktionalität haben Lam, Chan \& Chan (2008) herausgefunden, dass die Projektnatur, die Effizienz des Projektmanagement und die Anwendung innovativer Managementtechnologien einen erfolgreichen Projektabschluss begünstigen. Attraktive und komplexe Projekte würden die Projektmanager dazu veranlassen, mehr Effort für das Projekt zu leisten, da solche Projekte mit Prestige und Selbstverwirklichung verbunden sind (Lam, Chan \& Chan, 2008). Mittels logistischer Regressionsanalyse fanden Lu, Hua \& Zhang (2017) auf Basis einer Kostenperspektive heraus, dass die Fähigkeiten des Auftragsnehmers aufgrund von Erfahrungen und bezüglich der Teamfähigkeit sowie Kostenaffinität relevant für den Erfolg des Projekts sind. 
\newline Aus den vorangehenden Ausführungen geht hervor, dass trotz unterschiedlicher Erhebungsmethoden, Erfolgskriterien und Projektarten ähnliche Erfolgsfaktoren identifiziert wurden. Zusammenfassend kann deshalb postuliert werden, dass der Erfolg mit den Projektteilnehmern und der Projektkultur korreliert. Dabei stellen Attribute wie, Fehlerkultur, Teamfähigkeit, Konfliktlösen, Vertrauen, gemeinsame Mission oder Wertschätzung nur eine Auswahl dar, um die Projektteilnehmer und das Arbeitsklima zu beschreiben. Alam \& Gühl (2016) sprechen in diesem Zusammenhang auch von denjenigen Anforderungen, die während jeder Projektphase gegeben sein müssen, damit Projekte erfolgreich bearbeitet werden können. Der Erfolg von Projekten bestimmt sich auf Basis der gewählten Erfolgsdefinition, wobei die Vergleichbarkeit zwischen den Projekten aufgrund ihrer Einzigartigkeit nur bedingt möglich ist. Das vorherrschende Erfolgsparadigma, das magische Dreieck, wird jedoch kritisch gewürdigt, da bisher nicht-monetäre Wertgenerierung aussser Acht gelassen wurde. Denn ausschliessliche Fokussierung der Kosten- und Zeitvorgaben berücksichtigt die Änderungen des Projektinhalts aufgrund neuer Ereignisse während der Projektlaufzeit  nicht.
%%
%%part PM der Bühler AG 
%%subpart Prozess: Customer Project Prozess
%%subpart Einflussfaktoren: Summarische Erläuterung, Kategorien & Begründung, Erfolgskriterium nochmals erwähnen?, Verweis auf Kapitel 3, Hypothese: Dass Variablen Attribute von erfolgreichen respektive nicht erfolgreichen sein können.
\section{Projektmanagement der Bühler AG}
In diesem Kapitel wird der Bühler Projektmanagement-Prozess, der die Basis für die Ergründung der möglichen Einflussfaktoren bildete, erläutert. Die nachfolgenden Ausführungen basieren auf den intern dokumentierten Prozessbeschreibungen des C2C-Prozesses, Customer Project (CP). Er gliedert sich in zwei Kernprozesse, den Sales \& Quotation- und den Fulfilment-Prozess. Anschliessend hat das zweite Unterkapitel zum Ziel, die Faktoren, welche möglicherweise den Projekterfolg beeinflussen oder Charakteristiken von nicht-erfolgreichen Projekten bilden sowie ihre Bedeutung erklärt.
\subsection{Projektmanagementprozess der Bühler AG}\label{zweieins}
Die nachfolgende Abbildung \ref{fig: processcp} zeigt die zwei Kernprozesse, wobei beide Prozesse durch das Hand-over-Meeting (HOM) direkt ineinander übergehen. 
\begin{figure}[H]
	\centering
	\includegraphics[width=5cm]{processcp.png}
	\caption{Prozess Customer Project der Bühler AG}
	\label{fig:processcp}
\end{figure}
Links in der Abbildung \ref{fig:processcp} ist der SQ-Prozess dargestellt, der durch die Übergabe des Projekts in den FF-Prozess mündet. In der Folge sollen beide Prozesse zusammengefasst beschrieben werden, wobei der Fokus auf denjenigen Bestandteilen liegt, die im Zusammenhang mit den Erfolgsdeterminanten stehen. Der Verkaufsprozess umfasst vier Phasen: Identify Potential, Set Priorities, Quote and Evaluate Risk und Close Order. Die Verantwortlichkeit für den Prozess liegt hauptsächlich beim Area Manager(AM). Die Business Unit (BU) und entsprechende Centers of Competences tragen eine Mitverantwortung. Das Fundament des CP-Prozess bilden die Milestones (MS), welche die Erreichung oder die Lieferung vordefinierter Ziele einfordern, bevor mit dem Prozess weitergefahren werden darf. Die Phase I und II konzentrieren sich darauf, Geschäftspotenziale und Kundenbedürfnisse zu identifizieren, Kontakte mit den Kunden aufzunehmen und letztendlich auf Basis von diversen Checks zu entscheiden,  welche Projekte fokussiert, das heisst, offeriert werden sollen. In der Phase III und IV gilt es die möglichen Projekte einer Detailprüfung in technischer, kommerzieller und finanzieller Hinsicht zu unterziehen sowie ein Basiskonzept auszuarbeiten. Nach den anschliessenden Vertragsverhandlungen zwischen dem Kunden und der BU respektive dem AM endet der Prozess mit dem MS Orders Released (OR.) Der Auftrag wird freigegeben und das Projekt wird nach kurzer Zeit beim HOM an die Projektabwicklung übergeben. Im Fulfillment werden fünf Phasen unterschieden, wie der Abbildung \ref{fig: processff} zu entnehmen ist.
\begin{figure}[H]
	\centering
	\includegraphics[width=5cm]{processff.png}
	\caption{Prozess Customer Project der Bühler AG}
	\label{fig: processff}
\end{figure}
\textbf{Phase I: Planning and Basic Engineering}
\newline
Bei der Übergabe des Projekts vom Verkauf an die Abwicklung sind typischerweise der AM, der PM und der Teamleiter involviert. Diese wichtige Schnittstelle dient dazu alle relevanten Informationen zu übergeben und offene Punkte zu klären. Die Phase I beinhaltet die Projektanalyse, die Ausarbeitung respektive Überarbeitung des Konzepts, die Projektplanung und das Kick-off-Meeting (KOM). Das Ziel der Projektanalyse ist die Realisierbarkeit mittels der Identifizierung von technischen als auch kommerziellen Risiken und Chancen sowie entsprechenden Massnahmen zu prüfen. Die anschliessende Konzeptphase beinhaltet die Ausarbeitung oder Nachbearbeitung des Maschinen- oder Anlagekonzept und sowohl die interne als auch die externe Genehmigung einzuholen. In der Projektplanung werden überwiegend organisatorische und administrative Aufgaben wie zum Beispiel die Planung der Deadline oder die Definition von Arbeitspaketen gemacht. Der letzte MS dieser Phase bildet MS5, das Kick-off-Meeting, welches der Schaffung eines gemeinsamen, identischen Verständnis unter sämtlichen Teilprozessverantwortlichen bildet. Im KOM werden in Abstimmung mit den vertraglichen Bedingungen, verbindliche Vereinbarungen bezüglich, Termine, Kosten, Qualität und Zuständigkeiten getroffen. Ausserdem bietet das KOM Raum zur Diskussion unklarer Punkte. In der Regel findet das KOM nach der kommerziellen respektive Gesamtfreigabe statt.
\newline
\textbf{Phase II: Engineering and Specification}
\newline
Charakteristisch für diese Phase ist die Ausarbeitung verbindlicher Pläne zur Anlagen- oder Maschinendisposition. Optimierungen am Maschinen- respektive Anlagenkonzept und der interne Review sind während dieser Zeit von zentraler Bedeutung. Nach der schriftlichen Einverständniserklärung des Kunden zum Layout der Prozessanlage sind Änderungen dem Risiko von Mehrkosten, Zeitverzögerung und anderen Anpassungen in der Projektstruktur ausgesetzt. Der MS 5 "Point of now return" ist ein interner Meilenstein, deren Erreichung die Verbindlichkeit der Liefertermine gegenüber dem Kunden sowie die Sicherstellung der Kundenzahlung, erfordert. 
\newline
\textbf{Phase III: Manufacturing, Procurement and Logisic Out}
\newline
Diese Phase beginnt mit der Fabrikation und endet mit der Lieferung der Maschine an den Sitz des Kunden respektive an den vereinbarten Lieferort. Die Einhaltung des Lieferversprechen sowie die vertragskonforme Ablieferung der Anlage oder Maschine(n) ist hierbei von besonderer Wichtigkeit. Der nachfolgende Prozessschritt 9 Project Documentation umfasst die Erstellung der Projektdokumentation für den Monteur und den Kunden, welche die Nachvollziehbarkeit der Änderungen garantiert. Das Ende dieser Phase wird durch den MS8 festgehalten, der erreicht wird, wenn die Dokumentation offiziell an den Kunden und den Monteur überreicht wurde. 
\newline
\textbf{Phase IV: Installation and Start up}
\newline
Die Installation, Inbetriebsetzung und Übergabe sind die elementaren Prozesschritte dieser Phase. Der Zusammenbau einer Anlage und die Inbetriebsetzung einer Maschine erfordert eine Instruktion des Montageteams, die zugleich eine unabdingbare Voraussetzung für ein gewisses Qualitätsniveau gewährt. Die Montageverantwortlichen werden durch die Projekt- und Verkaufsleiter ständig unterstützt, wobei gleichzeitig die Überwachung des Prozesses gewährleistet wird. Der Meilenstein 10 wir mit der kompletten Übergabe der Anlage an den Kunden nach dem Abschluss der Inbetriebsetzung erreicht. Hierbei ist darauf zu achten, dass möglichst alle vertraglich vereinbarten Anforderungen, wie zum Beispiel Tests und Inhalt, Umfang und Darstellung der Übergabedokumente erfüllt werden, da oftmals die letzten Kundenzahlungen an die Leistungserfüllung gekoppelt sind. 
\newline
\textbf{Phase V: Evaluation and Transfer}
\newline
Als letztes folgt das Debriefing, bei welchem im Sinne des kontinuierlichen Verbesserungsprozess, Rückmeldungen zur Optimierung der Projektabwicklung für künftige Projekt festgehalten werden, so dass die gleichen Fehler nicht wiederholt  werden. Der elfte Meilenstein legt intern den Projektabschlussfest. Danach beginnt die zweijährige Garantieperiode. 
\subsection{Einflussfaktoren im Projektmanagement der Bühler AG}\label{zweizwei}
Nachfolgend werden sämtliche Einflussdeterminaten und das Projektmanagement-Tool, das sämtlich Projektinformationen enthält, erläutert.
\newline
Das BPM-Cockpit enthält Informationen zu Kosten, Zeit, Verantwortlichkeiten, Risiken und Aktionen als auch zum Engineering inklusive verschiedener Projektstatus. Die Beurteilung der Projektperformance erfolgt anhand eines dreifarbigen Ampelsystems für Kosten, Zeit und Qualität. Die ersten beiden werden vom System automatisch gerechnet wohingegen die Qualität auf der subjektiven Einschätzung der Projektmanager beruht, die sie selbst berichten können. Basierend auf der internen Richtlinie für das BPM-Cockpit ist der Kostenstatus grün, wenn die konsolidierte Abweichung der Projektmarge zwischen dem Forecast und dem Budget mehr als -400 Prozentpunkte beträgt. Dieser Status ändert von grün auf gelb, sobald die Abweichung weniger als -4\% beträgt und von gelb auf rot, wenn die Schwelle von -10\% unterschritten wird. Die Zeitampel basiert einerseits auf der Differenz zwischen dem realisierten und dem geplanten Termin und anderseits auf der Eintragung im System. Sofern Angaben zum realisierten Termin im System enthalten sind, ist der Status grün, wenn er vor oder auf dem geplanten Datum liegt und gelb, wenn das Zeitversprechen nicht eingehalten wurde. Die rote Farbe impliziert, dass das BPM-Cockpit keine Informationen zur Erfüllung des Termins enthält und der geplante Termin vor dem heutigen Datum liegt. Die Farbe der Ampeln hängt davon ab, ob der Forecast für eine Reportingperiode angepasst wurde oder nicht. Je nach Grösse der Abweichung zwischen dem Budget und dem Forecast wird eine Erklärung vom entsprechenden Geschäftsbereich erwartet. Dies kann dazu führen, dass die Anpassung der ProgMse hinausgezögert wird.
\newline
Obwohl die Performance eines Bühler-Projekts mittels des magischen Dreiecks - Zeit, Kosten und Qualität - beurteilt wird, hat letztendlich der Kostenaspekt aus finanzieller Perspektive die relativ gewichtigere Bedeutung als die anderen zwei Dimensionen. Deshalb wurde in Zusammenarbeit mit der Bühler AG, das folgende Erfolgskriterium festgelegt:
\begin{equation}
	\text{Deviation DB1 \%} = \text{DB1 Act \%} - \text{DB1 Bud \%}
\end{equation}
Der KPI rechnet sich realisierte Marge (DB1 Act) in \% minus budgetierte Marge (DB1 Bud) in \% und wird jeweils am Projektende respektive nach Erreichung des MS11 (Kapitel \ref{zweieins}) kalkuliert. Die Kosten der Garantieperiode werden nicht direkt dem Projekt belastet sondern summarisch in einem anderen buchhalterischen Konto. Bei jedem Projekt wird präventiv ein Kostenpuffer im Bereich von 4\% bis 9\% einkalkuliert, der nach dem Projektende (MS11) bei einer Nullbeanspruchung im Projektergebnis realisiert wird.
\newline Die nachfolgende Abbildung \ref{Einflussfaktoren} zeigt die ursprünglich 93 Determinanten eingeteilt in sechs Kategorien. Gewisse Faktoren könnten ihrer Natur nach auch in eine andere Gruppe gegliedert werden. In der Folge wird die Bedeutung und Relevanz der Variablen erklärt.
%%%%%%%%%%%%%%%%%%%%%%%% WAraanty shit!
\begin{figure}[H]
	\centering
	\includegraphics[width=90mm]{Model.jpg}
	\caption{Einflussfaktoren
	\label{Einflussfaktoren}}
\end{figure}
\paragraph{Rahmenbedingungen:} In dieser Kategorie sind kategoriale Variablen zusammengefasst, die den eindeutigen Rahmen eines Projekts festlegen. Dazu gehören die Region (Region) respektive das Land (EquLoc), in welchem die Anlage gebaut wird, der Kunde (CuNo) und Geschäftsbereich (BA, BU und MS). Ausserdem zählt die relative Wichtigkeit eines Projekts (BAImportPr, BUImportPr und MSImportPr) für den entsprechenden Geschäftsbereich ebenso zu den Rahmenbedingungen. Die zugrundeliegende Hypothese unterstellt, dass bestimmte Kombinationen der Charakteristiken den Projekterfolg begünstigen. Kunden beispielsweise lassen sich bezügliche der individuellen Anlagespezifikationen, ihrer Bonität oder Kultur unterscheiden. Die Region in welcher die Anlage gebaut werden soll, birgt differenzierbare Risiken im Bereich der Politik, Wirtschaftsentwicklung oder länderspezifischer Handelsregelungen. Der Geschäftsbereich kann als eindeutiges Diversifikationskriterium der Anlage gewertet werden. Obwohl der Projektmanagementleitfaden intern universelle Gültigkeit hat, können während der Projektlaufzeit verschiedene Herausforderungen in Abhängigkeit der jeweiligen Anlage auftreten. Zudem kann davon ausgegangen werden, dass die Teamarbeit und Teamkultur pro Geschäftsbereich und -einheit verschieden sind und den Projekterfolg unterschiedlich beeinflussen. Die Wichtigkeit eines Projekts, das Umsatzbudget des Projekts im Verhältnis zum Median des Umsatzbudgets aller laufenden Projekte, kann als Indikator zur Konzentration von Ressourcen bei der Projektabwicklung interpretiert werden. Demzufolge müssten bedeutendere Projekte, die auch einen erheblichen Einfluss auf das Geschäftsbereichsergebnis haben, mehr Aufmerksamkeit in Bezug auf Risikominimierung erhalten.
\paragraph{Sales \& Quotation (SQ):} Der Verkaufsprozess geht unmittelbar in die Projektabwicklung über, weshalb die vorgelagerten Entscheidungen direkt oder indirekt den Projekterfolg beeinflussen können. Zum Beispiel beeinflussen die Qualität der Offerte sowie die vertraglichen Vereinbarungen die Rahmenbedingungen für die Projektabwicklung. Die Offertstellung und vorgängige Risikoanalysen des Projekts liegen im Aufgaben- und Verantwortungsbereich des Area Managers (AM und AMNo). Es wird davon ausgegangen, dass erfahrenere (AMAge) und langjährige (AMTen) Verkaufsmanager über mehr Kenntnisse zu den Projekten allgemein, als auch deren Risiken und dem internen Prozess verfügen und deshalb 'erfolgreichere' Projekte verkaufen. Die Incentivierung und Performancemessung der Verkaufsmanager erfolgt über das Auftragsvolumen des Geschäftsbereichs und der Region. Die Abweichung von den Budgetvorgaben auf regionaler Ebene und der Geschäftsbereichsstufe (BUORBudGap und RegioORBudGap) zum Zeitpunkt des Projektabschlusses kann den Zielerreichungsdruck vor allem zum Jahresende erhöhen. Die Schnittstelle zwischen Sales \& Quotation und dem Fulfillment ist für den Projekterfolg von zentraler Bedeutung, weshalb eine Zeitverzögerung zwischen der Auftragsfreigabe (ORDate) und dem Projektbeginn (PrStartDate) als Indiz für Komplikationen, Unklarheiten und Unsicherheiten interpretiert werden kann. 
\paragraph{Fulfillment (FF): } In dieser Kategorie werden sämtliche Faktoren im Zusammenhang mit dem Projektmanager (PM und PMNo), dem Forecastmanagement (FC-Management) und der Unternehmensverantwortung subsumiert, da sie den Projektabwicklungsprozess tangieren. Der Betriebszugehörigkeit der Projektmanager (PMTen) sowie dessen Erfahrungsschatz (PMAge) sind stellvertretende Variablen für die Kenntnisse der internen Prozesse und das vorhandene Wissen in Bezug auf den Beruf. Bei Unstimmigkeiten zwischen dem Kunden und dem Projektmanager, kann letzterer ersetzt werden (PMChange). Je nach Status des Projekts und Zeitpunkt des Wechsels können nicht alle Differenzen durch den neuen Projektmanager kompensiert werden, weshalb ein Austausch als Indiz für nicht-erfolgreiche Projekte betrachtet wird. In sehr seltenen Fällen muss die Funktion des Projektmanagers sogar mehrmals neu besetzt werden (NoPM), was den positiven Ausgang eines Projekts beeinträchtigen kann.
\newline Die organisatorische Verantwortung für das ganze Projekt (LeadSASPr) und den Abwicklungsprozess (LeadSASFF), kann bei einer Gesellschaft oder zwei verschiedenen Gesellschaften (LeadSAS.PrFF) angesiedelt sein. Die zusätzliche Schnittstelle erhöht den Komplexitätsgrad eines Projekts und kann deshalb nachteilig auf den Projekterfolg wirken. Die Zusammenarbeit sowohl zwischen den Gesellschaften als auch innerhalb der Unternehmen kann sich voneinander unterscheiden, weshalb einige Gesellschaften wahrscheinlich mehr Erfolg im Projektmanagement aufweisen.
\newline Das  Forecastmanagement liegt im Verantwortungsbereich des Projektmanagers und bezieht sich auf die Prognose des Projektumsatzes, der -kosten sowie der -marge, welche monatlich geprüft und entsprechend angepasst werden muss. Das frühzeitige Erkennen von drohenden Mehrkosten kann deren Verminderung oder Vermeidung begünstigen. Deshalb wurde pro Projektphase, Mechnical Supply \gls{abk:MS}, Mechnical Engineering \gls{abk:ME}, Plant \&Automation \gls{abk:PA} und Installation \gls{abk:IS} erhoben, ob der Forecast angepasst wurde. Dabei wurde zwischen 'nur Mehrkosten' und 'Mehrkosten inklusive Umsatzerhöhung' unterschieden, (CostFCadj). Zudem wurde die Anzahl Monate zwischen dem Projektabschluss (MS11) und der negativsten FC-Anpassung (CostmostnegFCadj) gemessen. Die Abweichungen von den Vorgaben in Bezug auf Zeit und Kosten wird systemisch automatisch berechnet und  durch das dreistufige Ampelsystem des Bühler Projektmanagement-Cockpit (BPM-Cockpit) reflektiert. Obwohl intern vorgeschrieben wird, dass jede mögliche Veränderung in der monatlichen Prognose unverzüglich einfliessen muss, wird aufgrund des Begründungszwangs bei hohen Abweichungen versucht, die Kommunikation der negativen Veränderung so lange wie möglich hinauszuzögern.  % Erklärung wieso HOM
Deshalb wurde die Periode zwischen dem erstmaligen gelben respektive roten Status und dem Projektbeginn HOM gemessen und ins Verhältnis zur vereinbarten Projektzeit (HOMRed/YellowCost/Time/Quality) gesetzt.  %Auf diese Weise kann herausgefunden werden..
\paragraph{Kosten:} Die monetären Aspekte eines Projekts umfassen Umsatz (TO), Kosten (Cost), Marge (DB1), Budget (Bud) und realisierte Zahlen (Act), deren Vergleich und die monetären Abweichungen zwischen dem letzten Forecast und den Istzahlen. Ein höheres Umsatzbudget (TOBud) wird mit komplexeren und umfangreicheren Projekten, die ein höheres Mass an Planung, Ressourcen sowie Betreuung erfordern, assoziiert. Ausserdem ist ihr finanzieller Einfluss auf das Geschäftsbereich- bzw. Regionenergebnis von besonderer Wichtigkeit. Die Kostenabweichungen (CostActBud) pro Projektphase in absoluten und relativen Grössen sollen Aufschluss über die Verlustbereiche geben. Die Zusammensetzung der budgetierten Kosten pro Projektphase in Relation zum Umsatz (BudMS/ME/PA/ISTot) kann zudem Aufschluss über die Projektart geben, da beispielsweise ein hoher Engineering-Anteil erwartungsgemäss eher mit Mehrkosten einhergeht als ein hoher MS-Anteil. Es wurden zusätzlich die Kosten aus Nachlieferungen infolge Nichteinhaltung des vereinbarten Liefertermins in das Modell mit einbezogen, da hypothetisch vermutet wird, dass dieser Kostenanteil in Bezug zum Umsatzbudget bei nicht-erfolgreichen Projekten höher sein muss. Im Zusammenhang mit der Projektmarge liegt der Fokus vor allem auf Projekten mit einem Budget nahe des intern festgelegten Grenzwertes von 23\%. Denn sämtliche Projekte, deren budgetierte Projektmarge unter diesem Wert liegt, bedarf einer Zustimmung zur Eingehung dieses Risikos sprich der Projektdurchführung durch die nächst höhere Managementstufe. Davon ausgehend, dass aufgrund des Budgetdrucks versucht wird diesen Genehmigungsprozess zu umgehen, wird vermutete, dass risikoreichere Projekte verkauft werden, die letztendlich eher mit negativer Performance einhergehen. Die Abweichung der realisierten Kosten vom letzten Kostenforecast (DeltaLastFCAct) pro Projektphase soll zudem erfassen, wie hoch die Mehrkosten kurz vor dem Ende der Projektlaufzeit waren, um Rückschlüsse auf den Zeitpunkt der Herausforderungen machen zu können. %%%%%%%%%%%%%%%%%%%%%%%%%%%%%%%%%%%Prüfen und Bühler besprechen
\paragraph{Zeit: } In dieser Gruppe sind alle Variablen, die in Verbindung mit der Projektlaufzeit und den Milestones stehen zusammengefasst. Die Einhaltung des Liefertermins sowie die Lokalisierung von Zeitverzögerungen anhand ausgewählter Milestones sind hierbei von grossem Interesse. Dazu wurde die Zeitdifferenz (PrTimeDelay) zwischen der vereinbarten (PrTimeBaseline) und der erreichten Projektlaufzeit(PrTimeAct) für das ganze Projekte und die folgenden Milestones gemessen: MS2 Concept approved, MS5 Point of no return, MS8 Documented, MS10 Takeover und MS11 Project.
\paragraph{Komplexität:} Die Komplexität eines Projekts kann unterschiedliche Dimensionen betreffen, so zum Beispiel können die technische Anforderung an die Anlage, die Anzahl involvierter Parteien, die Zusammenarbeit mit externen Partnern sowie neuartige Prozesse den Komplexitätsgrad eines Projekts erhöhen. Zur Abbildung der Komplexität wurden als sogenannte Proxyvariablen die Anzahl involvierter Zulieferer respektive Partner pro Projektphase (NoSupplSAS) und Verträge (NoContr) pro Projekt erhoben. Gewisse Projekte werden im Konsortium (ConPart), das heisst mit einem externen Partner abgewickelt, da dieser beispielsweise mehr oder ergänzende Expertise in Bezug auf die Anlage hat.
%%
%%part Frühwarnsystem im PM: theoretischer Rahmen für Diskussion
%Frühwarnsystem: Definition, Anwendung hauptsälich, Grund: Anwendung im PM, Anpassung Begrifflichkeiten
%Früherkennung: Definition (Watch Redundanzen), Methoden: Generationen, Fokus der Forschung 3. Generation
%Früherkennung: Einführung und Verwendung Begriff Frühwarnsignale, Methodische Ansätze summarisch,
%Früherkennung: Getestet und Bewährte Methoden
%
\section{Frühwarnsystem im Projektmanagement}
Die Notwendigkeit von Früherkennung im Projektmanagement lässt mit dem Auftauchen neuer Herausforderungen während der Projektlaufzeit sowie der sich kontinuierlich verändernden Projektumwelt begründen. Die Anforderungen an die Flexibilität im Projektmanagement und die Fähigkeit zukünftige Ereignisse "vorauszusehen" sind gestiegen. Früherkennung hat zum Ziel aufkommende Gefahren und Chancen frühzeitig zu identifizieren, so dass rechtzeitig entsprechende Massnahmen eingeleitet werden können. In gegenwärtigen Managementsystem der Bühler AG wird ein sich verschlechternder Projektstatus erst bei fortgeschrittener Projektlaufzeit ersichtlich, weshalb beabsichtigt wird mittels der Implementierung eines Frühwarnsystems, dieser Situation entgegenzuwirken. Dadurch sollen mehr Projekte erfolgreich abgeschlossen und das die Marge des Anlagegeschäfts verbessert werden. Anschliessend folgt die Erläuterung der drei Generationen von Frühwarnsystem und sowie deren die Anforderungen. Danach werden Ideen und Ansätze für Früherkennung bei der Bühler AG diskutiert.
\subsection{Frühwarnsysteme und Frühwarnindikatoren}\label{viereins}
Die Voraussetzungen zur Anwendung eines Frühwarnsystems sind gemäss Jacobs, Riegler \& Matter (2012, S. 23), die Möglichkeit ambivalenter Ausgänge, die Gefährdung dominanter Ziele sowie der prozessuale Ablauf einer drohenden "Krise". Obwohl diese Kriterien im Zusammenhang mit Unternehmenskrisen ausgearbeitet wurden, können sie für das Projektmanagement ebenso angewendet werden. Grundsätzlich werden folgende drei Generationen unterschieden.
\begin{description}
	\item[1. Generation:] Die erste Generation orientiert sich an traditionellen Kennzahlen- und Hochrechnungen. Vergleiche der Prognosen mit Sollwerten werden dann als Frühwarnsignale verwendet.
	\item[2. Generation:] Die zweite Generation basiert auf der Anwendung von Indikatoren und Prognosen auf Basis von Faktorenmodelle, die statistisch signifikante Zusammenhänge aufweisen mit dem gewählten Indikator haben.
	\item[3. Generation:] Die dritte Generation stützt sich auf die Theorie der schwachen Signale, die intuitiver und unstrukturierter Natur sind. 
\end{description}
Die erste und zweite Generation haben den Nachteil, dass die Aussagekraft der vergangenheits- und gegenwartsorientierten Datengrundlage relativ beschränkt ist und anderseits durch die Selektion relevante Faktoren unbeachtet bleiben. Aus diesen Gründen sind Diskontinuitäten aus Basis der Hochrechnungen nur schwer erkennbar und an die zugrunde liegenden mathematischen respektive statistischen Modelle gebunden   (Jacobs, Riniger \& Matter, 2012, S. 26- 28). Die dritte Generation versucht die Schwächen der vorangehenden Frühwarnsystem zu kompensieren. Gemäss Ansoff (1967, S.129ff), kündigen sich Diskontinuitäten nicht plötzlich sonder relativ früh durch sogenannte schwache Signale, die als frühe Hinweise zu bevorstehenden einflussreichen Ereignisse zu verstehen sind (Ansoff in Haji-Kazemi \& Anderson, 2013). Diesem Ansatz liegt die Prämisse zugrunde, dass Unternehmen die wahrscheinlichsten Faktoren, welche das Scheitern des Projekts begünstigen sowie die Anzeichen eines bevorstehenden Scheiterns, bereits kennen. Allerdings wird erst in nachgelagerten Projekt Assessments dieses Bewusstsein gefördert. Unter diesem Aspekt erscheint es relativ unverständlich, weshalb diese ignoriert wurden.
\newline
Die Ansätze, wie Projektmanager solche Signale erkennen und zu ihren Gunsten interpretieren können sind vielfältig, wie die Tabelle \ref{tab:Ans} aufzeigt.
%Tabelle mit möglichen Ansätzen
\begin{table}[H]
	\centering
	\caption{Ansätze zur Identifikation von Frühwarnsignalen
		\newline in Anlehnung an Haji-Kazemi, Andersen \& Krane (2013)}\label{tab:Ans}	
	\begin{tabular}{l|l}
		Risikomanagement & Past Project Consultation\\
		Earned Value Management & Cause-Effect-Analyse\\
		Projekt Assessment Ansätze & Gut feelings \\
		Performance Management & Interface Analysis\\
		Stakeholder Analyse & Project Analysis \\
		Maturity Assessment & Project Surrounding Analysis\\
	\end{tabular}		
\end{table}
Die Wahl der Methode ist abhängig von der Projektart und des Unternehmens. Haji-Kazemi, Andersen \& Krane (2013, S. 59) haben mittels mehrere Fallstudien eruiert, dass das Projekt Assessment sowie 'Gut feelings' in der Praxis die bewährt haben, wobei Experten die Überzeugung haben, dass Frühwarnsignale qualitativer Natur sind und eher durch Intuition Erfahrungswissen entdeckt werden. Klakegg, et.al, (2010) haben mittels formalen Assessments die Anzahl fehlender Informationen, fehlende Beurteilungen und Dokumentationen sowie unklare Anforderung der Meilenstein und verspätete Bericht als mögliche Frühwarnsignale ergründet. Missverständnisse bezüglich der Bedürfnisse, sowie mangelnde Offenheit der Unternehmenskultur und Kommunikationsbereitschaft zwischen den Projektteilnehmer, sowie angespannte Projektatomsphäre wurden in Fallstudien mittels der "Gut feelings"-Ansätzen als wichtige Früherkennungshinweise erkannt. Diese Erkenntnisse bestätigen unter anderem die Ansichten der befragten Experten aus anderen Studien. Denn bei der Evaluation von Assessments während der Projektlaufzeit können zwar wichtige Hinweise für nachfolgende Projekte ausgearbeitet, die aber im aktuellen Projekt nicht mehr berücksichtigt werden können. Während diese Indikatoren qualitativen Charakter haben und eher schwierig zu messen sind, konnten Haji-Kazem \& Anderson (2013) im Rahmen des Performance Management die Überwachung der Schnittstellenmassnahmen, die Mitarbeiterzufriedenheit und Risikoüberwachung als effiziente Quellen von Frühwarnsignalen erheben. Ihre Gemeinsamkeiten sind die quantitative und kontinuierliche Messbarkeit sowie die Funktion als sogenannte "leading" Indikatoren, die es ermöglichen in der Ursachen-Wirkungs-kette möglichst früh Hinweise zu möglichen Gefahren zu erhalten. "Lagging" Faktoren liefern dementsprechend eher spät oder zu spät Signale zu möglichen Risiken. Die Herausforderung bei der Identifikation von Frühwarnsignalen mit dem Performancemanagement-Ansatz ist die Selektion der zu überwachenden Faktoren. Ausserdem kann der Einfluss von Drittvariablen unentdeckt respektive unterschätzt werden.
% Faktoren von Klakegg und Kommentieren
% Faktoren von Performance Ansatz, 
% Leading Lagging Faktor,  
% wan ist früh, leading not lagging indicators, IdentifyAc says that performance only measures lagging 38
% possible early warning signs culture, lack of an outsiders perspective on the project, anchoring in the permanent organization, lack of consistency between stakeholders ambition and certain organizations. gut felt signs: detection of unrealism, lack of clarity, misalignment btw qualitative and quantitaive risk analysis 42
%no early signs in later stages, change...not used 43
%problems: difficult to stop projects despite EWS
%need for formalized proces for finding ealry warning signs, outside of the box thinking


\newpage
%Methodenteil
% !TEX root = MA.tex
\section{Analye der Erfolgsfaktoren des Bühler Projektmanagements}\label{drei}
In diesem Kapitel wird zuerst das analytische Vorgehen erläutert und anschliessend die Ergebnisse präsentiert sowie kritisch gewürdigt. In vergangene Studien wurden die Erfolgsfaktoren von Projekten mittels der statistischen Auswertung von Einschätzungen zu deren Relevanz für den Projekterfolg erhoben. Die nachfolgende Analyse unterscheidet sich insofern, da versucht wird auf Basis der unternehmensspezifische Daten  Charakteristiken nicht-erfolgreicher Projekte zu ergründen.
\subsection{Analytisches Vorgehen}
Die untersuchte Stichprobe enthält alle Projekte, die im Zeitraum zwischen 2013 und 2105 abgeschlossen wurden. Die eindeutigen Abgrenzungskriterien bilden der Projektstatus und das Datum des Project Closure (MS11). Zuerst wurden alle Projekte mit einem MS11-Datum zwischen dem 1.1.2013 und dem 31.12.2015 eingegrenzt. Anschliessend wurde mittles dem Projektstatus sichergestellt, dass das Projekt auch aus finanzieller Sicht abgeschlossen war. Denn gewisse Projekte sind zwar operativ bereits beendet, gelten aber aufgrund ausstehender Rechnungen aus finanzieller Sicht als 'nicht abgeschlossen'.
\newline Nach der ersten Datenexploration und Prüfung der Annahmen für lineare statistische Modelle, wurde festgestellt, dass die ursprünglich geplante Methodenwahl nicht angewendet werden konnte. Denn die unabhängigen Daten hatten geringe bis keine Korrelation mit der abhängigen Variable sprich dem Erfolgskriterium. Die lineare Variablentransformationen und andere Methoden, um eine Verteilungskurve zu simulieren führten nur zu kleineren Verbesserung. Dieser Umstand und die Tatsache, dass Erfolgsfaktoren bereits sehr gut erforscht wurden, hat die Entscheidung auf Inferenzstatistik zu verzichten bestärkt. Die nachfolgende Analyse ist deshalb deskriptiver Natur und hat ausserdem das Ziel, die finanziellen Einbussen von sogenannten nicht erfolgreichen Projekten zu untersuchen. Die Aussagekraft der Ergebnisse wird dadurch so eingeschränkt, dass da keine Rückschlüsse auf die Grundgesamtheit (sämtliche Projekte der Bühler AG) gemacht werden können.  Die Ergebnisse haben nur in Bezug auf die die untersuchte Stichprobe Gültigkeit. Es ist jedoch denkbar, auf Basis der Ergebnisse neue Hypothesen zu formulieren, welche mittels anderer, geeigneter statistischer Methoden geprüft werden können. Die erstmalige Auswertung der Projektdaten kann zudem Erkenntnisse zu möglichen Charakteristiken nicht-erfolgreicher Projekte liefern.
\newline Das Erfolgskriterium (DB1BudDev) wurde in Zusammenarbeit mit der Bühler AG festgelegt. Aus finanzieller und interne Perspektive ist die Abweichung der relativen Projektmarge (DB1Act) vom Budget (DB1Bud) von zentraler Bedeutung. Denn sowohl die Finanzziele wie auch die Incentivierung der Projekt- und Verkaufsmanager sowie der Geschäftsbereichsleitung basieren auf DB1 und den entsprechenden Budgetvorgaben. Die relative Projektmarge errechnet sich aus Umsatz minus Kosten in Relation zum Umsatz. Anhand der Differenz zwischen Act und Bud wird der Erfolg ($Differenz \geq 0$) respektive Nicht-Erfolg ($Differenz < 0$) von Projekten gemessen. Der DB1BudDev wurde zu Analysezwecken in eine binäre Variable (Success) transformiert. Daraus folgt, dass alle positiven (negativen) Differenzen als erfolgreiche (nicht-erfolgreiche) Projekte betrachtet werden. Im Folgenden werden erfolgreiche Projekte und Success-Projekte sowie nicht-erfolgreiche Projekte und Fail-Projekte als Synonyme verwendet. Obwohl retrospektive Erkenntnisse und Erfahrungen aufgrund des Projekts einen Gewinn für das Unternehmen darstellen können, wird diesem Aspekt in dieser Analyse nicht Rechnung getragen. 
\newline\newline $Erfolgsquote = Anzahl erfolgreicher Projekte/Anzahl nicht-erfolgreicher Projekte$
\newline\newline\textbf{Datenaufbereitung:} Der Rohdatensatz enthält sämtliche Daten zu den Faktoren der untersuchten Projekte (Stichprobe). Er setzt sich aus drei Datensätzen zusammen, die separat aus den Bühler-System extrahiert wurden. Das Alter und die Betriebszugehörigkeit der Projekt- und Areamanager mussten korrigiert werden, da der ursprüngliche Datensatz die Unterscheidung zwischen fehlenden Werten und Nullwerten nicht zu liess.
\newline\newline $Stichprobenumfang N = 1471$ und $Anzahl Faktoren i = 93$.
\newline\newline
Es wurden alle vorhandenen, unplausiblen und Berechnungsfaktoren vom Datensatz entfernt. Anschliessend wurde die Anzahl fehlender Daten pro Faktor ausgewertet und zusätzlich alle Determinanten mit mehr als 300 fehlender Datensätze von der weiteren Analyse ausgeschlossen. Zusätzlich bleibt die Variablen AMNo unberücksichtigt, da durch den Ausschluss der verbundenen Variablen (AMTen und AMAge) wenig Informationsgewinn erwartet wird. Ausserdem mussten alle Variablen, welche die Zeitdifferenz zwischen dem letzten Kostenforecast und dem Projektende messen, aufgrund fragwürdiger Plausibilität und Korrektheit der Daten von der Analyse ausgeschlossen werden. Mittels diesem Vorgehen kann der Datenverlust aufgrund fehlender Daten in Grenzen gehalten werden. Der neue Stichprobenumfang beträgt $N = 1076$ und die Anzahl Faktoren $i = 71$
\begin{table}[htbp]
	\centering
	\caption{Anzahl NA's pro Variable (Ausschnitt)}
	\begin{tabular}{lr}
		\textbf{Variale} & \multicolumn{1}{l}{\textbf{Anzahl NA}} \\
		PrTimeDelayMS5 & 538 \\
		AMAge2 & 444 \\
		AMTen2 & 444 \\
	\end{tabular}%
	\label{tab:addlabel}%
\end{table}%
Im Anschluss wurden die Datensätze auf ihre Plausibilität getestet und Ausreisser entfernt. Die Plausibilitätsüberlegungen basieren auf der logischen Interpretation und Herleitung der Indikatoren. Die Tabellen mit den Begründungen der unplausiblen Werte und Ausreisser befinden sich im Anhang. Die Outliers wurden mit Hilfe von Boxplots, Histogramme und der 'Interquartile Ranges' (IQR) der numerischen Variablen identifiziert. Zur quantitativen Bestimmung der Ausreisser wurde folgendes Entscheidungskalkül angewendet:
\newline\newline
\begin{centering}
	$ Werte < Q1 - 1.5 * IQR$ und $ Werte > Q3 + 1-5 * IQR$
\end{centering}
\newline\newline
\begin{centering}
	$ Werte < Q1 - 3 * IQR$ und $ Werte > Q3 + 3 * IQR$
\end{centering}
\newline\newline
Je nach Zweck der Analyse und untersuchten Objekten sind Ausreisser unterschiedlich einzustufen. Die Geschäftsbereiche der Bühler AG verkaufen unterschiedliche Anlangen, weshalb die Datenbereiche der Faktoren stark variieren können. Die realisierte Projektmarge (DB1Act), wurde auf die Werte des doppelten IQR berichtigt, da extreme negative Margen auf sogenannte Crash-Projects schliessen lassen, welche bereits mittels internem Audit untersucht wurden und die Stichprobenergebnisse unnötige verzerren können. Extreme positive DB1Act sind bei einer durchschnittlichen Projektmarge von ca. 30\% relativ unwahrscheinlich und lassen Zweifel zur Richtigkeit der Kostenverbuchung zu. Bei den relativen Kostenabweichungen für PA und IS wurden jeweils einzelne Extremalwerte nur dann entfernt, wenn kein entsprechendes Budget geplant wurde. Denn es wurde davon ausgegangen, dass die Budgetierung der Projektkosten nicht korrekt verlaufen ist, was letztendlich zu extremalen relativen Kostenabweichung geführt hat. Es wurden keine weiteren Ausreisser eliminiert, selbst wenn einige Werte ausserhalb des Entscheidungskalküls lagen. Nach der Datenbereinigung umfasst die zu untersuchende Stichprobe $N = 966$ Projekte und die Anzahl verfügbarer Faktoren entspricht $ i = 71$. 
\newline\newline
\textbf{Zusätzliche Variablen:} Nach dem Datenbereinigungsprozess wurden zu analytischen Zwecken kategoriale Variablen auf Basis der vorhandene Daten erhoben. Die nachfolgende Tabelle zeigt sämtliche verbleibende (s. Kapitel 2 für alle Faktoren) inklusive der hinzugefügten Faktoren nach ihrer Kategorie strukturiert. Sämtliche Berechnungsformeln sowie die Interpretationen der Faktoren sind im Anhang enthalten.
\begin{table}[htbp]
	\centering
	\caption{Übersicht der Faktoren}
	\begin{tabular}{lrrr}
		\textbf{Erfolgskriterium} &       &       &  \\
		DB1BudDev &       &       &  \\
		Success &       &       &  \\
		\textbf{Rahmenbedingungen} & \multicolumn{1}{l}{\textbf{Zeitmanagement}} & \multicolumn{1}{l}{\textbf{Sales \& Quoatation}} & \multicolumn{1}{l}{\textbf{Komplexität}} \\
		CuNo  & \multicolumn{1}{l}{PrTimeBase} & \multicolumn{1}{l}{BUORBudGapAbs} & \multicolumn{1}{l}{ConPart} \\
		EquLoc & \multicolumn{1}{l}{PrTimeAct} & \multicolumn{1}{l}{BUORBudGapRel} & \multicolumn{1}{l}{NoSupplSAS} \\
		BA    & \multicolumn{1}{l}{PrTimeDelay} & \multicolumn{1}{l}{RegiORBudGapAbs} & \multicolumn{1}{l}{NoSupplSASMS} \\
		BU    & \multicolumn{1}{l}{PrTimeDelayMS2} & \multicolumn{1}{l}{RegiORBudGapRel} & \multicolumn{1}{l}{NoSupplSASME} \\
		MS    & \multicolumn{1}{l}{PrTimeDelayMS8} &       & \multicolumn{1}{l}{NoSupplSASPA} \\
		Region & \multicolumn{1}{l}{PrTimeDelayMS10} &       & \multicolumn{1}{l}{NoSupplSASIS} \\
		& \multicolumn{1}{l}{PrTimeDelayMS11} &       & \multicolumn{1}{l}{NoContr} \\
		& \multicolumn{1}{l}{Delay} &       &  \\
		\textbf{Kostenmanagement} &       & \multicolumn{1}{l}{\textbf{Fulfillment}} &  \\
		TOBud & \multicolumn{1}{l}{CostActBudISRel} & \multicolumn{1}{l}{PMNo} & \multicolumn{1}{l}{CostFCadjPA} \\
		BudMSTot & \multicolumn{1}{l}{DeltaLastFCAct} & \multicolumn{1}{l}{PMAge2} & \multicolumn{1}{l}{CostFCadjIS} \\
		BudMETot & \multicolumn{1}{l}{DeltaLastFCActMS} & \multicolumn{1}{l}{PMTen2} & \multicolumn{1}{l}{HOMYellCost} \\
		BudPATot & \multicolumn{1}{l}{DeltaLastFCActME} & \multicolumn{1}{l}{PMChange} & \multicolumn{1}{l}{HOMYellQual} \\
		BudISTot & \multicolumn{1}{l}{DeltaLastFCActPA} & \multicolumn{1}{l}{NoPM} & \multicolumn{1}{l}{HOMYellTime} \\
		DB1Bud & \multicolumn{1}{l}{DeltaLastFCActIS} & \multicolumn{1}{l}{LeadSASPr} & \multicolumn{1}{l}{HOMRedCost} \\
		DB1Act & \multicolumn{1}{l}{TOAct} & \multicolumn{1}{l}{LeadSAS.PrFF} & \multicolumn{1}{l}{HOMRedQual} \\
		CostActBudMSabs & \multicolumn{1}{l}{TOBudDevabs} & \multicolumn{1}{l}{NoLeadSASFF} & \multicolumn{1}{l}{HOMRedTime} \\
		CostActBudMEabs & \multicolumn{1}{l}{CostBud} & \multicolumn{1}{l}{CostFCadj} & \multicolumn{1}{l}{PrStartDate} \\
		CostActBudPAabs & \multicolumn{1}{l}{CostAct} & \multicolumn{1}{l}{CostFCadjMS} & \multicolumn{1}{l}{Cat\_age} \\
		CostActBudISabs & \multicolumn{1}{l}{CostBudDevabs} & \multicolumn{1}{l}{CostFCadjME} &  \\
		SUCostTO & \multicolumn{1}{l}{DB1Budabs} &       &  \\
		CostActBudRel & \multicolumn{1}{l}{DB1Actabs} &       &  \\
		CostActBudMSRel & \multicolumn{1}{l}{DB1BudDevabs} &       &  \\
		CostActBudMERel & \multicolumn{1}{l}{TOBudCat} &       &  \\
		CostActBudPARel &       &       &  \\
	\end{tabular}%
	\label{tab:addlabel}%
\end{table}%
\subsection{Ergebnisse und Interpretation}
Die Ergebnisse der finanziellen Analyse und der Untersuchung der Einflussfaktoren werden getrennt dargestellt. Die untersuchte Stichprobe enthält 966 Projekte, wovon 654 erfolgreich abgeschlossen wurden.
\begin{table}[htbp]
	\centering
	\caption{Übersicht Stichprobe}
	\begin{tabular} {l|r|r}
		\textbf{Stichprobe} & \textbf{absolut} & \textbf{relativ} \\\hline
		\textbf{Total} & 966 & 100\% \\
		\textbf{Success} & 654 & 68\% \\
		\textbf{Fail} & 312 & 32\% \\
	\end{tabular}
\end{table}
\subsubsection{Finanzielle Performance Analyse}
Zur Bewertung der finanziellen Performance wurden drei verschiedene Auswertungen gemacht: Abweichung Act-Bud, Zusammensetzung der Kosten sowie eine Auswertung pro Umsatzkategorie. Das Ziel besteht darin, die den finanziellen Verlust auf Basis des zugrundeliegenden Erfolgskriterium (DB1BudDev) zu quantifizieren. 
\begin{table}[htbp]
	\centering
	\caption{Übersicht Budget [TCHF]}
	\begin{tabular}{lrrrr}
		\textbf{Erfolgskriterium} & \textbf{TO Bud} & \textbf{Cost Bud} &
		\textbf{DB1 Bud} & \textbf{DB1 Bud [\%]} \\
	SUCCESS & 1'552'450 & -1'156'598 & 395'851 & 25.5\% \\
	FAIL  & 618'013 & -465'066 & 152'947 & 24.7\% \\
	Grand Total & 2'170'463 & -1'621'664 & 548'799 & 25.3\% \\
	\end{tabular}%
\label{bud}%
\end{table}%
\begin{table}[htbp]
	\centering
	\caption{Übersicht Actuals [TCHF]}
	\begin{tabular}{lrrrr}
		\textbf{Erfolgskriterium} & \textbf{TO Act} & \textbf{Cost Act} & \textbf{DB1 Act}&
		\textbf{DB1 Act-Bud [\%]} \\
			SUCCESS & 1'560'001 & -1'041'728 & 518'273 & 33.2\% \\
			FAIL  & 631'346 & -526'600 & 104'746 & 16.6\% \\
			Grand Total & 2'191'347 & -1'568'328 & 623'018 & 28.4\% \\
	\end{tabular}
\label{act}%
\end{table}%
\begin{table}[H]
\centering
\caption{Übersicht Abweichungen [TCHF] ($Act-Bud$)}
\begin{tabular}{lrrrr}
	\textbf{Erfolgskriterium} & \textbf{TO} & \textbf{Cost} & \textbf{DB1}&
	\textbf{DB1 [\%]} \\
	SUCCESS & 7'551 & 114'870 & 122'421 & 7.7\% \\
	FAIL  & 13'333 & -61'534 & -48'202 & -8.2\% \\
	Grand Total & 20'884 & 53'336 & 74'220 & 3.1\% \\
\end{tabular}
\label{Abw}%
\end{table}%
Die Tabelle \ref{Abw} wurde mittels Tabellen \ref{act} und \ref{bud} berechnet und zeigt, dass der realisierte Umsatz höher war als budgetiert wurde. Diese Abweichung kann auf Zusatzverkäufe oder die Verrechnung allfälliger Mehrkosten an den Kunden zurückgeführt werden. Der kumulierte DB1 der Fail-Projekte lag 48 Mio. CHF (-32\%) unter dem Budget und der Success-Projekte 122 Mio. CHF über dem Budget. Die positive Abweichung der Istkosten der Success-Projekte kann mittels der realisierten Kostenreserve, die üblicherweise pro Projekt einkalkuliert wird und je nach Geschäftsbereich zwischen 4\% und 9\% Kostenreserven beträg, zurückgeführt werden. Wenn die Kostenreserve aufgebraucht wird, resultieren Mehrkosten und die Kostenabweichung wird negative. Da die Reserve in dieser Betrachtung nicht ersichtlich ist, wäre die effektive Differenz für Fail-Projekte (Success-Projekte) tiefer (höher). Die realisierte Marge über alle Success-Projekte beträgt 33\% und liegt 7.7\% über der budgetierten Marge von 25.4\%. Demgegenüber beträgt der DB1Act der Fail-Projekte 16.6\% und liegt 8.2\% unter dem DB1 Bud von 24.7\%.
\newline
Die Aufschlüsselung der Kostenabweichung zeigt, dass die Installationsphase sowohl der erfolgreichen als auch der nicht-erfolgreich Projekte mit Mehrkosten verbunden ist. Die negative Kostenabweichung der Fail-Projekte kann zu einem Drittel auf die IS- und zu einem weiteren Drittel auf die MS-Kosten zurückgeführt werden. Bei den Success-Projekten kann ein gewisser 'Verlust'-Kompensationseffekt durch die positive Kostenabweichung der MS-Kosten festgestellt werden. Die kumulierte Kostendifferenz zwischen Act und Bud der Fail-Projekte kann fast vollständig durch die Kostendifferenz der vier Kostenarten MS, ME, PA und IS erklärt werden. Der Unterschied zu 'Total Cost' ist auf nicht-Abbildung der fehlenden Kostenarten zurückzuführen. Dieser Effekt ist bei den Success-Projekten ebenfalls sichtbar und kann zu einem Teil auf die Realisation des Kostenpuffers zurückgeführt werden.
\begin{table}[H]
	\centering
	\caption{Kostenabweichung [TCHF]}
	\begin{tabular}{lrrrrrr}
		\textbf{Erfolgskriterium} & \multicolumn{1}{l}{\textbf{Total Cost}} & \multicolumn{1}{l}{\textbf{MS}} & \multicolumn{1}{l}{\textbf{ME}} & \multicolumn{1}{l}{\textbf{PA}} & \multicolumn{1}{l}{\textbf{IS}} & \multicolumn{1}{l}{\textbf{Total MS bis IS}} \\
		SUCCESS & 114'870 & 47'615 & -2'159 & -908  & -7'114 & 37'434 \\
		FAIL  & -61'534 & -20'253 & -12'721 & -7'220 & -22'053 & -62'247 \\
		Grand Total & 53'336 & 27'363 & -14'880 & -8'128 & -29'167 & -24'812 \\
	\end{tabular}%
	\label{tab:addlabel}%
\end{table}%
Als Ergänzung wurde versucht zu eruieren, von welchem Projekttyp die Margeneinbusse der Fail-Projekte stammt. Dazu wurde die Häufigkeit und die absolute DB1 Abweichung pro Umsatzkategorie berechnet. Die Auswertung zeigt, dass ein Viertel des der DB1-Abweichung auf 23 Projekte mit einem Umsatzvolumen zwischen 5 und 10 Mio. CHF und 20\% auf 7 Projekte mit einem Umsatzvolumen von mehr als 10 Mio. CHF zurückzuführen ist. Die drittgrösste Abweichung stammt von der Umsatzkategorie mit den meisten Projekten.
\begin{table}[htbp]
	\centering
	\caption{DB1-Abweichung [TCHF]}
	\begin{tabular}{lrrr}
		& \multicolumn{1}{l}{\textbf{TOBud\_Cat}} &       & \multicolumn{1}{l}{\textbf{DB1BudDevabs}} \\
		\multicolumn{1}{r}{1} & \multicolumn{1}{l}{[13.2,500)} & 54    & -1'672 \\
		\multicolumn{1}{r}{2} & \multicolumn{1}{l}{[500,1e+03)} & 87    & -6'210 \\
		\multicolumn{1}{r}{3} & \multicolumn{1}{l}{[1e+03,1.5e+03)} & 54    & -3'966 \\
		\multicolumn{1}{r}{4} & \multicolumn{1}{l}{[1.5e+03,2e+03)} & 32    & -3'822 \\
		\multicolumn{1}{r}{5} & \multicolumn{1}{l}{[2e+03,2.5e+03)} & 17    & -3'161 \\
		\multicolumn{1}{r}{6} & \multicolumn{1}{l}{[2.5e+03,3e+03)} & 12    & -1'289 \\
		\multicolumn{1}{r}{7} & \multicolumn{1}{l}{[3e+03,3.5e+03)} & 8     & -1'359 \\
		\multicolumn{1}{r}{8} & \multicolumn{1}{l}{[3.5e+03,4e+03)} & 8     & -2'107 \\
		\multicolumn{1}{r}{9} & \multicolumn{1}{l}{[4e+03,4.5e+03)} & 4     & -1'360 \\
		\multicolumn{1}{r}{10} & \multicolumn{1}{l}{[4.5e+03,5e+03)} & 6     & -1'662 \\
		\multicolumn{1}{r}{11} & \multicolumn{1}{l}{[5e+03,1e+04)} & 23    & -12'043 \\
		\multicolumn{1}{r}{12} & \multicolumn{1}{l}{[1e+04,3.42e+04)} & 7     & -9'551 \\
		\textbf{Total} &       &       & \textbf{-48'202} \\
	\end{tabular}%
	\label{tab:addlabel}%
\end{table}%
\subsubsection{Erfolgsfaktoren}
In diesem Kapitel werden die Ergebnisse pro Variablenkategorie sowie mögliche Erklärungsansätze erläutert. Mittels Histogrammen, Häufigkeitstabellen und Mittelwerten wurde versucht, die Charakteristiken vergangener Fail-Projekte zu ergründen. Die Stichprobe wurde hierfür gemäss Erfolgskriterium in zwei Datensets unterteilt. Zur Evaluation von kategorialen Variablen wurde ein weiteres Kriterium die Erfolgsquote $(Anzahl Success-Projekte) / (Anzahl Fail-Projekte)$ hinzugezogen, um beispielsweise Geschäftsbereiche oder Region untereinander vergleichen zu können.
\newline\newline\textbf{Rahmenbedingungen:} Die Analyse der Rahmenbedingungen eines Projekts geben Hinweise darauf, in welchen Geschäftsbereichen und Regionen und mit welchen Kunden nicht-erfolgreiche gemäss dem Erfolgskriterium realisiert wurden. Da die Bühler AG in einer Matrix-Organisation organsiert ist, wurde nebst den Einzelauswertungen für die Region und die Business Area, der Regionen-BA Split für die Häufigkeit der Success- und Fail-Projekte erstellt.
\begin{table}[H]
	\centering
	\caption{Erfolgsquote pro Region}
	\begin{tabular}{lrrrrrr}
		\textbf{Region} & \multicolumn{1}{l}{\textbf{Erfolgsquote}} & \multicolumn{1}{l}{\textbf{Success}} & \multicolumn{1}{l}{\textbf{Fail}} & \multicolumn{1}{l}{\textbf{Fail [\%]}} & \multicolumn{1}{l}{\textbf{Total}} & \multicolumn{1}{l}{\textbf{Total [\%]}} \\ \hline
		East\_Asia & 6.7   & 20    & 3     & 13.0\% & 23    & 2.4\% \\
		EU    & \textbf{1.7}   & 240   & 145   & 37.7\% & 385   & 39.9\% \\
		MEA\_Afr & 2.7   & 112   & 42    & 27.3\% & 154   & 15.9\% \\
		North\_Ame & \textbf{1.4}   & 54    & 38    & 41.3\% & 92    & 9.5\% \\
		SAS\_BCHI & 2.9   & 119   & 41    & 25.6\% & 160   & 16.6\% \\
		South\_Ame & 1.9   & 58    & 31    & 34.8\% & 89    & 9.2\% \\
		South\_Asia & 4.3   & 51    & 12    & 19.0\% & 63    & 6.5\% \\ \hline
		\textbf{Total} & \textbf{2.1} & \textbf{654} & \textbf{312} & \textbf{32.3\%} & \textbf{966} & \textbf{100.0\%} \\
	\end{tabular}%
	\label{freg}%
\end{table}%  
\begin{table}[H]
	\centering
	\caption{Erfolgsquote pro Geschäftsbereich}
	\begin{tabular}{lrrrrrr}
		\textbf{BA}   & \multicolumn{1}{l}{\textbf{Erfolgsquote}} & \multicolumn{1}{l}{\textbf{Success}} & \multicolumn{1}{l}{\textbf{Fail}} & \multicolumn{1}{l}{\textbf{Fail [\%]}} & \multicolumn{1}{l}{\textbf{Total}} & \multicolumn{1}{l}{\textbf{Total [\%]}} \\ \hline
		CF    & 2.8   & 118   & 42    & 26.3\% & 160   & 16.6\% \\
		DC    & 5.6   & 96    & 17    & 15.0\% & 113   & 11.7\% \\
		GD    & 2.3   & 7     & 3     & 30.0\% & 10    & 1.0\% \\
		GL    & 1.2   & 39    & 32    & 45.1\% & 71    & 7.3\% \\
		GM    & 1.9   & 226   & 122   & 35.1\% & 348   & 36.0\% \\
		LO    & 1.4   & 30    & 21    & 41.2\% & 51    & 5.3\% \\
		SR    & 5.0   & 35    & 7     & 16.7\% & 42    & 4.3\% \\
		TP    & NA      & 8     & 0     & 0.0\% & 8     & 0.8\% \\
		VN    & 1.4   & 95    & 68    & 41.7\% & 163   & 16.9\% \\\hline
		\textbf{Total } & \textbf{2.1} & \textbf{654} & \textbf{312} & \textbf{32.3\%} & \textbf{966} & \textbf{100.0\%} \\
	\end{tabular}%
	\label{fba}%
\end{table}%
Die Ergebnisse der Tabellen \ref{freg} und \ref{fba} reflektieren die Tatsache, dass Europa der grösste Absatzmarkt und GM die grösste Business Area der Bühler AG ist. Die niedrigste Erfolgsquote hat NAM als viertgrösste Region (in Abhängigkeit der Anzahl Projekte), gefolgt von Europa. Die Anzahl der Fail-Projekte in den Regionen EU, MEA und SAS\_BCHI beträgt 73\% ($(145+42+41)/312$), weshalb die Erfolgsquote von allen Projekten hauptsächlich durch diese drei Regionen bestimmt wird. Die kleinsten Regionen haben die besten Erfolgsquoten. Die Geschäftsbereichen CF, VN, GL und GM ($(42+68+32+122)/312$) umfassen zusammen 84\% aller Fail-Projekte, wobei die drei letzt genannten zugleich die niedrigsten Erfolgsquoten ausweisen. Die Anzahl untersuchter Projekte der letzten drei Jahre der Geschäftsbereiche CF und VN ist faktisch identisch, allerdings weist VN eine viel tiefere Erfolgsquote aus als CF.
\begin{table}[H]
	\centering
	\caption{Erfolgsquote pro Geschäftseinheit}
	\begin{tabular}{llrrrrr}
		\textbf{BA} & \textbf{BU} & \multicolumn{1}{l}{\textbf{Erfolgsquote}} & \multicolumn{1}{l}{\textbf{Success}} & \multicolumn{1}{l}{\textbf{Fail}} & \multicolumn{1}{l}{\textbf{Fail [\%]}} & \multicolumn{1}{l}{\textbf{Total}} \\\hline
		GL    & GC    & NA    & 1     & 0     & 0.0\% & 1 \\
		GL    & GS    & 1.2   & 36    & 29    & 44.6\% & 65 \\
		GL    & MT    & 0.7   & 2     & 3     & 60.0\% & 5 \\\hline
		\textbf{GL} &  & \textbf{1.2} & \textbf{39} &\textbf{32} & \textbf{45.1\%} & \textbf{71}\\
		      &       &       &       &       &        &   \\
		GM    & BA    & 1.5   & 17    & 11    & 39.3\% & 28 \\
		GM    & BR    & 0.9   & 12    & 14    & 53.8\% & 26 \\
		GM    & IM    & 2.1   & 185   & 87    & 32.0\% & 272 \\
		GM    & SM    & 1.2   & 12    & 10    & 45.5\% & 22 \\\hline
		\textbf{GM} &  & \textbf{1.9} & \textbf{226} &\textbf{122} & \textbf{35.1\%} & \textbf{348}\\
		      &       &       &       &       &        &   \\
		VN    & AG    & 0.8   & 11    & 14    & 56.0\% & 25 \\
		VN    & FE    & 1.2   & 27    & 22    & 44.9\% & 49 \\
		VN    & NU    & 1.3   & 27    & 21    & 43.8\% & 48 \\
		VN    & OL    & 2.0   & 6     & 3     & 33.3\% & 9 \\
		VN    & PN    & 3.0   & 24    & 8     & 25.0\% & 32 \\\hline
		\textbf{VN} &  & \textbf{1.4} & \textbf{95} &\textbf{68} & \textbf{41.7\%} & \textbf{163}\\
	\end{tabular}%
	\label{fbabu}%
\end{table}%
Bei der Auswertung der Geschäftseinheiten (s. Tabelle \ref{fbabu}) für GL, GM und VN, konnte festgestellt werden, dass die kleineren BU's von GM eine verhältnismässig tiefe Erfolgsquote hatten. Dennoch wird das Verhältnis zwischen erfolgreichen und nicht-erfolgreichen Projekten fast ausschliesslich durch IM, die grösste BU in GM bestimmt. In der Business Area VN, sind die Erfolgsquoten mit Ausnahme von PN und OL grundsätzlich tief. Die Business Unit Grain Storage determiniert die Geschäftsbereichserfolgsquote von GL.
\begin{table}[H]
	\centering
	\caption{Ausschnitt Häufigkeitsverteilung Regionen-BA Split}
	\begin{tabular}{llrrrrr}
		\textbf{Region} & \textbf{BA}    & \multicolumn{1}{l}{\textbf{Erfolgsquote}} & \multicolumn{1}{l}{\textbf{Success}} & \multicolumn{1}{l}{\textbf{Fail}} & \multicolumn{1}{l}{\textbf{Fail [\%]}} & \multicolumn{1}{l}{\textbf{Total}} \\\hline
		EU    & CF    & 1.9   & 58    & 31    & 34.8\% & 89 \\
		EU    & DC    & 5.0   & 45    & 9     & 16.7\% & 54 \\
		EU    & GD    & NA    & 2     & 0     & 0.0\% & 2 \\
		EU    & \textbf{GL}    & 1.0   & 24    & 23    & 48.9\% & 47 \\
		EU    & \textbf{GM}  & 1.2   & 58    & 50    & 46.3\% & 108 \\
		EU    & LO    & 2.5   & 10    & 4     & 28.6\% & 14 \\
		EU    & SR    & 2.5   & 5     & 2     & 28.6\% & 7 \\
		EU    & \textbf{VN}     & 1.5   & 38    & 26    & 40.6\% & 64 \\\hline
		North\_Ame & CF    & 2.0   & 10    & 5     & 33.3\% & 15 \\
		North\_Ame & DC    & 1.0   & 2     & 2     & 50.0\% & 4 \\
		North\_Ame & GL    & 1.0   & 1     & 1     & 50.0\% & 2 \\
		North\_Ame & \textbf{GM}   & 1.5   & 24    & 16    & 40.0\% & 40 \\
		North\_Ame & LO    & 4.0   & 4     & 1     & 20.0\% & 5 \\
		North\_Ame & SR    & 1.0   & 1     & 1     & 50.0\% & 2 \\
		North\_Ame & \textbf{VN}  & 1.0   & 12    & 12    & 50.0\% & 24 \\
	\end{tabular}%
	\label{fregba}%
\end{table}%
Im Regionen-BA Split der Tabelle \ref{fregba} sind für diejenigen Regionen mit den niedrigsten Erfolgsquoten, EU und NAM, sind jene BA's mit den niedrigsten Erfolgsquoten zu finden. Die Kombination EU-GM, EU-GL, EU-VN mit den tiefen Erfolgsquoten machen knapp 30\% ($(50+23+26)/312$) aller Fail-Projekte aus. Die niedrige Erfolgsquote von NAM stammt vor allem aus VN- und GM-Projekten, wobei VN noch vor GM weniger gut abschneidet.
\newline
Zusammenfassend lässt sich aussagen, dass ungefähr 60\% ($(122+68)/312$ respektive $(145+38)/312$) der Fail-Projekte entweder in den Geschäftsbereichen VN und GM respektive in den Regionen EU und NAM liegen. Zudem wird die Erfolgsquote aller Projekte zu 30\% durch europäische Projekte in den Geschäftsbereichen GM, GL und VN bestimmt wird. 
\newline\newline\textbf{Kosten} Das Umsatzvolumen soll Aufschluss über die Grösse und Wichtigkeit eines Projekts geben. Die zugrundeliegende Prämisse postuliert, dass Projekte mit höherem Umsatzvolumen risikoreicher sind und deshalb eher unter Budget beendet werden. Die Gegenhypothese unterstellt, dass grössere Projekte (hoher TOBud) relativ mehr Beachtung erhalten, da sie den Erfolg eines Geschäftsbereich respektive einer Region mehr beeinflussten als kleinere Projekte, und deshalb erfolgreicher abschliessen.
\newline\textbf{Einfügung Histogram TOBud\_cat - SWEAVE}
\newline
Die Verteilung des Umsatzvolumen ist linksschief und zeigt dass der Grossteil der Projekte ein Umsatzvolumen von weniger als 10 Mio. CHF haben.
\newline\textbf{Histogram}Das Histogramm für die TOBud\_Cat zeigt, dass ca. zwei Drittel aller untersuchten Projekte ein Umsatzvolumen von bis und mit 2 Mio. CHF hat. Die Anzahl Fail-Projekte konzentriert sich folglich in diesen vier untersten Kategorien. Die Auswertung der Erfolgsquote pro Klasse ergab, dass Projekte mit einem Umsatzbudget im Bereich von 2 bis 5 Mio. relativ erfolgreich abgeschlossen wurden. Demgegenüber ist die Erfolgsquote von Projekten mit einem Umsatzbudget zwischen 5 und 10 Mio. tiefer. Basierend auf diesen Erkenntnissen lässt sich die folgende Hypothese formulieren: Das Umsatzvolumen begünstigt bis zu einem gewissen Schwellenwert, die Gegenhypothese und ab diesem Schwellenwert die ursprüngliche These.
\newline\newline Die absoluten und relativen Abweichungen zwischen den aktuellen und den budgetierten Kosten war bereits Bestandteil der finanziellen Analyse.
\newline\newline Die Zusammensetzung der Projektkosten soll Hinweise zur Natur der Projekte liefern, beispielsweise, ob Unterschiede zwischen den untersuchten Gruppen festzustellen sind. Die nachfolgende Tabelle zeigt die Mittelwerte pro relativem Kostenanteil. Es lassen sich keine auffallende Unterschiede feststellen.
\begin{table}[htbp]
	\centering
	\caption{Arithmetisches Mittel der relativen Anteile am Gesamtkostenbudget je Kostenart [\%]}
	\begin{tabular}{lrrrr}
		\textbf{Success} & \multicolumn{1}{l}{\textbf{BudMSTot}} & \multicolumn{1}{l}{\textbf{BudMETot}} & \multicolumn{1}{l}{\textbf{BudPATot}} & \multicolumn{1}{l}{\textbf{BudISTot}} \\
		FALSE & 67.1  & 6.2   & 5.9   & 7.7 \\
		TRUE  & 67.9  & 5.4   & 5.1   & 6.8 \\
	\end{tabular}%
	\label{tab:addlabel}%
\end{table}%
Der durchschnittliche budgetierte MS-Anteil am Kostenbudget des Projekts beträgt 67\%. Der durchschnittliche ME-Anteil und PA-Anteil ist für Failprojekten um etwa 80 Prozentpunkt höher. Der IS-Anteil von Fail-Projekten ist um 0.9\% höher als bei Success-Projekten. 
\newline\newline Nachlieferungen können einerseits ein Indiz für die Nicht-Einhaltung der vorgegeben Lieferzeit und anderseits für Fehlkonstruktionen sein. Allfällige Mehrkosten werden bei Verschulden der Bühler AG von der Bühler AG übernommen. Vermutungsweise ist der Anteil der Kosten aus Nachlieferungen bei Fail-Projekten höher als bei Success-Projekten. Die Auswertung des arithmetischen Mittels der prozentualen SU Kosten am Umsatz bestätigt die erwartete Vermutung.
\begin{table}[htbp]
	\centering
	\caption{Arithmetisches Mittel der SUCostTO [\%]}
	\begin{tabular}{lr}
		\textbf{Success} & \multicolumn{1}{l}{\textbf{SUCostTO}} \\
		FALSE & -0.81 \\
		TRUE  & -0.36 \\
	\end{tabular}%
	\label{tab:addlabel}%
\end{table}%

\begin{table}[htbp]
	\centering
	\caption{Arithmetisches Mittel der SUCostTO [\%] pro TO-Kategorie}
	\begin{tabular}{llr}
		\textbf{Success} & \textbf{TOBud\_Cat} & \multicolumn{1}{l}{\textbf{SUCostTO}} \\
		FALSE & [4.5e+03,5e+03) & -6.88 \\
		FALSE & [13.2,500) & -0.93 \\
		FALSE & [2.5e+03,3e+03) & -0.82 \\
		FALSE & [5e+03,1e+04) & -0.82 \\
		FALSE & [1.5e+03,2e+03) & -0.72 \\
		FALSE & [500,1e+03) & -0.69 \\
		FALSE & [3.5e+03,4e+03) & -0.69 \\
		FALSE & [2e+03,2.5e+03) & -0.57 \\
		FALSE & [4e+03,4.5e+03) & -0.52 \\
		FALSE & [1e+03,1.5e+03) & -0.47 \\
		FALSE & [1e+04,3.42e+04) & -0.36 \\
		FALSE & [3e+03,3.5e+03) & -0.31 \\
	\end{tabular}%
	\label{tab:addlabel}%
\end{table}%
Die Analyse der SUCostTO pro TO-Kategorie zeigt, dass für Projekte mit einem Umsatzvolumen zwischen 13.2 TCHF und 500 TCHF die Nachlieferungskosten in Relation zum Umsatz am höchsten war. Der Wert 6.9\% kann als Anomalie betrachtet werden, ein Projekt mit einem SUCostTO-Wert von ca. 40\% ein Einzelfall darstellt.
\newline\newline Tendenziell wird die Anpassung des Forecast für die Projektkosten bei erwarteten Mehrkosten möglichst lange hinausgezögert. Einerseits kann mit diesem Vorgehen, die Erklärungsdirektive umgangen werden und anderseits besteht wahrscheinlich, dass die Projektkosten sich wieder normalisieren. Deshalb wird erwartet, dass die Differenz zwischen der letzten FC-Anpassung und den tatsächlichen Kosten bei Fail-Projekten höher ist. Die tatsächlichen Kosten waren durchschnittlich höher als beim letzten Kostenforecast. Die durchschnittliche Differenz bei den IS-Kosten war für Fail-Projekte doppelt so hoch. Dies könnte ein Indiz sein, dass bei Fail-Projekten die Installation kostenintensiver verlief. Mögliche Gründe könnte die unzureichende Vorbereitung durch den Kunden oder mangelnde personelle Ressourcen, die zu Mehrkosten in der letzten Projektphase führen. 
\begin{table}[htbp]
	\centering
	\caption{Arithmetisches Mittel der Abweichung der effektiven Kosten vom FC [TCHF]}
	\begin{tabular}{lrrr}
		\textbf{Success} & \multicolumn{1}{l}{\textbf{DeltaLastFCAct}} & \multicolumn{1}{l}{\textbf{DeltaLastFCActMS}} & \multicolumn{1}{l}{\textbf{DeltaLastFCActME}} \\
		FALSE & -490.54 & -445.53 & 7.48 \\
		TRUE  & -436.41 & -454.24 & 7.93 \\
	\end{tabular}%
	\label{tab:addlabel}%
\end{table}%
\begin{table}[htbp]
	\centering
	\caption{Arithmetisches Mittel der Abweichung der effektiven Kosten vom FC [TCHF]}
	\begin{tabular}{lrr}
		\textbf{Success} & \multicolumn{1}{l}{\textbf{DeltaLastFCActPA}} & \multicolumn{1}{l}{\textbf{DeltaLastFCActIS}} \\
		FALSE & -12.87 & -14.52 \\
		TRUE  & -13.41 & -6.49 \\
	\end{tabular}%
	\label{tab:addlabel}%
\end{table}%
\textbf{FF-Variablen:} Der bedeutenste Einflussfaktor im Projektmanagement ist der Projektmanager selbst. Die Evaluation der realisierten Projekte pro Projektmanager inklusive der Erfolgsquote hat ergeben, dass die 966 Projekte von 301 unterschiedlichen Projektmanager abgewickelt wurde. 145 Projektmanager haben ihre Projekte aussschliesslich erfolgreich beendet, wohingegen gerade einmal 45 PM nur unzureichend Projekte abgewickelt hat. Die detaillierte Liste ist im Anhang zu finden.
\newline\newline Der Wechsel des Projektmanagers kann ein Indiz für konfligierende Verhältnisse zwischen den Vertragsparteien sein, weshalb hypothetisch vermutet wird, dass Fail-Projekte eher mit einem PMChange einhergehen. 
\begin{table}[htbp]
	\centering
	\caption{Häufigkeit PMChange}
	\begin{tabular}{lrrrr}
		\textbf{PMChange} & \multicolumn{1}{l}{\textbf{Success}} & \multicolumn{1}{l}{\textbf{Fail}} & \multicolumn{1}{l}{\textbf{Fail [\%]}} & \multicolumn{1}{l}{\textbf{Total}} \\
		no    & 628   & 295   & 31.96\% & 923 \\
		yes   & 26    & 17    & 39.53\% & 43 \\
		\textbf{Total} & \textbf{654} & \textbf{312} &       & \textbf{966} \\
	\end{tabular}%
	\label{tab:addlabel}%
\end{table}%
Insgesamt wurden 43 Projekte mit einem Wechsel des Projektmanagers über die letzten drei Jahre abgewickelt. Davon sind 17 gescheitert und 26 wurden erfolgreich abgeschlossen.
\newline Die Anzahl Projektmanager ist direkt mit der Variable PMChange verbunden, das ein PMChange zwei Projektmanager währen des Projektverlaufs impliziert. Die nachfolgende Tabelle zur Häufigkeitsübersicht reflektiert diese Relation. Es gab eine kleine Anzahl Projekte (insgesamt 6 Projekte) bei denen zweimal ein Wechsel des Projektmanager erfolgte, wovon 5 nicht erfolgreich abgeschlossen werden konnten.
\newline Das Alter des Projektmanagers ist eine Proxyvarialbe für die Lebens- und Berufserfahrung generell. Hierbei wird unterstellt, dass je erfahrener der Projektmanager ist, desto eher können die Projekte erfolgreich abgeschlossen werden. Die Betriebszugehörigkeit des Projektmanagers (PMTen) ist eine Proxyvariable für die Kenntnisse der Bühlerwelt. Die Varialbe postuliert einen Zusammenhang zwischen der Erfolgschance und den Kenntnissen über die Bühlerwelt, folglich müssten jüngere Mitglieder der Bühler-Familie weniger Erfolg im Projektmanagement haben. Das durchschnittliche Alter der untersuchten Stichprobe beträgt 39 Jahre. Diesem Durchschnitt kann unterstellt werden, dass relativ erfahrene Projektmanager während der Zeit von 2013 bis 2015 bei der Bühler AG gearbeitet haben. Die durchschnittliche Betriebszugehörigkeit beträgt 10 Jahre, womit sich postulieren lässt, dass die PM der betrachteten Stichprobe relativ gute Kenntnisse von den Bühler-Praktiken hatten.
\begin{table}[htbp]
	\centering
	\caption{Durchschnittswerte PMAge und PMTen}
	\begin{tabular}{lrr}
		\textbf{Success} & \multicolumn{1}{l}{\textbf{Age}} & \multicolumn{1}{l}{\textbf{Ten}} \\
		FALSE & 41.1 & 12.4 \\
		TRUE  & 39.5 & 11.7 \\
	\end{tabular}%
	\label{tab:addlabel}%
\end{table}%
 Die Lead SAS des Projekts trägt die Gesamtverantwortung. Einige Gesellschaften sind bessere Projektmanager als anderen. Der Vergleich Erfolgsquoten pro SAS zeigt, dass die europäischen Gesellschaften n den letzten drei Jahren eine unterdurchschnittlich Erfolgsrate hatten. Es lässt sich eine Übereinstimmung mit den Befunden aus der Regionen-Analyse feststellen.
\newline Die LeadSASFF ist verantwortlich für die Projektabwicklung, wobei sie sich von der LeadSASPr unterscheiden kann. Da bei geteilter Verantwortlichkeiten die Anforderungen an die Kommunikation zwischen den Schnittstellen steigt, wird vermutet, dass bei getrennter Verantwortlichkeiten ein Merkmal von Fail-Projekten sind. Die nachfolgende Informationen der Tabelle deuten an, dass das Gegenteil wahr ist. 
\begin{table}[htbp]
	\centering
	\caption{Häufigkeit geteilter Verantwortlicheit [yes]}
	\begin{tabular}{lrrrr}
		\textbf{LeadSAS.PrFF} & \multicolumn{1}{l}{\textbf{Success}} & \multicolumn{1}{l}{\textbf{Fail}} & \multicolumn{1}{l}{\textbf{Fail [\%]}} & \multicolumn{1}{l}{\textbf{Total}} \\
		No    & 569   & 296   & 34.2\% & 865 \\
		Yes   & 85    & 16    & 15.8\% & 101 \\
		\textbf{Total} & \textbf{654} & \textbf{312} &       & \textbf{966} \\
	\end{tabular}%
	\label{tab:addlabel}%
\end{table}%
In der Stichprobe war die Projektverantwortung für ungefähr 90\% zentralisiert, wovon 34.2\% nicht erfolgreich waren. Bei den restlichen 101 Projekten mit geteilter Projektverantwortung wurden lediglich 16\% mit einem DB1Act unter Budget abgeschlossen. Die Anzahl der LeadSASFF steht in direkter Verbindung zum Faktor LeadSAS.PrFF, da die Ausprägung 'No' impliziert, dass nur eine SAS die Projektverantwortung inne hat. Deshalb liefert diese Determinante keine zusätzlichen Informationen. 
\newline\newline\textbf{Zeit:} Die Beurteilung des Zeitmanagement hängt von der Einhaltung des vereinbarten Liefertermins ab. Mehrkosten und Zeitverzug gehen oftmals einher, weshalb unterstellt wird, dass Fail-Projekte den vereinbarten Projektabschluss nicht einhalten konnten. Ferner soll ergründet werden, ab welchem Zeitpunkt respektive bei Milestone der Zeitverzug üblicherweise eintritt. 
\begin{table}[htbp]
	\centering
	\caption{Projektlaufzeiten und Zeitverzug [in Monaten]}
	\begin{tabular}{lrrrrrrr}
		\textbf{Success} & \multicolumn{1}{l}{\textbf{Base}} & \multicolumn{1}{l}{\textbf{Act}} & \multicolumn{1}{l}{\textbf{Delay}} & \multicolumn{1}{l}{\textbf{MS2}} & \multicolumn{1}{l}{\textbf{MS8}} & \multicolumn{1}{l}{\textbf{MS10}} & \multicolumn{1}{l}{\textbf{MS11}} \\
		TRUE  & 11.9  & 17.3  & -5.4  & -0.1  & -2.0  & -5.0  & -5.5 \\
		FALSE & 11.4  & 18.7  & -7.2  & -0.2  & -1.7  & -5.7  & -7.3 \\
	\end{tabular}%
	\label{tab:addlabel}%
\end{table}%
Die durchschnittliche budgetierte Projektlaufzeit unterscheidet sich zwischen erfolgreichen und  nicht-erfolgreichen Projekten kaum wohingegen die effektive Projektlaufzeit der Fail-Projekte einen Monat mehr betrug. Gemäss der Tabelle sind Success-Projekte ca. 2 Monate weniger zeitverzögert. Die Termineinhaltung beim MS 2 Concept approved bewegt sich im vernachlässigbaren Bereich. Demgegenüber steigt der durchschnittliche Zeitverzug nach MS8 Documented auf zwei und nach MS10 Takeover auf 5-6 Monate an. Bei den Fail-Projekten stieg die durchschnittliche Zeitverzögerung auf 7.3 Monate an. Ein möglicher Erklärungsansatz wäre, dass bei der Übergabe Mängel beanstandet wurden und nachgebessert werden musste.
\begin{table}[htbp]
	\centering
	\caption{Add caption}
	\begin{tabular}{lrrrrrrrrrr}
		\textbf{Success} & \multicolumn{1}{l}{\textbf{Delay}} & \multicolumn{1}{l}{\textbf{Total}} & \multicolumn{1}{l}{\textbf{DelayMS2}} & \multicolumn{1}{l}{\textbf{onTimeMS2}} & \multicolumn{1}{l}{\textbf{DelayMS8}} & \multicolumn{1}{l}{\textbf{onTimeMS8}} & \multicolumn{1}{l}{\textbf{DelayMS10}} & \multicolumn{1}{l}{\textbf{onTimeMS10}} & \multicolumn{1}{l}{\textbf{DelayMS11}} & \multicolumn{1}{l}{\textbf{onTimeMS11}} \\
		FALSE & TRUE  & 268   & 39    & 229   & 187   & 81    & 249   & 19    & 267   & 1 \\
		FALSE & FALSE & 44    & 4     & 40    & 22    & 22    & 26    & 18    & 3     & 41 \\
		\textbf{Total FALSE} &       & \textbf{312} & \textbf{43} & \textbf{269} & \textbf{209} & \textbf{103} & \textbf{275} & \textbf{37} & \textbf{270} & \textbf{42} \\
		TRUE  & TRUE  & 515   & 63    & 452   & 353   & 162   & 476   & 39    & 513   & 2 \\
		TRUE  & FALSE & 139   & 12    & 127   & 55    & 84    & 56    & 83    & 3     & 136 \\
		\textbf{Total TRUE} &       & \textbf{654} & \textbf{75} & \textbf{579} & \textbf{408} & \textbf{246} & \textbf{532} & \textbf{122} & \textbf{516} & \textbf{138} \\
		\textbf{Grand Total} &       & \textbf{966} & \textbf{118} & \textbf{848} & \textbf{617} & \textbf{349} & \textbf{807} & \textbf{159} & \textbf{786} & \textbf{180} \\
	\end{tabular}%
	\label{tab:addlabel}%
\end{table}%
Die Mehrheit der untersuchten Projekte konnte die Zeitvereinbarungen im MS2 einhalten. Dieses Verhältnis ändert sich bei Erreichung des MS8 und steigt bei MS10 so an, dass letztendlich der Grossteil der Projekte zeitverzögert abgeschlossen wird (783 Projekte respektive 86\%). Mittels Dummyvariablen pro Milestone wurde ausgewertet, ob sich ein anfängliche Verspätung sich durch die Projektlaufzeit durchzieht. Die meisten Projekte hatten die Eigenschaft, bei einer Verspätung im MS8 ebenso auch im MS11 verspätet zu sein. Die zweithäufigste Gruppe war diejenige, die erstmalig ab dem MS10 den Liefertermin bis zum Projektabschluss nicht mehr einhalten konnte. Auf Basis der Tabellen lassen sich jedoch keine Aussagen machen, ob bei einer Nichteinhaltung des vereinbarten Termin, das Projekt zwangsläufig unter Budget abschliessen wird. 
\newline\newline\textbf{SQ:} Die Einflussdeterminante des SQ-Prozess sind einerseits der Stand im Bezug auf das OR-Budget und die Erfahrung sowie Betriebszugehörigkeit des Verkaufsmanager in Jahren. Allerdings konnten letztere aufgrund fehlender Datensätze nicht ausgewertet werden. Grundsätzlich wird vermutete, dass ein Budgetdruck im Zeitpunkt des Verkaufsabschlusses, den Verkauf von risikoreicheren Projekten begünstigt. Die mittlere Abweichung des OR vom Budget des Geschäftseinheit und der Region war für Fail-Projekte in absoluten und relativen Grössen höher als für Success Projekte.  
\newline\newline\textbf{Komplexität:} Die Komplexität drückt den Schwierigkeitsgrad eines Projekts aus. Die vorherrschende Auffassung lautete, dass komplexere Projekte weniger erfolgreich abschliessen. Da die Komplexität nicht direkt gemessen werde kann, wurden die Anzahl Aufträge sowie involvierter Parteien bei den unterschiedlichen Projektphasen und Konsortien als Proxyvariablen festgelegt. Die Anzahl Projekte in der Stichprobe, die in einem Konsortium abgewickelt wurden, beträgt 78 , wovon 47 erfolgreich und 31 unter Budget abgeschlossen wurden. Die nachfolgende Tabelle zeigt die Verteilung der Konsortium-Projekte pro Region und BA. In EU und SAS\_BCHi werden 51 Projekte im Konsortium abgewickelt. In Europa konnte bei den GL- und VN-Projekten gerade einmal 50\% der Projekte oder weniger erfolgreich abgeschlossen werden. In China und MEA hingegen sind es die GM-Projekte, welche unter Budget abgeschlossen haben. 
\begin{table}[htbp]
	\centering
	\caption{Add caption}
	\begin{tabular}{lrrrrr}
		& \multicolumn{1}{l}{Region} & \multicolumn{1}{l}{BA} & \multicolumn{1}{l}{Success} & \multicolumn{1}{l}{Fail} & \multicolumn{1}{l}{Total} \\
		& \multicolumn{1}{l}{East\_Asia} & \multicolumn{1}{l}{GL} & 1     & 0     & 1 \\
		&       &       &       &       &  \\
		& \multicolumn{1}{l}{EU} & \multicolumn{1}{l}{CF} & 5     & 0     & 5 \\
		& \multicolumn{1}{l}{EU} & \multicolumn{1}{l}{DC} & 0     & 1     & 1 \\
		& \multicolumn{1}{l}{EU} & \multicolumn{1}{l}{GL} & 6     & 5     & 11 \\
		& \multicolumn{1}{l}{EU} & \multicolumn{1}{l}{GM} & 1     & 3     & 4 \\
		& \multicolumn{1}{l}{EU} & \multicolumn{1}{l}{VN} & 1     & 5     & 6 \\
		\textbf{Total EU} &       &       & \textbf{13} & \textbf{14} & \textbf{27} \\
		& \multicolumn{1}{l}{MEA\_Afr} & \multicolumn{1}{l}{GL} & 2     & 2     & 4 \\
		& \multicolumn{1}{l}{MEA\_Afr} & \multicolumn{1}{l}{GM} & 6     & 5     & 11 \\
		& \multicolumn{1}{l}{MEA\_Afr} & \multicolumn{1}{l}{VN} & 1     & 0     & 1 \\
		\textbf{Total MEA} &       &       & \textbf{9} & \textbf{7} & \textbf{16} \\
		& \multicolumn{1}{l}{North\_Ame} & \multicolumn{1}{l}{CF} & 1     & 0     & 1 \\
		& \multicolumn{1}{l}{North\_Ame} & \multicolumn{1}{l}{GM} & 1     & 1     & 2 \\
		\textbf{Total NAM} &       &       & \textbf{2} & \textbf{1} & \textbf{3} \\
		& \multicolumn{1}{l}{SAS\_BCHI} & \multicolumn{1}{l}{CF} & 2     & 0     & 2 \\
		& \multicolumn{1}{l}{SAS\_BCHI} & \multicolumn{1}{l}{DC} & 11    & 1     & 12 \\
		& \multicolumn{1}{l}{SAS\_BCHI} & \multicolumn{1}{l}{GM} & 4     & 6     & 10 \\
		\textbf{Total SAS\_BCHI} &       &       & \textbf{17} & \textbf{7} & \textbf{24} \\
		& \multicolumn{1}{l}{South\_Ame} & \multicolumn{1}{l}{GM} & 3     & 1     & 4 \\
		& \multicolumn{1}{l}{South\_Ame} & \multicolumn{1}{l}{VN} & 1     & 1     & 2 \\
		\textbf{Total SAM} &       &       & \textbf{4} & \textbf{2} & \textbf{6} \\
		& \multicolumn{1}{l}{South\_Asia} & \multicolumn{1}{l}{CF} & 1     & 0     & 1 \\
		\textbf{Grant Total} &       &       & \textbf{47} & \textbf{31} & \textbf{78} \\
	\end{tabular}%
	\label{tab:addlabel}%
\end{table}%
\newline Die Anzahl Aufträge pro Projekt variiert zwischen eins und 10 wobei die Mehrheit aller Projekte arbeitet auf Basis von einem Auftrag, von 312 nicht-erfolgreichen Projekten hatten 70\% einen Auftrag. Zudem sind von allen Projekte mit mehr als einem Auftrag 2 von 3 Projekten gut gelaufen. Mögliche Erklärungsansätze könnte sein, dass bei mehr Aufträgen die Übersichtlichkeit verloren gehen kann.
\begin{table}[htbp]
	\centering
	\caption{Add caption}
	\begin{tabular}{lrr}
		& \multicolumn{1}{l}{\textbf{Success}} & \multicolumn{1}{l}{\textbf{Fail}} \\
		\textbf{1 Vertrag} & 449   & 228 \\
		\textbf{> 1 Vertrag} & 205   & 84 \\
		\textbf{Total} & 654   & 312 \\
	\end{tabular}%
	\label{tab:addlabel}%
\end{table}%
Die Anzahl involvierte SAS bei der Zulieferung liegt im Bereich null und zehn, wobei null mit Eigenproduktion oder Zulieferung durch Dritte gleichzusetzen ist. In den übrigen Projektphasen sind maximal drei andere Bühler-Gesellschaften involviert. Eine Einzelauswertung pro Projektphase ergibt relativ wenig Aufschluss, weshalb die Häufigkeit der Kombinationen untersucht wurde. Die nachfolgende Tabelle zeigt, dass die Eigenproduktion respektive die Arbeit mit Drittlieferanten während allen Projektphasen das häufigste Charakteristika ist. Es fällt auf, dass sich die Hälfte der Datenpunkte in einem Bereich von maximal zwei involvierten SAS-Gesellschaften befindet. Der Vergleich der Erfolgsquoten der Kombinaten wäre für Projekte die bei der Zulieferung mit zwei weiteren SAS arbeiten jedoch bei den anderen Projektphasen keine weitere Parteien miteinbeziehen am höchsten. Die Zusammenarbeit mit einer anderen Bühler-Partei während der gesamten Projektlaufzeit scheint weniger erfolgreich zu sein.
\begin{table}[htbp]
	\centering
	\caption{Anzahl involvierter Parteien}
	\begin{tabular}{rrrrrrrr}
		\multicolumn{1}{l}{NoSupplSAS} & \multicolumn{1}{l}{NoSupplSASMS} & \multicolumn{1}{l}{NoSupplSASME} & \multicolumn{1}{l}{NoSupplSASPA} & \multicolumn{1}{l}{NoSupplSASIS} & \multicolumn{1}{l}{Dummy\_Success} & \multicolumn{1}{l}{Dummy\_Fail} & \multicolumn{1}{l}{total} \\
		0     & 0     & 0     & 0     & 0     & 180   & 89    & 269 \\
		1     & 1     & 0     & 0     & 0     & 73    & 26    & 99 \\
		2     & 2     & 0     & 0     & 0     & 38    & 13    & 51 \\
		2     & 2     & 1     & 0     & 0     & 23    & 13    & 36 \\
		1     & 1     & 1     & 0     & 0     & 17    & 11    & 28 \\
		1     & 1     & 1     & 1     & 1     & 15    & 12    & 27 \\
	\end{tabular}%
	\label{tab:addlabel}%
\end{table}%
\newline Die Analyse der Anzahl Aufträge in Verbindung mit der involvierten Parteien zeigt, dass ein Drittel der Projekte basierend auf einem ein Produktionsauftrag arbeiten und die Herstellung (MS) mittels Eigenproduktion oder in Zusammenarbeit mit einer anderen Bühler Gesellschaft erfolgte.
\begin{table}[htbp]
	\centering
	\caption{Add caption}
	\begin{tabular}{rrrrrrrrr}
		\multicolumn{1}{l}{NoSupplSAS} & \multicolumn{1}{l}{NoSupplSASMS} & \multicolumn{1}{l}{NoSupplSASME} & \multicolumn{1}{l}{NoSupplSASPA} & \multicolumn{1}{l}{NoSupplSASIS} & \multicolumn{1}{l}{NoContr} & \multicolumn{1}{l}{Dummy\_Success} & \multicolumn{1}{l}{Dummy\_Fail} & \multicolumn{1}{l}{Total} \\
		0     & 0     & 0     & 0     & 0     & 1     & 161   & 83    & 244 \\
		1     & 1     & 0     & 0     & 0     & 1     & 47    & 20    & 67 \\
		2     & 2     & 0     & 0     & 0     & 1     & 28    & 7     & 35 \\
		2     & 2     & 1     & 0     & 0     & 1     & 16    & 10    & 26 \\
		1     & 1     & 1     & 0     & 0     & 1     & 14    & 11    & 25 \\
		1     & 1     & 1     & 1     & 1     & 1     & 14    & 9     & 23 \\
		1     & 1     & 0     & 0     & 0     & 2     & 15    & 5     & 20 \\
		0     & 0     & 0     & 0     & 0     & 2     & 14    & 5     & 19 \\
		1     & 0     & 0     & 0     & 0     & 1     & 18    & 1     & 19 \\
		1     & 1     & 1     & 0     & 1     & 2     & 16    & 2     & 18 \\
		1     & 1     & 0     & 1     & 0     & 1     & 14    & 1     & 15 \\
	\end{tabular}%
	\label{tab:addlabel}%
\end{table}%
\newpage
\subsection{Kritische Würdigung der Ergebnisse}
\newpage	
	


\newpage
%Ergebnisse
\chapter{Ergebnisse}\label{sec: Ergeb}
Die Ergebnisse der finanziellen Analyse und der Untersuchung der Einflussfaktoren werden getrennt dargestellt. Die untersuchte Stichprobe enthält 966 Projekte, wovon 654 erfolgreich abgeschlossen wurden.
%%Ergenbnisse pro Kategorie: Hypothese Nennen, Kommentierung, Ergebnisse präsentieren, Tabellen etc
%%Rahmenbedingungen: Region, BA und allenfalls BU: MS zu feine Gliederung, EquLoc too maany Einzelwerte, CuNo too many Einzelwerte
%%Sales & Quotation: Budget Druck, Rest ging nicht
%%Fullfillment: PMNo to many Einzelwerte, PMAge gemäss Kategorie, PMTen (EVTL.), NoPM not, da Analyse zeigt dass nur einmal 3 PM gab, PMChange,
%%Fulfillment: LeadSASPr, LeadSASPrFF, LeadSAS.PrFF,  NoLeadSASFF nicht, da zu wenig Fälle
%%Fulfillment: CostFCajd alle und HOM
%%Kosten: Tabelle: finanzielle Performanceanalyse: Umsatz, Marge, Erklärung, Erkenntnisse (Bud, Ac, Dev)
%%Kosten: Tabelle: finanzielle Performancenaalyse: Kosten(Bud, Act, Abw und Abw Schlüssel)
%%Kosten: evtl. Tabelle: pro Umsatzkategorie Anz. Cost Dev und: mit wenig, viel Verlust
%%Kosten: Mittelwerte für alle Variablen
%%Zeit: alle, aber ohne Delay
%%Komplexität: alle 

%Tabelle Übersicht Stichprobe
\begin{table}[htbp]
	\centering
	\caption{Übersicht Stichprobe}
	\begin{tabular} {l|r|r}
		\textbf{Erfolgskriterium} & \textbf{absolut} & \textbf{relativ} \\\hline
		SUCCESS & 654 & 68\% \\
		FAIL & 312 & 32\% \\\hline
		\textbf{Total} & \textbf{966} & \textbf{100\%} \\
	\end{tabular}
\end{table}
\section{Finanzielle Performance Analyse}
Zur Bewertung der finanziellen Performance wurden drei verschiedene Auswertungen gemacht: Abweichung Act-Bud, Zusammensetzung der Kosten sowie eine Auswertung pro Umsatzkategorie. Das Ziel besteht darin, den finanziellen Verlust in Abhängigkeit des Erfolgskriterium (DB1BudDev) zu quantifizieren. 
%Bud TO Cost DB1
\begin{table}[H]
	\centering
	\caption{Budget TO, Cost und DB1 [TCHF]}
	\begin{tabular}{lrrrr}
		\textbf{Erfolgskriterium} & \textbf{TO Bud} & \textbf{Cost Bud} &
		\textbf{DB1 Bud} & \textbf{DB1 Bud [\%]} \\\hline
		SUCCESS & 1'552'450 & -1'156'598 & 395'851 & 25.5\% \\
		FAIL  & 618'013 & -465'066 & 152'947 & 24.7\% \\\hline
		\textbf{Total} & \textbf{2'170'463} & \textbf{-1'621'664} & \textbf{548'799} & \textbf{25.3\%}\\
	\end{tabular}%
	\label{bud}%
\end{table}%
%Act TO Cost DB+
\begin{table}[H]
	\centering
	\caption{Actuals TO, Cost und DB1 [TCHF]}
	\begin{tabular}{lrrrr}
		\textbf{Erfolgskriterium} & \textbf{TO Act} & \textbf{Cost Act} & \textbf{DB1 Act}&
		\textbf{DB1 Act [\%]} \\\hline
		SUCCESS & 1'560'001 & -1'041'728 & 518'273 & 33.2\% \\
		FAIL  & 631'346 & -526'600 & 104'746 & 16.6\% \\\hline
		\textbf{Total} & \textbf{2'191'347} & \textbf{-1'568'328} & \textbf{623'018} & \textbf{28.4\%} \\
	\end{tabular}
	\label{act}%
\end{table}%
%Deviation TO Cost DB1
\begin{table}[H]
	\centering
	\caption{Abweichungen ($Act-Bud$) TO, Cost und DB1 [TCHF] }
	\begin{tabular}{lrrrr}
		\textbf{Erfolgskriterium} & \textbf{TO} & \textbf{Cost} & \textbf{DB1}&
		\textbf{DB1 [\%]} \\\hline
		SUCCESS & 7'551 & 114'870 & 122'421 & 7.7\% \\
		FAIL  & 13'333 & -61'534 & -48'202 & -8.2\% \\\hline
		\textbf{Total} & \textbf{20'884} & \textbf{53'336} & \textbf{74'220} & \textbf{3.1\%} \\
	\end{tabular}
	\label{Abw}%
\end{table}%
Die Tabelle \ref{Abw} wurde mittels Tabellen \ref{bud} und \ref{act} berechnet und zeigt, dass der realisierte Umsatz höher war als budgetiert wurde. Diese Abweichung kann auf Zusatzverkäufe oder die Verrechnung allfälliger Mehrkosten an den Kunden zurückgeführt werden. Der kumulierte DB1 der Fail-Projekte lag 48 Mio. CHF (-32\%) unter dem Budget und bei den Success-Projekten 122 Mio. CHF über dem Budget. Die positive Abweichung der Istkosten der Success-Projekte kann mittels der realisierten Kostenreserve, die üblicherweise pro Projekt einkalkuliert wird und je nach Geschäftsbereich zwischen 4\% und 9\% beträgt, zurückgeführt werden. Wenn die Kostenreserve aufgebraucht wird, resultieren Mehrkosten und die Kostenabweichung wird negativ. Da die Reserve in dieser Betrachtung nicht ersichtlich ist, wäre die effektive Differenz für Fail-Projekte (Success-Projekte) tiefer (höher). Die realisierte Marge über alle Success-Projekte beträgt 33\% und liegt 7.7\% über der budgetierten Marge von 25.4\%. Demgegenüber beträgt der DB1Act der Fail-Projekte 16.6\% und liegt 8.2\% unter dem DB1 Bud von 24.7\%. Nachfolgend wird die Kostenperformance näher betrachtet, um herauszufinden, bei welcher Projektphase die Mehrkosten entstehen.
%Aufschlüsselung Kostenabweichung gemäss Projektphase
\begin{table}[H]
	\centering
	\caption{Aufschlüsselung der Kosten nach der Projektphase [TCHF]}
	\begin{tabular}{lrrrrr|r}
		\textbf{Erfolgskriterium} & \multicolumn{1}{l}{\textbf{Total Cost}} & \multicolumn{1}{l}{\textbf{MS}} & \multicolumn{1}{l}{\textbf{ME}} & \multicolumn{1}{l}{\textbf{PA}} & \multicolumn{1}{l}{\textbf{IS}} & \multicolumn{1}{l}{\textbf{Summe}} \\\hline
		SUCCESS & 114'870 & 47'615 & -2'159 & -908  & -7'114 & 37'434 \\
		FAIL  & -61'534 & -20'253 & -12'721 & -7'220 & -22'053 & -62'247 \\\hline
		\textbf{ Total} & \textbf{53'336} & \textbf{ 27'363} & \textbf{ -14'880} & 
		\textbf{ -8'128} & \textbf{ -29'167} & \textbf{ -24'812} \\
	\end{tabular}%
	\label{stocostdb1dev}%
\end{table}%
Die Aufschlüsselung der Kostenabweichung der Tabelle \ref{stocostdb1dev} zeigt, dass die Installationsphase sowohl der erfolgreichen als auch der nicht-erfolgreich Projekte mit Mehrkosten verbunden ist. Die negative Kostenabweichung der Fail-Projekte kann zu einem Drittel auf die IS- und zu einem weiteren Drittel auf die MS-Kosten zurückgeführt werden. Bei den Success-Projekten kann ein gewisser 'Verlust'-Kompensationseffekt durch die positive Kostenabweichung der MS-Kosten festgestellt werden. Die Kostendifferenz zwischen Act und Bud der Fail-Projekte kann fast vollständig durch Summe der Kostenabweichungen der vier Projektphasen MS, ME, PA und IS erklärt werden. Der Unterschied zu 'Total Cost' ist auf fehlende Abbildung der anderen Kostenarten der Projektstruktur zurückzuführen. Dieser Effekt ist bei den Success-Projekten ebenfalls sichtbar und kann zu einem Teil auf die Realisation des Kostenpuffers zurückgeführt werden.
\newline
Als Ergänzung wurde versucht zu eruieren, von welchem Projekttyp in Bezug auf das Umsatzvolumen die Margeneinbusse der Fail-Projekte stammt. Dazu wurde die Häufigkeit und der DB1 Abweichung pro Umsatzkategorie ausgewertet.
% Table generated by Excel2LaTeX from sheet 'TOBud_cat'
\begin{table}[htpb]
	\centering
	\caption{DB1 und Häufigkeit pro Umsatzkategorie (TOBud\_Cat) [TCHF]}
	\begin{tabular}{lrcrrrr}
		\multicolumn{1}{l}{\textbf{Kat.}} & \multicolumn{1}{l}{\textbf{TOBud\_Cat}} & \multicolumn{1}{c}{\textbf{ Anz.}} & \multicolumn{1}{l}{\textbf{DB1 Act}} & \multicolumn{1}{l}{\textbf{DB1 Bud }} & \multicolumn{1}{l}{\textbf{DB1BudDevabs}} &  \multicolumn{1}{c}{\textbf{\%}}\\\hline
		\multicolumn{1}{r}{1} & \multicolumn{1}{l}{[13.2,500)} & 54    & 4'508 & 6'181 & -1'672 & -27\% \\
		\multicolumn{1}{r}{2} & \multicolumn{1}{l}{[500,1e+03)} & 87    & 11'956 & 18'167 & -6'210 & -34\% \\
		\multicolumn{1}{r}{3} & \multicolumn{1}{l}{[1e+03,1.5e+03)} & 54    & 13'510 & 17'476 & -3'966 & -23\% \\
		\multicolumn{1}{r}{4} & \multicolumn{1}{l}{[1.5e+03,2e+03)} & 32    & 11'106 & 14'928 & -3'822 & -26\% \\
		\multicolumn{1}{r}{5} & \multicolumn{1}{l}{[2e+03,2.5e+03)} & 17    & 6'286 & 9'447 & -3'161 & -33\% \\
		\multicolumn{1}{r}{6} & \multicolumn{1}{l}{[2.5e+03,3e+03)} & 12    & 6'987 & 8'276 & -1'289 & -16\% \\
		\multicolumn{1}{r}{7} & \multicolumn{1}{l}{[3e+03,3.5e+03)} & 8     & 5'419 & 6'778 & -1'359 & -20\% \\
		\multicolumn{1}{r}{8} & \multicolumn{1}{l}{[3.5e+03,4e+03)} & 8     & 4'338 & 6'445 & -2'107 & -33\% \\
		\multicolumn{1}{r}{9} & \multicolumn{1}{l}{[4e+03,4.5e+03)} & 4     & 3'191 & 4'551 & -1'360 & -30\% \\
		\multicolumn{1}{r}{10} & \multicolumn{1}{l}{[4.5e+03,5e+03)} & 6     & 2'686 & 4'349 & -1'662 & -38\% \\
		\multicolumn{1}{r}{11} & \multicolumn{1}{l}{[5e+03,1e+04)} & 23    & 23'602 & 35'644 & -12'043 & -34\% \\
		\multicolumn{1}{r}{12} & \multicolumn{1}{l}{[1e+04,3.42e+04)} & 7     & 11'155 & 20'706 & -9'551 & -46\% \\\hline
		\textbf{Total} &       &    \textbf{312}   & \textbf{104'746} & \textbf{152'947} & \textbf{-48'202} & \textbf{-32\%} \\
	\end{tabular}%
	\label{tab:ftobudcat}%
\end{table}%
\newline In der Tabelle \ref{tab:ftobudcat} wird ersichtlich, dass ein Viertel des der DB1-Abweichung auf 23 Projekte mit einem Umsatzvolumen zwischen 5 und 10 Mio. CHF und 20\% auf 7 Projekte mit einem Umsatzvolumen von mehr als 10 Mio. CHF zurückzuführen ist. Die drittgrösste Abweichung stammt von der Umsatzkategorie mit den meisten Projekten.
%%%%%%%%%%%%%%%%%%%%%%%%%%%%%%%%%%%%%%%%%%%%%%%%%%%%%%%%%%%%%%%%%%%%%%%%%
%%%%%%%%%%%%%%%%%%%%%%%%%%%%%%%%%%%%%%%%%%%%%%%%%%%%%%%%%%%%%%%%%%%%%%%%%
%Auswertung Erfolgsfaktoren
%%%%%%%%%%%%%%%%%%%%%%%%%%%%%%%%%%%%%%%%%%%%%%%%%%%%%%%%%%%%%%%%%%%%%%%%%
%%%%%%%%%%%%%%%%%%%%%%%%%%%%%%%%%%%%%%%%%%%%%%%%%%%%%%%%%%%%%%%%%%%%%%%%%
\section{Erfolgsfaktoren}
In diesem Kapitel werden die Ergebnisse pro Variablenkategorie sowie mögliche Erklärungsansätze erläutert. Mittels Histogrammen, Häufigkeitstabellen und Mittelwerten wurde versucht, die Charakteristiken vergangener Fail-Projekte zu ergründen. Die Stichprobe wurde hierfür gemäss Erfolgskriterium in zwei Datensets unterteilt. Zur Evaluation von kategorialen Variablen wurde ein weiteres Kriterium, die Erfolgsquote hinzugezogen, um beispielsweise Geschäftsbereiche oder Region untereinander vergleichen zu können.
\newline
\begin{equation}
\text{Erfolgsquote} = \frac{\text{{Anzahl Success-Projekte}}}{\text{Anzahl Fail-Projekte}}
\end{equation}

\paragraph{Rahmenbedingungen}
Die Analyse der Rahmenbedingungen eines Projekts geben Hinweise darauf, in welchen Geschäftsbereichen und Regionen und mit welchen Kunden nicht-erfolgreiche gemäss dem Erfolgskriterium realisiert wurden. Da die Bühler AG in einer Matrix-Organisation organsiert ist, wurde nebst den Einzelauswertungen für die Region und die Business Area, der Regionen-BA Split für die Häufigkeit der Success- und Fail-Projekte erstellt.
%Tabelle Auswertung Häufigkeit pro Region
\begin{table}[H]
	\centering
	\caption{Erfolgsquote und Häufigkeitsverteilung pro Region}
	\begin{tabular}{lrrrrrr}
		\textbf{Region} & \multicolumn{1}{l}{\textbf{Erfolgsquote}} & \multicolumn{1}{l}{\textbf{Success}} & \multicolumn{1}{l}{\textbf{Fail}} & \multicolumn{1}{l}{\textbf{Fail [\%]}} & \multicolumn{1}{l}{\textbf{Total}} & \multicolumn{1}{l}{\textbf{Total [\%]}} \\ \hline
		East\_Asia & 6.7   & 20    & 3     & 13.0\% & 23    & 2.4\% \\
		EU    & \textbf{1.7}   & 240   & 145   & 37.7\% & 385   & 39.9\% \\
		MEA\_Afr & 2.7   & 112   & 42    & 27.3\% & 154   & 15.9\% \\
		North\_Ame & \textbf{1.4}   & 54    & 38    & 41.3\% & 92    & 9.5\% \\
		SAS\_BCHI & 2.9   & 119   & 41    & 25.6\% & 160   & 16.6\% \\
		South\_Ame & 1.9   & 58    & 31    & 34.8\% & 89    & 9.2\% \\
		South\_Asia & 4.3   & 51    & 12    & 19.0\% & 63    & 6.5\% \\ \hline
		\textbf{Total} & \textbf{2.1} & \textbf{654} & \textbf{312} & \textbf{32.3\%} & \textbf{966} & \textbf{100.0\%} \\
	\end{tabular}%
	\label{freg}%
\end{table}% 
Die Ergebnisse der Tabellen \ref{freg} und \ref{fba} auf der nachfolgenden Seite reflektieren die Tatsache, dass Europa der grösste Absatzmarkt und GM die grösste Business Area der Bühler AG ist. Die niedrigste Erfolgsquote hat NAM als viertgrösste Region (in Abhängigkeit der Anzahl Projekte), gefolgt von Europa. Die Anzahl der Fail-Projekte in den Regionen EU, MEA und SAS\_BCHI beträgt 73\% ($(145+42+41)/312$), weshalb die Erfolgsquote von allen Projekten hauptsächlich durch diese drei Regionen bestimmt wird. Die kleinsten Regionen haben die besten Erfolgsquoten. Die Geschäftsbereichen CF, VN, GL und GM ($(42+68+32+122)/312$) umfassen zusammen 84\% aller Fail-Projekte, wobei die drei letzt genannten zugleich die niedrigsten Erfolgsquoten ausweisen. Die Anzahl untersuchter Projekte der letzten drei Jahre der Geschäftsbereiche CF und VN ist faktisch identisch, allerdings weist VN eine viel tiefere Erfolgsquote aus als CF.
%Auswertung Häufigkeit und Erfolgsquote pro Region
\begin{table}[H]
	\centering
	\caption{Erfolgsquote und Häufigkeitsverteilung pro Geschäftsbereich}
	\begin{tabular}{lrrrrrr}
		\textbf{BA}   & \multicolumn{1}{l}{\textbf{Erfolgsquote}} & \multicolumn{1}{l}{\textbf{Success}} & \multicolumn{1}{l}{\textbf{Fail}} & \multicolumn{1}{l}{\textbf{Fail [\%]}} & \multicolumn{1}{l}{\textbf{Total}} & \multicolumn{1}{l}{\textbf{Total [\%]}} \\ \hline
		CF    & 2.8   & 118   & 42    & 26.3\% & 160   & 16.6\% \\
		DC    & 5.6   & 96    & 17    & 15.0\% & 113   & 11.7\% \\
		GD    & 2.3   & 7     & 3     & 30.0\% & 10    & 1.0\% \\
		GL    & \textbf{1.2}  & 39    & 32    & 45.1\% & 71    & 7.3\% \\
		GM    & \textbf{1.9}   & 226   & 122   & 35.1\% & 348   & 36.0\% \\
		LO    & 1.4   & 30    & 21    & 41.2\% & 51    & 5.3\% \\
		SR    & 5.0   & 35    & 7     & 16.7\% & 42    & 4.3\% \\
		TP    & NA      & 8     & 0     & 0.0\% & 8     & 0.8\% \\
		VN    & \textbf{1.4}   & 95    & 68    & 41.7\% & 163   & 16.9\% \\\hline
		\textbf{Total } & \textbf{2.1} & \textbf{654} & \textbf{312} & \textbf{32.3\%} & \textbf{966} & \textbf{100.0\%} \\
	\end{tabular}%
	\label{fba}%
\end{table}%
Bei der Auswertung der Geschäftseinheiten auf Basis der Tabelle \ref{fbabu} für GL, GM und VN, konnte festgestellt werden, dass die kleineren BU's von GM eine verhältnismässig tiefe Erfolgsquote hatten. Dennoch wird das Verhältnis zwischen erfolgreichen und nicht-erfolgreichen Projekten fast ausschliesslich durch IM, die grösste BU in GM bestimmt. In der Business Area VN, sind die Erfolgsquoten mit Ausnahme von PN und OL grundsätzlich tief. Die Business Unit Grain Storage determiniert die Geschäftsbereichserfolgsquote von GL.
%Auswertung BU und BA
\begin{table}[htpb]
	\centering
	\caption{Erfolgsquote und Häufigkeitsverteilung pro Geschäftseinheit für GL, GM und VN}
	\begin{tabular}{llrrrrr}
		\textbf{BA} & \textbf{BU} & \multicolumn{1}{l}{\textbf{Erfolgsquote}} & \multicolumn{1}{l}{\textbf{Success}} & \multicolumn{1}{l}{\textbf{Fail}} & \multicolumn{1}{l}{\textbf{Fail [\%]}} & \multicolumn{1}{l}{\textbf{Total}} \\\hline
		GL    & GC    & NA    & 1     & 0     & 0.0\% & 1 \\
		GL    & GS    & 1.2   & 36    & 29    & 44.6\% & 65 \\
		GL    & MT    & 0.7   & 2     & 3     & 60.0\% & 5 \\\hline
		\textbf{GL} &  & \textbf{1.2} & \textbf{39} &\textbf{32} & \textbf{45.1\%} & \textbf{71}\\
		&       &       &       &       &        &   \\
		GM    & BA    & 1.5   & 17    & 11    & 39.3\% & 28 \\
		GM    & BR    & 0.9   & 12    & 14    & 53.8\% & 26 \\
		GM    & IM    & 2.1   & 185   & 87    & 32.0\% & 272 \\
		GM    & SM    & 1.2   & 12    & 10    & 45.5\% & 22 \\\hline
		\textbf{GM} &  & \textbf{1.9} & \textbf{226} &\textbf{122} & \textbf{35.1\%} & \textbf{348}\\
		&       &       &       &       &        &   \\
		VN    & AG    & 0.8   & 11    & 14    & 56.0\% & 25 \\
		VN    & FE    & 1.2   & 27    & 22    & 44.9\% & 49 \\
		VN    & NU    & 1.3   & 27    & 21    & 43.8\% & 48 \\
		VN    & OL    & 2.0   & 6     & 3     & 33.3\% & 9 \\
		VN    & PN    & 3.0   & 24    & 8     & 25.0\% & 32 \\\hline
		\textbf{VN} &  & \textbf{1.4} & \textbf{95} &\textbf{68} & \textbf{41.7\%} & \textbf{163}\\
	\end{tabular}%
	\label{fbabu}%
\end{table}%
\newline Im Regionen-BA Split der Tabelle \ref{tab:fregba} sind für diejenigen Regionen mit den niedrigsten Erfolgsquoten, EU und NAM jene BA's mit den niedrigsten Erfolgsquoten zu finden. Die Kombination EU-GM, EU-GL, EU-VN mit den tiefen Erfolgsquoten machen knapp 30\% ($(50+23+26)/312$) aller Fail-Projekte aus. Die niedrige Erfolgsquote von NAM stammt vor allem aus VN- und GM-Projekten, wobei VN noch vor GM weniger gut abschneidet.
% Auswertung BA-Region
\begin{table}[H]
	\centering
	\caption{Erfolgsquote und Häufigkeitsverteilung Regionen-BA für EU und North\_Ame}
	\begin{tabular}{llrrrrr}
		\textbf{Region} & \textbf{BA}    & \multicolumn{1}{l}{\textbf{Erfolgsquote}} & \multicolumn{1}{l}{\textbf{Success}} & \multicolumn{1}{l}{\textbf{Fail}} & \multicolumn{1}{l}{\textbf{Fail [\%]}} & \multicolumn{1}{l}{\textbf{Total}} \\\hline
		EU    & CF    & 1.9   & 58    & 31    & 34.8\% & 89 \\
		EU    & DC    & 5.0   & 45    & 9     & 16.7\% & 54 \\
		EU    & GD    & NA    & 2     & 0     & 0.0\% & 2 \\
		EU    & \textbf{GL}    & 1.0   & 24    & 23    & 48.9\% & 47 \\
		EU    & \textbf{GM}  & 1.2   & 58    & 50    & 46.3\% & 108 \\
		EU    & LO    & 2.5   & 10    & 4     & 28.6\% & 14 \\
		EU    & SR    & 2.5   & 5     & 2     & 28.6\% & 7 \\
		EU    & \textbf{VN}     & 1.5   & 38    & 26    & 40.6\% & 64 \\\hline
		North\_Ame & CF    & 2.0   & 10    & 5     & 33.3\% & 15 \\
		North\_Ame & DC    & 1.0   & 2     & 2     & 50.0\% & 4 \\
		North\_Ame & GL    & 1.0   & 1     & 1     & 50.0\% & 2 \\
		North\_Ame & \textbf{GM}   & 1.5   & 24    & 16    & 40.0\% & 40 \\
		North\_Ame & LO    & 4.0   & 4     & 1     & 20.0\% & 5 \\
		North\_Ame & SR    & 1.0   & 1     & 1     & 50.0\% & 2 \\
		North\_Ame & \textbf{VN}  & 1.0   & 12    & 12    & 50.0\% & 24 \\
	\end{tabular}%
	\label{tab:fregba}%
\end{table}%
Zusammenfassend lässt sich aussagen, dass ungefähr 60\% ($(122+68)/312$ respektive $(145+38)/312$) der Fail-Projekte entweder in den Geschäftsbereichen VN und GM oder in den Regionen EU und NAM liegen. Zudem wird die Erfolgsquote aller Projekte zu 30\% durch europäische Projekte von den Geschäftsbereichen GM, GL und VN bestimmt wird.
%
%Analyse der Kosten
%
\paragraph{Kosten} Das Umsatzvolumen (TOBud) soll Aufschluss über die Grösse und Wichtigkeit eines Projekts geben. Die zugrundeliegende Prämisse postuliert, dass Projekte mit höherem Umsatzvolumen risikoreicher sind und deshalb eher unter Budget beendet werden. Die Gegenhypothese unterstellt, dass grössere Projekte (hoher TOBud) relativ mehr Beachtung erhalten, da sie den Erfolg eines Geschäftsbereich respektive einer Region mehr beeinflussen als kleinere Projekte. Deshalb liege der Fokus auf der Einhaltung der Budgetvorgaben.
%
%Histogram of TOBud_cat
%
\begin{figure}[H]
	\centering
	\includegraphics[width=5cm]{test.pdf}
	\caption{Histogram Umsatzkategorie}
	\label{fig: htobudcat}
	\text{1	= [13.2,500), 2 = [500,1e+03), 3 = [1e+03,1.5e+03), 4 = [1.5e+03,2e+03), 5 = [2e+03,2.5e+03)}
	\text{6 =	[2.5e+03,3e+03), 7 = [3e+03,3.5e+03), 8 = [3.5e+03,4e+03), 9 = [4e+03,4.5e+03)}
	\text{10 = [4.5e+03,5e+03), 11 = [5e+03,1e+04), 12 = [1e+04,3.42e+04)}
\end{figure}
Die Verteilung Umsatzvolumen mit Hilfe der Umsatzkategorie ist linksschief und zeigt dass der Grossteil der Projekte ein Umsatzvolumen von weniger als 10 Mio. CHF haben. Die Anzahl Fail-Projekte konzentriert sich folglich in den vier untersten Kategorien (vgl. auch Tabelle \ref{tab:ftobudcat}).
%%
%%Relative und Asolute Kostenabweichung
%%
\newline\newline\textit{Absolute und relative Kostenabweichung (CostActBud):} Die absoluten und relativen Abweichungen zwischen den aktuellen und den budgetierten Kosten sind direkt mit dem Erfolgskriterium korreliert und sind erwartungsgemäss für Fail-Projekte höher. 
% Mean absolute Kostenabweichungen
\begin{table}[H]
	\centering
	\caption{Arithmetisches Mittel der absoluten Kostenabweichungen pro Kostenart [TCHF]}
	\begin{tabular}{lrrrrr}
		\textbf{Success} & \multicolumn{1}{l}{\textbf{Total Cost}} & \multicolumn{1}{l}{\textbf{MS}} & \multicolumn{1}{l}{\textbf{ME}} & \multicolumn{1}{l}{\textbf{PA}} & \multicolumn{1}{l}{\textbf{IS}}
		\\\hline
		FALSE & -197  & -65   & -41   & -23   & -71 \\
		TRUE  & 176   & 73    & -3    & -1    & -11 \\
	\end{tabular}%
	\label{mcostabs}%
\end{table}%
% Mean relative Kostenabweichung
\begin{table}[H]
	\centering
	\caption{Arithmetisches Mittel der relativen Kostenabweichungen pro Kostenart [\%]}
	\begin{tabular}{lrrrrr}
		\textbf{Success} & \multicolumn{1}{l}{\textbf{Total Cost}} & \multicolumn{1}{l}{\textbf{MS}} & \multicolumn{1}{l}{\textbf{ME}} & \multicolumn{1}{l}{\textbf{PA}} & \multicolumn{1}{l}{\textbf{IS}}
		\\\hline
		FALSE & 12    & 9     & 206   & 79    & 93 \\
		TRUE  & -10   & 4     & 56    & 17    & 28 \\
	\end{tabular}%
	\label{mcostrel}%
\end{table}%
Die Tabellen \ref{mcostabs} und \ref{mcostrel} zeigen, dass die durchschnittlichen Abweichung der Projektkosten von nicht-erfolgreichen Projekten in absoluten und relativen Grössen über derjenigen der Success-Projekte liegt. Dabei fällt vor allem der Durchschnittswert der relativen Differenz Kostenabweichung der ME-Kosten auf.
\newline\newline\textit{Relative Anteile des Kostenbudgets:} Die Zusammensetzung der budgetierten Projektkosten soll Hinweise zur Projektart aufzeigen und ob sie sich zwischen den zwei Projektgruppen unterscheidet.
\begin{table}[htbp]
	\centering
	\caption{Arithmetisches Mittel der relativen Anteile am Gesamtkostenbudget je Kostenart [\%]}
	\begin{tabular}{lrrrr}
		\textbf{Success} & \multicolumn{1}{l}{\textbf{BudMSTot}} & \multicolumn{1}{l}{\textbf{BudMETot}} & \multicolumn{1}{l}{\textbf{BudPATot}} & \multicolumn{1}{l}{\textbf{BudISTot}} \\\hline
		FALSE & 67.1  & 6.2   & 5.9   & 7.7 \\
		TRUE  & 67.9  & 5.4   & 5.1   & 6.8 \\
	\end{tabular}%
	\label{mbudtot}%
\end{table}%
Die dargestellten Mittelwerte in der Tabelle \ref{mbudtot} liegen für nicht-erfolgreiche Projekte bei ME, PA und IS etwas höher. In der vorherigen Auswertung der durchschnittlichen Kostenabweichung wurde exakt bei diesen Projektphasen für Fail-Projekte deutlich höhere Werte festgestellt.
%
%Auswertung Nachlieferung
%
\newline\newline\textit{Nachlieferung SUCostTO:} Nachlieferungen verursachen zusätzliche Kosten, die aufgrund der zeitlichen Verzögerung bei der Fabrikation der Maschine oder Installation entstehen können. Es wird spekuliert, dass der Anteil der Kosten aus Nachlieferungen im Verhältnis zum Umsatzbudget bei Fail-Projekten höher ist als bei Success-Projekten.
%Mean SU Cost
\begin{table}[H]
	\centering
	\caption{Arithmetisches Mittel der SUCostTO [\%]}
	\begin{tabular}{lr}
		\textbf{Success} & \multicolumn{1}{l}{\textbf{SUCostTO}} \\\hline
		FALSE & -0.81 \\
		TRUE  & -0.36 \\
	\end{tabular}%
	\label{msu}%
\end{table}%
Die Auswertung (s. Tabelle \ref{msu}) des arithmetischen Mittels der prozentualen SU-Kosten am Umsatz bestätigt diese erwartete Vermutung.
%Mean SU Cost pro Umsatzkategorie
\begin{table}[H]
	\centering
	\caption{Arithmetisches Mittel der SUCostTO [\%] pro TO-Kategorie}
	\begin{tabular}{llr}
		\textbf{Success} & \textbf{TOBud\_Cat} & \multicolumn{1}{l}{\textbf{SUCostTO}} \\\hline
		FALSE & [13.2,500) & -0.93 \\
		FALSE & [500,1e+03) & -0.69 \\
		FALSE & [1e+03,1.5e+03) & -0.47 \\
		FALSE & [1.5e+03,2e+03) & -0.72 \\
		FALSE & [2e+03,2.5e+03) & -0.57 \\		
		FALSE & [2.5e+03,3e+03) & -0.82 \\
		FALSE & [3e+03,3.5e+03) & -0.31 \\
		FALSE & [3.5e+03,4e+03) & -0.69 \\
		FALSE & [4e+03,4.5e+03) & -0.52 \\
		FALSE & [4.5e+03,5e+03) & -6.88 \\
		FALSE & [5e+03,1e+04) & -0.82 \\
		FALSE & [1e+04,3.42e+04) & -0.36 \\
	\end{tabular}%
	\label{msutocat}%
\end{table}%
Die Mittelwertauswertung von SUCostTO in der Tabelle \ref{msutocat} pro TO-Kategorie zeigt, dass für Projekte mit einem Umsatzvolumen zwischen 13.2 TCHF und 500 TCHF die Nachlieferungskosten in Relation zum Umsatz am höchsten war. Der Wert 6.9\% kann als Anomalie betrachtet werden und ist auf ein Projekt mit einem SUCostTO-Wert von ca. 40\% als Einzelfall zu betrachten.
%
%Auswertung DeltaLastFCAct
%
\newline\newline\textit{Abweichung zwischen dem letzten FC und Act DeltaLastFCAct:} Tendenziell wird die Anpassung des Forecast für die Projektkosten bei erwarteten Mehrkosten möglichst lange hinausgezögert. Einerseits kann mit diesem Vorgehen, die Erklärungsdirektive umgangen werden und anderseits besteht wahrscheinlich die Hoffnung, dass die Projektkosten sich wieder normalisieren. Deshalb wird erwartet, dass die Differenz zwischen der letzten FC-Anpassung und den tatsächlichen Kosten bei Fail-Projekten höher war. 
%Tabelle Mean DeltaLastFCAct  
\begin{table}[H]
	\centering
	\caption{Arithmetisches Mittel DeltaLastFCAct für MS, ME, PA und IS [TCHF]}
	\begin{tabular}{lrrrrr}
		\textbf{Success} & \multicolumn{1}{c}{\textbf{Total FC}} & \multicolumn{1}{c}{\textbf{MS}} & \multicolumn{1}{c}{\textbf{ME}} & \multicolumn{1}{c}{\textbf{PA}} & \multicolumn{1}{c}{\textbf{IS}}
		\\\hline
		FALSE & -490.54 & -445.53 & 7.48 & -12.87 & -14.52 \\
		TRUE  & -436.41 & -454.24 & 7.93 & -13.41 & -6.49\\
	\end{tabular}%
	\label{mdeltalastfcact}%
\end{table}%
Die Ergebnisse der Tabelle \ref{mdeltalastfcact} bestätigen diese Vermutung. Die durchschnittliche Differenz bei den IS-Kosten war für Fail-Projekte doppelt so hoch.
%%
%%
%%Auswertung FF-Variablen
%%
%%
\paragraph{Fulfillment} Der bedeutenste Einflussfaktor im Projektmanagement ist der Projektmanager selbst. Die Evaluation der realisierten Projekte pro Projektmanager inklusive der Erfolgsquote hat ergeben, dass die 966 Projekte von 301 unterschiedlichen Projektmanager abgewickelt wurde. 145 Projektmanager haben ihre Projekte aussschliesslich erfolgreich beendet, wohingegen gerade einmal 45 PM nur unzureichend Projekte abgewickelt hat (s. Tablle \ref{tab:pmno} im Anhang).
%
%Auswertung Projektmanager Change
%
\newline\newline\textit{Projektmanager:} Der Wechsel des Projektmanagers wird mit konfligierende Verhältnisse zwischen den Vertragsparteien assoziiert, weshalb hypothetisch vermutet wird, dass Fail-Projekte eher mit einem PMChange einhergehen. 
\begin{table}[H]
	\centering
	\caption{Häufigkeitsverteilung PMChange}
	\begin{tabular}{lrrrrr}
		\textbf{PMChange} & \multicolumn{1}{l}{\textbf{Success}} & \multicolumn{1}{l}{\textbf{Fail}} & \multicolumn{1}{l}{\textbf{Fail [\%]}} & \multicolumn{1}{l}{\textbf{Total}} &
		\multicolumn{1}{l}{\textbf{Total [\%]}} \\\hline
		no    & 628   & 295   & 32.0\% & 923 & 96\% \\
		yes   & 26    & 17    & 39.5\% & 43  & 4\% \\\hline
		\textbf{Total} & \textbf{654} & \textbf{312} & \textbf{32.3\%} & \textbf{966} & \textbf{100.0\%} \\
	\end{tabular}%
	\label{pmchange}%
\end{table}%
Insgesamt wurden 43 Projekte mit einem Wechsel des Projektmanagers über die letzten drei Jahre abgewickelt, wie der Tabelle \ref{pmchange} zu entnehmen ist. Davon sind 17 gescheitert und 26 wurden erfolgreich abgeschlossen. Die Anzahl Projektmanager (NoPM) ist direkt mit der Variable PMChange verbunden und weist wenig Informationsgehalt auf. Es gab in der Stichprobe sechs Projekte, bei denen der PM zweimal ausgetauscht wurde, davon sind fünf Projekte gescheitert.
%
% Auswertung Alter und Tenuer PM
%
\newline Das Alter (PMAge2) und die Dienstjahre (PMTen2) des Projektmanagers sind Proxyvariablen für die Lebens- und Berufserfahrung sowie die Kenntnisse der Bühlerwelt. Erfahrenere (ältere) sowie langjährige Mitarbeitende müssten mehr Erfolg im Projektmanagement haben, da sie mehr Praxiserfahrung mit der Bühler-Welt einerseits und dem Projektmanagement anderseits haben sollten. Die Durchschnittswerte des Alter und der Dienstjahre in der Tabelle \ref{ageten} im Anhang sind für Fail-Projekte und Success-Projekte faktisch identisch. Sie betragen gerundet 40 beziehungsweise 12 Jahre.
% Table generated by Excel2LaTeX from sheet 'fagecat'
\begin{table}[htbp]
	\centering
	\caption{Erfolgsquote und Häufigkeitsverteilung pro Alterskategorie}
	\begin{tabular}{lrrrrrr}
		\textbf{Cat\_age} & \multicolumn{1}{l}{\textbf{Erfolgsquote}} & \multicolumn{1}{l}{\textbf{Success}} & \multicolumn{1}{l}{\textbf{Fail}} & \multicolumn{1}{l}{\textbf{Fail [\%]}} & \multicolumn{1}{l}{\textbf{Total}} & \multicolumn{1}{l}{\textbf{Total [\%]}} \\\hline
		20-25 & 2.5   & 15    & 6     & 28.6\% & 21    & 2.2\% \\
		26-30 & 2.3   & 130   & 57    & 30.5\% & 187   & 19.4\% \\
		31-35 & 3.0   & 130   & 44    & 25.3\% & 174   & 18.0\% \\
		36-40 & 2.3   & 97    & 43    & 30.7\% & 140   & 14.5\% \\
		41-45 & \textbf{1.8} & 105   & 57    & 35.2\% & 162   & 16.8\% \\
		46-50 & 2.3   & 65    & 28    & 30.1\% & 93    & 9.6\% \\
		51-55 & \textbf{1.3} & 69    & 52    & 43.0\% & 121   & 12.5\% \\
		56-60 & \textbf{1.4} & 26    & 19    & 42.2\% & 45    & 4.7\% \\
		61-63 & 2.8   & 17    & 6     & 26.1\% & 23    & 2.4\% \\\hline
		\textbf{Total} & \textbf{2.1} & \textbf{654} & \textbf{312} & \textbf{32.3\%} & \textbf{966} & \textbf{100.0\%} \\
	\end{tabular}%
	\label{fagecat}%
\end{table}%
Die Tabelle zeigt, dass in den Alterskategorien (51-55) und (56-60) relativ am meisten nicht-erfolgreiche' Projekte. Folglich liegt ein Indiz für die Gegenhypothese vor, wobei angemerkt werden muss, dass Verteilung der Projekte diese Quote beeinflussen kann. Beispielsweise könnte den erfahreneren Mitarbeitenden, die herausfordernden Projekte zugewiesen werden, da sie über mehr fundiertes Wisse im Projektmanagement verfügen.
%
% Auswertung Lead SAS
%
\newline\newline\textit{Lead SAS}
Die Bühler AG unterscheidet zwei Typen von Lead SAS: die Lead SAS für das gesamte Projekt (LeadSASPr) und die Lead SAS für die Projektabwicklung (LeadSASFF). Lead legt in dieser Hinsicht die Verantwortung fest. Da die Verantwortlichkeit auf zwei Gesellschaften aufgeteilt werden kann, wurde zudem eruiert, ob sich die LeadSASFF von der LeadSASPr unterscheidet (LeadSAS.PrFF). Es wird postuliert, dass einige Gesellschaften Projekte erfolgreicher managen. Zudem wird angenommen, dass bei zwei Verantwortungsparteien, die Erfolgsquote höher sein muss, da die jeweiligen Aufgabenbereich besser fokussiert werden kann.
% Table frequency Lead SASPr
\begin{table}[H]
	\centering
	\caption{Erfolgsquoten und Häufigkeitsverteilung pro LeadSASPr}
	\begin{tabular}{lrrrrrr}
		\textbf{LeadSASPr} & \multicolumn{1}{l}{\textbf{Erfolgsquote}} & \multicolumn{1}{l}{\textbf{Success}} & \multicolumn{1}{l}{\textbf{Fail}} & \multicolumn{1}{l}{\textbf{Fail [\%]}} & \multicolumn{1}{l}{\textbf{Total}} & \multicolumn{1}{l}{\textbf{Total [\textbackslash{}\%]}} \\\hline
		BJHB  & 0.6   & 10    & 17    & 63.0\% & 27    & 2.8\% \\
		\textbf{BBS}   & 0.8   & 27    & 35    & 56.5\% & 62    & 6.4\% \\
		\textbf{BMIL}  & 0.9   & 13    & 15    & 53.6\% & 28    & 2.9\% \\
		BRAL  & 0.9   & 12    & 13    & 52.0\% & 25    & 2.6\% \\
		\textbf{BLOA}  & 1.0   & 19    & 19    & 50.0\% & 38    & 3.9\% \\
		BPRI  & 1.0   & 2     & 2     & 50.0\% & 4     & 0.4\% \\
		\textbf{BBAR}  & 1.1   & 8     & 7     & 46.7\% & 15    & 1.6\% \\
		BJOI  & 1.2   & 19    & 16    & 45.7\% & 35    & 3.6\% \\
		BSSE  & 1.2   & 27    & 22    & 44.9\% & 49    & 5.1\% \\
		BMIN  & 1.2   & 31    & 25    & 44.6\% & 56    & 5.8\% \\
		BBIN  & 1.3   & 20    & 15    & 42.9\% & 35    & 3.6\% \\
		BPAR  & 1.3   & 16    & 12    & 42.9\% & 28    & 2.9\% \\
		BBAI  & 2.0   & 2     & 1     & 33.3\% & 3     & 0.3\% \\
		BMAD  & 2.1   & 15    & 7     & 31.8\% & 22    & 2.3\% \\
		BMEX  & 2.5   & 5     & 2     & 28.6\% & 7     & 0.7\% \\
		BLON  & 3.0   & 3     & 1     & 25.0\% & 4     & 0.4\% \\
		BUZ   & 3.6   & 310   & 86    & 21.7\% & 396   & 41.0\% \\
		BLOC  & 5.5   & 11    & 2     & 15.4\% & 13    & 1.3\% \\
		BYOK  & 5.5   & 11    & 2     & 15.4\% & 13    & 1.3\% \\
		BCHI  & 5.7   & 34    & 6     & 15.0\% & 40    & 4.1\% \\
		BBAN  & 7.3   & 51    & 7     & 12.1\% & 58    & 6.0\% \\
		BDAG  & \multicolumn{1}{r}{NA} & 6     & 0     & 0.0\% & 6     & 0.6\% \\
		BTEH  & \multicolumn{1}{r}{NA} & 2     & 0     & 0.0\% & 2     & 0.2\% \\\hline
		\textbf{Total} &       & \textbf{654} & \textbf{312} & \textbf{32.3\%} & \textbf{966} & \textbf{100.0\%} \\
	\end{tabular}%
	\label{fleadsas}%
\end{table}%
Der Vergleich Erfolgsquoten pro SAS der Tabelle \ref{fleadsas} zeigt, dass die europäischen Gesellschaften im Vergleich tiefere Erfolgsraten haben. BJHB, BBS, BMIL und BRAL, die zusammen 15\% aller Projekte abwickeln, haben eine unterdurchschnittliche Erfolgsbilanz.
%Tabelle LeadSAS.PrFF
\begin{table}[H]
	\centering
	\caption{Häufigkeitsverteilung LeadSAS.PrFF}
	\begin{tabular}{lrrrrr}
		\textbf{LeadSAS.PrFF} & \multicolumn{1}{l}{\textbf{Success}} & \multicolumn{1}{l}{\textbf{Fail}} & \multicolumn{1}{l}{\textbf{Fail [\%]}} & \multicolumn{1}{l}{\textbf{Total}} & \multicolumn{1}{l}{\textbf{Total [\%]}}
		\\\hline
		identisch    & 569   & 296   & 34.2\% & 865 & 89.5\% \\
		verschieden   & 85    & 16    & 15.8\% & 101 & 10.5\% \\\hline
		\textbf{Total} & \textbf{654} & \textbf{312} &       & \textbf{966} \\
	\end{tabular}%
	\label{fleadsasprff}%
\end{table}%
Die Resultate der Tabelle \ref{fleadsasprff} implizieren, dass der überwiegende Anteil der Projekte eine Verantwortungspartei hatte. Von den 101 Projekten mit geteilter Projektverantwortung wurden lediglich 16\% mit einem prozentualen DB1Act unter Budget abgeschlossen. Das Ergebnis kann folglich als Indiz zu Gunsten der Hypothese gewertet werden.
%%
%%
%%Zeit
%%
%%
\paragraph{Zeit} Die Beurteilung des Zeitmanagement hängt von der Einhaltung des vereinbarten Liefertermins ab. Mehrkosten und Zeitverzug gehen oftmals einher, weshalb unterstellt wird, dass Fail-Projekte den vereinbarten Projektabschluss nicht einhalten konnten. Ferner soll ergründet werden, ab welchem Zeitpunkt respektive Milestone der Zeitverzug üblicherweise eintritt.
%Table durchschnittlicher PrTimeDelay und pro MS
\begin{table}[H]
	\centering
	\caption{Arithmetisches Mittel der Projektlaufzeit und Zeitverzögerung [Monate]}
	\begin{tabular}{lrrrrrrr}
		\textbf{Success} & \multicolumn{1}{l}{\textbf{Base}} & \multicolumn{1}{l}{\textbf{Act}} & \multicolumn{1}{l}{\textbf{Delay}} & \multicolumn{1}{l}{\textbf{MS2}} & \multicolumn{1}{l}{\textbf{MS8}} & \multicolumn{1}{l}{\textbf{MS10}} & \multicolumn{1}{l}{\textbf{MS11}} \\ \hline
		TRUE  & 11.9  & 17.3  & -5.4  & -0.1  & -2.0  & -5.0  & -5.5 \\
		FALSE & 11.4  & 18.7  & -7.2  & -0.2  & -1.7  & -5.7  & -7.3 \\
	\end{tabular}%
	\label{mtime}%
\end{table}%
Die durchschnittliche budgetierte Projektlaufzeit unterscheidet sich zwischen erfolgreichen und  nicht-erfolgreichen Projekten kaum wohingegen die effektive Projektlaufzeit der Fail-Projekte einen Monat mehr betrug (s. Tabelle \ref{mtime}). Beide Projektgruppen konnten im Durchschnitt den Liefertermin nicht einhalten, wobei die Success-Projekte ca. zwei Monate weniger zeitverzögert waren. Die Termineinhaltung beim MS2 Concept approved bewegt sich im vernachlässigbaren Bereich. Demgegenüber steigt der durchschnittliche Zeitverzug nach MS8 Documented auf zwei, nach MS10 Takeover auf fünf bis sechs Monate und liegt bei Projektabschluss (MS11) zwischen gerundet sechs und sieben Monaten.
%Table Häufigkeit Delay False/Success
\begin{table}[H]
	\centering
	\caption{Häufigkeitsverteilung zeitverzögerter Projekte}
	\begin{tabular}{lrrrrr}
		\textbf{Delay} & \multicolumn{1}{l}{\textbf{Success}} & \multicolumn{1}{l}{\textbf{Fail}} & \multicolumn{1}{l}{\textbf{Fail [\%]}} & \multicolumn{1}{l}{\textbf{Total}} & \multicolumn{1}{l}{\textbf{Total [\%]}} \\ \hline
		TRUE  & 515   & 268   & 34.2\% & 783   & 81.1\% \\
		FALSE & 139   & 44    & 24.0\% & 183   & 18.9\% \\\hline
		\textbf{Total} & \textbf{654} & \textbf{312} & \textbf{32.3\%} & \textbf{966} & \textbf{100.0\%} \\
	\end{tabular}%
	\label{fdelay}%
\end{table}%
Insgesamt wurde gemäss der Tabelle \ref{fdelay} der Liefertermin bei 783 Projekten nicht eingehalten, wobei das Verhältnis zwischen erfolgreichen und nicht-erfolgreichen zwei zu eins beträgt. Der prozentuale Anteil der Projekte, bei denen die Anlage pünktlich an den Kunden übergeben werden konnte, beträgt 20. 
%Table Häufigkeit Delay pro MS
\begin{table}[H]
	\centering
	\caption{Absolute Häufigkeitsverteilung zeitverzögerter Projekte pro Milestone}
	\begin{tabular}{lrrrr}
		\textbf{Success} & \multicolumn{1}{l}{\textbf{DelayMS2}} & \multicolumn{1}{l}{\textbf{DelayMS8}} & \multicolumn{1}{l}{\textbf{DelayMS10}} & \multicolumn{1}{l}{\textbf{DelayMS11}} \\\hline
		FALSE & 43    & 209   & 275   & 270 \\
		TRUE  & 75    & 408   & 532   & 516 \\\hline
		\textbf{Total} & \textbf{118} & \textbf{717} & \textbf{807} &  \textbf{786}
	\end{tabular}%
	\label{fdelayms}%
\end{table}
Die Tabelle \ref{fdelayms} zeigt, dass die Mehrheit der Projekte bei MS2 noch im Zeitplan agierte und nach MS8 bereits hinter dem vereinbarten Liefertermin lag. Mittels Dummyvariablen pro Milestone wurde ausgewertet, ob sich ein anfängliche Verspätung durch die Projektlaufzeit durchzieht.
% Table generated by Excel2LaTeX from sheet 'time_sukz'
\begin{table}[H]
	\centering
	\caption{Absolute Häufigkeitsverteilung von Kombinationen der Zeitverzögerung}
	\begin{tabular}{lrrrrrr}
		\textbf{Success} & \multicolumn{1}{l}{\textbf{Delay}} & \multicolumn{1}{l}{\textbf{DelayMS2}} & \multicolumn{1}{l}{\textbf{DelayMS8}} & \multicolumn{1}{l}{\textbf{DelayMS10}} & \multicolumn{1}{l}{\textbf{DelayMS11}} & \multicolumn{1}{l}{\textbf{freq}} \\
		FALSE & \multicolumn{1}{l}{TRUE} & 0     & 1     & 1     & 1     & 147 \\
		FALSE & \multicolumn{1}{l}{TRUE} & 0     & 0     & 1     & 1     & 64 \\
		FALSE & \multicolumn{1}{l}{TRUE} & 1     & 1     & 1     & 1     & 33 \\
		FALSE & \multicolumn{1}{l}{TRUE} & 0     & 0     & 0     & 1     & 10 \\
		FALSE & \multicolumn{1}{l}{TRUE} & 0     & 1     & 0     & 1     & 7 \\
		FALSE & \multicolumn{1}{l}{TRUE} & 1     & 0     & 1     & 1     & 5 \\
		FALSE & \multicolumn{1}{l}{TRUE} & 0     & 0     & 0     & 0     & 1 \\
		FALSE & \multicolumn{1}{l}{TRUE} & 1     & 0     & 0     & 1     & 1 \\\hline
		\textbf{Total} &       &       &       &       &       & \textbf{268} \\
		TRUE  & \multicolumn{1}{l}{TRUE} & 0     & 1     & 1     & 1     & 290 \\
		TRUE  & \multicolumn{1}{l}{TRUE} & 0     & 0     & 1     & 1     & 126 \\
		TRUE  & \multicolumn{1}{l}{TRUE} & 1     & 1     & 1     & 1     & 42 \\
		TRUE  & \multicolumn{1}{l}{TRUE} & 0     & 1     & 0     & 1     & 19 \\
		TRUE  & \multicolumn{1}{l}{TRUE} & 1     & 0     & 1     & 1     & 17 \\
		TRUE  & \multicolumn{1}{l}{TRUE} & 0     & 0     & 0     & 1     & 16 \\
		TRUE  & \multicolumn{1}{l}{TRUE} & 1     & 1     & 0     & 1     & 2 \\
		TRUE  & \multicolumn{1}{l}{TRUE} & 0     & 0     & 1     & 0     & 1 \\
		TRUE  & \multicolumn{1}{l}{TRUE} & 1     & 0     & 0     & 0     & 1 \\
		TRUE  & \multicolumn{1}{l}{TRUE} & 1     & 0     & 0     & 1     & 1 \\\hline
		\textbf{Total} &       &       &       &       &       & \textbf{515} \\
	\end{tabular}%
	\label{fdleay}%
\end{table}%
Die meisten Projekte hatten die Eigenschaft, bei einer Verspätung im MS8 ebenso auch im MS11 verspätet zu sein. Die zweithäufigste Gruppe war diejenige, die erstmalig ab dem MS10 den Liefertermin bis zum Projektabschluss nicht mehr einhalten konnte. Erst an dritter Stelle folgt die Gruppe, die bereits ab MS2 bis MS11 hinter dem Zeitplan lag. Auf Basis der Tabellen lassen sich jedoch keine Aussagen machen, ob bei einer Nichteinhaltung des vereinbarten Termin, das Projekt zwangsläufig unter Budget abschliessen wird. 
%%
%%
%% Auswertung SQ
%%
%%
\paragraph[short title]{Sales and Quotation} Die Einflussdeterminanten des SQ-Prozess sind einerseits der Stand im Bezug auf das OR-Budget bei Projektabschluss und die Erfahrung (AMAge2) sowie Dienstjahre (AMTen2) des Verkaufsmanager. Allerdings konnten letztere aufgrund fehlender Datensätze nicht ausgewertet werden. Grundsätzlich wird vermutet, dass ein Budgetdruck im Zeitpunkt des Verkaufsabschlusses, den Verkauf von risikoreicheren Projekten begünstigt. 
% Table generated by Excel2LaTeX from sheet 'sq mean'
\begin{table}[H]
	\centering
	\caption{Arithmetisches Mittel des BudGapOR der BU und Region}
	\begin{tabular}{lrr|rr}
		\textbf{Success} & \multicolumn{1}{l}{\textbf{BU [TCHF]}} & \multicolumn{1}{l}{\textbf{BU [\%]}} & \multicolumn{1}{l}{\textbf{Region [TCHF]}} & \multicolumn{1}{l}{\textbf{Region [\%]}} \\\hline
		FALSE & -7'244.2 & -11.3 & -20'845.5 & -9.3 \\
		TRUE  & -1'207.2 & -4.0  & -14'886.9 & -6.8 \\
	\end{tabular}%
	\label{msq}%
\end{table}%
Die mittlere Abweichung des OR vom Budget des Geschäftseinheit und der Region der Tabelle \ref{msq} waren für Fail-Projekte in absoluten und relativen Grössen höher als für Success-Projekte.  
%%
%%
%% Auswertung Komplexität
%%
%%
\newline\newline\textbf{Komplexität:} Die Komplexität kann anhand unterschiedlicher Dimensionen gemessen werden wie zum Beispiel, dem Inhalt, den Zielen, den Beteiligten oder dem Umfeld. In der nachfolgenden Auswertung wurden die Anzahl Aufträge sowie involvierter Parteien bei den unterschiedlichen Projektphasen und Konsortien als Proxyvariablen für Komplexität untersucht. Die vorherrschende Auffassung bezüglich Komplexität, ist ihr negativer Einfluss auf den Projekterfolg.
%
%Auswertung Konsortium
%
Die Anzahl Projekte in der Stichprobe, die in einem Konsortium abgewickelt wurden, beträgt 78 , wovon 47 erfolgreich und 31 unter Budget abgeschlossen wurden, wie der Tabelle \ref{fcons} zu entnehmen ist. In EU und SAS\_BCHI wurden 51 Projekte im Konsortium abgewickelt.
% Table generated by Excel2LaTeX from sheet 'con_part_reg'
\begin{table}[H]
	\centering
	\caption{Absolute Häufigkeitsverteilung der Projekte mit Konsortium pro Region}
	\begin{tabular}{lrrrr}
		\textbf{ConPart} & \multicolumn{1}{l}{\textbf{Region}} & \multicolumn{1}{l}{\textbf{Success}} & \multicolumn{1}{l}{\textbf{Fail}} & \multicolumn{1}{l}{\textbf{Erfolgsquote}} \\\hline
		TRUE  & \multicolumn{1}{l}{East\_Asia} & 1     & 0     & NA \\
		TRUE  & \multicolumn{1}{l}{\textbf{EU}} & 13    & 14    & 0.9 \\
		TRUE  & \multicolumn{1}{l}{MEA\_Afr} & 9     & 7     & 1.3 \\
		TRUE  & \multicolumn{1}{l}{North\_Ame} & 2     & 1     & 2.0 \\
		TRUE  & \multicolumn{1}{l}{\textbf{SAS\_BCHI}} & 17    & 7     & 2.4 \\
		TRUE  & \multicolumn{1}{l}{South\_Ame} & 4     & 2     & 2.0 \\
		TRUE  & \multicolumn{1}{l}{South\_Asia} & 1     & 0     & NA \\\hline
		\textbf{Total} &       & \textbf{47} & \textbf{31} & \textbf{1.5} \\
	\end{tabular}%
	\label{fcons}%
\end{table}%
Die meisten Konsortium-Projekte in Europa stammen vom Geschäftsbereich GL wovon nur jedes zweite erfolgreich zu Ende geführt werden konnte (s. Tabelle \ref{consregba}). Demgegenüber stehen 11 Projekte mit Konsortium in Middle East \& Africa des Geschäftsbereichs GM, wovon ebenso die Hälfte erfolgreich abgeschlossen werden konnte. Die DC-Projekte weisen gegenüber den GM-Projekten in SAS\_BCHI die bessere Erfolgsbilanz aus.
% Table generated by Excel2LaTeX from sheet 'cons_par_reg_Ba'
\begin{table}[htbp]
	\centering
	\caption{Erfolgsquote und absolute Häufigkeit von ConPart für EU, MEA\_Afr und SAS\_BCHI}
	\begin{tabular}{llrrr}
		\textbf{Region} & \textbf{BA} & \multicolumn{1}{l}{\textbf{Success}} & \multicolumn{1}{l}{\textbf{Fail}} & \multicolumn{1}{l}{\textbf{Erfolgsquote}} \\\hline
		EU  & CF    & 5     & 0     & NA \\
		& DC    & 0     & 1     & 0.0 \\
		& GL    & 6     & 5     & 1.2 \\
		& GM    & 1     & 3     & 0.3 \\
		&VN    & 1     & 5     & 0.2 \\\hline
		\textbf{Total} &  & \textbf{13} & \textbf{14} & \textbf{0.9} \\
		&  &  & & \\
		MEA\_Afr &	GL    & 2     & 2     & 1.0 \\
		& 	GM    & 6     & 5     & 1.2 \\
		& VN    & 1     & 0     & NA \\\hline
		\textbf{Total} &  & \textbf{9} & \textbf{7} & \textbf{1.3} \\
		& & & & \\
		SAS\_BCHI & \multicolumn{1}{l}{CF} & 2     & 0     & NA \\
		& \multicolumn{1}{l}{DC} & 11    & 1     & 11.0 \\
		& \multicolumn{1}{l}{GM} & 4     & 6     & 0.7 \\\hline
		\textbf{Total} &       &  \textbf{17}   & \textbf{7}     & \textbf{2.4} \\
	\end{tabular}%
	\label{consregba}%
\end{table}%
%
%Auswertung Anzahl Aufträge
%
\newpage
Die Anzahl Aufträge pro Projekt (NoContr) kann ein Indiz für die Anzahl Komponenten der Maschine oder involvierter Parteien sein. In der Stichprobe variiert sie zwischen eins und zehn, wobei die Mehrheit aller Projekte einen einzigen Auftrag pflegte (s. Tabelle \ref{fnocontr}). Weitere 20\% aller Projekte werden mittels zwei Aufträgen abgewickelt. Die Erfolgsquote der Projekte mit zwei Verträgen liegt leicht über derjenigen von Projekten mit einem Auftrag. Allerdings wird das Verhältnis zwischen Fail- und Success-Projekten der gesamten Stichprobe durch Projekte mit einem Vertrag bestimmt.
% Table generated by Excel2LaTeX from sheet 'nocontr'
\begin{table}[H]
	\centering
	\caption{Erfolgsquote und Häufigkeitsverteilung von NoContr}
	\begin{tabular}{lrrrrrr}
		\textbf{NoContr} & \multicolumn{1}{l}{\textbf{Erfolgsquote}} & \multicolumn{1}{l}{\textbf{Success}} & \multicolumn{1}{l}{\textbf{Fail}} & \multicolumn{1}{l}{\textbf{Fail [\%]}} & \multicolumn{1}{l}{\textbf{Total}} & \multicolumn{1}{l}{\textbf{Total [\%]}} \\\hline
		\multicolumn{1}{r}{1} & 2.0   & 449   & 228   & 33.7\% & 677   & 70.1\% \\
		\multicolumn{1}{r}{2} & 2.6   & 136   & 52    & 27.7\% & 188   & 19.5\% \\
		\multicolumn{1}{r}{3} & 1.7   & 38    & 22    & 36.7\% & 60    & 6.2\% \\
		\multicolumn{1}{r}{4} & 3.6   & 18    & 5     & 21.7\% & 23    & 2.4\% \\
		\multicolumn{1}{r}{5} & 4.0   & 4     & 1     & 20.0\% & 5     & 0.5\% \\
		\multicolumn{1}{r}{6} & 3.0   & 6     & 2     & 25.0\% & 8     & 0.8\% \\
		\multicolumn{1}{r}{7} & NA    & 1     & 0     & 0.0\% & 1     & 0.1\% \\
		\multicolumn{1}{r}{8} & NA    & 1     & 0     & 0.0\% & 1     & 0.1\% \\
		\multicolumn{1}{r}{9} & NA    & 1     & 0     & 0.0\% & 1     & 0.1\% \\
		\multicolumn{1}{r}{10} & 0.0   & 0     & 2     & 100.0\% & 2     & 0.2\% \\\hline
		\textbf{Total} & \textbf{2.1} & \textbf{654} & \textbf{312} & \textbf{32.3\%} & \textbf{966} & \textbf{100.0\%} \\
	\end{tabular}%
	\label{fnocontr}%
\end{table}%
%
%Auswertung Anzahl SAS
%
Die Anzahl involvierte SAS (NoSupplSAS) bei der Zulieferung liegt im Bereich null und zehn, wobei null mit Eigenproduktion oder Zulieferung durch Dritte gleichzusetzen ist. In den übrigen Projektphasen sind maximal drei andere Bühler-Gesellschaften involviert. Eine Einzelauswertung pro Projektphase ergibt relativ wenig Aufschluss, weshalb die Häufigkeit der Kombinationen untersucht wurde. 
% Table generated by Excel2LaTeX from sheet 'nosas'
\begin{table}[H]
	\centering
	\caption{Erfolgsquote und Häufigkeitsverteilung der NoSupplySAS-Kombinationen (Ausschnitt)}
	\begin{tabular}{lrrrrrrrr}
		\textbf{NoSupplSAS} & \multicolumn{1}{l}{\textbf{MS}} & \multicolumn{1}{l}{\textbf{ME}} & \multicolumn{1}{l}{\textbf{PA}} & \multicolumn{1}{l}{\textbf{IS}} & \multicolumn{1}{l}{\textbf{Success}} & \multicolumn{1}{l}{\textbf{Fail}} & \multicolumn{1}{l}{\textbf{Total}} & \multicolumn{1}{l}{\textbf{Erfolgsquote}} \\ \hline
		\multicolumn{1}{r}{0} & 0     & 0     & 0     & 0     & 180   & 89    & 269 & 2.0 \\
		\multicolumn{1}{r}{1} & 1     & 0     & 0     & 0     & 73    & 26    & 99  & 2.8 \\
		\multicolumn{1}{r}{2} & 2     & 0     & 0     & 0     & 38    & 13    & 51  & 2.9 \\
		\multicolumn{1}{r}{2} & 2     & 1     & 0     & 0     & 23    & 13    & 36  & 1.8 \\
		\multicolumn{1}{r}{1} & 1     & 1     & 1     & 1     & 15    & 12    & 27  & 1.3 \\
		\multicolumn{1}{r}{1} & 1     & 1     & 0     & 0     & 17    & 11    & 28  & 1.5 \\ \hline
		\textbf{Total} &       &       &       &       & \textbf{346} & \textbf{164} & \textbf{510} & 2.1 \\
	\end{tabular}%
	\label{fnosas}%
\end{table}%
Die obige Tabelle \ref{fnosas} zeigt, dass die Eigenfertigung respektive die Zusammenarbeit mit Drittlieferanten während allen Projektphasen das häufigste Charakteristika aller Projekte war. Es fällt auf, dass sich mehr als die Hälfte der Datenpunkte im Bereich mit maximal zwei involvierten SAS-Gesellschaften während der gesamten Projektlaufzeit konzentriert. Aufgrund der Erfolgsquoten kann postuliert werden, dass eine Zusammenarbeit mit mindestens einer weiteren Partei im MS, den Projekterfolg begünstigen könnte.
\newpage
%%%%%%%%%%%%%%%%%%%%%%%%%%%%%%%%%%%%%%%%%%%%%%%%%%%%%%%%%%%%%%%%%%%%%%%%%%%%%%%%%%%
%%
%%Beurteilung der Ergebnis
%%
%%%%%%%%%%%%%%%%%%%%%%%%%%%%%%%%%%%%%%%%%%%%%%%%%%%%%%%%%%%%%%%%%%%%%%%%%%%%%%%%%%%
\newpage
%Diskussion
% !TEX root = MA.tex

\section{Fazit}\label{sec:fazit}
\section*{Annex: Datentabellen}\label{sec:annex}
Die nachfolgenden Tabellen enthalten sämtliche Informationen zu den Variablen, ihrer Berechnung und Interpretation. Mit einem * gekennzeichnete Variablen wurden für die Auswertung zu Berechnungszwecken zusätzlich erhoben, liefern allerdings keinen neuen Informationsgehalt. 
%%
%%Die Bühler AG
%%
Bühler AG
 Die Bühler AG wird mittels einer Matrixorganisation geführt, das heisst es gibt sowohl Regionen- und Geschäftsbereichsverantwortliche. Die Bühler AG ist in sechs Kontinente in über 140 Länder tätig und hat zwei grosse Business Cluster, Grains \& Foods (GF) mit den fünf Geschäftsbereichen Grain Milling (GM), Value Nutrition (NU), Consumer Foos (CF), Sortex (SR) und Grain Logistics (GL) und Advanced Materials (AM) mit den drei Geschäftsbereichen, Druckguss (DC), Grinding \& Dispersion (GD) und Leybold Optics (LO). Die Projektmanagementorganisation ist in den Geschäftsbereichen, wohingegen das Projektreporting für die Gruppe im Verantwortungsbereich des Bühler Project Management (BPM) Team angesiedelt ist. Das BPM ist Bestandteil der Supportfunktion Coporate Finance. Seine Aufgabe ist das monatliche Projektreporting mittels dem BPM-Cockpit für die Gruppe aufzubereiten. Das BPM-Cockpit ist das Management- und Reporting-Tool, in welchem sämtliche verfügbaren Projektinformationen während des Verlaufs zusammengezogen werden. Es zeigt zugleich die historische und gegenwärtige Sicht aller laufenden und abgeschlossenen Projekte. Allerdings arbeiten nicht alle Gesellschaften mit dem BPM-Cockpit, sondern nur diejenigen, bei welchen sich die Implementierung und die Umstellung in Abhängigkeit der Grösse und des Projektumschlages gelohnt hat. Die Verantwortung für die Vollständigkeit der Daten liegt beim Geschäftsbereich und die Sicherstellung der Funktionsfähigkeit bei der IT.
%%
%%Übersicht verwendeter Varialen
%%
% Table generated by Excel2LaTeX from sheet 'Codebook'
\begin{longtable}[ht]{p{0.05\textwidth} p{0.25\textwidth}p{0.6\textwidth}}
	\caption{Übersicht der verwendeten Variablen}\\
		\textbf{No.} & \textbf{Variablencode} & \textbf{Variablenname} \\\hline\endhead
    1     & CuNo  & Customer Number \\
2     & EquLoc & Equipment Location \\
3     & PMNo  & Project Manager User ID \\
4     & PMChange & Project Manager Change \\
5     & BA    & Business Area \\
6     & BU    & Business Unit \\
7     & MS    & Market Segement \\
8     & LeadSASPr & Lead SAS Project (overall) \\
9     & LeadSAS.PrFF & Lead SAS Project (overall) different from Lead SAS FF \\
10    & ConPart & Consortial Part \\
11    & PrStartDate & Project Start Date = HOM \\
12    & TOBud & Turnover Bud \\
13    & BudMSTot & Cost Bud MS in relation to total cost Bud \\
14    & BudMETot & Cost Bud ME in relation to total cost Bud  \\
15    & BudPATot & Cost Bud PA in relation to total cost Bud \\
16    & BudISTot & Cost Bud IS in relation to total cost Bud \\
17    & DB1Bud & DB1 Bud \\
18    & DB1Act & DB1 Act \\
19    & DB1BudDev & Deviation between DB1 Act DB1 Bud \\
20    & CostActBudMSabs & Cost deviation between Act and Bud absolute MS \\
21    & CostActBudMEabs & Cost deviation between Act and Bud absolute ME \\
22    & CostActBudPAabs & Cost deviation between Act and Bud absolute PA \\
23    & CostActBudISabs & Cost deviation between Act and Bud absolute IS \\
24    & SUCostTO & Subsequent Delivery in Relation to TO \\
25    & CostActBudRel & Cost deviation between Act and Bud relative for the whole project \\
26    & CostActBudMSRel & Cost deviation between Act and Bud relative MS \\
27    & CostActBudMERel & Cost deviation between Act and Bud relative ME\\
28    & CostActBudPARel & Cost deviation between Act and Bud relative PA \\
29    & CostActBudISRel & Cost deviation between Act and Bud relative IS \\
30    & CostFCadj & Cost forecast adjustment Project \\
31    & CostFCadjMS & Cost forecast adjustment MS \\
32    & CostFCadjME & Cost forecast adjustment ME \\
33    & CostFCadjPA & Cost forecast adjustment PA \\
34    & CostFCadjIS & Cost forecast adjustment IS \\
35    & HOMYellCost & Months between HOM and first yellow status cost \\
36    & HOMYellQual & Months between HOM and fist yellow status quality \\
37    & HOMYellTime & Months between HOM and first yellow status time \\
38    & HOMRedCost & Months between HOM first red status cost \\
39    & HOMRedQual & Months between HOM and first red status quality \\
40    & HOMRedTime & Months between HOM and first red status time \\
41    & DeltaLastFCAct & Delta between last forecast and actual  \\
42    & DeltaLastFCActMS & Delta between last forecast and actual MS \\
43    & DeltaLastFCActME & Delta between last forecast and actual ME \\
44    & DeltaLastFCActPA & Delta between last forecast and actual PA \\
45    & DeltaLastFCActIS & Delta between last forecast and actual IS \\
46    & BUORBudGapAbs & Business Unit OR Bud gap absolute \\
47    & BUORBudGapRel & Business Unit OR Bud gap relative \\
48    & RegiORBudGapAbs & Region OR Bud gap absolute \\
49    & RegiORBudGapRel & Region OR Bud gap relative \\
50    & PrTimeBase & Project time baseline \\
51    & PrTimeAct & Project time actual \\
52    & PrTimeDelay & Project time delay \\
53    & PrTimeDelayMS2 & Project time delay for MS2 \\
54    & PrTimeDelayMS8 & Project time delay for MS8 \\
55    & PrTimeDelayMS10 & Project time delay for MS10 \\
56    & PrTimeDelayMS11 & Project time delay for MS11 \\
57    & NoPM  & Number of Project Manager during project life time \\
58    & NoLeadSASFF & Number of Lead SAS FF \\
59    & NoSupplSAS & Number of supplying SAS \\
60    & NoSupplSASMS & Number of supplying SAS MS \\
61    & NoSupplSASME & Number of supplying SAS ME \\
62    & NoSupplSASPA & Number of supplying SAS PA \\
63    & NoSupplSASIS & Number of supplying SAS IS \\
64    & NoContr & Number of contracts \\
65    & Region & Region \\
66    & TOAct & Turnover Act \\
67    & DB1Budabs & DB1 Bud absolute \\
68    & DB1Actabs & DB1 Act absolute \\
69    & PMAge2 & Project Manager Age \\
70    & PMTen2 & Project Manager Tenure \\
71 *   & Success & Success \\
72 *  & Dummy\_Success & Dummyvariable Success \\
73 *   & Dummy\_Fail & Dummyvariable Fail \\
74 *   & Delay & Logical delay of project \\
75 *   & TOBud\_Cat & Categorical Turnover Bud \\
76 *   & TOBudDevabs & Deviation TO Bud from TO Act \\
77 *   & DB1BudDevabs & Deviation DB1 Bud from DB1 Act \\
78 *   & CostBudDevabs & Deviation Cost Bud from Cost Act \\
79 *   & CostAct & Cost Act \\
80 *   & CostBud & Cost Bud \\
81 *   & Cat\_age & Categorical Project Manager Age \\
	\label{tab:addlabel}%
\end{longtable}%
%%
%%Übersicht nicht verwendeter Variablen
%%
\newpage
% Table generated by Excel2LaTeX from sheet 'Codebook'
\begin{longtable}[ht]{p{0.36\textwidth} p{0.6\textwidth}}
	\caption{Übersicht der nicht verwendeten Variablen}\\
		\textbf{Variable Code} & \textbf{Variable Name} \\\hline\endhead
		 & \\
		\textbf{Anzahl NA's} &  \\\hline
		AMNo  & Area Manager User ID \\
		AMAge & Area Manager Age \\
		AMTen & Area Manager Tenure \\
		PrTimeDelayMS5 & Project runtime delay (baseline - actual) MS5 \\
		&\\
		\textbf{Berechnungsvariable} &  \\\hline
		monthsbetw.HOMand1strcstb & \# of months between HOM and first red status of cost \\
		monthsbetw.HOMand1strqtyb & \# of months between HOM and first red status of quality \\
		monthsbetw.HOMand1strtimeb & \# of months between HOM and first red status of time \\
		HOMvsyellow/redstatuscostsb & \# of months between HOM and first yellow status of cost \\
		HOMvsyellow/redstatusqualityb & \# of months between HOM and first yellow status of quality \\
		HOMvsyellow/redstatustimeb & \# of months between HOM and first yellow status of time \\
		Medianofavg.BAprojectb & Median of turnover budget for BA \\
		Medianofavg.BUprojectb & Median of turnover budget for BU \\
		Medianofavg.MSprojectb & Median of turnover budget for MS \\
		\textbf{Doppelte Variablen} &  \\\hline
		CuName & Customer Name \\
		PM    & Project Manager \\
		AM    & Area Manager \\
		&\\
		\textbf{Unplausible Variablen} &  \\\hline
		CostFirstadj & \# of months between first negative cost FC adj. (any) and project closure \\
		CostMostnegFCadj & \# of months between most negative cost FC adj. (any) and project closure \\
		CostMostnegFCadjPA & \# of months between most negative cost FC adj. PA (any) and project closure \\
		CostMostnegFCadjIS & \# of months between most negative cost FC adj. IS (any) and project closure \\
		CostMostnegFCadjMS & \# of months between most negative cost FC adj. MS (any) and project closure \\
		CostMostnegFCadjME & \# of months between most negative cost FC adj. ME (any) and project closure \\
		BAImportPr & Business Area Importance of Project \\
		BUImportPr & Business Unit Importance of Project \\
		MSImportPr & Market Segement Importance  of Project \\
		&\\
		\textbf{Kein relevanter Informationsgehalt} & \\\hline
		BPMID & Project Management ID \\
		ORDate & Orders Released Date \\
	\label{tab:addlabel}%
\end{longtable}%
%%
%%Wertebereiche der Daten
%%
%Daten mit Kategorie mit Wertebereich von Codebook
\newpage
\begin{longtable}[ht]{p{0.05\textwidth} p{0.25\textwidth} p{0.25\textwidth} p{0.4\textwidth}}
	\caption{Wertebereiche und Kategorien der verwendeten Variablen}\\
	\textbf{No.} & \textbf{Variablencode} & \textbf{Variable Kategorie} & \textbf{Masseinheit/Wertebereich}
	\\\hline\endhead
	5     & BA    & Rahmenbedingungen &  \\
	6     & BU    & Rahmenbedingungen &  \\
	1     & CuNo  & Rahmenbedingungen &  \\
	2     & EquLoc & Rahmenbedingungen &  \\
	7     & MS    & Rahmenbedingungen &  \\
	65    & Region & Rahmenbedingungen &  \\
	16    & BudISTot & Kosten & \% \\
	14    & BudMETot & Kosten & \% \\
	13    & BudMSTot & Kosten & \% \\
	15    & BudPATot & Kosten & \% \\
	79    & CostAct * & Kosten & TCHF \\
	23    & CostActBudISabs & Kosten & TCHF \\
	29    & CostActBudISRel & Kosten & \% \\
	21    & CostActBudMEabs & Kosten & TCHF \\
	27    & CostActBudMERel & Kosten & \% \\
	20    & CostActBudMSabs & Kosten & TCHF \\
	26    & CostActBudMSRel & Kosten & \% \\
	22    & CostActBudPAabs & Kosten & TCHF \\
	28    & CostActBudPARel & Kosten & \% \\
	25    & CostActBudRel & Kosten & \% \\
	80    & CostBud *& Kosten & TCHF \\
	78    & CostBudDevabs * & Kosten & TCHF \\
	18    & DB1Act & Kosten & \% \\
	68    & DB1Actabs & Kosten & TCHF \\
	17    & DB1Bud & Kosten & \% \\
	67    & DB1Budabs & Kosten & TCHF \\
	77    & DB1BudDevabs * & Kosten & TCHF \\
	41    & DeltaLastFCAct & Kosten & TCHF \\
	45    & DeltaLastFCActIS & Kosten & TCHF \\
	43    & DeltaLastFCActME & Kosten & TCHF \\
	42    & DeltaLastFCActMS & Kosten & TCHF \\
	44    & DeltaLastFCActPA & Kosten & TCHF \\
	24    & SUCostTO & Kosten & \% \\
	66    & TOAct & Kosten & TCHF \\
	12    & TOBud & Kosten & TCHF \\
	75    & TOBud\_Cat * & Kosten &  \\
	76    & TOBudDevabs * & Kosten & TCHF \\
	81    & Cat\_age * & Fulfillment &  \\
	30    & CostFCadj & Fulfillment & \{0,1\} \\
	34    & CostFCadjIS & Fulfillment & \{0,1,2\} \\
	32    & CostFCadjME & Fulfillment & \{0,1,2\}\\
	31    & CostFCadjMS & Fulfillment & \{0,1,2\} \\
	33    & CostFCadjPA & Fulfillment & \{0,1,2\} \\
	38    & HOMRedCost & Fulfillment & Monate \\
	39    & HOMRedQual & Fulfillment & Monate \\
	40    & HOMRedTime & Fulfillment & Monate \\
	35    & HOMYellCost & Fulfillment & Monate \\
	36    & HOMYellQual & Fulfillment & Monate \\
	37    & HOMYellTime & Fulfillment & Monate \\
	9     & LeadSAS.PrFF & Fulfillment & \{NO,YES\} \\
	8     & LeadSASPr & Fulfillment &  \\
	58    & NoLeadSASFF & Fulfillment & Anzahl\\
	57    & NoPM  & Fulfillment & Anzahl \\
	69    & PMAge2 & Fulfillment & Monate \\
	4     & PMChange & Fulfillment & \{YES,NO\} \\
	3     & PMNo  & Fulfillment &  \\
	70    & PMTen2 & Fulfillment & Monate \\
	11    & PrStartDate & Fulfillment & DD-MM-YYYY \\
	74    & Delay * & Zeit  & \{TRUE,FALSE\} \\
	51    & PrTimeAct & Zeit  & Monate \\
	50    & PrTimeBase & Zeit  & Monate \\
	52    & PrTimeDelay & Zeit  & Monate \\
	55    & PrTimeDelayMS10 & Zeit  & Monate \\
	56    & PrTimeDelayMS11 & Zeit  & Monate \\
	53    & PrTimeDelayMS2 & Zeit  & Monate \\
	54    & PrTimeDelayMS8 & Zeit  & Monate \\
	10    & ConPart & Komplexität & \{TRUE,FALSE\} \\
	64    & NoContr & Komplexität & Anzahl \\
	59    & NoSupplSAS & Komplexität & Anzahl \\
	63    & NoSupplSASIS & Komplexität & Anzahl \\
	61    & NoSupplSASME & Komplexität & Anzahl \\
	60    & NoSupplSASMS & Komplexität & Anzahl \\
	62    & NoSupplSASPA & Komplexität & Anzahl \\
	19    & DB1BudDev & Erfolgskriterium & \% \\
	73    & Dummy\_Fail * & Erfolgskriterium & \{0,1\} \\
	72    & Dummy\_Success * & Erfolgskriterium & \{0,1\} \\
	71    & Success * & Erfolgskriterium & \{YES,NO\} \\
	46    & BUORBudGapAbs & Sales \& Qotation & TCHF \\
	47    & BUORBudGapRel & Sales \& Qotation & \% \\
	48    & RegiORBudGapAbs & Sales \& Qotation & TCHF \\
	49    & RegiORBudGapRel & Sales \& Qotation & \% \\
\end{longtable}
%%
%%Variablen mit Formel
%%
\newpage
\begin{longtable}{p{0.5cm}|p{4cm}|p{3.5cm}|p{6cm}}
	\caption{Berechnungsformel der verwendeten Variablen}\\
	\textbf{No.} & \textbf{Kategorie} & {\textbf{Variablencode}} & {\textbf{Berechnungsformel}} \\\hline\endhead
	19    & Erfolgskriterium & DB1BudDev & $DB1Act - DB1Bud$ \\
	71    & Erfolgskriterium & Success *& $DB1Act \geq 0 = TRUE$ \\
	72    & Erfolgskriterium & Dummy\_Success *& $DB1Act \geq 0 = 1$\\
	73    & Erfolgskriterium & Dummy\_Fail *& $DB1Act < 0 = 1$\\\hline
	1     & Rahmenbedingungen & CuNo & \\
	2     & Rahmenbedingungen & EquLoc & \\
	5     & Rahmenbedingungen & BA & \\
	6     & Rahmenbedingungen & BU & \\
	7     & Rahmenbedingungen & MS  &\\
	65    & Rahmenbedingungen & Region & \\\hline
	12    & Kosten & TOBud &\\
	13    & Kosten & BudMSTot & $\frac{Cost Bud_{MS}}{Cost Bud_{Total}}$\\ [3mm]
	14    & Kosten & BudMETot & $\frac{Cost Bud_{ME}}{Cost Bud_{Total}}$ \\[3mm] 
	15    & Kosten & BudPATot & $\frac{Cost Bud_{PA}}{Cost Bud_{Total}}$\\ [3mm]
	16    & Kosten & BudISTot & $\frac{Cost Bud_{IS}}{Cost Bud_{Total}}$\\ [3mm]
	17    & Kosten & DB1Bud & $\frac{(TOBud - DB1Bud)}{TOBud}$ \\ [3mm]
	18    & Kosten & DB1Act & $\frac{(TOAct - DB1Act)}{TOAct}$ \\ [3mm]
	20    & Kosten & CostActBudMSabs & $(-CostAct_{MS})-(-CostBud_{MS})$ \\ 
	21    & Kosten & CostActBudMEabs & $(-CostAct_{ME})-(-CostBud_{ME})$ \\
	22    & Kosten & CostActBudPAabs & $(-CostAct_{PA})-(-CostBud_{PA})$ \\
	23    & Kosten & CostActBudISabs & $(-CostAct_{IS})-(-CostBud_{IS})$ \\
	24    & Kosten & SUCostTO & $\frac{(-SUAct)}{TOAct}$ \\ [3mm]
	25    & Kosten & CostActBudRel & $\frac{(-CostAct_{Total}) - (-CostBud_{Total})}{(-CostBud_{Total})}$ \\ [3mm]
	26    & Kosten & CostActBudMSRel & $\frac{(-CostAct_{MS}) - (-CostBud_{MS})}{(-CostBud_{MS})}$ \\ [3mm]
	27    & Kosten & CostActBudMERel & $\frac{(-CostAct_{ME}) - (-CostBud_{MS})}{(-CostBud_{ME})}$ \\ [3mm]
	28    & Kosten & CostActBudPARel & $\frac{(-CostAct_{PA}) - (-CostBud_{MS})}{(-CostBud_{PA})}$ \\ [3mm]
	29    & Kosten & CostActBudISRel & $\frac{(-CostAct_{IS}) - (-CostBud_{MS})}{(-CostBud_{IS})}$ \\ [3mm]
	41    & Kosten & DeltaLastFCAct & $Last Cost FC-Cost Act$ \\
	42    & Kosten & DeltaLastFCActMS & $Last Cost FC_{MS}-Cost Act_{MS}$ \\
	43    & Kosten & DeltaLastFCActME & $Last Cost FC_{ME}-Cost Act_{ME}$ \\
	44    & Kosten & DeltaLastFCActPA & $Last Cost FC_{PA}-Cost Act_{PA}$\\
	45    & Kosten & DeltaLastFCActIS & $Last Cost FC_{IS}-Cost Act_{IS}$\\
	66    & Kosten & TOAct &\\
	67    & Kosten & DB1Budabs & $TOBud-CostBud$\\
	68    & Kosten & DB1Actabs & $TOAct-CostAct$\\
	75    & Kosten & TOBud\_Cat * &
\small Klasse 1: $13.2 \geq TOBud < 500$ \newline Klassen 2 bis 10: \newline $500 \geq TOBud < 5'000$\newline Klassenbreite = 500 \newline Klasse 11: $5000 \geq TOBud < 10'000$\newline Klasse 12: $ TOBud \geq 10'000$\\
	76    & Kosten & TOBudDevabs * & $TOAct-TOBud$\\
	77    & Kosten & DB1BudDevabs * & $DB1Actabs-DB1Budabs$\\
	78    & Kosten & CostBudDevabs * & $(-CostAct)-(-CostBud)$ \\
	79    & Kosten & CostAct *& $TOAct-DB1Actabs$\\
	80    & Kosten & CostBud * & $TOBud-DB1Budabs$\\\hline
	3     & Fulfillment & PMNo & \\
	4     & Fulfillment & PMChange & YES = Wechsel des Projektmanager \\
	8     & Fulfillment & LeadSASPr & \\
	9     & Fulfillment & LeadSAS.PrFF & NO = LeadSASPr verschieden von LeadSASFF  \\
	11    & Fulfillment & PrStartDate & \\
	30    & Fulfillment & CostFCadj & 
\small0 = keine FC-Anpassung oder Umsatz und Kosten wurden angepasst\newline 1 = FC-Anpassung\\ [3mm]
	31    & Fulfillment & CostFCadjMS &
\small 0 = keine FC-Anpassung\newline 1 = FC-Anpassung für weniger Kosten\newline2 = FC-Anpassung für mehr Kosten  \\[3mm]
	32    & Fulfillment & CostFCadjME & 
\small 0 = keine FC-Anpassung\newline 1 = FC-Anpassung für weniger Kosten\newline2 = FC-Anpassung für mehr Kosten
\\[3mm]
	33    & Fulfillment & CostFCadjPA &
\small 0 = keine FC-Anpassung\newline 1 = FC-Anpassung für weniger Kosten\newline2 = FC-Anpassung für mehr Kosten
\\[3mm]
	34    & Fulfillment & CostFCadjIS &
\small 0 = keine FC-Anpassung\newline 1 = FC-Anpassung für weniger Kosten\newline2 = FC-Anpassung für mehr Kosten
\\[3mm]
	35    & Fulfillment & HOMYellCost & $\frac{(Date(first Yellow Status)_{Cost} - (Date(HOM))}{Project baseline}$ \\ [3mm]
	36    & Fulfillment & HOMYellQual & $\frac{(Date(first Yellow Status)_{Quality} - (Date(HOM))}{Project baseline}$ \\ [3mm]
	37    & Fulfillment & HOMYellTime & $\frac{(Date(first Yellow Status)_{Time} - (Date(HOM))}{Project baseline}$ \\ [3mm]
	38    & Fulfillment & HOMRedCost & $\frac{(Date(first Red Status)_{Cost} - (Date(HOM))}{Project baseline}$ \\ [3mm]
	39    & Fulfillment & HOMRedQual & $\frac{(Date(first Red Status)_{Quality} - (Date(HOM))}{Project baseline}$ \\ [3mm]
	40    & Fulfillment & HOMRedTime & $\frac{(Date(first Red Status)_{Time} - (Date(HOM))}{Project baseline}$ \\
	57    & Fulfillment & NoPM & \\
	58    & Fulfillment & NoLeadSASFF & \\
	69    & Fulfillment & PMAge2 & \\
	70    & Fulfillment & PMTen2 & \\
	81    & Fulfillment & Cat\_age * & \small min = 20, max = 63, Klassenbreite = 5\\\hline
	46    & Sales \& Qotation & BUORBudGapAbs & $OR Act_{BU}-OR Bud_{BU}$ \\
	47    & Sales \& Qotation & BUORBudGapRel & $\frac{(ORAct_{BU}-ORBud_{BU})}{ORBud_{BU}}$\\
	48    & Sales \& Qotation & RegiORBudGapAbs & $OR Act_{Region}-OR Bud_{Region}$\\
	49    & Sales \& Qotation & RegiORBudGapRel & $\frac{(ORAct_{Region}-ORBud_{Region})}{ORBud_{Region}}$\\\hline
	50    & Zeit  & PrTimeBase & $Baseline_{MS11}-Baseline_{MS1}$ \\
	51    & Zeit  & PrTimeAct & $Act_{MS10}-Act_{MS1}$ \\
	52    & Zeit  & PrTimeDelay & $PrTimeBase - PrTimeAct$\\
	53    & Zeit  & PrTimeDelayMS2 & $Baseline_{MS2}-Act_{MS2}$\\
	54    & Zeit  & PrTimeDelayMS8 & $Baseline_{MS8}-Act_{MS8}$\\
	55    & Zeit  & PrTimeDelayMS10 & $Baseline_{MS10}-Act_{MS10}$\\
	56    & Zeit  & PrTimeDelayMS11 & $Baseline_{MS11}-Act_{MS11}$\\
	74    & Zeit  & Delay * & $PrTimeDelay \geq 0 = TRUE$\\\hline
	10    & Komplexität & ConPart & \\
	59    & Komplexität & NoSupplSAS & \\
	60    & Komplexität & NoSupplSASMS & \\
	61    & Komplexität & NoSupplSASME & \\
	62    & Komplexität & NoSupplSASPA & \\
	63    & Komplexität & NoSupplSASIS & \\
	64    & Komplexität & NoContr & \\	
\end{longtable}
%%
%%%Tabelle mit Anzahl NA's
%%
\newpage
Nachfolgend werden die Tabellen von Auswertungen, welche nicht im Kapitel 3 enthalten sind aufgeführt.
% Table generated by Excel2LaTeX from sheet 'routput'
\begin{table}[htbp]
	
	\centering
	\caption{Anzahl NA's pro Variable}
	\begin{tabular}{lr}
		\textbf{Variablencode} & \multicolumn{1}{l}{\textbf{Anzahl NA's}} \\\hline
		PrTimeDelayMS5 & 538 \\
		AMAge2 & 444 \\
		AMTen2 & 444 \\
		PrTimeDelay & 254 \\
		PrTimeDelayMS11 & 227 \\
		PrTimeDelayMS10 & 214 \\
		PrTimeAct & 192 \\
		PrTimeDelayMS2 & 156 \\
		PrTimeDelayMS8 & 139 \\
		AMNo  & 132 \\
		PrTimeBase & 118 \\
		PMAge2 & 98 \\
		PMTen2 & 98 \\
		PrStartDate & 13 \\
		PMNo  & 6 \\
		BA    & 6 \\
		BU    & 6 \\
		TOAct & 6 \\
		DB1Budabs & 6 \\
		DB1Actabs & 6 \\
		EquLoc & 2 \\
	\end{tabular}%
	\label{tab:na}%
\end{table}%
\newpage
%Tabelle 2: Plausibilität
%Vorab muss erläutert werden, dass der Wert $1'111'111$ bei den CostMostnegFCajd-Variablen keinen Ausreisser darstellt sondern angibt, dass das Projekt nur positive FC-Anpassungen gehabt hat. Dies impliziert, dass der Forecast für die Kosten gesunken sind und somit weniger Kosten erwartet wurden, wobei der Umsatz konstant geblieben oder gestiegen ist. Bei den HOM-YellowStatus und die HOM-RedStatus drückt der Wert $1'111'111$ aus, dass der entsprechende Status nicht als erstes oder gar nicht aufgetreten ist. Die Interpretation wäre somit, dass bei den HOMYellow-Status Variablen der Status immer grün war oder zuerst respektive direkt den roten Status hatte. Eine ähnliche Interpretation gilt für HOMRedStatus-Variablen, somit hätte dieses Projekt, den roten Status gar nicht erst erreicht. Da diese Interpretationen valide sind und somit keine fehlende Werte darstellen, verbleiben sie im Datensatz. 
%Tabelle 3: Ausreisser
%Diagramme: Boxplots Histogramme aller numerischen Variablen
%Tabelle 4: Zusätzliche Variablen: faktisch Übersicht aller Variablen 
%Tabelle 5: Berechnungsformeln
%Tabelle 6: Interpretation
%%%%%%%%%%%%%%%%%%%%%%%%%%%%%%%%%%%%%%%%%%%%%%%%%%%%%%%%%%%%%%%%%%%%%%%%%%%%%%
%%
%%Weitere Tabellen, die nicht im Text erhalten sind
%%
%%Rahmenbedingungen
%%Kosten
\textbf{Fulfillment}
%Erfolgsquote und Häufigkeitsverteilung der Projektmanager
\begin{longtable}{lrrrrr}
	\caption{Erfolgsquote und Häufigkeitsverteilung pro Projektmanager}\\
	\textbf{PMNo} & \multicolumn{1}{c}{\textbf{Erfolgsquote}} & {\textbf{Success}} &
	\textbf{Fail} & \textbf{Fail [\%]} & \textbf{Total} \\\hline\endhead
	        \multicolumn{1}{r}{712067} & 0.6   & 8     & 14    & 63.6\% & 22 \\
	    \multicolumn{1}{r}{616048} & 0.7   & 4     & 6     & 60.0\% & 10 \\
	    \multicolumn{1}{r}{7110189} & 0.7   & 4     & 6     & 60.0\% & 10 \\
	    \multicolumn{1}{r}{1110883} & 0.8   & 4     & 5     & 55.6\% & 9 \\
	    \multicolumn{1}{r}{1515253} & 0.2   & 1     & 5     & 83.3\% & 6 \\
	    \multicolumn{1}{r}{21747} & 4.0   & 20    & 5     & 20.0\% & 25 \\
	    \multicolumn{1}{r}{28160} & 4.0   & 20    & 5     & 20.0\% & 25 \\
	    \multicolumn{1}{r}{618240} & 0.4   & 2     & 5     & 71.4\% & 7 \\
	    \multicolumn{1}{r}{1110920} & 0.3   & 1     & 4     & 80.0\% & 5 \\
	    \multicolumn{1}{r}{1910333} & 1.5   & 6     & 4     & 40.0\% & 10 \\
	    \multicolumn{1}{r}{20071} & 0.8   & 3     & 4     & 57.1\% & 7 \\
	    \multicolumn{1}{r}{20405} & 0.8   & 3     & 4     & 57.1\% & 7 \\
	    \multicolumn{1}{r}{25005} & 0.5   & 2     & 4     & 66.7\% & 6 \\
	    \multicolumn{1}{r}{5116088} & 1.5   & 6     & 4     & 40.0\% & 10 \\
	    \multicolumn{1}{r}{616291} & 1.8   & 7     & 4     & 36.4\% & 11 \\
	    \multicolumn{1}{r}{618787} & 0.0   & 0     & 4     & 100.0\% & 4 \\
	    \multicolumn{1}{r}{712066} & 2.5   & 10    & 4     & 28.6\% & 14 \\
	    \multicolumn{1}{r}{1519603} & 2.3   & 7     & 3     & 30.0\% & 10 \\
	    \multicolumn{1}{r}{2119231} & 1.0   & 3     & 3     & 50.0\% & 6 \\
	    \multicolumn{1}{r}{2119285} & 0.0   & 0     & 3     & 100.0\% & 3 \\
	    \multicolumn{1}{r}{22784} & 0.7   & 2     & 3     & 60.0\% & 5 \\
	    \multicolumn{1}{r}{27135} & 1.7   & 5     & 3     & 37.5\% & 8 \\
	    \multicolumn{1}{r}{28371} & 0.0   & 0     & 3     & 100.0\% & 3 \\
	    \multicolumn{1}{r}{4610121} & 0.0   & 0     & 3     & 100.0\% & 3 \\
	    \multicolumn{1}{r}{510134} & 0.3   & 1     & 3     & 75.0\% & 4 \\
	    \multicolumn{1}{r}{510204} & 0.0   & 0     & 3     & 100.0\% & 3 \\
	    \multicolumn{1}{r}{510288} & 0.3   & 1     & 3     & 75.0\% & 4 \\
	    \multicolumn{1}{r}{5116453} & 0.7   & 2     & 3     & 60.0\% & 5 \\
	    \multicolumn{1}{r}{616174} & 1.0   & 3     & 3     & 50.0\% & 6 \\
	    \multicolumn{1}{r}{617096} & 0.3   & 1     & 3     & 75.0\% & 4 \\
	    \multicolumn{1}{r}{617861} & 0.7   & 2     & 3     & 60.0\% & 5 \\
	    \multicolumn{1}{r}{618270} & 0.3   & 1     & 3     & 75.0\% & 4 \\
	    \multicolumn{1}{r}{7110014} & 1.3   & 4     & 3     & 42.9\% & 7 \\
	    \multicolumn{1}{r}{7110287} & 0.3   & 1     & 3     & 75.0\% & 4 \\
	    \multicolumn{1}{r}{811918} & 0.3   & 1     & 3     & 75.0\% & 4 \\
	    \multicolumn{1}{r}{1110966} & 1.5   & 3     & 2     & 40.0\% & 5 \\
	    \multicolumn{1}{r}{1210380} & 0.0   & 0     & 2     & 100.0\% & 2 \\
	    \multicolumn{1}{r}{1512545} & 0.0   & 0     & 2     & 100.0\% & 2 \\
	    \multicolumn{1}{r}{1512802} & 1.5   & 3     & 2     & 40.0\% & 5 \\
	    \multicolumn{1}{r}{1513083} & 0.5   & 1     & 2     & 66.7\% & 3 \\
	    \multicolumn{1}{r}{1514618} & 1.5   & 3     & 2     & 40.0\% & 5 \\
	    \multicolumn{1}{r}{1517859} & 1.5   & 3     & 2     & 40.0\% & 5 \\
	    \multicolumn{1}{r}{15508} & 0.0   & 0     & 2     & 100.0\% & 2 \\
	    \multicolumn{1}{r}{17670} & 0.0   & 0     & 2     & 100.0\% & 2 \\
	    \multicolumn{1}{r}{1910417} & 1.0   & 2     & 2     & 50.0\% & 4 \\
	    \multicolumn{1}{r}{1910439} & 0.0   & 0     & 2     & 100.0\% & 2 \\
	    \multicolumn{1}{r}{1910485} & 1.0   & 2     & 2     & 50.0\% & 4 \\
	    \multicolumn{1}{r}{19148} & 0.0   & 0     & 2     & 100.0\% & 2 \\
	    \multicolumn{1}{r}{2112169} & 2.0   & 4     & 2     & 33.3\% & 6 \\
	    \multicolumn{1}{r}{2112187} & 1.5   & 3     & 2     & 40.0\% & 5 \\
	    \multicolumn{1}{r}{2119240} & 2.0   & 4     & 2     & 33.3\% & 6 \\
	    \multicolumn{1}{r}{2119306} & 0.0   & 0     & 2     & 100.0\% & 2 \\
	    \multicolumn{1}{r}{2210237} & 0.0   & 0     & 2     & 100.0\% & 2 \\
	    \multicolumn{1}{r}{25339} & 1.0   & 2     & 2     & 50.0\% & 4 \\
	    \multicolumn{1}{r}{25432} & 1.0   & 2     & 2     & 50.0\% & 4 \\
	    \multicolumn{1}{r}{25445} & 0.5   & 1     & 2     & 66.7\% & 3 \\
	    \multicolumn{1}{r}{26360} & 3.0   & 6     & 2     & 25.0\% & 8 \\
	    \multicolumn{1}{r}{27080} & 1.0   & 2     & 2     & 50.0\% & 4 \\
	    \multicolumn{1}{r}{3611931} & 1.0   & 2     & 2     & 50.0\% & 4 \\
	    \multicolumn{1}{r}{4510504} & 1.5   & 3     & 2     & 40.0\% & 5 \\
	    \multicolumn{1}{r}{4510613} & 1.0   & 2     & 2     & 50.0\% & 4 \\
	    \multicolumn{1}{r}{4533123} & 1.5   & 3     & 2     & 40.0\% & 5 \\
	    \multicolumn{1}{r}{4610078} & 0.0   & 0     & 2     & 100.0\% & 2 \\
	    \multicolumn{1}{r}{4610118} & 2.5   & 5     & 2     & 28.6\% & 7 \\
	    \multicolumn{1}{r}{510038} & 0.0   & 0     & 2     & 100.0\% & 2 \\
	    \multicolumn{1}{r}{510232} & 0.5   & 1     & 2     & 66.7\% & 3 \\
	    \multicolumn{1}{r}{510254} & 0.5   & 1     & 2     & 66.7\% & 3 \\
	    \multicolumn{1}{r}{5116095} & 1.0   & 2     & 2     & 50.0\% & 4 \\
	    \multicolumn{1}{r}{5116518} & 0.0   & 0     & 2     & 100.0\% & 2 \\
	    \multicolumn{1}{r}{5116583} & 1.0   & 2     & 2     & 50.0\% & 4 \\
	    \multicolumn{1}{r}{616193} & 2.0   & 4     & 2     & 33.3\% & 6 \\
	    \multicolumn{1}{r}{617531} & 0.5   & 1     & 2     & 66.7\% & 3 \\
	    \multicolumn{1}{r}{617709} & 1.0   & 2     & 2     & 50.0\% & 4 \\
	    \multicolumn{1}{r}{618107} & 1.0   & 2     & 2     & 50.0\% & 4 \\
	    \multicolumn{1}{r}{618392} & 2.5   & 5     & 2     & 28.6\% & 7 \\
	    \multicolumn{1}{r}{618499} & 0.5   & 1     & 2     & 66.7\% & 3 \\
	    \multicolumn{1}{r}{619070} & 0.5   & 1     & 2     & 66.7\% & 3 \\
	    \multicolumn{1}{r}{7110678} & 0.0   & 0     & 2     & 100.0\% & 2 \\
	    \multicolumn{1}{r}{7110792} & 1.0   & 2     & 2     & 50.0\% & 4 \\
	    \multicolumn{1}{r}{712069} & 0.0   & 0     & 2     & 100.0\% & 2 \\
	    \multicolumn{1}{r}{811738} & 2.0   & 4     & 2     & 33.3\% & 6 \\
	    \multicolumn{1}{r}{1110986} & 1.0   & 1     & 1     & 50.0\% & 2 \\
	    \multicolumn{1}{r}{1110988} & 0.0   & 0     & 1     & 100.0\% & 1 \\
	    \multicolumn{1}{r}{11910} & 3.0   & 3     & 1     & 25.0\% & 4 \\
	    \multicolumn{1}{r}{1211259} & 1.0   & 1     & 1     & 50.0\% & 2 \\
	    \multicolumn{1}{r}{1211260} & 1.0   & 1     & 1     & 50.0\% & 2 \\
	    \multicolumn{1}{r}{12266} & 1.0   & 1     & 1     & 50.0\% & 2 \\
	    \multicolumn{1}{r}{13591} & 1.0   & 1     & 1     & 50.0\% & 2 \\
	    \multicolumn{1}{r}{13718} & 0.0   & 0     & 1     & 100.0\% & 1 \\
	    \multicolumn{1}{r}{13889} & 1.0   & 1     & 1     & 50.0\% & 2 \\
	    \multicolumn{1}{r}{1512883} & 1.0   & 1     & 1     & 50.0\% & 2 \\
	    \multicolumn{1}{r}{1513753} & 0.0   & 0     & 1     & 100.0\% & 1 \\
	    \multicolumn{1}{r}{1515461} & 0.0   & 0     & 1     & 100.0\% & 1 \\
	    \multicolumn{1}{r}{1517857} & 1.0   & 1     & 1     & 50.0\% & 2 \\
	    \multicolumn{1}{r}{17819} & 2.0   & 2     & 1     & 33.3\% & 3 \\
	    \multicolumn{1}{r}{18570} & 0.0   & 0     & 1     & 100.0\% & 1 \\
	    \multicolumn{1}{r}{18918} & 0.0   & 0     & 1     & 100.0\% & 1 \\
	    \multicolumn{1}{r}{1910272} & 0.0   & 0     & 1     & 100.0\% & 1 \\
	    \multicolumn{1}{r}{1910335} & 1.0   & 1     & 1     & 50.0\% & 2 \\
	    \multicolumn{1}{r}{1910509} & 0.0   & 0     & 1     & 100.0\% & 1 \\
	    \multicolumn{1}{r}{19350} & 4.0   & 4     & 1     & 20.0\% & 5 \\
	    \multicolumn{1}{r}{19401} & 0.0   & 0     & 1     & 100.0\% & 1 \\
	    \multicolumn{1}{r}{19815} & 0.0   & 0     & 1     & 100.0\% & 1 \\
	    \multicolumn{1}{r}{20108} & 0.0   & 0     & 1     & 100.0\% & 1 \\
	    \multicolumn{1}{r}{20562} & 0.0   & 0     & 1     & 100.0\% & 1 \\
	    \multicolumn{1}{r}{20810} & 1.0   & 1     & 1     & 50.0\% & 2 \\
	    \multicolumn{1}{r}{2111889} & 3.0   & 3     & 1     & 25.0\% & 4 \\
	    \multicolumn{1}{r}{2112077} & 0.0   & 0     & 1     & 100.0\% & 1 \\
	    \multicolumn{1}{r}{2112154} & 0.0   & 0     & 1     & 100.0\% & 1 \\
	    \multicolumn{1}{r}{2112322} & 0.0   & 0     & 1     & 100.0\% & 1 \\
	    \multicolumn{1}{r}{2119305} & 1.0   & 1     & 1     & 50.0\% & 2 \\
	    \multicolumn{1}{r}{21611} & 0.0   & 0     & 1     & 100.0\% & 1 \\
	    \multicolumn{1}{r}{22029} & 8.0   & 8     & 1     & 11.1\% & 9 \\
	    \multicolumn{1}{r}{23236} & 3.0   & 3     & 1     & 25.0\% & 4 \\
	    \multicolumn{1}{r}{23489} & 0.0   & 0     & 1     & 100.0\% & 1 \\
	    \multicolumn{1}{r}{24090} & 1.0   & 1     & 1     & 50.0\% & 2 \\
	    \multicolumn{1}{r}{24994} & 1.0   & 1     & 1     & 50.0\% & 2 \\
	    \multicolumn{1}{r}{25000} & 0.0   & 0     & 1     & 100.0\% & 1 \\
	    \multicolumn{1}{r}{25159} & 1.0   & 1     & 1     & 50.0\% & 2 \\
	    \multicolumn{1}{r}{25561} & 0.0   & 0     & 1     & 100.0\% & 1 \\
	    \multicolumn{1}{r}{25687} & 2.0   & 2     & 1     & 33.3\% & 3 \\
	    \multicolumn{1}{r}{25964} & 2.0   & 2     & 1     & 33.3\% & 3 \\
	    \multicolumn{1}{r}{26036} & 0.0   & 0     & 1     & 100.0\% & 1 \\
	    \multicolumn{1}{r}{26921} & 1.0   & 1     & 1     & 50.0\% & 2 \\
	    \multicolumn{1}{r}{27016} & 3.0   & 3     & 1     & 25.0\% & 4 \\
	    \multicolumn{1}{r}{27034} & 1.0   & 1     & 1     & 50.0\% & 2 \\
	    \multicolumn{1}{r}{310266} & 0.0   & 0     & 1     & 100.0\% & 1 \\
	    \multicolumn{1}{r}{410316} & 1.0   & 1     & 1     & 50.0\% & 2 \\
	    \multicolumn{1}{r}{4512231} & 1.0   & 1     & 1     & 50.0\% & 2 \\
	    \multicolumn{1}{r}{4512234} & 0.0   & 0     & 1     & 100.0\% & 1 \\
	    \multicolumn{1}{r}{4513192} & 0.0   & 0     & 1     & 100.0\% & 1 \\
	    \multicolumn{1}{r}{4523184} & 1.0   & 1     & 1     & 50.0\% & 2 \\
	    \multicolumn{1}{r}{4533116} & 2.0   & 2     & 1     & 33.3\% & 3 \\
	    \multicolumn{1}{r}{4533138} & 1.0   & 1     & 1     & 50.0\% & 2 \\
	    \multicolumn{1}{r}{4543141} & 0.0   & 0     & 1     & 100.0\% & 1 \\
	    \multicolumn{1}{r}{4553178} & 8.0   & 8     & 1     & 11.1\% & 9 \\
	    \multicolumn{1}{r}{4563170} & 2.0   & 2     & 1     & 33.3\% & 3 \\
	    \multicolumn{1}{r}{4610177} & 3.0   & 3     & 1     & 25.0\% & 4 \\
	    \multicolumn{1}{r}{510166} & 0.0   & 0     & 1     & 100.0\% & 1 \\
	    \multicolumn{1}{r}{510187} & 3.0   & 3     & 1     & 25.0\% & 4 \\
	    \multicolumn{1}{r}{617097} & 0.0   & 0     & 1     & 100.0\% & 1 \\
	    \multicolumn{1}{r}{617408} & 0.0   & 0     & 1     & 100.0\% & 1 \\
	    \multicolumn{1}{r}{617628} & 1.0   & 1     & 1     & 50.0\% & 2 \\
	    \multicolumn{1}{r}{617767} & 0.0   & 0     & 1     & 100.0\% & 1 \\
	    \multicolumn{1}{r}{617823} & 1.0   & 1     & 1     & 50.0\% & 2 \\
	    \multicolumn{1}{r}{617949} & 0.0   & 0     & 1     & 100.0\% & 1 \\
	    \multicolumn{1}{r}{618323} & 2.0   & 2     & 1     & 33.3\% & 3 \\
	    \multicolumn{1}{r}{710120} & 0.0   & 0     & 1     & 100.0\% & 1 \\
	    \multicolumn{1}{r}{7110106} & 0.0   & 0     & 1     & 100.0\% & 1 \\
	    \multicolumn{1}{r}{7110109} & 2.0   & 2     & 1     & 33.3\% & 3 \\
	    \multicolumn{1}{r}{7110186} & 0.0   & 0     & 1     & 100.0\% & 1 \\
	    \multicolumn{1}{r}{7110289} & 0.0   & 0     & 1     & 100.0\% & 1 \\
	    \multicolumn{1}{r}{7110299} & 0.0   & 0     & 1     & 100.0\% & 1 \\
	    \multicolumn{1}{r}{7110609} & 1.0   & 1     & 1     & 50.0\% & 2 \\
	    \multicolumn{1}{r}{712312} & 10.0  & 10    & 1     & 9.1\% & 11 \\
	    \multicolumn{1}{r}{811974} & 0.0   & 0     & 1     & 100.0\% & 1 \\
	    \multicolumn{1}{r}{9910051} & 0.0   & 0     & 1     & 100.0\% & 1 \\
	    \multicolumn{1}{r}{9910221} & 1.0   & 1     & 1     & 50.0\% & 2 \\
	    \multicolumn{1}{r}{9910250} & 1.0   & 1     & 1     & 50.0\% & 2 \\
	    \multicolumn{1}{r}{9910375} & 5.0   & 5     & 1     & 16.7\% & 6 \\
	    \multicolumn{1}{r}{1110867} & NA    & 2     & 0     & 0.0\% & 2 \\
	    \multicolumn{1}{r}{11890} & NA    & 5     & 0     & 0.0\% & 5 \\
	    \multicolumn{1}{r}{1211275} & NA    & 1     & 0     & 0.0\% & 1 \\
	    \multicolumn{1}{r}{1211289} & NA    & 4     & 0     & 0.0\% & 4 \\
	    \multicolumn{1}{r}{1211314} & NA    & 1     & 0     & 0.0\% & 1 \\
	    \multicolumn{1}{r}{12340} & NA    & 1     & 0     & 0.0\% & 1 \\
	    \multicolumn{1}{r}{12349} & NA    & 1     & 0     & 0.0\% & 1 \\
	    \multicolumn{1}{r}{13392} & NA    & 8     & 0     & 0.0\% & 8 \\
	    \multicolumn{1}{r}{13497} & NA    & 1     & 0     & 0.0\% & 1 \\
	    \multicolumn{1}{r}{13563} & NA    & 1     & 0     & 0.0\% & 1 \\
	    \multicolumn{1}{r}{15064} & NA    & 5     & 0     & 0.0\% & 5 \\
	    \multicolumn{1}{r}{1511240} & NA    & 1     & 0     & 0.0\% & 1 \\
	    \multicolumn{1}{r}{1512077} & NA    & 1     & 0     & 0.0\% & 1 \\
	    \multicolumn{1}{r}{1512612} & NA    & 1     & 0     & 0.0\% & 1 \\
	    \multicolumn{1}{r}{1513595} & NA    & 1     & 0     & 0.0\% & 1 \\
	    \multicolumn{1}{r}{1516955} & NA    & 2     & 0     & 0.0\% & 2 \\
	    \multicolumn{1}{r}{1517907} & NA    & 1     & 0     & 0.0\% & 1 \\
	    \multicolumn{1}{r}{1518993} & NA    & 1     & 0     & 0.0\% & 1 \\
	    \multicolumn{1}{r}{15803} & NA    & 1     & 0     & 0.0\% & 1 \\
	    \multicolumn{1}{r}{16046} & NA    & 1     & 0     & 0.0\% & 1 \\
	    \multicolumn{1}{r}{17397} & NA    & 4     & 0     & 0.0\% & 4 \\
	    \multicolumn{1}{r}{17638} & NA    & 3     & 0     & 0.0\% & 3 \\
	    \multicolumn{1}{r}{17667} & NA    & 1     & 0     & 0.0\% & 1 \\
	    \multicolumn{1}{r}{17675} & NA    & 2     & 0     & 0.0\% & 2 \\
	    \multicolumn{1}{r}{17678} & NA    & 2     & 0     & 0.0\% & 2 \\
	    \multicolumn{1}{r}{18119} & NA    & 1     & 0     & 0.0\% & 1 \\
	    \multicolumn{1}{r}{18194} & NA    & 6     & 0     & 0.0\% & 6 \\
	    \multicolumn{1}{r}{18231} & NA    & 1     & 0     & 0.0\% & 1 \\
	    \multicolumn{1}{r}{18299} & NA    & 1     & 0     & 0.0\% & 1 \\
	    \multicolumn{1}{r}{18511} & NA    & 1     & 0     & 0.0\% & 1 \\
	    \multicolumn{1}{r}{18628} & NA    & 3     & 0     & 0.0\% & 3 \\
	    \multicolumn{1}{r}{18938} & NA    & 1     & 0     & 0.0\% & 1 \\
	    \multicolumn{1}{r}{1910240} & NA    & 1     & 0     & 0.0\% & 1 \\
	    \multicolumn{1}{r}{1910503} & NA    & 2     & 0     & 0.0\% & 2 \\
	    \multicolumn{1}{r}{1910517} & NA    & 2     & 0     & 0.0\% & 2 \\
	    \multicolumn{1}{r}{19405} & NA    & 1     & 0     & 0.0\% & 1 \\
	    \multicolumn{1}{r}{19421} & NA    & 3     & 0     & 0.0\% & 3 \\
	    \multicolumn{1}{r}{19982} & NA    & 1     & 0     & 0.0\% & 1 \\
	    \multicolumn{1}{r}{19984} & NA    & 1     & 0     & 0.0\% & 1 \\
	    \multicolumn{1}{r}{19985} & NA    & 3     & 0     & 0.0\% & 3 \\
	    \multicolumn{1}{r}{20006} & NA    & 1     & 0     & 0.0\% & 1 \\
	    \multicolumn{1}{r}{20022} & NA    & 4     & 0     & 0.0\% & 4 \\
	    \multicolumn{1}{r}{20028} & NA    & 2     & 0     & 0.0\% & 2 \\
	    \multicolumn{1}{r}{20585} & NA    & 3     & 0     & 0.0\% & 3 \\
	    \multicolumn{1}{r}{20770} & NA    & 4     & 0     & 0.0\% & 4 \\
	    \multicolumn{1}{r}{2111985} & NA    & 1     & 0     & 0.0\% & 1 \\
	    \multicolumn{1}{r}{2112002} & NA    & 2     & 0     & 0.0\% & 2 \\
	    \multicolumn{1}{r}{2112146} & NA    & 1     & 0     & 0.0\% & 1 \\
	    \multicolumn{1}{r}{2112175} & NA    & 2     & 0     & 0.0\% & 2 \\
	    \multicolumn{1}{r}{2119270} & NA    & 5     & 0     & 0.0\% & 5 \\
	    \multicolumn{1}{r}{21207} & NA    & 1     & 0     & 0.0\% & 1 \\
	    \multicolumn{1}{r}{21223} & NA    & 7     & 0     & 0.0\% & 7 \\
	    \multicolumn{1}{r}{21508} & NA    & 3     & 0     & 0.0\% & 3 \\
	    \multicolumn{1}{r}{21623} & NA    & 3     & 0     & 0.0\% & 3 \\
	    \multicolumn{1}{r}{21633} & NA    & 6     & 0     & 0.0\% & 6 \\
	    \multicolumn{1}{r}{21828} & NA    & 5     & 0     & 0.0\% & 5 \\
	    \multicolumn{1}{r}{21893} & NA    & 1     & 0     & 0.0\% & 1 \\
	    \multicolumn{1}{r}{21939} & NA    & 7     & 0     & 0.0\% & 7 \\
	    \multicolumn{1}{r}{2210239} & NA    & 2     & 0     & 0.0\% & 2 \\
	    \multicolumn{1}{r}{2210241} & NA    & 4     & 0     & 0.0\% & 4 \\
	    \multicolumn{1}{r}{2210258} & NA    & 2     & 0     & 0.0\% & 2 \\
	    \multicolumn{1}{r}{2210261} & NA    & 3     & 0     & 0.0\% & 3 \\
	    \multicolumn{1}{r}{22321} & NA    & 4     & 0     & 0.0\% & 4 \\
	    \multicolumn{1}{r}{22499} & NA    & 3     & 0     & 0.0\% & 3 \\
	    \multicolumn{1}{r}{22686} & NA    & 2     & 0     & 0.0\% & 2 \\
	    \multicolumn{1}{r}{22780} & NA    & 1     & 0     & 0.0\% & 1 \\
	    \multicolumn{1}{r}{22854} & NA    & 4     & 0     & 0.0\% & 4 \\
	    \multicolumn{1}{r}{23221} & NA    & 1     & 0     & 0.0\% & 1 \\
	    \multicolumn{1}{r}{23244} & NA    & 1     & 0     & 0.0\% & 1 \\
	    \multicolumn{1}{r}{23875} & NA    & 7     & 0     & 0.0\% & 7 \\
	    \multicolumn{1}{r}{23950} & NA    & 1     & 0     & 0.0\% & 1 \\
	    \multicolumn{1}{r}{24297} & NA    & 1     & 0     & 0.0\% & 1 \\
	    \multicolumn{1}{r}{24411} & NA    & 1     & 0     & 0.0\% & 1 \\
	    \multicolumn{1}{r}{24636} & NA    & 1     & 0     & 0.0\% & 1 \\
	    \multicolumn{1}{r}{25130} & NA    & 1     & 0     & 0.0\% & 1 \\
	    \multicolumn{1}{r}{25175} & NA    & 2     & 0     & 0.0\% & 2 \\
	    \multicolumn{1}{r}{25342} & NA    & 1     & 0     & 0.0\% & 1 \\
	    \multicolumn{1}{r}{25355} & NA    & 1     & 0     & 0.0\% & 1 \\
	    \multicolumn{1}{r}{25356} & NA    & 3     & 0     & 0.0\% & 3 \\
	    \multicolumn{1}{r}{25476} & NA    & 7     & 0     & 0.0\% & 7 \\
	    \multicolumn{1}{r}{25645} & NA    & 4     & 0     & 0.0\% & 4 \\
	    \multicolumn{1}{r}{25914} & NA    & 2     & 0     & 0.0\% & 2 \\
	    \multicolumn{1}{r}{25952} & NA    & 1     & 0     & 0.0\% & 1 \\
	    \multicolumn{1}{r}{26949} & NA    & 1     & 0     & 0.0\% & 1 \\
	    \multicolumn{1}{r}{27038} & NA    & 1     & 0     & 0.0\% & 1 \\
	    \multicolumn{1}{r}{27104} & NA    & 7     & 0     & 0.0\% & 7 \\
	    \multicolumn{1}{r}{27163} & NA    & 1     & 0     & 0.0\% & 1 \\
	    \multicolumn{1}{r}{27698} & NA    & 2     & 0     & 0.0\% & 2 \\
	    \multicolumn{1}{r}{29228} & NA    & 1     & 0     & 0.0\% & 1 \\
	    \multicolumn{1}{r}{310261} & NA    & 3     & 0     & 0.0\% & 3 \\
	    \multicolumn{1}{r}{4025} & NA    & 3     & 0     & 0.0\% & 3 \\
	    \multicolumn{1}{r}{410051} & NA    & 3     & 0     & 0.0\% & 3 \\
	    \multicolumn{1}{r}{410315} & NA    & 3     & 0     & 0.0\% & 3 \\
	    \multicolumn{1}{r}{410341} & NA    & 2     & 0     & 0.0\% & 2 \\
	    \multicolumn{1}{r}{410366} & NA    & 1     & 0     & 0.0\% & 1 \\
	    \multicolumn{1}{r}{4510521} & NA    & 1     & 0     & 0.0\% & 1 \\
	    \multicolumn{1}{r}{4512235} & NA    & 1     & 0     & 0.0\% & 1 \\
	    \multicolumn{1}{r}{4513162} & NA    & 7     & 0     & 0.0\% & 7 \\
	    \multicolumn{1}{r}{4513184} & NA    & 2     & 0     & 0.0\% & 2 \\
	    \multicolumn{1}{r}{4523134} & NA    & 2     & 0     & 0.0\% & 2 \\
	    \multicolumn{1}{r}{4523135} & NA    & 3     & 0     & 0.0\% & 3 \\
	    \multicolumn{1}{r}{4523177} & NA    & 4     & 0     & 0.0\% & 4 \\
	    \multicolumn{1}{r}{4533130} & NA    & 1     & 0     & 0.0\% & 1 \\
	    \multicolumn{1}{r}{4533142} & NA    & 1     & 0     & 0.0\% & 1 \\
	    \multicolumn{1}{r}{4543128} & NA    & 2     & 0     & 0.0\% & 2 \\
	    \multicolumn{1}{r}{4553190} & NA    & 5     & 0     & 0.0\% & 5 \\
	    \multicolumn{1}{r}{4563187} & NA    & 5     & 0     & 0.0\% & 5 \\
	    \multicolumn{1}{r}{4573142} & NA    & 3     & 0     & 0.0\% & 3 \\
	    \multicolumn{1}{r}{4583153} & NA    & 1     & 0     & 0.0\% & 1 \\
	    \multicolumn{1}{r}{510029} & NA    & 2     & 0     & 0.0\% & 2 \\
	    \multicolumn{1}{r}{510206} & NA    & 1     & 0     & 0.0\% & 1 \\
	    \multicolumn{1}{r}{510394} & NA    & 1     & 0     & 0.0\% & 1 \\
	    \multicolumn{1}{r}{616217} & NA    & 3     & 0     & 0.0\% & 3 \\
	    \multicolumn{1}{r}{617616} & NA    & 3     & 0     & 0.0\% & 3 \\
	    \multicolumn{1}{r}{618213} & NA    & 2     & 0     & 0.0\% & 2 \\
	    \multicolumn{1}{r}{618512} & NA    & 2     & 0     & 0.0\% & 2 \\
	    \multicolumn{1}{r}{618625} & NA    & 1     & 0     & 0.0\% & 1 \\
	    \multicolumn{1}{r}{619245} & NA    & 13    & 0     & 0.0\% & 13 \\
	    \multicolumn{1}{r}{7110026} & NA    & 1     & 0     & 0.0\% & 1 \\
	    \multicolumn{1}{r}{7110066} & NA    & 1     & 0     & 0.0\% & 1 \\
	    \multicolumn{1}{r}{7110074} & NA    & 3     & 0     & 0.0\% & 3 \\
	    \multicolumn{1}{r}{7110090} & NA    & 1     & 0     & 0.0\% & 1 \\
	    \multicolumn{1}{r}{7110093} & NA    & 1     & 0     & 0.0\% & 1 \\
	    \multicolumn{1}{r}{7110110} & NA    & 1     & 0     & 0.0\% & 1 \\
	    \multicolumn{1}{r}{7110261} & NA    & 1     & 0     & 0.0\% & 1 \\
	    \multicolumn{1}{r}{7110630} & NA    & 2     & 0     & 0.0\% & 2 \\
	    \multicolumn{1}{r}{7110757} & NA    & 3     & 0     & 0.0\% & 3 \\
	    \multicolumn{1}{r}{712065} & NA    & 1     & 0     & 0.0\% & 1 \\
	    \multicolumn{1}{r}{712248} & NA    & 1     & 0     & 0.0\% & 1 \\
	    \multicolumn{1}{r}{811849} & NA    & 2     & 0     & 0.0\% & 2 \\
	    \multicolumn{1}{r}{811896} & NA    & 2     & 0     & 0.0\% & 2 \\
	    \multicolumn{1}{r}{811929} & NA    & 3     & 0     & 0.0\% & 3 \\
	    \multicolumn{1}{r}{811976} & NA    & 1     & 0     & 0.0\% & 1 \\
	    \multicolumn{1}{r}{9910123} & NA    & 3     & 0     & 0.0\% & 3 \\
	    \multicolumn{1}{r}{9910135} & NA    & 4     & 0     & 0.0\% & 4 \\
	    \multicolumn{1}{r}{9910153} & NA    & 3     & 0     & 0.0\% & 3 \\
	    \multicolumn{1}{r}{9910183} & NA    & 1     & 0     & 0.0\% & 1 \\
	    \multicolumn{1}{r}{9910210} & NA    & 1     & 0     & 0.0\% & 1 \\
	    \multicolumn{1}{r}{9910212} & NA    & 1     & 0     & 0.0\% & 1 \\
	    \multicolumn{1}{r}{9910222} & NA    & 8     & 0     & 0.0\% & 8 \\
	    \multicolumn{1}{r}{9910229} & NA    & 1     & 0     & 0.0\% & 1 \\
	    \multicolumn{1}{r}{9910274} & NA    & 2     & 0     & 0.0\% & 2 \\
	    \multicolumn{1}{r}{9910292} & NA    & 1     & 0     & 0.0\% & 1 \\
	    \multicolumn{1}{r}{9910294} & NA    & 1     & 0     & 0.0\% & 1 \\
	    \multicolumn{1}{r}{9910816} & NA    & 1     & 0     & 0.0\% & 1 \\
	    \textbf{Total} & \textbf{2.1} & \textbf{654} & \textbf{312} & \textbf{32.3\%} & \textbf{966} \\
	    \label{tab:pmno}
\end{longtable}
%Tabelle PMAge2 und PMTen Projektmanager
\begin{table}[H]
	\centering
	\caption{Arithmetisches Mittel PMAge und PMTen [Jahre]}
	\begin{tabular}{lrr}
		\textbf{Success} & \multicolumn{1}{l}{\textbf{Age}} & \multicolumn{1}{l}{\textbf{Ten}} \\\hline
		FALSE & 41.1 & 12.4 \\
		TRUE  & 39.5 & 11.7 \\
	\end{tabular}%
	\label{ageten}%
\end{table}%

%%
%%Zeit
%%Komplexität
%%%%
%%% Bei der letzteren Variable wird die gesamte Arbeitszeit bei der Bühler AG berücksichtigt, was die Aussagekraft insofern abschwächt, als dass die Erfahrung im Projektmanagement bei der Bühler AG nicht explizit erfasst wird.\newpage
%Fazit
\chapter{Fazit}\label{sec:fazit}\newpage
%Literaturverzeichnis
\bibliographystyle{apacite} %Wähle Zitierstandard
\bibliography{Literaturverzeichnis}
%\addbibresource{C:/Users/Michèle/Dropbox/Master/MA\06\_MA\_Files/MA\_Latex/Literaturverzeichnis.bib}
\newpage
%Anhang
\renewcommand\appendixtocname{Anhang}
\begin{appendices}
	%Anhang
\section*{Anhang}
\label{sec:annex}
Die nachfolgenden Tabellen enthalten sämtliche Informationen zu den Variablen, ihrer Berechnung und Interpretation. Mit einem * gekennzeichnete Variablen wurden für die Auswertung zu Berechnungszwecken zusätzlich erhoben, liefern allerdings keinen neuen Informationsgehalt. 
%%
%%Die Bühler AG
%%
Bühler AG
Die Bühler AG wird mittels einer Matrixorganisation geführt, das heisst es gibt sowohl Regionen- und Geschäftsbereichsverantwortliche. Die Bühler AG ist in sechs Kontinente in über 140 Länder tätig und hat zwei grosse Business Cluster, Grains \& Foods (GF) mit den fünf Geschäftsbereichen Grain Milling (GM), Value Nutrition (NU), Consumer Foos (CF), Sortex (SR) und Grain Logistics (GL) und Advanced Materials (AM) mit den drei Geschäftsbereichen, Druckguss (DC), Grinding \& Dispersion (GD) und Leybold Optics (LO). Die Projektmanagementorganisation ist in den Geschäftsbereichen, wohingegen das Projektreporting für die Gruppe im Verantwortungsbereich des Bühler Project Management (BPM) Team angesiedelt ist. Das BPM ist Bestandteil der Supportfunktion Coporate Finance. Seine Aufgabe ist das monatliche Projektreporting mittels dem BPM-Cockpit für die Gruppe aufzubereiten. Das BPM-Cockpit ist das Management- und Reporting-Tool, in welchem sämtliche verfügbaren Projektinformationen während des Verlaufs zusammengezogen werden. Es zeigt zugleich die historische und gegenwärtige Sicht aller laufenden und abgeschlossenen Projekte. Allerdings arbeiten nicht alle Gesellschaften mit dem BPM-Cockpit, sondern nur diejenigen, bei welchen sich die Implementierung und die Umstellung in Abhängigkeit der Grösse und des Projektumschlages gelohnt hat. Die Verantwortung für die Vollständigkeit der Daten liegt beim Geschäftsbereich und die Sicherstellung der Funktionsfähigkeit bei der IT.
%%
%%Übersicht verwendeter Varialen
%%
% Table generated by Excel2LaTeX from sheet 'Codebook'
\begin{longtable}[ht]{p{0.05\textwidth} p{0.25\textwidth}p{0.6\textwidth}}
	\caption{Übersicht der verwendeten Variablen}\\
	\textbf{No.} & \textbf{Variablencode} & \textbf{Variablenname} \\\hline\endfirsthead
	1     & CuNo  & Customer Number \\
	2     & EquLoc & Equipment Location \\
	3     & PMNo  & Project Manager User ID \\
	4     & PMChange & Project Manager Change \\
	5     & BA    & Business Area \\
	6     & BU    & Business Unit \\
	7     & MS    & Market Segement \\
	8     & LeadSASPr & Lead SAS Project (overall) \\
	9     & LeadSAS.PrFF & Lead SAS Project (overall) different from Lead SAS FF \\
	10    & ConPart & Consortial Part \\
	11    & PrStartDate & Project Start Date = HOM \\
	12    & TOBud & Turnover Bud \\
	13    & BudMSTot & Cost Bud MS in relation to total cost Bud \\
	14    & BudMETot & Cost Bud ME in relation to total cost Bud  \\
	15    & BudPATot & Cost Bud PA in relation to total cost Bud \\
	16    & BudISTot & Cost Bud IS in relation to total cost Bud \\
	17    & DB1Bud & DB1 Bud \\
	18    & DB1Act & DB1 Act \\
	19    & DB1BudDev & Deviation between DB1 Act DB1 Bud \\
	20    & CostActBudMSabs & Cost deviation between Act and Bud absolute MS \\
	21    & CostActBudMEabs & Cost deviation between Act and Bud absolute ME \\
	22    & CostActBudPAabs & Cost deviation between Act and Bud absolute PA \\
	23    & CostActBudISabs & Cost deviation between Act and Bud absolute IS \\
	24    & SUCostTO & Subsequent Delivery in Relation to TO \\
	25    & CostActBudRel & Cost deviation between Act and Bud relative for the whole project \\
	26    & CostActBudMSRel & Cost deviation between Act and Bud relative MS \\
	27    & CostActBudMERel & Cost deviation between Act and Bud relative ME\\
	28    & CostActBudPARel & Cost deviation between Act and Bud relative PA \\
	29    & CostActBudISRel & Cost deviation between Act and Bud relative IS \\
	30    & CostFCadj & Cost forecast adjustment Project \\
	31    & CostFCadjMS & Cost forecast adjustment MS \\
	32    & CostFCadjME & Cost forecast adjustment ME \\
	33    & CostFCadjPA & Cost forecast adjustment PA \\
	34    & CostFCadjIS & Cost forecast adjustment IS \\
	35    & HOMYellCost & Months between HOM and first yellow status cost \\
	36    & HOMYellQual & Months between HOM and fist yellow status quality \\
	37    & HOMYellTime & Months between HOM and first yellow status time \\
	38    & HOMRedCost & Months between HOM first red status cost \\
	39    & HOMRedQual & Months between HOM and first red status quality \\
	40    & HOMRedTime & Months between HOM and first red status time \\
	41    & DeltaLastFCAct & Delta between last forecast and actual  \\
	42    & DeltaLastFCActMS & Delta between last forecast and actual MS \\
	43    & DeltaLastFCActME & Delta between last forecast and actual ME \\
	44    & DeltaLastFCActPA & Delta between last forecast and actual PA \\
	45    & DeltaLastFCActIS & Delta between last forecast and actual IS \\
	46    & BUORBudGapAbs & Business Unit OR Bud gap absolute \\
	47    & BUORBudGapRel & Business Unit OR Bud gap relative \\
	48    & RegiORBudGapAbs & Region OR Bud gap absolute \\
	49    & RegiORBudGapRel & Region OR Bud gap relative \\
	50    & PrTimeBase & Project time baseline \\
	51    & PrTimeAct & Project time actual \\
	52    & PrTimeDelay & Project time delay \\
	53    & PrTimeDelayMS2 & Project time delay for MS2 \\
	54    & PrTimeDelayMS8 & Project time delay for MS8 \\
	55    & PrTimeDelayMS10 & Project time delay for MS10 \\
	56    & PrTimeDelayMS11 & Project time delay for MS11 \\
	57    & NoPM  & Number of Project Manager during project life time \\
	58    & NoLeadSASFF & Number of Lead SAS FF \\
	59    & NoSupplSAS & Number of supplying SAS \\
	60    & NoSupplSASMS & Number of supplying SAS MS \\
	61    & NoSupplSASME & Number of supplying SAS ME \\
	62    & NoSupplSASPA & Number of supplying SAS PA \\
	63    & NoSupplSASIS & Number of supplying SAS IS \\
	64    & NoContr & Number of contracts \\
	65    & Region & Region \\
	66    & TOAct & Turnover Act \\
	67    & DB1Budabs & DB1 Bud absolute \\
	68    & DB1Actabs & DB1 Act absolute \\
	69    & PMAge2 & Project Manager Age \\
	70    & PMTen2 & Project Manager Tenure \\
	71 *   & Success & Success \\
	72 *  & Dummy\_Success & Dummyvariable Success \\
	73 *   & Dummy\_Fail & Dummyvariable Fail \\
	74 *   & Delay & Logical delay of project \\
	75 *   & TOBud\_Cat & Categorical Turnover Bud \\
	76 *   & TOBudDevabs & Deviation TO Bud from TO Act \\
	77 *   & DB1BudDevabs & Deviation DB1 Bud from DB1 Act \\
	78 *   & CostBudDevabs & Deviation Cost Bud from Cost Act \\
	79 *   & CostAct & Cost Act \\
	80 *   & CostBud & Cost Bud \\
	81 *   & Cat\_age & Categorical Project Manager Age \\
	\label{tab:addlabel}%
\end{longtable}%
%%
%%Übersicht nicht verwendeter Variablen
%%
\newpage
% Table generated by Excel2LaTeX from sheet 'Codebook'
\begin{longtable}[ht]{p{0.36\textwidth} p{0.6\textwidth}}
	\caption{Übersicht der nicht verwendeten Variablen}\\
	\textbf{Variable Code} & \textbf{Variable Name} \\\hline\endhead
	& \\
	\textbf{Anzahl NA's} &  \\\hline
	AMNo  & Area Manager User ID \\
	AMAge & Area Manager Age \\
	AMTen & Area Manager Tenure \\
	PrTimeDelayMS5 & Project runtime delay (baseline - actual) MS5 \\
	&\\
	\textbf{Berechnungsvariable} &  \\\hline
	monthsbetw.HOMand1strcstb & \# of months between HOM and first red status of cost \\
	monthsbetw.HOMand1strqtyb & \# of months between HOM and first red status of quality \\
	monthsbetw.HOMand1strtimeb & \# of months between HOM and first red status of time \\
	HOMvsyellow/redstatuscostsb & \# of months between HOM and first yellow status of cost \\
	HOMvsyellow/redstatusqualityb & \# of months between HOM and first yellow status of quality \\
	HOMvsyellow/redstatustimeb & \# of months between HOM and first yellow status of time \\
	Medianofavg.BAprojectb & Median of turnover budget for BA \\
	Medianofavg.BUprojectb & Median of turnover budget for BU \\
	Medianofavg.MSprojectb & Median of turnover budget for MS \\
	\textbf{Doppelte Variablen} &  \\\hline
	CuName & Customer Name \\
	PM    & Project Manager \\
	AM    & Area Manager \\
	&\\
	\textbf{Unplausible Variablen} &  \\\hline
	CostFirstadj & \# of months between first negative cost FC adj. (any) and project closure \\
	CostMostnegFCadj & \# of months between most negative cost FC adj. (any) and project closure \\
	CostMostnegFCadjPA & \# of months between most negative cost FC adj. PA (any) and project closure \\
	CostMostnegFCadjIS & \# of months between most negative cost FC adj. IS (any) and project closure \\
	CostMostnegFCadjMS & \# of months between most negative cost FC adj. MS (any) and project closure \\
	CostMostnegFCadjME & \# of months between most negative cost FC adj. ME (any) and project closure \\
	BAImportPr & Business Area Importance of Project \\
	BUImportPr & Business Unit Importance of Project \\
	MSImportPr & Market Segement Importance  of Project \\
	&\\
	\textbf{Kein relevanter Informationsgehalt} & \\\hline
	BPMID & Project Management ID \\
	ORDate & Orders Released Date \\
	\label{tab:addlabel}%
\end{longtable}%
%%
%%Wertebereiche der Daten
%%
%Daten mit Kategorie mit Wertebereich von Codebook
\newpage
\begin{longtable}[ht]{p{0.05\textwidth} p{0.25\textwidth} p{0.25\textwidth} p{0.4\textwidth}}
	\caption{Wertebereiche und Kategorien der verwendeten Variablen}\\
	\textbf{No.} & \textbf{Variablencode} & \textbf{Variable Kategorie} & \textbf{Masseinheit/Wertebereich}
	\\\hline\endhead
	5     & BA    & Rahmenbedingungen &  \\
	6     & BU    & Rahmenbedingungen &  \\
	1     & CuNo  & Rahmenbedingungen &  \\
	2     & EquLoc & Rahmenbedingungen &  \\
	7     & MS    & Rahmenbedingungen &  \\
	65    & Region & Rahmenbedingungen &  \\
	16    & BudISTot & Kosten & \% \\
	14    & BudMETot & Kosten & \% \\
	13    & BudMSTot & Kosten & \% \\
	15    & BudPATot & Kosten & \% \\
	79    & CostAct * & Kosten & TCHF \\
	23    & CostActBudISabs & Kosten & TCHF \\
	29    & CostActBudISRel & Kosten & \% \\
	21    & CostActBudMEabs & Kosten & TCHF \\
	27    & CostActBudMERel & Kosten & \% \\
	20    & CostActBudMSabs & Kosten & TCHF \\
	26    & CostActBudMSRel & Kosten & \% \\
	22    & CostActBudPAabs & Kosten & TCHF \\
	28    & CostActBudPARel & Kosten & \% \\
	25    & CostActBudRel & Kosten & \% \\
	80    & CostBud *& Kosten & TCHF \\
	78    & CostBudDevabs * & Kosten & TCHF \\
	18    & DB1Act & Kosten & \% \\
	68    & DB1Actabs & Kosten & TCHF \\
	17    & DB1Bud & Kosten & \% \\
	67    & DB1Budabs & Kosten & TCHF \\
	77    & DB1BudDevabs * & Kosten & TCHF \\
	41    & DeltaLastFCAct & Kosten & TCHF \\
	45    & DeltaLastFCActIS & Kosten & TCHF \\
	43    & DeltaLastFCActME & Kosten & TCHF \\
	42    & DeltaLastFCActMS & Kosten & TCHF \\
	44    & DeltaLastFCActPA & Kosten & TCHF \\
	24    & SUCostTO & Kosten & \% \\
	66    & TOAct & Kosten & TCHF \\
	12    & TOBud & Kosten & TCHF \\
	75    & TOBud\_Cat * & Kosten &  \\
	76    & TOBudDevabs * & Kosten & TCHF \\
	81    & Cat\_age * & Fulfillment &  \\
	30    & CostFCadj & Fulfillment & \{0,1\} \\
	34    & CostFCadjIS & Fulfillment & \{0,1,2\} \\
	32    & CostFCadjME & Fulfillment & \{0,1,2\}\\
	31    & CostFCadjMS & Fulfillment & \{0,1,2\} \\
	33    & CostFCadjPA & Fulfillment & \{0,1,2\} \\
	38    & HOMRedCost & Fulfillment & Monate \\
	39    & HOMRedQual & Fulfillment & Monate \\
	40    & HOMRedTime & Fulfillment & Monate \\
	35    & HOMYellCost & Fulfillment & Monate \\
	36    & HOMYellQual & Fulfillment & Monate \\
	37    & HOMYellTime & Fulfillment & Monate \\
	9     & LeadSAS.PrFF & Fulfillment & \{NO,YES\} \\
	8     & LeadSASPr & Fulfillment &  \\
	58    & NoLeadSASFF & Fulfillment & Anzahl\\
	57    & NoPM  & Fulfillment & Anzahl \\
	69    & PMAge2 & Fulfillment & Monate \\
	4     & PMChange & Fulfillment & \{YES,NO\} \\
	3     & PMNo  & Fulfillment &  \\
	70    & PMTen2 & Fulfillment & Monate \\
	11    & PrStartDate & Fulfillment & DD-MM-YYYY \\
	74    & Delay * & Zeit  & \{TRUE,FALSE\} \\
	51    & PrTimeAct & Zeit  & Monate \\
	50    & PrTimeBase & Zeit  & Monate \\
	52    & PrTimeDelay & Zeit  & Monate \\
	55    & PrTimeDelayMS10 & Zeit  & Monate \\
	56    & PrTimeDelayMS11 & Zeit  & Monate \\
	53    & PrTimeDelayMS2 & Zeit  & Monate \\
	54    & PrTimeDelayMS8 & Zeit  & Monate \\
	10    & ConPart & Komplexität & \{TRUE,FALSE\} \\
	64    & NoContr & Komplexität & Anzahl \\
	59    & NoSupplSAS & Komplexität & Anzahl \\
	63    & NoSupplSASIS & Komplexität & Anzahl \\
	61    & NoSupplSASME & Komplexität & Anzahl \\
	60    & NoSupplSASMS & Komplexität & Anzahl \\
	62    & NoSupplSASPA & Komplexität & Anzahl \\
	19    & DB1BudDev & Erfolgskriterium & \% \\
	73    & Dummy\_Fail * & Erfolgskriterium & \{0,1\} \\
	72    & Dummy\_Success * & Erfolgskriterium & \{0,1\} \\
	71    & Success * & Erfolgskriterium & \{YES,NO\} \\
	46    & BUORBudGapAbs & Sales \& Qotation & TCHF \\
	47    & BUORBudGapRel & Sales \& Qotation & \% \\
	48    & RegiORBudGapAbs & Sales \& Qotation & TCHF \\
	49    & RegiORBudGapRel & Sales \& Qotation & \% \\
\end{longtable}
%%
%%Variablen mit Formel
%%
\newpage
\begin{longtable}{p{0.5cm}|p{4cm}|p{3.5cm}|p{6cm}}
	\caption{Berechnungsformel der verwendeten Variablen}\\
	\textbf{No.} & \textbf{Kategorie} & {\textbf{Variablencode}} & {\textbf{Berechnungsformel}} \\\hline\endhead
	19    & Erfolgskriterium & DB1BudDev & $DB1Act - DB1Bud$ \\
	71    & Erfolgskriterium & Success *& $DB1Act \geq 0 = TRUE$ \\
	72    & Erfolgskriterium & Dummy\_Success *& $DB1Act \geq 0 = 1$\\
	73    & Erfolgskriterium & Dummy\_Fail *& $DB1Act < 0 = 1$\\\hline
	1     & Rahmenbedingungen & CuNo & \\
	2     & Rahmenbedingungen & EquLoc & \\
	5     & Rahmenbedingungen & BA & \\
	6     & Rahmenbedingungen & BU & \\
	7     & Rahmenbedingungen & MS  &\\
	65    & Rahmenbedingungen & Region & \\\hline
	12    & Kosten & TOBud &\\
	13    & Kosten & BudMSTot & $\frac{Cost Bud_{MS}}{Cost Bud_{Total}}$\\ [3mm]
	14    & Kosten & BudMETot & $\frac{Cost Bud_{ME}}{Cost Bud_{Total}}$ \\[3mm] 
	15    & Kosten & BudPATot & $\frac{Cost Bud_{PA}}{Cost Bud_{Total}}$\\ [3mm]
	16    & Kosten & BudISTot & $\frac{Cost Bud_{IS}}{Cost Bud_{Total}}$\\ [3mm]
	17    & Kosten & DB1Bud & $\frac{(TOBud - DB1Bud)}{TOBud}$ \\ [3mm]
	18    & Kosten & DB1Act & $\frac{(TOAct - DB1Act)}{TOAct}$ \\ [3mm]
	20    & Kosten & CostActBudMSabs & $(-CostAct_{MS})-(-CostBud_{MS})$ \\ 
	21    & Kosten & CostActBudMEabs & $(-CostAct_{ME})-(-CostBud_{ME})$ \\
	22    & Kosten & CostActBudPAabs & $(-CostAct_{PA})-(-CostBud_{PA})$ \\
	23    & Kosten & CostActBudISabs & $(-CostAct_{IS})-(-CostBud_{IS})$ \\
	24    & Kosten & SUCostTO & $\frac{(-SUAct)}{TOAct}$ \\ [3mm]
	25    & Kosten & CostActBudRel & $\frac{(-CostAct_{Total}) - (-CostBud_{Total})}{(-CostBud_{Total})}$ \\ [3mm]
	26    & Kosten & CostActBudMSRel & $\frac{(-CostAct_{MS}) - (-CostBud_{MS})}{(-CostBud_{MS})}$ \\ [3mm]
	27    & Kosten & CostActBudMERel & $\frac{(-CostAct_{ME}) - (-CostBud_{MS})}{(-CostBud_{ME})}$ \\ [3mm]
	28    & Kosten & CostActBudPARel & $\frac{(-CostAct_{PA}) - (-CostBud_{MS})}{(-CostBud_{PA})}$ \\ [3mm]
	29    & Kosten & CostActBudISRel & $\frac{(-CostAct_{IS}) - (-CostBud_{MS})}{(-CostBud_{IS})}$ \\ [3mm]
	41    & Kosten & DeltaLastFCAct & $Last Cost FC-Cost Act$ \\
	42    & Kosten & DeltaLastFCActMS & $Last Cost FC_{MS}-Cost Act_{MS}$ \\
	43    & Kosten & DeltaLastFCActME & $Last Cost FC_{ME}-Cost Act_{ME}$ \\
	44    & Kosten & DeltaLastFCActPA & $Last Cost FC_{PA}-Cost Act_{PA}$\\
	45    & Kosten & DeltaLastFCActIS & $Last Cost FC_{IS}-Cost Act_{IS}$\\
	66    & Kosten & TOAct &\\
	67    & Kosten & DB1Budabs & $TOBud-CostBud$\\
	68    & Kosten & DB1Actabs & $TOAct-CostAct$\\
	75    & Kosten & TOBud\_Cat * &
	\small Klasse 1: $13.2 \geq TOBud < 500$ \newline Klassen 2 bis 10: \newline $500 \geq TOBud < 5'000$\newline Klassenbreite = 500 \newline Klasse 11: $5000 \geq TOBud < 10'000$\newline Klasse 12: $ TOBud \geq 10'000$\\
	76    & Kosten & TOBudDevabs * & $TOAct-TOBud$\\
	77    & Kosten & DB1BudDevabs * & $DB1Actabs-DB1Budabs$\\
	78    & Kosten & CostBudDevabs * & $(-CostAct)-(-CostBud)$ \\
	79    & Kosten & CostAct *& $TOAct-DB1Actabs$\\
	80    & Kosten & CostBud * & $TOBud-DB1Budabs$\\\hline
	3     & Fulfillment & PMNo & \\
	4     & Fulfillment & PMChange & YES = Wechsel des Projektmanager \\
	8     & Fulfillment & LeadSASPr & \\
	9     & Fulfillment & LeadSAS.PrFF & NO = LeadSASPr verschieden von LeadSASFF  \\
	11    & Fulfillment & PrStartDate & \\
	30    & Fulfillment & CostFCadj & 
	\small0 = keine FC-Anpassung oder Umsatz und Kosten wurden angepasst\newline 1 = FC-Anpassung\\ [3mm]
	31    & Fulfillment & CostFCadjMS &
	\small 0 = keine FC-Anpassung\newline 1 = FC-Anpassung für weniger Kosten\newline2 = FC-Anpassung für mehr Kosten  \\[3mm]
	32    & Fulfillment & CostFCadjME & 
	\small 0 = keine FC-Anpassung\newline 1 = FC-Anpassung für weniger Kosten\newline2 = FC-Anpassung für mehr Kosten
	\\[3mm]
	33    & Fulfillment & CostFCadjPA &
	\small 0 = keine FC-Anpassung\newline 1 = FC-Anpassung für weniger Kosten\newline2 = FC-Anpassung für mehr Kosten
	\\[3mm]
	34    & Fulfillment & CostFCadjIS &
	\small 0 = keine FC-Anpassung\newline 1 = FC-Anpassung für weniger Kosten\newline2 = FC-Anpassung für mehr Kosten
	\\[3mm]
	35    & Fulfillment & HOMYellCost & $\frac{(Date(first Yellow Status)_{Cost} - (Date(HOM))}{Project baseline}$ \\ [3mm]
	36    & Fulfillment & HOMYellQual & $\frac{(Date(first Yellow Status)_{Quality} - (Date(HOM))}{Project baseline}$ \\ [3mm]
	37    & Fulfillment & HOMYellTime & $\frac{(Date(first Yellow Status)_{Time} - (Date(HOM))}{Project baseline}$ \\ [3mm]
	38    & Fulfillment & HOMRedCost & $\frac{(Date(first Red Status)_{Cost} - (Date(HOM))}{Project baseline}$ \\ [3mm]
	39    & Fulfillment & HOMRedQual & $\frac{(Date(first Red Status)_{Quality} - (Date(HOM))}{Project baseline}$ \\ [3mm]
	40    & Fulfillment & HOMRedTime & $\frac{(Date(first Red Status)_{Time} - (Date(HOM))}{Project baseline}$ \\
	57    & Fulfillment & NoPM & \\
	58    & Fulfillment & NoLeadSASFF & \\
	69    & Fulfillment & PMAge2 & \\
	70    & Fulfillment & PMTen2 & \\
	81    & Fulfillment & Cat\_age * & \small min = 20, max = 63, Klassenbreite = 5\\\hline
	46    & Sales \& Qotation & BUORBudGapAbs & $OR Act_{BU}-OR Bud_{BU}$ \\
	47    & Sales \& Qotation & BUORBudGapRel & $\frac{(ORAct_{BU}-ORBud_{BU})}{ORBud_{BU}}$\\
	48    & Sales \& Qotation & RegiORBudGapAbs & $OR Act_{Region}-OR Bud_{Region}$\\
	49    & Sales \& Qotation & RegiORBudGapRel & $\frac{(ORAct_{Region}-ORBud_{Region})}{ORBud_{Region}}$\\\hline
	50    & Zeit  & PrTimeBase & $Baseline_{MS11}-Baseline_{MS1}$ \\
	51    & Zeit  & PrTimeAct & $Act_{MS10}-Act_{MS1}$ \\
	52    & Zeit  & PrTimeDelay & $PrTimeBase - PrTimeAct$\\
	53    & Zeit  & PrTimeDelayMS2 & $Baseline_{MS2}-Act_{MS2}$\\
	54    & Zeit  & PrTimeDelayMS8 & $Baseline_{MS8}-Act_{MS8}$\\
	55    & Zeit  & PrTimeDelayMS10 & $Baseline_{MS10}-Act_{MS10}$\\
	56    & Zeit  & PrTimeDelayMS11 & $Baseline_{MS11}-Act_{MS11}$\\
	74    & Zeit  & Delay * & $PrTimeDelay \geq 0 = TRUE$\\\hline
	10    & Komplexität & ConPart & \\
	59    & Komplexität & NoSupplSAS & \\
	60    & Komplexität & NoSupplSASMS & \\
	61    & Komplexität & NoSupplSASME & \\
	62    & Komplexität & NoSupplSASPA & \\
	63    & Komplexität & NoSupplSASIS & \\
	64    & Komplexität & NoContr & \\	
\end{longtable}
%%
%%%Tabelle mit Anzahl NA's
%%
\newpage
Nachfolgend werden die Tabellen von Auswertungen, welche nicht im Kapitel 3 enthalten sind aufgeführt.
% Table generated by Excel2LaTeX from sheet 'routput'
\begin{table}[htbp]
	
	\centering
	\caption{Anzahl NA's pro Variable}
	\begin{tabular}{lr}
		\textbf{Variablencode} & \multicolumn{1}{l}{\textbf{Anzahl NA's}} \\\hline
		PrTimeDelayMS5 & 538 \\
		AMAge2 & 444 \\
		AMTen2 & 444 \\
		PrTimeDelay & 254 \\
		PrTimeDelayMS11 & 227 \\
		PrTimeDelayMS10 & 214 \\
		PrTimeAct & 192 \\
		PrTimeDelayMS2 & 156 \\
		PrTimeDelayMS8 & 139 \\
		AMNo  & 132 \\
		PrTimeBase & 118 \\
		PMAge2 & 98 \\
		PMTen2 & 98 \\
		PrStartDate & 13 \\
		PMNo  & 6 \\
		BA    & 6 \\
		BU    & 6 \\
		TOAct & 6 \\
		DB1Budabs & 6 \\
		DB1Actabs & 6 \\
		EquLoc & 2 \\
	\end{tabular}%
	\label{tab:na}%
\end{table}%
\newpage
%Tabelle 2: Plausibilität
%Vorab muss erläutert werden, dass der Wert $1'111'111$ bei den CostMostnegFCajd-Variablen keinen Ausreisser darstellt sondern angibt, dass das Projekt nur positive FC-Anpassungen gehabt hat. Dies impliziert, dass der Forecast für die Kosten gesunken sind und somit weniger Kosten erwartet wurden, wobei der Umsatz konstant geblieben oder gestiegen ist. Bei den HOM-YellowStatus und die HOM-RedStatus drückt der Wert $1'111'111$ aus, dass der entsprechende Status nicht als erstes oder gar nicht aufgetreten ist. Die Interpretation wäre somit, dass bei den HOMYellow-Status Variablen der Status immer grün war oder zuerst respektive direkt den roten Status hatte. Eine ähnliche Interpretation gilt für HOMRedStatus-Variablen, somit hätte dieses Projekt, den roten Status gar nicht erst erreicht. Da diese Interpretationen valide sind und somit keine fehlende Werte darstellen, verbleiben sie im Datensatz. 
%Tabelle 3: Ausreisser
%Diagramme: Boxplots Histogramme aller numerischen Variablen
%Tabelle 4: Zusätzliche Variablen: faktisch Übersicht aller Variablen 
%Tabelle 5: Berechnungsformeln
%Tabelle 6: Interpretation
%%%%%%%%%%%%%%%%%%%%%%%%%%%%%%%%%%%%%%%%%%%%%%%%%%%%%%%%%%%%%%%%%%%%%%%%%%%%%%
%%
%%Weitere Tabellen, die nicht im Text erhalten sind
%%
%%Rahmenbedingungen
%%Kosten
\textbf{Fulfillment}
%Erfolgsquote und Häufigkeitsverteilung der Projektmanager
\begin{longtable}{lrrrrr}
	\caption{Erfolgsquote und Häufigkeitsverteilung pro Projektmanager}\\
	\textbf{PMNo} & \multicolumn{1}{c}{\textbf{Erfolgsquote}} & {\textbf{Success}} &
	\textbf{Fail} & \textbf{Fail [\%]} & \textbf{Total} \\\hline\endhead
	\multicolumn{1}{r}{712067} & 0.6   & 8     & 14    & 63.6\% & 22 \\
	\multicolumn{1}{r}{616048} & 0.7   & 4     & 6     & 60.0\% & 10 \\
	\multicolumn{1}{r}{7110189} & 0.7   & 4     & 6     & 60.0\% & 10 \\
	\multicolumn{1}{r}{1110883} & 0.8   & 4     & 5     & 55.6\% & 9 \\
	\multicolumn{1}{r}{1515253} & 0.2   & 1     & 5     & 83.3\% & 6 \\
	\multicolumn{1}{r}{21747} & 4.0   & 20    & 5     & 20.0\% & 25 \\
	\multicolumn{1}{r}{28160} & 4.0   & 20    & 5     & 20.0\% & 25 \\
	\multicolumn{1}{r}{618240} & 0.4   & 2     & 5     & 71.4\% & 7 \\
	\multicolumn{1}{r}{1110920} & 0.3   & 1     & 4     & 80.0\% & 5 \\
	\multicolumn{1}{r}{1910333} & 1.5   & 6     & 4     & 40.0\% & 10 \\
	\multicolumn{1}{r}{20071} & 0.8   & 3     & 4     & 57.1\% & 7 \\
	\multicolumn{1}{r}{20405} & 0.8   & 3     & 4     & 57.1\% & 7 \\
	\multicolumn{1}{r}{25005} & 0.5   & 2     & 4     & 66.7\% & 6 \\
	\multicolumn{1}{r}{5116088} & 1.5   & 6     & 4     & 40.0\% & 10 \\
	\multicolumn{1}{r}{616291} & 1.8   & 7     & 4     & 36.4\% & 11 \\
	\multicolumn{1}{r}{618787} & 0.0   & 0     & 4     & 100.0\% & 4 \\
	\multicolumn{1}{r}{712066} & 2.5   & 10    & 4     & 28.6\% & 14 \\
	\multicolumn{1}{r}{1519603} & 2.3   & 7     & 3     & 30.0\% & 10 \\
	\multicolumn{1}{r}{2119231} & 1.0   & 3     & 3     & 50.0\% & 6 \\
	\multicolumn{1}{r}{2119285} & 0.0   & 0     & 3     & 100.0\% & 3 \\
	\multicolumn{1}{r}{22784} & 0.7   & 2     & 3     & 60.0\% & 5 \\
	\multicolumn{1}{r}{27135} & 1.7   & 5     & 3     & 37.5\% & 8 \\
	\multicolumn{1}{r}{28371} & 0.0   & 0     & 3     & 100.0\% & 3 \\
	\multicolumn{1}{r}{4610121} & 0.0   & 0     & 3     & 100.0\% & 3 \\
	\multicolumn{1}{r}{510134} & 0.3   & 1     & 3     & 75.0\% & 4 \\
	\multicolumn{1}{r}{510204} & 0.0   & 0     & 3     & 100.0\% & 3 \\
	\multicolumn{1}{r}{510288} & 0.3   & 1     & 3     & 75.0\% & 4 \\
	\multicolumn{1}{r}{5116453} & 0.7   & 2     & 3     & 60.0\% & 5 \\
	\multicolumn{1}{r}{616174} & 1.0   & 3     & 3     & 50.0\% & 6 \\
	\multicolumn{1}{r}{617096} & 0.3   & 1     & 3     & 75.0\% & 4 \\
	\multicolumn{1}{r}{617861} & 0.7   & 2     & 3     & 60.0\% & 5 \\
	\multicolumn{1}{r}{618270} & 0.3   & 1     & 3     & 75.0\% & 4 \\
	\multicolumn{1}{r}{7110014} & 1.3   & 4     & 3     & 42.9\% & 7 \\
	\multicolumn{1}{r}{7110287} & 0.3   & 1     & 3     & 75.0\% & 4 \\
	\multicolumn{1}{r}{811918} & 0.3   & 1     & 3     & 75.0\% & 4 \\
	\multicolumn{1}{r}{1110966} & 1.5   & 3     & 2     & 40.0\% & 5 \\
	\multicolumn{1}{r}{1210380} & 0.0   & 0     & 2     & 100.0\% & 2 \\
	\multicolumn{1}{r}{1512545} & 0.0   & 0     & 2     & 100.0\% & 2 \\
	\multicolumn{1}{r}{1512802} & 1.5   & 3     & 2     & 40.0\% & 5 \\
	\multicolumn{1}{r}{1513083} & 0.5   & 1     & 2     & 66.7\% & 3 \\
	\multicolumn{1}{r}{1514618} & 1.5   & 3     & 2     & 40.0\% & 5 \\
	\multicolumn{1}{r}{1517859} & 1.5   & 3     & 2     & 40.0\% & 5 \\
	\multicolumn{1}{r}{15508} & 0.0   & 0     & 2     & 100.0\% & 2 \\
	\multicolumn{1}{r}{17670} & 0.0   & 0     & 2     & 100.0\% & 2 \\
	\multicolumn{1}{r}{1910417} & 1.0   & 2     & 2     & 50.0\% & 4 \\
	\multicolumn{1}{r}{1910439} & 0.0   & 0     & 2     & 100.0\% & 2 \\
	\multicolumn{1}{r}{1910485} & 1.0   & 2     & 2     & 50.0\% & 4 \\
	\multicolumn{1}{r}{19148} & 0.0   & 0     & 2     & 100.0\% & 2 \\
	\multicolumn{1}{r}{2112169} & 2.0   & 4     & 2     & 33.3\% & 6 \\
	\multicolumn{1}{r}{2112187} & 1.5   & 3     & 2     & 40.0\% & 5 \\
	\multicolumn{1}{r}{2119240} & 2.0   & 4     & 2     & 33.3\% & 6 \\
	\multicolumn{1}{r}{2119306} & 0.0   & 0     & 2     & 100.0\% & 2 \\
	\multicolumn{1}{r}{2210237} & 0.0   & 0     & 2     & 100.0\% & 2 \\
	\multicolumn{1}{r}{25339} & 1.0   & 2     & 2     & 50.0\% & 4 \\
	\multicolumn{1}{r}{25432} & 1.0   & 2     & 2     & 50.0\% & 4 \\
	\multicolumn{1}{r}{25445} & 0.5   & 1     & 2     & 66.7\% & 3 \\
	\multicolumn{1}{r}{26360} & 3.0   & 6     & 2     & 25.0\% & 8 \\
	\multicolumn{1}{r}{27080} & 1.0   & 2     & 2     & 50.0\% & 4 \\
	\multicolumn{1}{r}{3611931} & 1.0   & 2     & 2     & 50.0\% & 4 \\
	\multicolumn{1}{r}{4510504} & 1.5   & 3     & 2     & 40.0\% & 5 \\
	\multicolumn{1}{r}{4510613} & 1.0   & 2     & 2     & 50.0\% & 4 \\
	\multicolumn{1}{r}{4533123} & 1.5   & 3     & 2     & 40.0\% & 5 \\
	\multicolumn{1}{r}{4610078} & 0.0   & 0     & 2     & 100.0\% & 2 \\
	\multicolumn{1}{r}{4610118} & 2.5   & 5     & 2     & 28.6\% & 7 \\
	\multicolumn{1}{r}{510038} & 0.0   & 0     & 2     & 100.0\% & 2 \\
	\multicolumn{1}{r}{510232} & 0.5   & 1     & 2     & 66.7\% & 3 \\
	\multicolumn{1}{r}{510254} & 0.5   & 1     & 2     & 66.7\% & 3 \\
	\multicolumn{1}{r}{5116095} & 1.0   & 2     & 2     & 50.0\% & 4 \\
	\multicolumn{1}{r}{5116518} & 0.0   & 0     & 2     & 100.0\% & 2 \\
	\multicolumn{1}{r}{5116583} & 1.0   & 2     & 2     & 50.0\% & 4 \\
	\multicolumn{1}{r}{616193} & 2.0   & 4     & 2     & 33.3\% & 6 \\
	\multicolumn{1}{r}{617531} & 0.5   & 1     & 2     & 66.7\% & 3 \\
	\multicolumn{1}{r}{617709} & 1.0   & 2     & 2     & 50.0\% & 4 \\
	\multicolumn{1}{r}{618107} & 1.0   & 2     & 2     & 50.0\% & 4 \\
	\multicolumn{1}{r}{618392} & 2.5   & 5     & 2     & 28.6\% & 7 \\
	\multicolumn{1}{r}{618499} & 0.5   & 1     & 2     & 66.7\% & 3 \\
	\multicolumn{1}{r}{619070} & 0.5   & 1     & 2     & 66.7\% & 3 \\
	\multicolumn{1}{r}{7110678} & 0.0   & 0     & 2     & 100.0\% & 2 \\
	\multicolumn{1}{r}{7110792} & 1.0   & 2     & 2     & 50.0\% & 4 \\
	\multicolumn{1}{r}{712069} & 0.0   & 0     & 2     & 100.0\% & 2 \\
	\multicolumn{1}{r}{811738} & 2.0   & 4     & 2     & 33.3\% & 6 \\
	\multicolumn{1}{r}{1110986} & 1.0   & 1     & 1     & 50.0\% & 2 \\
	\multicolumn{1}{r}{1110988} & 0.0   & 0     & 1     & 100.0\% & 1 \\
	\multicolumn{1}{r}{11910} & 3.0   & 3     & 1     & 25.0\% & 4 \\
	\multicolumn{1}{r}{1211259} & 1.0   & 1     & 1     & 50.0\% & 2 \\
	\multicolumn{1}{r}{1211260} & 1.0   & 1     & 1     & 50.0\% & 2 \\
	\multicolumn{1}{r}{12266} & 1.0   & 1     & 1     & 50.0\% & 2 \\
	\multicolumn{1}{r}{13591} & 1.0   & 1     & 1     & 50.0\% & 2 \\
	\multicolumn{1}{r}{13718} & 0.0   & 0     & 1     & 100.0\% & 1 \\
	\multicolumn{1}{r}{13889} & 1.0   & 1     & 1     & 50.0\% & 2 \\
	\multicolumn{1}{r}{1512883} & 1.0   & 1     & 1     & 50.0\% & 2 \\
	\multicolumn{1}{r}{1513753} & 0.0   & 0     & 1     & 100.0\% & 1 \\
	\multicolumn{1}{r}{1515461} & 0.0   & 0     & 1     & 100.0\% & 1 \\
	\multicolumn{1}{r}{1517857} & 1.0   & 1     & 1     & 50.0\% & 2 \\
	\multicolumn{1}{r}{17819} & 2.0   & 2     & 1     & 33.3\% & 3 \\
	\multicolumn{1}{r}{18570} & 0.0   & 0     & 1     & 100.0\% & 1 \\
	\multicolumn{1}{r}{18918} & 0.0   & 0     & 1     & 100.0\% & 1 \\
	\multicolumn{1}{r}{1910272} & 0.0   & 0     & 1     & 100.0\% & 1 \\
	\multicolumn{1}{r}{1910335} & 1.0   & 1     & 1     & 50.0\% & 2 \\
	\multicolumn{1}{r}{1910509} & 0.0   & 0     & 1     & 100.0\% & 1 \\
	\multicolumn{1}{r}{19350} & 4.0   & 4     & 1     & 20.0\% & 5 \\
	\multicolumn{1}{r}{19401} & 0.0   & 0     & 1     & 100.0\% & 1 \\
	\multicolumn{1}{r}{19815} & 0.0   & 0     & 1     & 100.0\% & 1 \\
	\multicolumn{1}{r}{20108} & 0.0   & 0     & 1     & 100.0\% & 1 \\
	\multicolumn{1}{r}{20562} & 0.0   & 0     & 1     & 100.0\% & 1 \\
	\multicolumn{1}{r}{20810} & 1.0   & 1     & 1     & 50.0\% & 2 \\
	\multicolumn{1}{r}{2111889} & 3.0   & 3     & 1     & 25.0\% & 4 \\
	\multicolumn{1}{r}{2112077} & 0.0   & 0     & 1     & 100.0\% & 1 \\
	\multicolumn{1}{r}{2112154} & 0.0   & 0     & 1     & 100.0\% & 1 \\
	\multicolumn{1}{r}{2112322} & 0.0   & 0     & 1     & 100.0\% & 1 \\
	\multicolumn{1}{r}{2119305} & 1.0   & 1     & 1     & 50.0\% & 2 \\
	\multicolumn{1}{r}{21611} & 0.0   & 0     & 1     & 100.0\% & 1 \\
	\multicolumn{1}{r}{22029} & 8.0   & 8     & 1     & 11.1\% & 9 \\
	\multicolumn{1}{r}{23236} & 3.0   & 3     & 1     & 25.0\% & 4 \\
	\multicolumn{1}{r}{23489} & 0.0   & 0     & 1     & 100.0\% & 1 \\
	\multicolumn{1}{r}{24090} & 1.0   & 1     & 1     & 50.0\% & 2 \\
	\multicolumn{1}{r}{24994} & 1.0   & 1     & 1     & 50.0\% & 2 \\
	\multicolumn{1}{r}{25000} & 0.0   & 0     & 1     & 100.0\% & 1 \\
	\multicolumn{1}{r}{25159} & 1.0   & 1     & 1     & 50.0\% & 2 \\
	\multicolumn{1}{r}{25561} & 0.0   & 0     & 1     & 100.0\% & 1 \\
	\multicolumn{1}{r}{25687} & 2.0   & 2     & 1     & 33.3\% & 3 \\
	\multicolumn{1}{r}{25964} & 2.0   & 2     & 1     & 33.3\% & 3 \\
	\multicolumn{1}{r}{26036} & 0.0   & 0     & 1     & 100.0\% & 1 \\
	\multicolumn{1}{r}{26921} & 1.0   & 1     & 1     & 50.0\% & 2 \\
	\multicolumn{1}{r}{27016} & 3.0   & 3     & 1     & 25.0\% & 4 \\
	\multicolumn{1}{r}{27034} & 1.0   & 1     & 1     & 50.0\% & 2 \\
	\multicolumn{1}{r}{310266} & 0.0   & 0     & 1     & 100.0\% & 1 \\
	\multicolumn{1}{r}{410316} & 1.0   & 1     & 1     & 50.0\% & 2 \\
	\multicolumn{1}{r}{4512231} & 1.0   & 1     & 1     & 50.0\% & 2 \\
	\multicolumn{1}{r}{4512234} & 0.0   & 0     & 1     & 100.0\% & 1 \\
	\multicolumn{1}{r}{4513192} & 0.0   & 0     & 1     & 100.0\% & 1 \\
	\multicolumn{1}{r}{4523184} & 1.0   & 1     & 1     & 50.0\% & 2 \\
	\multicolumn{1}{r}{4533116} & 2.0   & 2     & 1     & 33.3\% & 3 \\
	\multicolumn{1}{r}{4533138} & 1.0   & 1     & 1     & 50.0\% & 2 \\
	\multicolumn{1}{r}{4543141} & 0.0   & 0     & 1     & 100.0\% & 1 \\
	\multicolumn{1}{r}{4553178} & 8.0   & 8     & 1     & 11.1\% & 9 \\
	\multicolumn{1}{r}{4563170} & 2.0   & 2     & 1     & 33.3\% & 3 \\
	\multicolumn{1}{r}{4610177} & 3.0   & 3     & 1     & 25.0\% & 4 \\
	\multicolumn{1}{r}{510166} & 0.0   & 0     & 1     & 100.0\% & 1 \\
	\multicolumn{1}{r}{510187} & 3.0   & 3     & 1     & 25.0\% & 4 \\
	\multicolumn{1}{r}{617097} & 0.0   & 0     & 1     & 100.0\% & 1 \\
	\multicolumn{1}{r}{617408} & 0.0   & 0     & 1     & 100.0\% & 1 \\
	\multicolumn{1}{r}{617628} & 1.0   & 1     & 1     & 50.0\% & 2 \\
	\multicolumn{1}{r}{617767} & 0.0   & 0     & 1     & 100.0\% & 1 \\
	\multicolumn{1}{r}{617823} & 1.0   & 1     & 1     & 50.0\% & 2 \\
	\multicolumn{1}{r}{617949} & 0.0   & 0     & 1     & 100.0\% & 1 \\
	\multicolumn{1}{r}{618323} & 2.0   & 2     & 1     & 33.3\% & 3 \\
	\multicolumn{1}{r}{710120} & 0.0   & 0     & 1     & 100.0\% & 1 \\
	\multicolumn{1}{r}{7110106} & 0.0   & 0     & 1     & 100.0\% & 1 \\
	\multicolumn{1}{r}{7110109} & 2.0   & 2     & 1     & 33.3\% & 3 \\
	\multicolumn{1}{r}{7110186} & 0.0   & 0     & 1     & 100.0\% & 1 \\
	\multicolumn{1}{r}{7110289} & 0.0   & 0     & 1     & 100.0\% & 1 \\
	\multicolumn{1}{r}{7110299} & 0.0   & 0     & 1     & 100.0\% & 1 \\
	\multicolumn{1}{r}{7110609} & 1.0   & 1     & 1     & 50.0\% & 2 \\
	\multicolumn{1}{r}{712312} & 10.0  & 10    & 1     & 9.1\% & 11 \\
	\multicolumn{1}{r}{811974} & 0.0   & 0     & 1     & 100.0\% & 1 \\
	\multicolumn{1}{r}{9910051} & 0.0   & 0     & 1     & 100.0\% & 1 \\
	\multicolumn{1}{r}{9910221} & 1.0   & 1     & 1     & 50.0\% & 2 \\
	\multicolumn{1}{r}{9910250} & 1.0   & 1     & 1     & 50.0\% & 2 \\
	\multicolumn{1}{r}{9910375} & 5.0   & 5     & 1     & 16.7\% & 6 \\
	\multicolumn{1}{r}{1110867} & NA    & 2     & 0     & 0.0\% & 2 \\
	\multicolumn{1}{r}{11890} & NA    & 5     & 0     & 0.0\% & 5 \\
	\multicolumn{1}{r}{1211275} & NA    & 1     & 0     & 0.0\% & 1 \\
	\multicolumn{1}{r}{1211289} & NA    & 4     & 0     & 0.0\% & 4 \\
	\multicolumn{1}{r}{1211314} & NA    & 1     & 0     & 0.0\% & 1 \\
	\multicolumn{1}{r}{12340} & NA    & 1     & 0     & 0.0\% & 1 \\
	\multicolumn{1}{r}{12349} & NA    & 1     & 0     & 0.0\% & 1 \\
	\multicolumn{1}{r}{13392} & NA    & 8     & 0     & 0.0\% & 8 \\
	\multicolumn{1}{r}{13497} & NA    & 1     & 0     & 0.0\% & 1 \\
	\multicolumn{1}{r}{13563} & NA    & 1     & 0     & 0.0\% & 1 \\
	\multicolumn{1}{r}{15064} & NA    & 5     & 0     & 0.0\% & 5 \\
	\multicolumn{1}{r}{1511240} & NA    & 1     & 0     & 0.0\% & 1 \\
	\multicolumn{1}{r}{1512077} & NA    & 1     & 0     & 0.0\% & 1 \\
	\multicolumn{1}{r}{1512612} & NA    & 1     & 0     & 0.0\% & 1 \\
	\multicolumn{1}{r}{1513595} & NA    & 1     & 0     & 0.0\% & 1 \\
	\multicolumn{1}{r}{1516955} & NA    & 2     & 0     & 0.0\% & 2 \\
	\multicolumn{1}{r}{1517907} & NA    & 1     & 0     & 0.0\% & 1 \\
	\multicolumn{1}{r}{1518993} & NA    & 1     & 0     & 0.0\% & 1 \\
	\multicolumn{1}{r}{15803} & NA    & 1     & 0     & 0.0\% & 1 \\
	\multicolumn{1}{r}{16046} & NA    & 1     & 0     & 0.0\% & 1 \\
	\multicolumn{1}{r}{17397} & NA    & 4     & 0     & 0.0\% & 4 \\
	\multicolumn{1}{r}{17638} & NA    & 3     & 0     & 0.0\% & 3 \\
	\multicolumn{1}{r}{17667} & NA    & 1     & 0     & 0.0\% & 1 \\
	\multicolumn{1}{r}{17675} & NA    & 2     & 0     & 0.0\% & 2 \\
	\multicolumn{1}{r}{17678} & NA    & 2     & 0     & 0.0\% & 2 \\
	\multicolumn{1}{r}{18119} & NA    & 1     & 0     & 0.0\% & 1 \\
	\multicolumn{1}{r}{18194} & NA    & 6     & 0     & 0.0\% & 6 \\
	\multicolumn{1}{r}{18231} & NA    & 1     & 0     & 0.0\% & 1 \\
	\multicolumn{1}{r}{18299} & NA    & 1     & 0     & 0.0\% & 1 \\
	\multicolumn{1}{r}{18511} & NA    & 1     & 0     & 0.0\% & 1 \\
	\multicolumn{1}{r}{18628} & NA    & 3     & 0     & 0.0\% & 3 \\
	\multicolumn{1}{r}{18938} & NA    & 1     & 0     & 0.0\% & 1 \\
	\multicolumn{1}{r}{1910240} & NA    & 1     & 0     & 0.0\% & 1 \\
	\multicolumn{1}{r}{1910503} & NA    & 2     & 0     & 0.0\% & 2 \\
	\multicolumn{1}{r}{1910517} & NA    & 2     & 0     & 0.0\% & 2 \\
	\multicolumn{1}{r}{19405} & NA    & 1     & 0     & 0.0\% & 1 \\
	\multicolumn{1}{r}{19421} & NA    & 3     & 0     & 0.0\% & 3 \\
	\multicolumn{1}{r}{19982} & NA    & 1     & 0     & 0.0\% & 1 \\
	\multicolumn{1}{r}{19984} & NA    & 1     & 0     & 0.0\% & 1 \\
	\multicolumn{1}{r}{19985} & NA    & 3     & 0     & 0.0\% & 3 \\
	\multicolumn{1}{r}{20006} & NA    & 1     & 0     & 0.0\% & 1 \\
	\multicolumn{1}{r}{20022} & NA    & 4     & 0     & 0.0\% & 4 \\
	\multicolumn{1}{r}{20028} & NA    & 2     & 0     & 0.0\% & 2 \\
	\multicolumn{1}{r}{20585} & NA    & 3     & 0     & 0.0\% & 3 \\
	\multicolumn{1}{r}{20770} & NA    & 4     & 0     & 0.0\% & 4 \\
	\multicolumn{1}{r}{2111985} & NA    & 1     & 0     & 0.0\% & 1 \\
	\multicolumn{1}{r}{2112002} & NA    & 2     & 0     & 0.0\% & 2 \\
	\multicolumn{1}{r}{2112146} & NA    & 1     & 0     & 0.0\% & 1 \\
	\multicolumn{1}{r}{2112175} & NA    & 2     & 0     & 0.0\% & 2 \\
	\multicolumn{1}{r}{2119270} & NA    & 5     & 0     & 0.0\% & 5 \\
	\multicolumn{1}{r}{21207} & NA    & 1     & 0     & 0.0\% & 1 \\
	\multicolumn{1}{r}{21223} & NA    & 7     & 0     & 0.0\% & 7 \\
	\multicolumn{1}{r}{21508} & NA    & 3     & 0     & 0.0\% & 3 \\
	\multicolumn{1}{r}{21623} & NA    & 3     & 0     & 0.0\% & 3 \\
	\multicolumn{1}{r}{21633} & NA    & 6     & 0     & 0.0\% & 6 \\
	\multicolumn{1}{r}{21828} & NA    & 5     & 0     & 0.0\% & 5 \\
	\multicolumn{1}{r}{21893} & NA    & 1     & 0     & 0.0\% & 1 \\
	\multicolumn{1}{r}{21939} & NA    & 7     & 0     & 0.0\% & 7 \\
	\multicolumn{1}{r}{2210239} & NA    & 2     & 0     & 0.0\% & 2 \\
	\multicolumn{1}{r}{2210241} & NA    & 4     & 0     & 0.0\% & 4 \\
	\multicolumn{1}{r}{2210258} & NA    & 2     & 0     & 0.0\% & 2 \\
	\multicolumn{1}{r}{2210261} & NA    & 3     & 0     & 0.0\% & 3 \\
	\multicolumn{1}{r}{22321} & NA    & 4     & 0     & 0.0\% & 4 \\
	\multicolumn{1}{r}{22499} & NA    & 3     & 0     & 0.0\% & 3 \\
	\multicolumn{1}{r}{22686} & NA    & 2     & 0     & 0.0\% & 2 \\
	\multicolumn{1}{r}{22780} & NA    & 1     & 0     & 0.0\% & 1 \\
	\multicolumn{1}{r}{22854} & NA    & 4     & 0     & 0.0\% & 4 \\
	\multicolumn{1}{r}{23221} & NA    & 1     & 0     & 0.0\% & 1 \\
	\multicolumn{1}{r}{23244} & NA    & 1     & 0     & 0.0\% & 1 \\
	\multicolumn{1}{r}{23875} & NA    & 7     & 0     & 0.0\% & 7 \\
	\multicolumn{1}{r}{23950} & NA    & 1     & 0     & 0.0\% & 1 \\
	\multicolumn{1}{r}{24297} & NA    & 1     & 0     & 0.0\% & 1 \\
	\multicolumn{1}{r}{24411} & NA    & 1     & 0     & 0.0\% & 1 \\
	\multicolumn{1}{r}{24636} & NA    & 1     & 0     & 0.0\% & 1 \\
	\multicolumn{1}{r}{25130} & NA    & 1     & 0     & 0.0\% & 1 \\
	\multicolumn{1}{r}{25175} & NA    & 2     & 0     & 0.0\% & 2 \\
	\multicolumn{1}{r}{25342} & NA    & 1     & 0     & 0.0\% & 1 \\
	\multicolumn{1}{r}{25355} & NA    & 1     & 0     & 0.0\% & 1 \\
	\multicolumn{1}{r}{25356} & NA    & 3     & 0     & 0.0\% & 3 \\
	\multicolumn{1}{r}{25476} & NA    & 7     & 0     & 0.0\% & 7 \\
	\multicolumn{1}{r}{25645} & NA    & 4     & 0     & 0.0\% & 4 \\
	\multicolumn{1}{r}{25914} & NA    & 2     & 0     & 0.0\% & 2 \\
	\multicolumn{1}{r}{25952} & NA    & 1     & 0     & 0.0\% & 1 \\
	\multicolumn{1}{r}{26949} & NA    & 1     & 0     & 0.0\% & 1 \\
	\multicolumn{1}{r}{27038} & NA    & 1     & 0     & 0.0\% & 1 \\
	\multicolumn{1}{r}{27104} & NA    & 7     & 0     & 0.0\% & 7 \\
	\multicolumn{1}{r}{27163} & NA    & 1     & 0     & 0.0\% & 1 \\
	\multicolumn{1}{r}{27698} & NA    & 2     & 0     & 0.0\% & 2 \\
	\multicolumn{1}{r}{29228} & NA    & 1     & 0     & 0.0\% & 1 \\
	\multicolumn{1}{r}{310261} & NA    & 3     & 0     & 0.0\% & 3 \\
	\multicolumn{1}{r}{4025} & NA    & 3     & 0     & 0.0\% & 3 \\
	\multicolumn{1}{r}{410051} & NA    & 3     & 0     & 0.0\% & 3 \\
	\multicolumn{1}{r}{410315} & NA    & 3     & 0     & 0.0\% & 3 \\
	\multicolumn{1}{r}{410341} & NA    & 2     & 0     & 0.0\% & 2 \\
	\multicolumn{1}{r}{410366} & NA    & 1     & 0     & 0.0\% & 1 \\
	\multicolumn{1}{r}{4510521} & NA    & 1     & 0     & 0.0\% & 1 \\
	\multicolumn{1}{r}{4512235} & NA    & 1     & 0     & 0.0\% & 1 \\
	\multicolumn{1}{r}{4513162} & NA    & 7     & 0     & 0.0\% & 7 \\
	\multicolumn{1}{r}{4513184} & NA    & 2     & 0     & 0.0\% & 2 \\
	\multicolumn{1}{r}{4523134} & NA    & 2     & 0     & 0.0\% & 2 \\
	\multicolumn{1}{r}{4523135} & NA    & 3     & 0     & 0.0\% & 3 \\
	\multicolumn{1}{r}{4523177} & NA    & 4     & 0     & 0.0\% & 4 \\
	\multicolumn{1}{r}{4533130} & NA    & 1     & 0     & 0.0\% & 1 \\
	\multicolumn{1}{r}{4533142} & NA    & 1     & 0     & 0.0\% & 1 \\
	\multicolumn{1}{r}{4543128} & NA    & 2     & 0     & 0.0\% & 2 \\
	\multicolumn{1}{r}{4553190} & NA    & 5     & 0     & 0.0\% & 5 \\
	\multicolumn{1}{r}{4563187} & NA    & 5     & 0     & 0.0\% & 5 \\
	\multicolumn{1}{r}{4573142} & NA    & 3     & 0     & 0.0\% & 3 \\
	\multicolumn{1}{r}{4583153} & NA    & 1     & 0     & 0.0\% & 1 \\
	\multicolumn{1}{r}{510029} & NA    & 2     & 0     & 0.0\% & 2 \\
	\multicolumn{1}{r}{510206} & NA    & 1     & 0     & 0.0\% & 1 \\
	\multicolumn{1}{r}{510394} & NA    & 1     & 0     & 0.0\% & 1 \\
	\multicolumn{1}{r}{616217} & NA    & 3     & 0     & 0.0\% & 3 \\
	\multicolumn{1}{r}{617616} & NA    & 3     & 0     & 0.0\% & 3 \\
	\multicolumn{1}{r}{618213} & NA    & 2     & 0     & 0.0\% & 2 \\
	\multicolumn{1}{r}{618512} & NA    & 2     & 0     & 0.0\% & 2 \\
	\multicolumn{1}{r}{618625} & NA    & 1     & 0     & 0.0\% & 1 \\
	\multicolumn{1}{r}{619245} & NA    & 13    & 0     & 0.0\% & 13 \\
	\multicolumn{1}{r}{7110026} & NA    & 1     & 0     & 0.0\% & 1 \\
	\multicolumn{1}{r}{7110066} & NA    & 1     & 0     & 0.0\% & 1 \\
	\multicolumn{1}{r}{7110074} & NA    & 3     & 0     & 0.0\% & 3 \\
	\multicolumn{1}{r}{7110090} & NA    & 1     & 0     & 0.0\% & 1 \\
	\multicolumn{1}{r}{7110093} & NA    & 1     & 0     & 0.0\% & 1 \\
	\multicolumn{1}{r}{7110110} & NA    & 1     & 0     & 0.0\% & 1 \\
	\multicolumn{1}{r}{7110261} & NA    & 1     & 0     & 0.0\% & 1 \\
	\multicolumn{1}{r}{7110630} & NA    & 2     & 0     & 0.0\% & 2 \\
	\multicolumn{1}{r}{7110757} & NA    & 3     & 0     & 0.0\% & 3 \\
	\multicolumn{1}{r}{712065} & NA    & 1     & 0     & 0.0\% & 1 \\
	\multicolumn{1}{r}{712248} & NA    & 1     & 0     & 0.0\% & 1 \\
	\multicolumn{1}{r}{811849} & NA    & 2     & 0     & 0.0\% & 2 \\
	\multicolumn{1}{r}{811896} & NA    & 2     & 0     & 0.0\% & 2 \\
	\multicolumn{1}{r}{811929} & NA    & 3     & 0     & 0.0\% & 3 \\
	\multicolumn{1}{r}{811976} & NA    & 1     & 0     & 0.0\% & 1 \\
	\multicolumn{1}{r}{9910123} & NA    & 3     & 0     & 0.0\% & 3 \\
	\multicolumn{1}{r}{9910135} & NA    & 4     & 0     & 0.0\% & 4 \\
	\multicolumn{1}{r}{9910153} & NA    & 3     & 0     & 0.0\% & 3 \\
	\multicolumn{1}{r}{9910183} & NA    & 1     & 0     & 0.0\% & 1 \\
	\multicolumn{1}{r}{9910210} & NA    & 1     & 0     & 0.0\% & 1 \\
	\multicolumn{1}{r}{9910212} & NA    & 1     & 0     & 0.0\% & 1 \\
	\multicolumn{1}{r}{9910222} & NA    & 8     & 0     & 0.0\% & 8 \\
	\multicolumn{1}{r}{9910229} & NA    & 1     & 0     & 0.0\% & 1 \\
	\multicolumn{1}{r}{9910274} & NA    & 2     & 0     & 0.0\% & 2 \\
	\multicolumn{1}{r}{9910292} & NA    & 1     & 0     & 0.0\% & 1 \\
	\multicolumn{1}{r}{9910294} & NA    & 1     & 0     & 0.0\% & 1 \\
	\multicolumn{1}{r}{9910816} & NA    & 1     & 0     & 0.0\% & 1 \\
	\textbf{Total} & \textbf{2.1} & \textbf{654} & \textbf{312} & \textbf{32.3\%} & \textbf{966} \\
	\label{tab:pmno}
\end{longtable}
%Tabelle PMAge2 und PMTen Projektmanager
\begin{table}[H]
	\centering
	\caption{Arithmetisches Mittel PMAge und PMTen [Jahre]}
	\begin{tabular}{lrr}
		\textbf{Success} & \multicolumn{1}{l}{\textbf{Age}} & \multicolumn{1}{l}{\textbf{Ten}} \\\hline
		FALSE & 41.1 & 12.4 \\
		TRUE  & 39.5 & 11.7 \\
	\end{tabular}%
	\label{ageten}%
\end{table}%

%%
%%Zeit
%%Komplexität
%%%%
%%% Bei der letzteren Variable wird die gesamte Arbeitszeit bei der Bühler AG berücksichtigt, was die Aussagekraft insofern abschwächt, als dass die Erfahrung im Projektmanagement bei der Bühler AG nicht explizit erfasst wird.
\end{appendices}

\end{document}

pdflatex MA.tex


kpsewhich lit.bib