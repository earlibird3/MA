%%Allgemeine Formatierung
\documentclass[11pt]{report} %Einstellung Design des Dokuments und Schriftgrösse 11
\usepackage[T1]{fontenc} %Ausgabe der Umlaute erlauben
\usepackage[utf8]{inputenc} %Eingabe der Umlaute erlauben
\usepackage[german]{babel} %Spracheinstellung Deutsch
\usepackage{hyphenat} %Ermögliche Trennung der Wörter im Blocksatz


%%Seiteneinstellung
\usepackage{lscape} %Erlaube Landschaft-Seiteneinstellung
\usepackage{geometry} %Erlaube Veränderung der Seiteneinstellungen
\geometry{a4paper,left=2.5cm,right=2.5cm,top=2.5cm,bottom=2cm} %Seiten einrichten
\usepackage{textcomp} %Blocksatz einrichten
\usepackage[onehalfspacing]{setspace} %Zeilenabstand setzen

%%Grafiken
\usepackage{graphicx,%Ermögliche Einbindung von Grafiken
	wrapfig} %Grafiken mit umschliessendem Text

%%Tabellen
\usepackage{longtable,%Tabellen über mehrere Seiten ermöglichen
float}  %Forcierung der Tabelle an der Stelle ermöglichen
\usepackage{array} %Erweiterung der Array und Tabellenumgebung

%%Mathematik
\usepackage{amsmath} %Mathematische Eingaben ermöglichen

%Inhaltsverzeichnis - inkludiere Abkürzungs-, Abbildungs- und Tabellenverzeichnis 
\usepackage[nottoc]{tocbibind}
\usepackage{tocloft} %Ermögliche Punkte 
\renewcommand{\cftchapfont}{\normalfont\bfseries}% titles in bold
\renewcommand{\cftchappagefont}{\normalfont\bfseries}% page numbers in bold
\renewcommand{\cftdotsep}{1}
\renewcommand{\cftchapleader}{\bfseries\cftdotfill{\cftsecdotsep}}% dot leaders in bold
\usepackage{titlesec}
\titleformat{\chapter}{\normalfont\LARGE\bfseries}{\thechapter}{1em}{}

%%Abkürzungsverzeichnis / Erklärung Abkürzungsverzeichnis nomencl http://strobelstefan.org/?p=153
\usepackage[acronym,nomain,nopostdot,nonumberlist,toc]{glossaries} 
\makeglossaries % Erstelle Abkürzungsverzeichnis, d.h. .gls Datei
%Einträge des Abkürzungsverzeichnis
\newacronym{abk:db1}{DB1}{Deckungsbeitrag 1}
\newacronym{abk:ME}{ME}{Mechnical Engineering}
\newacronym{abk:PA}{PA}{Plant Automation}
\newacronym{abk:IS}{IS}{Installation}
\newacronym{abk:MS}{MS}{Mechanical Supply}

%%Literaturverzeichnis
\usepackage[natbibapa]{apacite} %Setze APA Zietierungsstandard
\usepackage{hyperref} %Ermögliche Einstellungen für Hyperlinks
\hypersetup{hidelinks = true}

%%Anhang
\usepackage[titletoc,toc,title]{appendix} %Inkludiere Anhang

%Titel
\title{Einflussfaktoren und Frühwarnsystem im Projektmanagement der Bühler AG}
%Autor
\author{Michèle Schoch}
\date{21. August 2016}

%%Beginn des Dockuments
\begin{document}\selectlanguage{german}
	
%%Titelseite
\pagenumbering{roman}%römische Nummerierung für Seiten vor Einleitung
\begin{titlepage}
\maketitle
\end{titlepage}


%%Inhaltsverzeichnis
\setlength{\parindent}{0pt}
\tableofcontents\newpage

%Abkürzungsverzeichnis
\printglossary[style = super, title=Abkürzungsverzeichnis, toctitle = Abkürzungsverzeichnis]
\newpage

%Abbildungsverzeichnis
\listoffigures\newpage
%Tabellenverzeichnis
\listoftables\newpage
%Einleitung
\pagenumbering{arabic}
% !TEX root = MA.tex
\chapter{Einleitung}
Erfolg nimmt in der gegenwärtigen Gesellschaft eine zentrale Rolle ein, zum Beispiel im Privatleben bei der Partnersuche, in der Arbeitswelt zur Unternehmensführung oder während der Ausbildung hinsichtlich bestandener Prüfungen. Folglich kann postuliert werden, dass jedes Individuum während seiner Lebenszeit zwangsläufig mit Erfolg konfrontiert. Die Definition des Duden für Erfolg lautet "ein positives Ergebnis einer Bemühung; Eintreten einer beabsichtigten, erstrebten Wirkung". Die Auslegung des Begriffs umfasst somit ein positive Ergebnis einer Anstrengung zur Erreichung eines vorgängig definierten Ziel. Diese breite Definition kann mittels unterschiedlicher Attributen individualisiert werden, so dass sie auf unzählige Situationen anwendbar ist.
\newline\newline
 Die vorliegende Arbeit befasst sich mit dem Erfolg von Projekten und somit mit den Einflussfaktoren im Projektmanagement. Die traditionelle Projekterfolgsdefinition unabhängig der Projektart orientiert sich am magischen Dreieck Zeit, Kosten und Qualität (Atkinson, 1999, S. 337 \& Kerzner, 2014, S. 40), wobei eine minimale Abweichung hinsichtlich der Zielvorgaben dieser drei Grössen angestrebt wird. Zieldivergenzen und die Ungewissheit der Ursachen des Projektscheiterns gaben in Vergangenheit Anlass zur Erforschung der Faktoren, die den Projekterfolg begünstigen. Dabei wurden unterschiedliche Erfolgsdefinitionen, Projektmanagementansätze, Projektarten und Industrien berücksichtigt. Gemäss dem Projektmagazin (2014) kann der herkömmliche Projektmanagementansatz durch die Steuerung des Projektablaufs im magischen Dreieck und die Trennung der Projektphasen mittels Meilensteinen charakterisiert werden. Davon sind die agilen Techniken zu unterscheiden, welche einerseits die strikte Einhaltung von Budget, Zeit und Qualität weniger fokussieren und anderseits Veränderungen des Projektumfeldes oder Leistungsumfangs durch flexiblere Projektdurchführung berücksichtigen (Projektmagazin, 2014). Aufgrund der unterschiedlichen Managementmethoden, Projektarten und Industrien existiert eine beliebige Anzahl von mögliche Erfolgsfaktoren. Projektmanagement kann angesichts seiner Interdisziplinarität mit dem Management einer kleineren Organisation verglichen werden, welche die unbegrenzte Anzahl möglicher Einflussdeterminanten verdeutlicht. Die Herkulesaufgabe ist folglich die Identifizierung der wenigen, relevanten, den sogenannten kritischen Erfolgsfaktoren (Quelle finden). \cite{BeDeNov2015} weisen darauf hin, dass die Erfolgskriterien von den Erfolgsfaktoren abzugrenzen sind. Denn erstere beurteilen, ob das Projekt erfolgreich war, wohingegen letztere mit unabhängige Grössen die zur Erfolgswahrscheinlichkeit eines Projekts beitragen, assoziiert werden (Besteiero, de Souza Pinto \& Novaski, 2015). Demzufolge ermöglicht die Kenntnisse der kritischen Erfolgsdeterminanten den Projektmanagern gezielte Einflussnahme auf die Erfolgswahrscheinlichkeit. Somit kann postuliert werden, dass das übergeordnete Ziel im Projektmanagement letztendlich die Erhöhung der Anzahl erfolgreich abgeschlossener Projekte ist. Variablen des Projekterfolgs beeinflussen den Projektmanagementprozess und können somit auch als Einflussfaktoren im Projektmanagement benannt werden. Deshalb wird nachfolgend Einfluss- und Erfolgsfaktor als Synonyme verwendet. Im engen Zusammenhang mit der Theorie der Erfolgsfaktoren und dem Projekterfolg steht die Früherkennung respektive die Anwendung eines Frühwarnsystems. Wie der Terminus bereits impliziert hat sie zum Ziel, möglichst früh Gefahren und Chancen zu erkennen. Im Projektmanagement bedeutet dies, frühzeitig zu wissen, wann gewisse Projekte hinsichtlich des Erfolgs gefährdet sind, so dass schnellst möglich entsprechenden Gegenmassnahmen ergriffen werden können (Haji-Kazemi, Andersen \& Krane, 2013). Die Fokussierung dieses Konzepts im Projektmanagement wird damit begründet, einerseits trotz stetiger Verbesserung der Projektmanagement-Tools noch sehr viel Projekte misslingen und anders das Projektscheitern oftmals überraschend eintrat (Williams, et. al, 2012 und Haji-Kezmi \& Andersen, 2013). 
\newline\newline
In dieser Arbeit werden die Einflussfaktoren und Ansätze für Frühwarnsysteme im Projektmanagement der Bühler AG untersucht. Die Bühler AG ist ein Maschinentechnologiekonzern im Familienbesitz mit Hauptsitz in Uzwil. Sie hat eine führende Marktposition in der Herstellung von Maschinen für die Getreideverarbeitung für Mehl, Pasta, Schokolade, Reis, und auch für die Oberflächenbeschichtung. Die Herstellung von einzelnen Maschinen oder eines ganzen Maschinenparks wird mittels Projekten unter der Anwendung des vorhin beschriebenen traditionellen Projektmanagementansatzes abgewickelt. Infolge der unzureichenden Performance einiger Projekte und der Erkenntnis, dass eine drohende Verschlechterung der Projektleistung relativ spät im Bühler-Projektmanagement-Cockpit (BPM-Cockpit) ersichtlich war, entstand die Ambition diejenigen Variablen, welche den Projekterfolg beeinflussen können zu erfassen und untersuchen. Aus diesen Gründen erarbeitete 2016 das Bühler Projektmanagement in Zusammenarbeit mit dem Controlling eine Liste von Faktoren, die gemäss ihrer subjektiven Einschätzung und Erfahrung die Unterscheidung zwischen nicht-erfolgreichen und erfolgreichen Projekten ermöglichen. Die Erhebung erfolgte auf der Basis von den verfügbaren Daten des BPM-Cockpits, so dass auch die Berechnung zusätzlicher Indikatoren möglich gewesen ist. Ausserdem wurde der Projektmanagementprozess vom Verkauf bis zum Projektabschluss gesamtheitlich berücksichtigt. Es wurden sowohl finanzielle Faktoren zur Kosten- und Zeitperformance als auch Indikatoren zum Forecast (FC) Management und personelle Determinanten zum Projekt- und Verkaufsmanager ermittelt. Der Erfolg der Bühler-Projekte wird anhand der Zielgrössen Kosten, Zeit und Qualität beurteilt. Im BPM-Cockpit wird während der gesamten Projektlaufzeit der Status bezüglich der Zielerreichung mittels dem dreifarbigen Ampelsystem reflektiert. Dabei ändert die Ampelfarbe der Kosten- und Zeitampel gemäss hinterlegtem Schlüssel automatisch, wohingegen die Qualitätsampel erst mit der Eingabe der subjektiven Einschätzung des Projektmanagers die Farbe verändert. Obwohl alle drei Determinanten evaluiert werden, hat aus finanzieller Perspektive  die monetäre Zielerreichung ein wesentlicher Stellenwert. Der Residualwert aus Umsatz minus Kosten sprich die Projektmarge \gls{abk:db1} hat direkten Einfluss auf das Ergebnis des Anlagengeschäfts der Bühler AG. Ausserdem wird der DB1 in Prozent zur Incentivierung von Geschäftsbereichsleitern und Projektmanagern angewendet. Die Abweichung der prozentualen von der realisierten Projektmarge entscheidet demzufolge über den Erfolg eines Projekts und definiert somit das Erfolgskriterium der nachfolgenden Analyse der Bühler-Projektdaten.  
%%
%%
%%
\section{Ziele der Arbeit}\label{sec:zda}
Ziel dieser Arbeit ist es, die Einflussfaktoren des Projektmanagements der Bühler AG zu analysieren. Unter der Anwendung deskriptiver statistischer Methoden sollen auf Basis der Bühler-Projektdaten retrospektiv charakteristische Unterschiede zwischen erfolgreichen und nicht-erfolgreichen Anlageprojekte herausgearbeitet werden. Dazu sollen die Daten anhand des Erfolgskriterium in zwei Gruppen unterteilt werden und anschliessend auf der Basis von Häufigkeitsverteilungen und Mittelwertauswertungen die Charakteristiken nicht-erfolgreicher Projekte ergründet werden. Durch den Vergleich mit den erfolgreichen Projekten könnten allfällige Differenzen festgestellt werden. Zudem sollen Auswertungen der Finanzdaten zu Kostenabweichungen und Margeneinbussen Attribute von gescheiterten Projekten aufzeigen. Dieses Vorgehen dient vor allem dazu, die Verluste zu \glqq lokalisieren\grqq{} und der Finanzperspektive aufgrund des monetären Erfolgskriterium Rechnung tragen. Basierend auf den Ergebnissen sollen mögliche Erfolgsfaktoren im Bühler Projektmanagement zu erörtern. Darauf aufbauend soll auch evaluiert werden, ob sich gewisse Variablen zur Früherkennung von "bedrohten" Projekten eigen. Das Untersuchungsobjekt der Analyse bilde alle im Zeitraum zwischen 2013 und 2015 abgeschlossenen Projekte. Zudem werden ausschliesslich die von der Bühler AG zur Verfügung gestellten Daten untersucht, was die Erhebung zusätzlicher Daten ausschliesst. Aus den obigen Ausführungen leiten sich für diese Arbeit folgende zentrale Fragestellungen ab, die es zu prüfen gilt:
\newline\newline
Welche Eigenschaften unterscheiden vergangene nicht-erfolgreiche von erfolgreichen Projekten? Können auf Basis der Ergebnisse Erfolgsfaktoren und Frühwarnindikatoren der Bühler AG begründet werden? Wie hoch wären retrospektiv die finanziellen Ersparnisse unter Anwendung der Früherkennung gewesen?
\newline\newline
Der theoretische Rahmen der Arbeit bilden das Projektmanagement und deren Erfolgsfaktoren. Deshalb wird vor 
der quantitativen Analyse der Projektbegriff und das Projektmanagement eingehender erläutert und in Relation zum Unternehmensbeispiel gesetzt. Zudem sollen bisherige Erkenntnisse aus der Erfolgsfaktorenforschung aufgezeigt sowie die vorherrschende Erfolgsdefinition von Projekten und deren Wandlungstendenz erläutert werden. Im Anschluss soll der Bühler Projektmanagementprozess und die Einflussfaktoren aufgezeigt und erklärt werden, da er die Grundlage der Datenerfassung und der darauffolgenden Untersuchung bildet. 
\newline\newline
Nach der Erklärung der analytischen Vorgehensweise und Ergebnispräsentation soll das Thema der Früherkennung und Frühwarnsystem aus wissenschaftlicher Sicht eingeleitet werden. Basierend auf den theoretischen Ausführungen und unter der Berücksichtigung der Ergebnisse sollen konzeptionellen Ansätze zur Früherkennung im Projektmanagement der Bühler AG entwickelt werden. Danach soll abschliessend eine kritische Diskussion der gewonnenen Resultate in Verbindung mit den Aspekten der Forschung und der Zielsetzung dieser Arbeit stattfinden.
\newline\newline 
Der Umfang der Untersuchung wurde bereits vorgängig durch die Datenverfügbarkeit des internen Projektmanagementtool eingegrenzt, was zur Folge hat, dass Faktoren welche das Projektmanagement der Bühler AG auch beeinflussen können, im Rahmen dieser Arbeit nicht untersucht werden. Die Aussagekraft der Ergebnisse wird durch die unternehmensspezifische Daten begrenzt, so dass lediglich Rückschlüsse auf die Projekte und den Projektmanagementprozess der Bühler AG gemacht werden können.
\section{Methodik und Struktur der Arbeit}
Die Gliederung der Arbeit unterscheidet drei Abschnitte: der theoretische und unternehmensspezifische Rahmen, die Methodik und Ergebnisse sowie Anwendung und Diskussion der Erkenntnisse. Die Erarbeitung erfolgt dabei auf der Basis einer Kombination von Literaturrecherche, unternehmensspezifischem Wissen sowie statistischer Vorgehensweise.
\newline\newline
Im ersten Abschnitt wird eine theoretische Abhandlung zu Projekten, Projektmanagement und bisher erforschten Erfolgsfaktoren dargelegt sowie der Bühler Projektmanagementprozesse erläutert. Die definitorische Abgrenzung dient dazu den Rahmen der Begrifflichkeiten festzulegen und in Verbindung zu den internen Bestimmungen zu setzen. Auf die Unterschiede der Projektarten, verschiedenen Projektmanagementansätze und deren Kategorisierung wird in dieser Arbeit nicht näher eingegangen, da für das Verständnis der Einflussfaktoren vor allem der Bühler Projektmanagementprozess von Bedeutung ist. Die Ergründung der bisher identifizierten Erfolgsfaktoren dient dazu die dominierende Ansicht der Wissenschaft aufzuzeigen, wobei der Fokus auf den Konstruktionsprojekten liegt. Es wurden allerdings Erkenntnisse anderer Projektarten mitberücksichtigt, mit dem Ziel ein breites Spektrum an möglichen Erfolgsfaktoren zu erhalten. Diese Informationsbasis dient als Referenzpunkt für die Erkenntnisse aus den qualitativen Auswertungen und kann ergänzenden Variablen des Projekterfolges im Bühler Projektmanagement enthalten. Da sich beispielsweise die Faktoren in Abhängigkeit des Erfolgskriteriums unterscheiden können, ist der gewählten Projekterfolgsdefinition im Forschungsdesign eine entscheiden Rolle beizumessen. Deshalb werden unterschiedliche Betrachtungsweisen des Erfolgskriteriums und die mögliche Abwendung vom traditionellen Erfolgskalkül hervorgehoben. Die Prozessbeschreibung des Bühler Projektmanagement orientiert sich an den internen Dokumentationen und Darstellungen, wobei diejenigen Bestandteile eingehender erläutert werden, die dem Verständnis der Einflussfaktoren dienen.
\newline\newline
Der nächste Abschnitt umfasst die Beschreibung des analytischen Vorgehen, der gewählten Methoden und die Präsentation sowie die kritische Beurteilung der Ergebnisse. Die Analyse stützt sich dabei auf Ansätze der deskriptiven Statistik und der Exploration mittels des finanziellen Verlust bezüglich der Projektmarge, weshalb  die Auswertungen separat dargestellt werden. Die anschliessende Interpreation und Würdigung der Ergebnisse beabsichtigt einerseits den Bezug zum Unternehmen und den Prozessen herzustellen und anderseits Erfolgsfaktoren zu identifizieren. 
%Wahl der Methodik???? verzicht auf Befragungen der Bühler AG...externe Perspektive
\newline\newline
Der letzte Abschnitt konzentriert sich auf die Früherkennung und Frühwarnsysteme im Projektmanagement allgemein und der Bühler AG. Dazu wird das theoretische Wissen auf der Basis von Literatur erarbeitet, wobei die zentralen Aspekte zu den Begrifflichkeiten, Anforderungen und Methoden herausgearbeitet werden sollen. Die ausführliche Erläuterung der methodischen Ansätze der Früherkennung und des Implementierungsprozess sind nicht Teil dieser Arbeit, da der Schwerpunkt auf der konzeptionellen Ausarbeitung von Frühwarnindikatoren für das Bühler Projektmanagement unter der Berücksichtigung des Forschungserkenntnisse liegt. Die Ergebnisse der Analyse sowie die Prozessbeschreibungen aus dem ersten Abschnitt werden hierfür ebenso berücksichtigt, da erstere auf mögliche Frühwarnindikatoren hinweisen kann und letztere den Rahmen der Früherkennung festlegen. Danach folgt eine Diskussion und kritische Betrachtung des Konzepts und der Ergebnisse aus der quantitativen Analyse im Zusammenhang mit wissenschaftlichen Erläuterungen und den Zielen dieser Arbeit.
\newline\newline
Abschliesend werden im Fazit die wichtigsten Erkenntnisse der Arbeite zusammenfassend festgehalten und die eingangs erwähnten Forschungsfragen beantwortet. Zudem sollen Empfehlungen für weitere Forschungsthemen und Optimierungspotenziale festgehalten werden.

	




\newpage
%Theorieteil
% !TEX root = MA.tex
\chapter{Theoretischer Rahmen}\label{sec:theor}
Nachfolgend werden das Projektmanagement und die Tendenzen zu neuen Projekterfolgsdefinition erläutert. Im Anschluss werden bisher erforschte Erfolgsfaktoren unter der Berücksichtigung unterschiedlicher Projektarten und Brachen aufgezeigt. Im nächsten Unterkapitel wird der Bühler Projektmanagementprozess beschrieben. Zuletzt erfolgt eine Einführung in das Thema der Früherkennung die den theoretischen Rahmen für die Diskussion des \ref{sec:diskfru} bildet.
%%
%%part Erfolgsfaktoren und Projektmanagement
% Erläuterung Projekt: UT Software/Konstruktion, Fokus der Arbeit (FdA): Maschinen/Anlageproj der Bühler AG
% Projektmgmt: Definition, Komplexität, Interdisziplinarität, Grundprozess nach Din
% Projektmgmt: Methoden: Agile und Traditionelle Methode, FdA: traditionller Ansatz der Bühler AG
% Projektmgmt: Einflussfaktoren im PM, die Erfolg begünstigen können in Abh. Projektarten, Projektmgmtmethoden
% Erfolg: Definition, Unterscheidung Erfolg und Kriterien, Ansätze: traditionelle vs. andere: FdA Bühler AG
% Erfolgsfaktoren: Forschungsstand: Relevante Faktoren und weitere Einflussfaktoren, Überleitung zu Bühler Prozes, Bühler Einflussfaktoren unter der Berücksichtigung der Bühler Erfolgsdefinition
\section{Erfolgsfaktoren im Projektmanagement} \label{sec:erfprj}	
Gemäss dem Deutschen Institut für Normung (DIN)  ist ein Projekt: \glqq ein Vorhaben, das im Wesentlichen durch Einmaligkeit der Bedingungen in ihrer Gesamtheit gekennzeichnet ist, z.B. Zielvorgabe, zeitliche, finanzielle, personelle und andere Begrenzungen, Abgrenzung gegenüber anderen Vorhaben, projektspezifische Organisation \grqq{ } (69901-5, zit. in \citealp{alamg16})). Daraus folgt, dass Projekte an verschieden Vorgaben gebunden sind, die sich im Einzelnen voneinander unterscheiden können. Erst die Gesamtheit dieser Vorgaben begründet gemäss der obigen Definition die Einmaligkeitein Projekt. Infolge der internationalen Tätigkeit und des breiten Angebot an Maschinen zur Herstellung unterschiedliche Nahrungsmittel sind lediglich zwei Aspekte, welche die Einmaligkeit der Bühler-Projekte ausmachen können. Das Projektmanagement Handbuch \citeyear{pmhod} fügt als weitere Abgrenzungskriterien von Projekten die Ressourcenknappheit sowie die Notwendigkeit zur Teamarbeit an. Die Aufgabenstellung einiger Projekts kann sich zudem während der Laufzeit konkretisieren, das neue Informationen verarbeitet und die Vorstellung des Endprodukts klarer wird \citep[S.~1]{meyreh16}. Obwohl die Anlage und deren Spezifikation bei der Bühler AG im Vorfeld in Abstimmung mit dem Kunden festgelegt werden, können sich  sich Änderungen auf Wunsch des Kunden im Verlauf des Projekts ergeben. Basierend auf diesen Definition wird deutlich, dass die Abwicklung von Projekte eine geeignete Methode erfordert, sodass die zahlreichen Bedingungen eingehalten werden können. Das Projektmanagement ist ein \glqq generischer Managementprozess \grqq{ } der auf unterschiedliche Projekte angewendet werden kann (DIN 69 904 zit. in \citealp{pmhod}. Er bezeichnet die \glqq Gesamtheit von Führungsaufgaben, -organisation, -techniken und -mitteln für die Initiierung, Definition, Planung, Steuerung und den Abschluss von Projekten\grqq{ } (DIN 69901-5, 2009 zit. in \citealp[S.~3]{meyreh16}; \citealp{pmhod}). Das Management von Projekten ist eine interdisziplinäre Aufgabe, die nach \citealp[S.~2]{alamg16} auch "die Koordination von Menschen und der optimale Einsatz von Ressourcen zum Erreichen der Projektziele \grqq { }. Diese Definitionen verdeutlichen die umfassende Aufgaben des Projektmanagements und implizieren die Existenz unterschiedlicher Ansätze. An dieser Stelle wird nicht weiter auf die Methodiken eingegangen, da im Rahmen dieser Arbeit der Bühler Projektmanagementprozess des Kapitels \ref{sec:pmbueh} von zentraler Bedeutung ist, der sich grob an der Definition von DIN orientiert.
%%
%%Erfol und Erfolgsdef.
\newline\newline
Einleitend wurde das vorherrschende Paradigma zur Beurteilung des Projekterfolgs, das eiserne Dreieck Zeit, Kosten und Qualität erläutert. Aus einer finanziellen Perspektive liegt der Fokus auf dem Kostenaspekt, wobei dies die Zielerreichung der anderen primären Ziele beeinträchtigen kann. \citealp[S.~40]{kerz14} weist darauf hin, dass die Projekte selten innerhalb der Zielvorgaben abgeschlossen wurden, weshalb die historischen Definition um den Aspekt der Kundenakzeptanz erweitert wurde. Dadurch wurde die strikte Einhaltung der Dreierbedingung gelockert, da der Kunde trotz Mehrkosten oder Zeitverzug das Projekt akzeptieren konnte. Die Weiterentwicklung der Projektmanagementtechniken führte zur Erkenntnis, dass die Projekte an mehr als drei Bedingungen geknüpft sind,, die zusätzlich bei der Ermittlung des Erfolgs mittels Nebenzielen berücksichtigt werden sollen \citep[S.~41]{kerz14}. \citealp{Atk1999} bemängelte zudem, dass die traditionelle Definition langfristige Nutzen nicht berücksichtigt. Zudem argumentierte er, dass die vorherrschende Erfolgsdefinition vermutungsweise der Anwendung neuer Instrumente, Fähigkeiten oder Projektmanagementansätze nicht Rechnung trage \citep{Atk1999}. Aus diesen Gründen und der bedingten Vergleichbarkeit von Projekten aufgrund ihrer Einmaligkeit sind neue Ansätze zur Bestimmungen des Projekterfolgs gefragt. \citep{lchch08} schlagen einen gewichteten Erfolgsindex auf der Basis der Kosten-Zeit-Qualität-Bedingung zur Beurteilung des Erfolgs vor, um die Projekte trotz ihrer charakteristischen Eigenschaften vergleichen zu können. \citep{kerz14} entwickelt gänzlich einen neue Projekterfolgsdefinition, die sich auf die Erreichung des gewünschten Geschäftswertes innerhalb der sich konfligierenden Zielvorgaben konzentriert. Diese Geschäftswerten können unternehmensbezogen ausgearbeitet werden, wobei angesichts der Evaluierung von mehreren Bedingungen eine vierteilige Erfolgskategoriesierung wie sie \citealp[S.~48]{kerz14} vorschlägt, sinnvoll erscheint. Diese Ausführungen zeigen, dass letztendliche die Erfolgsdefinition vom Unternehmen abhängig ist und keinen Einschränkungen unterliegt. Der Analyse dieser Arbeit liegt einen finanzielle Erfolgsbeteiligung auf der Basis des Kostenaspekts der traditionellen Erfolgsdefinition zugrunde.
%%
%%Erfolgsfaktoren
\newline\newline
Nachfolgend werden bisher erforschte Erfolgsfaktoren dargelegt. Dabei kann sich die Beurteilung des Projekterfolges, der an mehrere Kriterien gebunden sein kann, zwischen den in den Studien untersuchten Unternehmen unterscheiden. Gemäss \citealp{iyerjha06} hat das Engagement der Projektmitarbeiter und die Fähigkeiten des Projekteigners einen positiven Zusammenhang mit der Zeitperformance, wohingegen sich Konflikte zwischen dem involvierten Projektteilnehmer (bspw. Projektmanager, Top-Management, Projekteigner, externe Parteien) die Einhaltung der Zeitvorgabe beeinträchtigen können. Die Projektperformance in Bezug auf Zeit und Kosten hat mit den Einsatz des Projektteams, der unter anderem mit Vertrauenskultur, Konfliktlösung und Verständnis der Projektziele assoziiert wird, und den Kompetenzen des Kunden eine positive Korrelation\citep{chahota01}. Diese Erfolgsfaktoren können zusammenfassend in Fähigkeiten der involvierten Projektteilnehmer und Aspekte der Projektkultur eingeordnet werden. Weitere Erfolgsfaktoren von Konstruktionsprojekten sind zudem die Kompetenzen des Auftragsnehmer und die Anwendung innovativer Technologien \citep{chahota01}. Der Erfolg von Projekten, der anhand eines gewichteten Erfolgskriterium der Kosten, Zeit, Qualität und Funktionalität, kann zudem von der Effizienz des Projektmanagements und der Projektnatur abhängen, wie \citealp{lchch08} bewiesen. Letztere gründet auf der Annahme, dass attraktive und komplexe Projekte (sprich die Projektnatur) mehr Aufmerksamkeit von den Projektmanagern erhalten, weil sie mit Prestige und Selbstverwirklichung verknüpft sind \citep{lchch08}. Die Erfolgsfaktoren, die positiv mit der Kostenperformance von Industrieprojekten korrelieren, sind gemäss \citealp{luhuazha17} die Fähigkeiten des Auftragsnehmers. Denn sein Aufgabenbereich umfasst, das Design, die Beschaffung und die Konstruktion, welche für die Fertigstellung des Projekts \citep{luhuazha17}. Ling et al. (2004 in \citealp{luhuazha17}) bemerkte zudem, dass fehlende finanzielle Kompetenzen des Auftragsnehmers die Kostenkontrolle erschweren.
\newline \citealp{BeDeNov2015} haben vier Kategorien von Erfolgstreiber gebildet, Managementfähigkeiten, kritische Erfolgsfaktoren, Projektcontrolling und \glqq Lessons Learned \grqq { } und festgestellt, dass Kommunikation in allen Bereichen des Projektmanagements von zentraler Bedeutung war. Der Erfolgsfaktor Kommunikation wurde bereits in früheren Studien ermittelt (Hyvräri, 2006 in \citealp{BeDeNov2015}). Die Analyse der Unterschiede zwischen Software- und Konstruktionsprojekten führte zur Schlussfolgerung, dass das Verständnis der Projektziele und die Projektplanung von beiden Projektarten kritische Erfolgsfaktoren sind \citep{VarDom14}. Demgegenüber war der Zusammenhang des Faktors \glqq Miteinbezug aller Projektteilnehmer \grqq { } bei Konstruktionsprojekten mit dem Projekterfolg höher als bei Software-Projekten. Folglich kann postuliert werden, trotz unterschiedlicher Projekttypen ähnliche Faktoren positiv mit dem Projekterfolg korrelieren.
\newline\newline Aus den vorangehenden Ausführungen geht hervor, dass trotz unterschiedlicher Beurteilung des Projekterfolgs, ähnliche Erfolgsfaktoren identifiziert wurden. Zusammenfassend kann deshalb postuliert werden, dass die im Projekt involvierten Personen und die Projektkultur zentrale Erfolgsfaktoren sind. Dabei bilden nachfolgende Attribute wie, Fehlerkultur, Teamfähigkeit, Konfliktfähigkeit, Vertrauen, gemeinsame Mission nur einen Teil derer Gesamtmenge ab, um die Art der Projektteilnehmer und das Arbeitsklima zu erfassen. Gemäss \citealp{alamg16} sind dies jene Anforderungen, die während jeder Projektphase gegeben sein müssen, damit Projekte erfolgreich bearbeitet werden können. Die bedingte Vergleichbarkeit von Projekten aufgrund ihrer Natur scheint demzufolge hinsichtlich der Erfolgsfaktoren von zweitrangiger Bedeutung zu sein. Der Erfolg von Projekten kann durch unterschiedliche Kriterien gemessen respektive beurteilt werden. Angesicht der Einzigartigkeit und Lerneffekten von Projekten, erscheint die Anwendung von mehreren Erfolgskategorien sinnvoll. 
%%
%%part PM der Bühler AG 
%%subpart Prozess: Customer Project Prozess
%%subpart Einflussfaktoren: Summarische Erläuterung, Kategorien & Begründung, Erfolgskriterium nochmals erwähnen?, Verweis auf Kapitel 3, Hypothese: Dass Variablen Attribute von erfolgreichen respektive nicht erfolgreichen sein können.
\section{Projektmanagementprozess der Bühler AG}\label{sec:pmbueh}
In der Folge wird der Projektmanagementprozess der Bühler AG erläutert, welcher die Grundlage der Datenerhebung bildete. Der Fokus liegt dabei auf denjenigen Bestandteilen, die dem Verständnis der Daten dienen. Der Kundenprojektprozess der Bühler AG gliedert sich in zwei Subprozesse, wie die Abbildung \ref{fig:processcp} illustriert: den \gls{abk:sq} und den \gls{abk:ff}, die nachfolgend auch als Projektverkaufsprozess und -abwicklungsprozess bezeichnet werden. Das Bindeglied bildet das \gls{abk:HOM}, bei welchem der Verkaufsmanager auch Area Manager genannt das Projekt an den Projektmanager übergibt. Die einzelnen Prozessphasen sind durch \gls{abk:mst}, bei denen gewisse Anforderungen erfüllt werden müssen, beispielsweise die Genehmigung des Projektzeitplans oder gewisse Risikochecks, getrennt.
\begin{figure}[H]
	\centering
	\includegraphics[width=8cm]{processcp.png}
	\caption{Kundenprojektprozess der Bühler AG}
	\label{fig:processcp}
\end{figure}
Der als Kreis dargestellte \gls{abk:sq}-Prozess (s. Abbildung \ref{fig:processcp}) umfasst vier Phasen: Potenzial identifizieren (I), Prioritäten setzen (II), Anbieten und Risiko (III) evaluieren und Auftrag abschliessen (IV). Die involvierten Parteien im Prozess sind der Area Manager und das Backoffice, wobei erstere in direktem Kundenkontakt steht und letztere für die Angebotsstellung verantwortlich ist. Die Phase I und II konzentrieren sich darauf, Geschäftspotenziale und Kundenbedürfnisse zu identifizieren, Kontakte mit den Kunden aufzunehmen und letztendlich auf Basis von diversen Checks zu entscheiden, welche Projekte fokussiert. In der Phase III werden die Projektmöglichkeiten detailliert in technischer, kommerzieller und finanzieller Hinsicht geprüft, so dass nach der Ausarbeitung des Basiskonzepts dem Kunden ein Angebot unterbreitet werden kann. In der vierten Phase erfolgt nach der Risikoprüfung und allfälligen Verhandlungen der vertragliche Abschluss des Auftrages mit dem Kunden. Der Output dieses Prozess wird als \gls{abk:OR} bezeichnet, was die Freigabe es Auftrages bedeutet. Dieser Auftrag wird beim \gls{abk:HOM} an den Projektmanager übergeben, der die Verantwortung für den Projektabwicklungsprozess hat. Der Fulfillment-Prozess der Abbildung \ref{fig: processff} ist in fünf Phasen unterteilt. 
\begin{figure}[H]
	\centering
	\includegraphics[width=8cm]{processff.png}
	\caption{Projektabwicklungsprozess der Bühler AG}
	\label{fig: processff}
\end{figure}
\textbf{Phase I: Planning and Basic Engineering}
\newline
Bei der Übergabe des Projekts vom Verkauf an die Abwicklung sind typischerweise der Kunde, der Verkaufs- und Projektmanager sowie der Teamleiter involviert. Diese wichtige Schnittstelle dient dazu alle relevanten Informationen zu übergeben und offene Punkte zu klären. In dieser Phase erfolgt Projektanalyse durch den Projektmanager, die Ausarbeitung respektive Überarbeitung des Konzepts, die Projektplanung und das Kick-off-Meeting. Das Ziel der Projektanalyse ist die Realisierbarkeit mittels der Identifizierung von technischen als auch kommerziellen Risiken und Chancen sowie entsprechenden Massnahmen zu prüfen. Die anschliessende Konzeptphase beinhaltet die Ausarbeitung oder Nachbearbeitung des Maschinen- oder Anlagekonzept, das die interne und  externe Genehmigung voraussetzt. In der Projektplanung werden überwiegend organisatorische und administrative Aufgaben wie zum Beispiel die Planung der Liefertermine pro Meilenstein oder die Definition von Arbeitspaketen. Der letzte Schritt dieser Phase bildet das Kick-off-Meeting, welches der Schaffung eines gemeinsamen und einheitlichen Verständnis unter sämtlichen Teilprozessverantwortlichen dient. Unter der Berücksichtigung der vertraglichen Bedingungen werden verbindliche Vereinbarungen bezüglich der Termine, Kosten, Qualität und Zuständigkeiten getroffen. Zeitlich findet dieses Treffen in der Regel vor Freigabe durch die kaufmännischen Berater statt und bietet einen Diskussionsraum für ungeklärte Aspekte.
\newline\newline
\textbf{Phase II: Engineering and Specifications}
\newline
Die Ausarbeitung verbindlicher Pläne zur Anlagen- oder Maschinendisposition, Optimierungen am Maschinen- respektive Anlagenkonzept bilden die zentralen Aufgaben dieser Phase. Während der Design Meetings wird mit dem Kunden das Einverständnis hinsichtlich der Spezifikationen und Pläne schriftlich protokolliert, so dass Änderungen dem Risiko von Mehrkosten und Zeitverzögerung ausgesetzt sind. Der \gls{abk:mst}5 \glqqPoint of now return\grqq{ } ist ein interner Meilenstein, bei dem die Liefertermine gegenüber dem Kunden verbindlich werden und eine Finanzierungslösung durch die Bühler AG sichergestellt sein muss.
\newline\newline
\textbf{Phase III: Manufacturing, Procurement \& Shipping}
\newline
Diese Prozessphase beginnt mit der Fabrikation und endet mit der Lieferung der Maschine an den vereinbarten Ort einschliesslich der der Beschaffung und der Dokumentation. Die Einhaltung des Liefertermins sowie die vertragskonforme Übergabe der Anlage ist hierbei von besonderer Wichtigkeit. Die Dokumentation wird bei der Installation der Maschine oder des Maschinenparkes benötigt und hat zudem die Gewährleistung der Nachvollziehbarkeit der Änderungen zum Zweck. Die Übergabe dieser Dokumentation an den Kunden und den Monteur begründet das Ende dieser Phase bei \gls{abk:mst}8 \glqq Documented \grqq.
\newline\newline
\textbf{Phase IV: Installation and Start up}
\newline
Die Installation, Inbetriebsetzung und Übergabe sind die elementaren Prozesschritte dieser Phase. Der Zusammenbau einer Anlage und die Inbetriebsetzung einer Maschine erfordert eine Instruktion des Montageteams, die zugleich eine unabdingbare Voraussetzung für ein gewisses Qualitätsniveau gewährt. Die Montageverantwortlichen werden durch die Projekt- und Verkaufsleiter laufend unterstützt, was die gleichzeitig die Überwachung des Installationsprozesses ermöglicht. Am Ende bei \gls{abk:mst}10 \glqq Take-over fulfilled\grqq{ } dieser Phase folgt nach abgeschlossener Inbetriebsetzung die Übergabe der Anlage an den Kunden. Dabei ist darauf zu achten, dass möglichst alle vertraglich vereinbarten Anforderungen, wie zum Beispiel Tests, Umfang und Darstellung der Übergabedokumente erfüllt werden, denn die letzten Kundenzahlungen sind oftmals an die Leistungserfüllung gekoppelt sind. 
\newline
\textbf{Phase V: Evaluation and Transfer}
\newline
Zuletzt findet das Debriefing statt, bei welchem Rückmeldungen zur Optimierung der Projektabwicklung für künftige Projekt festgehalten, so dass die gleichen Fehler nicht wiederholt werden. Das Projekt wird danach beim \gls{abk:mst} intern abgeschlossen, wobei gleichzeitig die Projektabschlussfest zweijährige Garantieperiode beginnt.
%%
%%part Frühwarnsystem im PM: theoretischer Rahmen für Diskussion
%Frühwarnsystem: Definition, Anwendung hauptsälich, Grund: Anwendung im PM, Anpassung Begrifflichkeiten
%Früherkennung: Definition (Watch Redundanzen), Methoden: Generationen, Fokus der Forschung 3. Generation
%Früherkennung: Einführung und Verwendung Begriff Frühwarnsignale, Methodische Ansätze summarisch,
%Früherkennung: Getestet und Bewährte Methoden
%
\section{Frühwarnsystem im Projektmanagement}
Die Notwendigkeit von Früherkennung im Projektmanagement lässt mit dem Auftauchen neuer Herausforderungen während der Projektlaufzeit sowie der sich kontinuierlich verändernden Projektumwelt begründen. Die Anforderungen an die Flexibilität im Projektmanagement und die Fähigkeit zukünftige Ereignisse "vorauszusehen" sind gestiegen. Früherkennung hat zum Ziel aufkommende Gefahren und Chancen frühzeitig zu identifizieren, so dass rechtzeitig entsprechende Massnahmen eingeleitet werden können. In gegenwärtigen Managementsystem der Bühler AG wird ein sich verschlechternder Projektstatus erst bei fortgeschrittener Projektlaufzeit ersichtlich, weshalb beabsichtigt wird mittels der Implementierung eines Frühwarnsystems, dieser Situation entgegenzuwirken. Dadurch sollen mehr Projekte erfolgreich abgeschlossen und das die Marge des Anlagegeschäfts verbessert werden. Anschliessend folgt die Erläuterung der drei Generationen von Frühwarnsystem und sowie deren die Anforderungen. Danach werden Ideen und Ansätze für Früherkennung bei der Bühler AG diskutiert.
%\subsection{Frühwarnsysteme und Frühwarnindikatoren}\label{viereins}
Die Voraussetzungen zur Anwendung eines Frühwarnsystems sind gemäss Jacobs, Riegler \& Matter (2012, S. 23), die Möglichkeit ambivalenter Ausgänge, die Gefährdung dominanter Ziele sowie der prozessuale Ablauf einer drohenden "Krise". Obwohl diese Kriterien im Zusammenhang mit Unternehmenskrisen ausgearbeitet wurden, können sie für das Projektmanagement ebenso angewendet werden. Grundsätzlich werden folgende drei Generationen unterschieden.
\begin{description}
	\item[1. Generation:] Die erste Generation orientiert sich an traditionellen Kennzahlen- und Hochrechnungen. Vergleiche der Prognosen mit Sollwerten werden dann als Frühwarnsignale verwendet.
	\item[2. Generation:] Die zweite Generation basiert auf der Anwendung von Indikatoren und Prognosen auf Basis von Faktorenmodelle, die statistisch signifikante Zusammenhänge aufweisen mit dem gewählten Indikator haben.
	\item[3. Generation:] Die dritte Generation stützt sich auf die Theorie der schwachen Signale, die intuitiver und unstrukturierter Natur sind. 
\end{description}
Die erste und zweite Generation haben den Nachteil, dass die Aussagekraft der vergangenheits- und gegenwartsorientierten Datengrundlage relativ beschränkt ist und anderseits durch die Selektion relevante Faktoren unbeachtet bleiben. Aus diesen Gründen sind Diskontinuitäten aus Basis der Hochrechnungen nur schwer erkennbar und an die zugrunde liegenden mathematischen respektive statistischen Modelle gebunden   (Jacobs, Riniger \& Matter, 2012, S. 26- 28). Die dritte Generation versucht die Schwächen der vorangehenden Frühwarnsystem zu kompensieren. Gemäss Ansoff (1967, S.129ff), kündigen sich Diskontinuitäten nicht plötzlich sonder relativ früh durch sogenannte schwache Signale, die als frühe Hinweise zu bevorstehenden einflussreichen Ereignisse zu verstehen sind (Ansoff in Haji-Kazemi \& Anderson, 2013). Diesem Ansatz liegt die Prämisse zugrunde, dass Unternehmen die wahrscheinlichsten Faktoren, welche das Scheitern des Projekts begünstigen sowie die Anzeichen eines bevorstehenden Scheiterns, bereits kennen. Allerdings wird erst in nachgelagerten Projekt Assessments dieses Bewusstsein gefördert. Unter diesem Aspekt erscheint es relativ unverständlich, weshalb diese ignoriert wurden.
\newline
Die Ansätze, wie Projektmanager solche Signale erkennen und zu ihren Gunsten interpretieren können sind vielfältig, wie die Tabelle \ref{tab:Ans} aufzeigt.
%Tabelle mit möglichen Ansätzen
\begin{table}[H]
	\centering
	\caption{Ansätze zur Identifikation von Frühwarnsignalen
		\newline in Anlehnung an Haji-Kazemi, Andersen \& Krane (2013)}\label{tab:Ans}	
	\begin{tabular}{l|l}
		Risikomanagement & Past Project Consultation\\
		Earned Value Management & Cause-Effect-Analyse\\
		Projekt Assessment Ansätze & Gut feelings \\
		Performance Management & Interface Analysis\\
		Stakeholder Analyse & Project Analysis \\
		Maturity Assessment & Project Surrounding Analysis\\
	\end{tabular}		
\end{table}
Die Wahl der Methode ist abhängig von der Projektart und des Unternehmens. Haji-Kazemi, Andersen \& Krane (2013, S. 59) haben mittels mehrere Fallstudien eruiert, dass das Projekt Assessment sowie 'Gut feelings' in der Praxis die bewährt haben, wobei Experten die Überzeugung haben, dass Frühwarnsignale qualitativer Natur sind und eher durch Intuition Erfahrungswissen entdeckt werden. Klakegg, et.al, (2010) haben mittels formalen Assessments die Anzahl fehlender Informationen, fehlende Beurteilungen und Dokumentationen sowie unklare Anforderung der Meilenstein und verspätete Bericht als mögliche Frühwarnsignale ergründet. Missverständnisse bezüglich der Bedürfnisse, sowie mangelnde Offenheit der Unternehmenskultur und Kommunikationsbereitschaft zwischen den Projektteilnehmer, sowie angespannte Projektatomsphäre wurden in Fallstudien mittels der "Gut feelings"-Ansätzen als wichtige Früherkennungshinweise erkannt. Diese Erkenntnisse bestätigen unter anderem die Ansichten der befragten Experten aus anderen Studien. Denn bei der Evaluation von Assessments während der Projektlaufzeit können zwar wichtige Hinweise für nachfolgende Projekte ausgearbeitet, die aber im aktuellen Projekt nicht mehr berücksichtigt werden können. Während diese Indikatoren qualitativen Charakter haben und eher schwierig zu messen sind, konnten Haji-Kazem \& Anderson (2013) im Rahmen des Performance Management die Überwachung der Schnittstellenmassnahmen, die Mitarbeiterzufriedenheit und Risikoüberwachung als effiziente Quellen von Frühwarnsignalen erheben. Ihre Gemeinsamkeiten sind die quantitative und kontinuierliche Messbarkeit sowie die Funktion als sogenannte "leading" Indikatoren, die es ermöglichen in der Ursachen-Wirkungs-kette möglichst früh Hinweise zu möglichen Gefahren zu erhalten. "Lagging" Faktoren liefern dementsprechend eher spät oder zu spät Signale zu möglichen Risiken. Die Herausforderung bei der Identifikation von Frühwarnsignalen mit dem Performancemanagement-Ansatz ist die Selektion der zu überwachenden Faktoren. Ausserdem kann der Einfluss von Drittvariablen unentdeckt respektive unterschätzt werden.
% Faktoren von Klakegg und Kommentieren
% Faktoren von Performance Ansatz, 
% Leading Lagging Faktor,  
% wan ist früh, leading not lagging indicators, IdentifyAc says that performance only measures lagging 38
% possible early warning signs culture, lack of an outsiders perspective on the project, anchoring in the permanent organization, lack of consistency between stakeholders ambition and certain organizations. gut felt signs: detection of unrealism, lack of clarity, misalignment btw qualitative and quantitaive risk analysis 42
%no early signs in later stages, change...not used 43
%problems: difficult to stop projects despite EWS
%need for formalized proces for finding ealry warning signs, outside of the box thinking


\newpage
%Methodneteil
% !TEX root = MA.tex
\section{Analye der Erfolgsfaktoren des Bühler Projektmanagements}\label{drei}
In diesem Kapitel wird zuerst das analytische Vorgehen erläutert und anschliessend die Ergebnisse präsentiert sowie kritisch gewürdigt. In vergangenen Studien wurden die Erfolgsfaktoren von Projekten mittels der statistischen Auswertung von Einschätzungen ausgewählter Attribute zu deren Relevanz für den Projekterfolg erhoben. Die nachfolgende Analyse unterscheidet sich insofern, da für die Ergründung der Charakteristiken nicht-erfolgreicher Projekte unternehmensspezifische Projektinformationen verwendet wurden.
\subsection{Analytisches Vorgehen}
Die untersuchte Stichprobe enthält alle Projekte, die im Zeitraum zwischen 2013 und 2105 abgeschlossen wurden. Die eindeutigen Abgrenzungskriterien bilden der Projektstatus und das Datum des Project Closure (MS11). Zuerst wurden alle Projekte mit einem MS11-Datum zwischen dem 1.1.2013 und dem 31.12.2015 eingegrenzt. Anschliessend wurde mittles dem Projektstatus sichergestellt, dass das Projekt auch aus finanzieller Sicht abgeschlossen war. Denn gewisse Projekte sind zwar operativ bereits beendet, gelten aber aufgrund ausstehender Rechnungen aus finanzieller Sicht als 'nicht abgeschlossen'.
\newline Nach der ersten Datenexploration und Prüfung der Annahmen für lineare statistische Modelle, wurde festgestellt, dass die ursprünglich geplante Methodenwahl nicht angewendet werden konnte. Denn die unabhängigen Daten hatten geringe bis keine Korrelation mit der abhängigen Variable sprich dem Erfolgskriterium. Die lineare Variablentransformationen und andere Methoden, um eine Verteilungskurve zu simulieren führten nur zu kleineren Verbesserung. Dieser Umstand und die Tatsache, dass Erfolgsfaktoren bereits sehr gut erforscht wurden, hat die Entscheidung auf Inferenzstatistik zu verzichten bestärkt. Die nachfolgende Analyse ist deshalb deskriptiver Natur und hat ausserdem das Ziel, die finanziellen Einbussen von sogenannten nicht erfolgreichen Projekten zu untersuchen. Die Aussagekraft der Ergebnisse wird dadurch so eingeschränkt, dass da keine Rückschlüsse auf die Grundgesamtheit (sämtliche Projekte der Bühler AG) gemacht werden können.  Die Ergebnisse haben nur in Bezug auf die die untersuchte Stichprobe Gültigkeit. Es ist jedoch denkbar, auf Basis der Ergebnisse neue Hypothesen zu formulieren, welche mittels anderer, geeigneter statistischer Methoden geprüft werden können. Die erstmalige Auswertung der Projektdaten kann zudem Erkenntnisse zu möglichen Charakteristiken nicht-erfolgreicher Projekte liefern.
\newline Das Erfolgskriterium (DB1BudDev) wurde in Zusammenarbeit mit der Bühler AG festgelegt. Aus finanzieller und interne Perspektive ist die Abweichung der relativen Projektmarge (DB1Act) vom Budget (DB1Bud) von zentraler Bedeutung. Denn sowohl die Finanzziele wie auch die Incentivierung der Projekt- und Verkaufsmanager sowie der Geschäftsbereichsleitung basieren auf DB1 und den entsprechenden Budgetvorgaben. Die relative Projektmarge errechnet sich aus Umsatz minus Kosten in Relation zum Umsatz. Anhand der Differenz zwischen Act und Bud wird der Erfolg ($Differenz \geq 0$) respektive Nicht-Erfolg ($Differenz < 0$) von Projekten gemessen. Der DB1BudDev wurde zu Analysezwecken in eine binäre Variable (Success) transformiert. Daraus folgt, dass alle positiven (negativen) Differenzen als erfolgreiche (nicht-erfolgreiche) Projekte betrachtet werden. Im Folgenden werden erfolgreiche Projekte und Success-Projekte sowie nicht-erfolgreiche Projekte und Fail-Projekte als Synonyme verwendet. Obwohl retrospektive Erkenntnisse und Erfahrungen aufgrund des Projekts einen Gewinn für das Unternehmen darstellen können, wird diesem Aspekt in dieser Analyse nicht Rechnung getragen. 
\newline\newline $Erfolgsquote = Anzahl erfolgreicher Projekte/Anzahl nicht-erfolgreicher Projekte$
\newline\newline\textbf{Datenaufbereitung:} Der Rohdatensatz enthält sämtliche Daten zu den Faktoren der untersuchten Projekte (Stichprobe). Er setzt sich aus drei Datensätzen zusammen, die separat aus den Bühler-System extrahiert wurden. Das Alter und die Betriebszugehörigkeit der Projekt- und Areamanager mussten korrigiert werden, da der ursprüngliche Datensatz die Unterscheidung zwischen fehlenden Werten und Nullwerten nicht zu liess.
\newline\newline $Stichprobenumfang N = 1471$ und $Anzahl Faktoren i = 93$.
\newline\newline
Es wurden alle vorhandenen, unplausiblen und Berechnungsfaktoren vom Datensatz entfernt. Anschliessend wurde die Anzahl fehlender Daten pro Faktor ausgewertet und zusätzlich alle Determinanten mit mehr als 300 fehlender Datensätze von der weiteren Analyse ausgeschlossen. Zusätzlich bleibt die Variablen AMNo unberücksichtigt, da durch den Ausschluss der verbundenen Variablen (AMTen und AMAge) wenig Informationsgewinn erwartet wird. Ausserdem mussten alle Variablen, welche die Zeitdifferenz zwischen dem letzten Kostenforecast und dem Projektende messen, aufgrund fragwürdiger Plausibilität und Korrektheit der Daten von der Analyse ausgeschlossen werden. Mittels diesem Vorgehen kann der Datenverlust aufgrund fehlender Daten in Grenzen gehalten werden. Der neue Stichprobenumfang beträgt $N = 1076$ und die Anzahl Faktoren $i = 71$
%Taelle zu Übersicht der Stichprobe NA
\begin{table}[htbp]
	\centering
	\caption{Anzahl NA's pro Variable (Ausschnitt)}
	\begin{tabular}{lr}
		\textbf{Variale} & \multicolumn{1}{l}{\textbf{Anzahl NA}} \\\hline
		PrTimeDelayMS5 & 538 \\
		AMAge2 & 444 \\
		AMTen2 & 444 \\
	\end{tabular}%
	\label{tab:na}%
\end{table}%
Im Anschluss wurden die Datensätze auf ihre Plausibilität getestet und Ausreisser entfernt. Die Plausibilitätsüberlegungen basieren auf der logischen Interpretation und Herleitung der Indikatoren. Die Tabellen mit den Begründungen der unplausiblen Werte und Ausreisser befinden sich im Anhang. Die Outliers wurden mit Hilfe von Boxplots, Histogramme und der 'Interquartile Ranges' (IQR) der numerischen Variablen identifiziert. Zur quantitativen Bestimmung der Ausreisser wurde folgendes Entscheidungskalkül angewendet:
\newline\newline
\begin{centering}
	$ Werte < Q1 - 1.5 * IQR$ und $ Werte > Q3 + 1-5 * IQR$
\end{centering}
\newline\newline
\begin{centering}
	$ Werte < Q1 - 3 * IQR$ und $ Werte > Q3 + 3 * IQR$
\end{centering}
\newline\newline
Je nach Zweck der Analyse und untersuchten Objekten sind Ausreisser unterschiedlich einzustufen. Die Geschäftsbereiche der Bühler AG verkaufen unterschiedliche Anlangen, weshalb die Datenbereiche der Faktoren stark variieren können. Die realisierte Projektmarge (DB1Act), wurde auf die Werte des doppelten IQR berichtigt, da extreme negative Margen auf sogenannte Crash-Projects schliessen lassen, welche bereits mittels internem Audit untersucht wurden und die Stichprobenergebnisse unnötige verzerren können. Extreme positive DB1Act sind bei einer durchschnittlichen Projektmarge von ca. 30\% relativ unwahrscheinlich und lassen Zweifel zur Richtigkeit der Kostenverbuchung zu. Bei den relativen Kostenabweichungen für PA und IS wurden jeweils einzelne Extremalwerte nur dann entfernt, wenn kein entsprechendes Budget geplant wurde. Denn es wurde davon ausgegangen, dass die Budgetierung der Projektkosten nicht korrekt verlaufen ist, was letztendlich zu extremalen relativen Kostenabweichung geführt hat. Es wurden keine weiteren Ausreisser eliminiert, selbst wenn einige Werte ausserhalb des Entscheidungskalküls lagen. Nach der Datenbereinigung umfasst die zu untersuchende Stichprobe $N = 966$ Projekte und die Anzahl verfügbarer Faktoren entspricht $ i = 71$. 
\newline\newline
%%
%%Zusätliche Variablen
%%
\textbf{Zusätzliche Variablen:} Nach dem Datenbereinigungsprozess wurden zu analytischen Zwecken vor allem kategoriale Variablen auf Basis der vorhandene Daten erhoben (mit * gekennzeichnet). Die nachfolgende Tabelle zeigt sämtliche verbleibende (s. Kapitel 2 für alle Faktoren) inklusive der hinzugefügten Faktoren nach ihrer Kategorie strukturiert. Sämtliche Berechnungsformeln sowie die Interpretationen der Faktoren sind im Anhang enthalten.
\begin{table}[htbp]
	\centering
	\caption{Übersicht der Faktoren}
	\begin{tabular}{llll}
		\textbf{Erfolgskriterium} &       &       &  \\\hline
		DB1BudDev &   Dummy\_Success *    &       &  \\
		Success * &     Dummy\_Fail * &       &  \\
		\textbf{Rahmenbedingungen} & \multicolumn{1}{l}{\textbf{Zeitmanagement}} & \multicolumn{1}{l}{\textbf{Sales \& Quoatation}} & \multicolumn{1}{l}{\textbf{Komplexität}} \\\hline
		CuNo  & \multicolumn{1}{l}{PrTimeBase} & \multicolumn{1}{l}{BUORBudGapAbs} & \multicolumn{1}{l}{ConPart} \\
		EquLoc & \multicolumn{1}{l}{PrTimeAct} & \multicolumn{1}{l}{BUORBudGapRel} & \multicolumn{1}{l}{NoSupplSAS} \\
		BA    & \multicolumn{1}{l}{PrTimeDelay} & \multicolumn{1}{l}{RegiORBudGapAbs} & \multicolumn{1}{l}{NoSupplSASMS} \\
		BU    & \multicolumn{1}{l}{PrTimeDelayMS2} & \multicolumn{1}{l}{RegiORBudGapRel} & \multicolumn{1}{l}{NoSupplSASME} \\
		MS    & \multicolumn{1}{l}{PrTimeDelayMS8} &       & \multicolumn{1}{l}{NoSupplSASPA} \\
		Region & \multicolumn{1}{l}{PrTimeDelayMS10} &       & \multicolumn{1}{l}{NoSupplSASIS} \\
		& \multicolumn{1}{l}{PrTimeDelayMS11} &       & \multicolumn{1}{l}{NoContr} \\
		& \multicolumn{1}{l}{Delay * } &       &  \\
		\textbf{Kostenmanagement} &       & \multicolumn{1}{l}{\textbf{Fulfillment}} &  \\\hline
		TOBud & \multicolumn{1}{l}{CostActBudISRel} & \multicolumn{1}{l}{PMNo} & \multicolumn{1}{l}{CostFCadjPA} \\
		BudMSTot & \multicolumn{1}{l}{DeltaLastFCAct} & \multicolumn{1}{l}{PMAge2} & \multicolumn{1}{l}{CostFCadjIS} \\
		BudMETot & \multicolumn{1}{l}{DeltaLastFCActMS} & \multicolumn{1}{l}{PMTen2} & \multicolumn{1}{l}{HOMYellCost} \\
		BudPATot & \multicolumn{1}{l}{DeltaLastFCActME} & \multicolumn{1}{l}{PMChange} & \multicolumn{1}{l}{HOMYellQual} \\
		BudISTot & \multicolumn{1}{l}{DeltaLastFCActPA} & \multicolumn{1}{l}{NoPM} & \multicolumn{1}{l}{HOMYellTime} \\
		DB1Bud & \multicolumn{1}{l}{DeltaLastFCActIS} & \multicolumn{1}{l}{LeadSASPr} & \multicolumn{1}{l}{HOMRedCost} \\
		DB1Act & \multicolumn{1}{l}{TOAct} & \multicolumn{1}{l}{LeadSAS.PrFF} & \multicolumn{1}{l}{HOMRedQual} \\
		CostActBudMSabs & \multicolumn{1}{l}{TOBudDevabs *} & \multicolumn{1}{l}{NoLeadSASFF} & \multicolumn{1}{l}{HOMRedTime} \\
		CostActBudMEabs & \multicolumn{1}{l}{CostBud *} & \multicolumn{1}{l}{CostFCadj} & \multicolumn{1}{l}{PrStartDate} \\
		CostActBudPAabs & \multicolumn{1}{l}{CostAct *} & \multicolumn{1}{l}{CostFCadjMS} & \multicolumn{1}{l}{Cat\_age * } \\
		CostActBudISabs & \multicolumn{1}{l}{CostBudDevabs *} & \multicolumn{1}{l}{CostFCadjME} &  \\
		SUCostTO & \multicolumn{1}{l}{DB1Budabs} &       &  \\
		CostActBudRel & \multicolumn{1}{l}{DB1Actabs} &       &  \\
		CostActBudMSRel & \multicolumn{1}{l}{DB1BudDevabs *} &       &  \\
		CostActBudMERel & \multicolumn{1}{l}{TOBudCat *} &       &  \\
		CostActBudPARel &       &       &  \\
	\end{tabular}%
	\label{tab:ovvar}%
\end{table}%
\subsection{Ergebnisse und Interpretation}
Die Ergebnisse der finanziellen Analyse und der Untersuchung der Einflussfaktoren werden getrennt dargestellt. Die untersuchte Stichprobe enthält 966 Projekte, wovon 654 erfolgreich abgeschlossen wurden.
\begin{table}[htbp]
	\centering
	\caption{Übersicht Stichprobe}
	\begin{tabular} {l|r|r}
		\textbf{Stichprobe} & \textbf{absolut} & \textbf{relativ} \\\hline
		\textbf{Total} & 966 & 100\% \\
		\textbf{Success} & 654 & 68\% \\
		\textbf{Fail} & 312 & 32\% \\
	\end{tabular}
\end{table}
\subsubsection{Finanzielle Performance Analyse}
Zur Bewertung der finanziellen Performance wurden drei verschiedene Auswertungen gemacht: Abweichung Act-Bud, Zusammensetzung der Kosten sowie eine Auswertung pro Umsatzkategorie. Das Ziel besteht darin, die den finanziellen Verlust auf Basis des zugrundeliegenden Erfolgskriterium (DB1BudDev) zu quantifizieren. 
\begin{table}[htbp]
	\centering
	\caption{Übersicht Budget [TCHF]}
	\begin{tabular}{lrrrr}
		\textbf{Erfolgskriterium} & \textbf{TO Bud} & \textbf{Cost Bud} &
		\textbf{DB1 Bud} & \textbf{DB1 Bud [\%]} \\
	SUCCESS & 1'552'450 & -1'156'598 & 395'851 & 25.5\% \\
	FAIL  & 618'013 & -465'066 & 152'947 & 24.7\% \\
	Grand Total & 2'170'463 & -1'621'664 & 548'799 & 25.3\% \\
	\end{tabular}%
\label{bud}%
\end{table}%
\begin{table}[htbp]
	\centering
	\caption{Übersicht Actuals [TCHF]}
	\begin{tabular}{lrrrr}
		\textbf{Erfolgskriterium} & \textbf{TO Act} & \textbf{Cost Act} & \textbf{DB1 Act}&
		\textbf{DB1 Act-Bud [\%]} \\
			SUCCESS & 1'560'001 & -1'041'728 & 518'273 & 33.2\% \\
			FAIL  & 631'346 & -526'600 & 104'746 & 16.6\% \\
			Grand Total & 2'191'347 & -1'568'328 & 623'018 & 28.4\% \\
	\end{tabular}
\label{act}%
\end{table}%
\begin{table}[H]
\centering
\caption{Übersicht Abweichungen [TCHF] ($Act-Bud$)}
\begin{tabular}{lrrrr}
	\textbf{Erfolgskriterium} & \textbf{TO} & \textbf{Cost} & \textbf{DB1}&
	\textbf{DB1 [\%]} \\
	SUCCESS & 7'551 & 114'870 & 122'421 & 7.7\% \\
	FAIL  & 13'333 & -61'534 & -48'202 & -8.2\% \\
	Grand Total & 20'884 & 53'336 & 74'220 & 3.1\% \\
\end{tabular}
\label{Abw}%
\end{table}%
Die Tabelle \ref{Abw} wurde mittels Tabellen \ref{act} und \ref{bud} berechnet und zeigt, dass der realisierte Umsatz höher war als budgetiert wurde. Diese Abweichung kann auf Zusatzverkäufe oder die Verrechnung allfälliger Mehrkosten an den Kunden zurückgeführt werden. Der kumulierte DB1 der Fail-Projekte lag 48 Mio. CHF (-32\%) unter dem Budget und der Success-Projekte 122 Mio. CHF über dem Budget. Die positive Abweichung der Istkosten der Success-Projekte kann mittels der realisierten Kostenreserve, die üblicherweise pro Projekt einkalkuliert wird und je nach Geschäftsbereich zwischen 4\% und 9\% Kostenreserven beträg, zurückgeführt werden. Wenn die Kostenreserve aufgebraucht wird, resultieren Mehrkosten und die Kostenabweichung wird negative. Da die Reserve in dieser Betrachtung nicht ersichtlich ist, wäre die effektive Differenz für Fail-Projekte (Success-Projekte) tiefer (höher). Die realisierte Marge über alle Success-Projekte beträgt 33\% und liegt 7.7\% über der budgetierten Marge von 25.4\%. Demgegenüber beträgt der DB1Act der Fail-Projekte 16.6\% und liegt 8.2\% unter dem DB1 Bud von 24.7\%.
\newline
Die Aufschlüsselung der Kostenabweichung zeigt, dass die Installationsphase sowohl der erfolgreichen als auch der nicht-erfolgreich Projekte mit Mehrkosten verbunden ist. Die negative Kostenabweichung der Fail-Projekte kann zu einem Drittel auf die IS- und zu einem weiteren Drittel auf die MS-Kosten zurückgeführt werden. Bei den Success-Projekten kann ein gewisser 'Verlust'-Kompensationseffekt durch die positive Kostenabweichung der MS-Kosten festgestellt werden. Die kumulierte Kostendifferenz zwischen Act und Bud der Fail-Projekte kann fast vollständig durch die Kostendifferenz der vier Kostenarten MS, ME, PA und IS erklärt werden. Der Unterschied zu 'Total Cost' ist auf nicht-Abbildung der fehlenden Kostenarten zurückzuführen. Dieser Effekt ist bei den Success-Projekten ebenfalls sichtbar und kann zu einem Teil auf die Realisation des Kostenpuffers zurückgeführt werden.
\begin{table}[H]
	\centering
	\caption{Kostenabweichung [TCHF]}
	\begin{tabular}{lrrrrrr}
		\textbf{Erfolgskriterium} & \multicolumn{1}{l}{\textbf{Total Cost}} & \multicolumn{1}{l}{\textbf{MS}} & \multicolumn{1}{l}{\textbf{ME}} & \multicolumn{1}{l}{\textbf{PA}} & \multicolumn{1}{l}{\textbf{IS}} & \multicolumn{1}{l}{\textbf{Total MS bis IS}} \\
		SUCCESS & 114'870 & 47'615 & -2'159 & -908  & -7'114 & 37'434 \\
		FAIL  & -61'534 & -20'253 & -12'721 & -7'220 & -22'053 & -62'247 \\
		Grand Total & 53'336 & 27'363 & -14'880 & -8'128 & -29'167 & -24'812 \\
	\end{tabular}%
	\label{stocostdb1dev}%
\end{table}%
Als Ergänzung wurde versucht zu eruieren, von welchem Projekttyp die Margeneinbusse der Fail-Projekte stammt. Dazu wurde die Häufigkeit und die absolute DB1 Abweichung pro Umsatzkategorie berechnet.
%Tabelle zu Häufigkeit und Verlust pro TOBUDCAT
\begin{table}[H]
	\centering
	\caption{DB1-Abweichung pro Umsatzkategorie (TOBud\_Cat) [TCHF]}
	\begin{tabular}{lrrr}
		& \multicolumn{1}{l}{\textbf{TOBud\_Cat}} &       & \multicolumn{1}{l}{\textbf{DB1BudDevabs}} \\
		\multicolumn{1}{r}{1} & \multicolumn{1}{l}{[13.2,500)} & 54    & -1'672 \\
		\multicolumn{1}{r}{2} & \multicolumn{1}{l}{[500,1e+03)} & 87    & -6'210 \\
		\multicolumn{1}{r}{3} & \multicolumn{1}{l}{[1e+03,1.5e+03)} & 54    & -3'966 \\
		\multicolumn{1}{r}{4} & \multicolumn{1}{l}{[1.5e+03,2e+03)} & 32    & -3'822 \\
		\multicolumn{1}{r}{5} & \multicolumn{1}{l}{[2e+03,2.5e+03)} & 17    & -3'161 \\
		\multicolumn{1}{r}{6} & \multicolumn{1}{l}{[2.5e+03,3e+03)} & 12    & -1'289 \\
		\multicolumn{1}{r}{7} & \multicolumn{1}{l}{[3e+03,3.5e+03)} & 8     & -1'359 \\
		\multicolumn{1}{r}{8} & \multicolumn{1}{l}{[3.5e+03,4e+03)} & 8     & -2'107 \\
		\multicolumn{1}{r}{9} & \multicolumn{1}{l}{[4e+03,4.5e+03)} & 4     & -1'360 \\
		\multicolumn{1}{r}{10} & \multicolumn{1}{l}{[4.5e+03,5e+03)} & 6     & -1'662 \\
		\multicolumn{1}{r}{11} & \multicolumn{1}{l}{[5e+03,1e+04)} & 23    & -12'043 \\
		\multicolumn{1}{r}{12} & \multicolumn{1}{l}{[1e+04,3.42e+04)} & 7     & -9'551 \\
		\textbf{Total} &       &       & \textbf{-48'202} \\
	\end{tabular}%
	\label{ftobudcat}%
\end{table}%
Die Auswertung (s.Tabelle \ref{ftobudcat}) zeigt, dass ein Viertel des der DB1-Abweichung auf 23 Projekte mit einem Umsatzvolumen zwischen 5 und 10 Mio. CHF und 20\% auf 7 Projekte mit einem Umsatzvolumen von mehr als 10 Mio. CHF zurückzuführen ist. Die drittgrösste Abweichung stammt von der Umsatzkategorie mit den meisten Projekten.
\subsubsection{Erfolgsfaktoren}
In diesem Kapitel werden die Ergebnisse pro Variablenkategorie sowie mögliche Erklärungsansätze erläutert. Mittels Histogrammen, Häufigkeitstabellen und Mittelwerten wurde versucht, die Charakteristiken vergangener Fail-Projekte zu ergründen. Die Stichprobe wurde hierfür gemäss Erfolgskriterium in zwei Datensets unterteilt. Zur Evaluation von kategorialen Variablen wurde ein weiteres Kriterium die Erfolgsquote $(Anzahl Success-Projekte) / (Anzahl Fail-Projekte)$ hinzugezogen, um beispielsweise Geschäftsbereiche oder Region untereinander vergleichen zu können.
\newline\newline\textbf{Rahmenbedingungen:} Die Analyse der Rahmenbedingungen eines Projekts geben Hinweise darauf, in welchen Geschäftsbereichen und Regionen und mit welchen Kunden nicht-erfolgreiche gemäss dem Erfolgskriterium realisiert wurden. Da die Bühler AG in einer Matrix-Organisation organsiert ist, wurde nebst den Einzelauswertungen für die Region und die Business Area, der Regionen-BA Split für die Häufigkeit der Success- und Fail-Projekte erstellt.
\begin{table}[H]
	\centering
	\caption{Erfolgsquote pro Region}
	\begin{tabular}{lrrrrrr}
		\textbf{Region} & \multicolumn{1}{l}{\textbf{Erfolgsquote}} & \multicolumn{1}{l}{\textbf{Success}} & \multicolumn{1}{l}{\textbf{Fail}} & \multicolumn{1}{l}{\textbf{Fail [\%]}} & \multicolumn{1}{l}{\textbf{Total}} & \multicolumn{1}{l}{\textbf{Total [\%]}} \\ \hline
		East\_Asia & 6.7   & 20    & 3     & 13.0\% & 23    & 2.4\% \\
		EU    & \textbf{1.7}   & 240   & 145   & 37.7\% & 385   & 39.9\% \\
		MEA\_Afr & 2.7   & 112   & 42    & 27.3\% & 154   & 15.9\% \\
		North\_Ame & \textbf{1.4}   & 54    & 38    & 41.3\% & 92    & 9.5\% \\
		SAS\_BCHI & 2.9   & 119   & 41    & 25.6\% & 160   & 16.6\% \\
		South\_Ame & 1.9   & 58    & 31    & 34.8\% & 89    & 9.2\% \\
		South\_Asia & 4.3   & 51    & 12    & 19.0\% & 63    & 6.5\% \\ \hline
		\textbf{Total} & \textbf{2.1} & \textbf{654} & \textbf{312} & \textbf{32.3\%} & \textbf{966} & \textbf{100.0\%} \\
	\end{tabular}%
	\label{freg}%
\end{table}%  
\begin{table}[H]
	\centering
	\caption{Erfolgsquote pro Geschäftsbereich}
	\begin{tabular}{lrrrrrr}
		\textbf{BA}   & \multicolumn{1}{l}{\textbf{Erfolgsquote}} & \multicolumn{1}{l}{\textbf{Success}} & \multicolumn{1}{l}{\textbf{Fail}} & \multicolumn{1}{l}{\textbf{Fail [\%]}} & \multicolumn{1}{l}{\textbf{Total}} & \multicolumn{1}{l}{\textbf{Total [\%]}} \\ \hline
		CF    & 2.8   & 118   & 42    & 26.3\% & 160   & 16.6\% \\
		DC    & 5.6   & 96    & 17    & 15.0\% & 113   & 11.7\% \\
		GD    & 2.3   & 7     & 3     & 30.0\% & 10    & 1.0\% \\
		GL    & 1.2   & 39    & 32    & 45.1\% & 71    & 7.3\% \\
		GM    & 1.9   & 226   & 122   & 35.1\% & 348   & 36.0\% \\
		LO    & 1.4   & 30    & 21    & 41.2\% & 51    & 5.3\% \\
		SR    & 5.0   & 35    & 7     & 16.7\% & 42    & 4.3\% \\
		TP    & NA      & 8     & 0     & 0.0\% & 8     & 0.8\% \\
		VN    & 1.4   & 95    & 68    & 41.7\% & 163   & 16.9\% \\\hline
		\textbf{Total } & \textbf{2.1} & \textbf{654} & \textbf{312} & \textbf{32.3\%} & \textbf{966} & \textbf{100.0\%} \\
	\end{tabular}%
	\label{fba}%
\end{table}%
Die Ergebnisse der Tabellen \ref{freg} und \ref{fba} reflektieren die Tatsache, dass Europa der grösste Absatzmarkt und GM die grösste Business Area der Bühler AG ist. Die niedrigste Erfolgsquote hat NAM als viertgrösste Region (in Abhängigkeit der Anzahl Projekte), gefolgt von Europa. Die Anzahl der Fail-Projekte in den Regionen EU, MEA und SAS\_BCHI beträgt 73\% ($(145+42+41)/312$), weshalb die Erfolgsquote von allen Projekten hauptsächlich durch diese drei Regionen bestimmt wird. Die kleinsten Regionen haben die besten Erfolgsquoten. Die Geschäftsbereichen CF, VN, GL und GM ($(42+68+32+122)/312$) umfassen zusammen 84\% aller Fail-Projekte, wobei die drei letzt genannten zugleich die niedrigsten Erfolgsquoten ausweisen. Die Anzahl untersuchter Projekte der letzten drei Jahre der Geschäftsbereiche CF und VN ist faktisch identisch, allerdings weist VN eine viel tiefere Erfolgsquote aus als CF.
\begin{table}[H]
	\centering
	\caption{Erfolgsquote pro Geschäftseinheit}
	\begin{tabular}{llrrrrr}
		\textbf{BA} & \textbf{BU} & \multicolumn{1}{l}{\textbf{Erfolgsquote}} & \multicolumn{1}{l}{\textbf{Success}} & \multicolumn{1}{l}{\textbf{Fail}} & \multicolumn{1}{l}{\textbf{Fail [\%]}} & \multicolumn{1}{l}{\textbf{Total}} \\\hline
		GL    & GC    & NA    & 1     & 0     & 0.0\% & 1 \\
		GL    & GS    & 1.2   & 36    & 29    & 44.6\% & 65 \\
		GL    & MT    & 0.7   & 2     & 3     & 60.0\% & 5 \\\hline
		\textbf{GL} &  & \textbf{1.2} & \textbf{39} &\textbf{32} & \textbf{45.1\%} & \textbf{71}\\
		      &       &       &       &       &        &   \\
		GM    & BA    & 1.5   & 17    & 11    & 39.3\% & 28 \\
		GM    & BR    & 0.9   & 12    & 14    & 53.8\% & 26 \\
		GM    & IM    & 2.1   & 185   & 87    & 32.0\% & 272 \\
		GM    & SM    & 1.2   & 12    & 10    & 45.5\% & 22 \\\hline
		\textbf{GM} &  & \textbf{1.9} & \textbf{226} &\textbf{122} & \textbf{35.1\%} & \textbf{348}\\
		      &       &       &       &       &        &   \\
		VN    & AG    & 0.8   & 11    & 14    & 56.0\% & 25 \\
		VN    & FE    & 1.2   & 27    & 22    & 44.9\% & 49 \\
		VN    & NU    & 1.3   & 27    & 21    & 43.8\% & 48 \\
		VN    & OL    & 2.0   & 6     & 3     & 33.3\% & 9 \\
		VN    & PN    & 3.0   & 24    & 8     & 25.0\% & 32 \\\hline
		\textbf{VN} &  & \textbf{1.4} & \textbf{95} &\textbf{68} & \textbf{41.7\%} & \textbf{163}\\
	\end{tabular}%
	\label{fbabu}%
\end{table}%
Bei der Auswertung der Geschäftseinheiten (s. Tabelle \ref{fbabu}) für GL, GM und VN, konnte festgestellt werden, dass die kleineren BU's von GM eine verhältnismässig tiefe Erfolgsquote hatten. Dennoch wird das Verhältnis zwischen erfolgreichen und nicht-erfolgreichen Projekten fast ausschliesslich durch IM, die grösste BU in GM bestimmt. In der Business Area VN, sind die Erfolgsquoten mit Ausnahme von PN und OL grundsätzlich tief. Die Business Unit Grain Storage determiniert die Geschäftsbereichserfolgsquote von GL.
\begin{table}[H]
	\centering
	\caption{Ausschnitt Häufigkeitsverteilung Regionen-BA Split}
	\begin{tabular}{llrrrrr}
		\textbf{Region} & \textbf{BA}    & \multicolumn{1}{l}{\textbf{Erfolgsquote}} & \multicolumn{1}{l}{\textbf{Success}} & \multicolumn{1}{l}{\textbf{Fail}} & \multicolumn{1}{l}{\textbf{Fail [\%]}} & \multicolumn{1}{l}{\textbf{Total}} \\\hline
		EU    & CF    & 1.9   & 58    & 31    & 34.8\% & 89 \\
		EU    & DC    & 5.0   & 45    & 9     & 16.7\% & 54 \\
		EU    & GD    & NA    & 2     & 0     & 0.0\% & 2 \\
		EU    & \textbf{GL}    & 1.0   & 24    & 23    & 48.9\% & 47 \\
		EU    & \textbf{GM}  & 1.2   & 58    & 50    & 46.3\% & 108 \\
		EU    & LO    & 2.5   & 10    & 4     & 28.6\% & 14 \\
		EU    & SR    & 2.5   & 5     & 2     & 28.6\% & 7 \\
		EU    & \textbf{VN}     & 1.5   & 38    & 26    & 40.6\% & 64 \\\hline
		North\_Ame & CF    & 2.0   & 10    & 5     & 33.3\% & 15 \\
		North\_Ame & DC    & 1.0   & 2     & 2     & 50.0\% & 4 \\
		North\_Ame & GL    & 1.0   & 1     & 1     & 50.0\% & 2 \\
		North\_Ame & \textbf{GM}   & 1.5   & 24    & 16    & 40.0\% & 40 \\
		North\_Ame & LO    & 4.0   & 4     & 1     & 20.0\% & 5 \\
		North\_Ame & SR    & 1.0   & 1     & 1     & 50.0\% & 2 \\
		North\_Ame & \textbf{VN}  & 1.0   & 12    & 12    & 50.0\% & 24 \\
	\end{tabular}%
	\label{tab:fregba}%
\end{table}%
Im Regionen-BA Split der Tabelle \ref{tab:fregba} sind für diejenigen Regionen mit den niedrigsten Erfolgsquoten, EU und NAM, sind jene BA's mit den niedrigsten Erfolgsquoten zu finden. Die Kombination EU-GM, EU-GL, EU-VN mit den tiefen Erfolgsquoten machen knapp 30\% ($(50+23+26)/312$) aller Fail-Projekte aus. Die niedrige Erfolgsquote von NAM stammt vor allem aus VN- und GM-Projekten, wobei VN noch vor GM weniger gut abschneidet.
\newline
Zusammenfassend lässt sich aussagen, dass ungefähr 60\% ($(122+68)/312$ respektive $(145+38)/312$) der Fail-Projekte entweder in den Geschäftsbereichen VN und GM respektive in den Regionen EU und NAM liegen. Zudem wird die Erfolgsquote aller Projekte zu 30\% durch europäische Projekte in den Geschäftsbereichen GM, GL und VN bestimmt wird.
%
%Analyse der Kosten
%
\newline\newline\textbf{Kosten} Das Umsatzvolumen soll Aufschluss über die Grösse und Wichtigkeit eines Projekts geben. Die zugrundeliegende Prämisse postuliert, dass Projekte mit höherem Umsatzvolumen risikoreicher sind und deshalb eher unter Budget beendet werden. Die Gegenhypothese unterstellt, dass grössere Projekte (hoher TOBud) relativ mehr Beachtung erhalten, da sie den Erfolg eines Geschäftsbereich respektive einer Region mehr beeinflussen als kleinere Projekte. Deshalb liege der Fokus auf der Einhaltung der Budgetvorgaben.
%
%include histogram of TOBud_cat
%
\begin{figure}[ht]
	\centering
	\includegraphics[scale=0.5]{test.pdf}
	\caption{Histogram Umsatzkategorie}
	\label{fig: htobudcat}
\end{figure}
Die Verteilung des Umsatzvolumen ist linksschief und zeigt dass der Grossteil der Projekte ein Umsatzvolumen von weniger als 10 Mio. CHF haben.
\newline\textbf{Histogram} Das Histogramm für die TOBud\_Cat zeigt, dass ca. zwei Drittel aller untersuchten Projekte ein Umsatzvolumen von bis und mit 2 Mio. CHF hat. Die Anzahl Fail-Projekte konzentriert sich folglich in den vier untersten Kategorien.
\newline\newline Die absoluten und relativen Abweichungen zwischen den aktuellen und den budgetierten Kosten sind direkt mit dem Erfolgskriterium korreliert und sind erwartungsgemäss für Fail-Projekte höher. 
% Mean absolute Kostenabweichungen
\begin{table}[H]
	\centering
	\caption{Arithmetisches Mittel der absoluten Kostenabweichungen pro Kostenart [TCHF]}
	\begin{tabular}{lrrrrr}
		\textbf{Success} & \multicolumn{1}{l}{\textbf{Total Cost}} & \multicolumn{1}{l}{\textbf{MS}} & \multicolumn{1}{l}{\textbf{ME}} & \multicolumn{1}{l}{\textbf{PA}} & \multicolumn{1}{l}{\textbf{IS}}
		\\\hline
		FALSE & -197  & -65   & -41   & -23   & -71 \\
		TRUE  & 176   & 73    & -3    & -1    & -11 \\
	\end{tabular}%
	\label{mcostabs}%
\end{table}%
% Mean relative Kostenabweichung
\begin{table}[H]
	\centering
	\caption{Arithmetisches Mittel der relativen Kostenabweichungen pro Kostenart [\%]}
	\begin{tabular}{lrrrrr}
		\textbf{Success} & \multicolumn{1}{l}{\textbf{Total Cost}} & \multicolumn{1}{l}{\textbf{MS}} & \multicolumn{1}{l}{\textbf{ME}} & \multicolumn{1}{l}{\textbf{PA}} & \multicolumn{1}{l}{\textbf{IS}}
		\\\hline
		FALSE & 12    & 9     & 206   & 79    & 93 \\
		TRUE  & -10   & 4     & 56    & 17    & 28 \\
	\end{tabular}%
	\label{mcostrel}%
\end{table}%
Die Tabellen \ref{mcostabs} und \ref{mcostrel} zeigen, dass die durchschnittlichen Abweichung der Projektkosten von nicht-erfolgreichen Projekten in absoluten und relativen Grössen über derjenigen der Success-Projekte liegt. Dabei fällt vor allem ist der Durchschnittswert der relativen Differenz Kostenabweichung der ME-Kosten auf.
\newline\newline Die Zusammensetzung der budgetierten Projektkosten soll Hinweise zur Projektart aufzeigen und ob sie sich zwischen den zwei Projektgruppen unterscheidet.
\begin{table}[htbp]
	\centering
	\caption{Arithmetisches Mittel der relativen Anteile am Gesamtkostenbudget je Kostenart [\%]}
	\begin{tabular}{lrrrr}
		\textbf{Success} & \multicolumn{1}{l}{\textbf{BudMSTot}} & \multicolumn{1}{l}{\textbf{BudMETot}} & \multicolumn{1}{l}{\textbf{BudPATot}} & \multicolumn{1}{l}{\textbf{BudISTot}} \\\hline
		FALSE & 67.1  & 6.2   & 5.9   & 7.7 \\
		TRUE  & 67.9  & 5.4   & 5.1   & 6.8 \\
	\end{tabular}%
	\label{mbudtot}%
\end{table}%
Die dargestellten Mittelwerte in der Tabelle \ref{mbudtot} liegen für nicht-erfolgreiche Projekte bei ME, PA und IS etwas höher. In der vorherigen Auswertung der durchschnittlichen Kostenabweichung wurde exakt bei diesen Projektphasen für Fail-Projekte deutlich höhere Werte festgestellt.
%
%Auswertung Nachlieferung
%
\newline\newline Nachlieferungen verursachen zusätzliche Kosten, die aufgrund der zeitlichen Verzögerung bei der Fabrikation der Maschine oder Installation entstehen können. Es wird spekuliert, dass der Anteil der Kosten aus Nachlieferungen im Verhältnis zum Umsatzbudget bei Fail-Projekten höher ist als bei Success-Projekten.
\begin{table}[H]
	\centering
	\caption{Arithmetisches Mittel der SUCostTO [\%]}
	\begin{tabular}{lr}
		\textbf{Success} & \multicolumn{1}{l}{\textbf{SUCostTO}} \\\hline
		FALSE & -0.81 \\
		TRUE  & -0.36 \\
	\end{tabular}%
	\label{msu}%
\end{table}%
Die Auswertung (s. Tabelle \ref{msu}) des arithmetischen Mittels der prozentualen SU-Kosten am Umsatz bestätigt die erwartete Vermutung.
\begin{table}[H]
	\centering
	\caption{Arithmetisches Mittel der SUCostTO [\%] pro TO-Kategorie}
	\begin{tabular}{llr}
		\textbf{Success} & \textbf{TOBud\_Cat} & \multicolumn{1}{l}{\textbf{SUCostTO}} \\\hline
		FALSE & [13.2,500) & -0.93 \\
		FALSE & [500,1e+03) & -0.69 \\
		FALSE & [1e+03,1.5e+03) & -0.47 \\
		FALSE & [1.5e+03,2e+03) & -0.72 \\
		FALSE & [2e+03,2.5e+03) & -0.57 \\		
		FALSE & [2.5e+03,3e+03) & -0.82 \\
		FALSE & [3e+03,3.5e+03) & -0.31 \\
		FALSE & [3.5e+03,4e+03) & -0.69 \\
		FALSE & [4e+03,4.5e+03) & -0.52 \\
		FALSE & [4.5e+03,5e+03) & -6.88 \\
		FALSE & [5e+03,1e+04) & -0.82 \\
		FALSE & [1e+04,3.42e+04) & -0.36 \\
	\end{tabular}%
	\label{msutocat}%
\end{table}%
Die Mittelwertauswertung von SUCostTO in der Tabelle \ref{msutocat} pro TO-Kategorie zeigt, dass für Projekte mit einem Umsatzvolumen zwischen 13.2 TCHF und 500 TCHF die Nachlieferungskosten in Relation zum Umsatz am höchsten war. Der Wert 6.9\% kann als Anomalie betrachtet werden, ein Projekt mit einem SUCostTO-Wert von ca. 40\% ein Einzelfall darstellt.
%
%Auswertung DeltaLastFCAct
%
\newline\newline Tendenziell wird die Anpassung des Forecast für die Projektkosten bei erwarteten Mehrkosten möglichst lange hinausgezögert. Einerseits kann mit diesem Vorgehen, die Erklärungsdirektive umgangen werden und anderseits besteht wahrscheinlich die Hoffnung, dass die Projektkosten sich wieder normalisieren. Deshalb wird erwartet, dass die Differenz zwischen der letzten FC-Anpassung und den tatsächlichen Kosten bei Fail-Projekten höher ist. 
%Tabelle Mean DeltaLastFCAct  
\begin{table}[H]
	\centering
	\caption{Arithmetisches Mittel DeltaLastFCAct für MS, ME, PA und IS [TCHF]}
	\begin{tabular}{lrrrrr}
		\textbf{Success} & \multicolumn{1}{c}{\textbf{Total FC}} & \multicolumn{1}{c}{\textbf{MS}} & \multicolumn{1}{c}{\textbf{ME}} & \multicolumn{1}{c}{\textbf{PA}} & \multicolumn{1}{c}{\textbf{IS}}
		\\\hline
		FALSE & -490.54 & -445.53 & 7.48 & -12.87 & -14.52 \\
		TRUE  & -436.41 & -454.24 & 7.93 & -13.41 & -6.49\\
	\end{tabular}%
	\label{mdeltalastfcact}%
\end{table}%
Die Ergebnisse der Tabelle \ref{mdeltalastfcact} bestätigen diese Vermutung. Die durchschnittliche Differenz bei den IS-Kosten war für Fail-Projekte doppelt so hoch.
%%
%%
%%Auswertung FF-Variablen
%%
%%
\newline\newline\textbf{Fulfillment:} Der bedeutenste Einflussfaktor im Projektmanagement ist der Projektmanager selbst. Die Evaluation der realisierten Projekte pro Projektmanager inklusive der Erfolgsquote hat ergeben, dass die 966 Projekte von 301 unterschiedlichen Projektmanager abgewickelt wurde. 145 Projektmanager haben ihre Projekte aussschliesslich erfolgreich beendet, wohingegen gerade einmal 45 PM nur unzureichend Projekte abgewickelt hat. Die detaillierte Liste ist im Anhang zu finden.
\newline\newline Der Wechsel des Projektmanagers wird mit konfligierende Verhältnisse zwischen den Vertragsparteien assoziiert, weshalb hypothetisch vermutet wird, dass Fail-Projekte eher mit einem PMChange einhergehen. 
\begin{table}[H]
	\centering
	\caption{Häufigkeit PMChange}
	\begin{tabular}{lrrrrr}
		\textbf{PMChange} & \multicolumn{1}{l}{\textbf{Success}} & \multicolumn{1}{l}{\textbf{Fail}} & \multicolumn{1}{l}{\textbf{Fail [\%]}} & \multicolumn{1}{l}{\textbf{Total}} &
		\multicolumn{1}{l}{\textbf{Total [\%]}} \\\hline
		no    & 628   & 295   & 32.0\% & 923 & 96\% \\
		yes   & 26    & 17    & 39.5\% & 43  & 4\% \\\hline
		\textbf{Total} & \textbf{654} & \textbf{312} & \textbf{32.3\%} & \textbf{966} & \textbf{100.0\%} \\
	\end{tabular}%
	\label{pmchange}%
\end{table}%
Insgesamt wurden 43 Projekte mit einem Wechsel des Projektmanagers über die letzten drei Jahre abgewickelt wie der Tabelle \ref{pmchange} zu entnehmen ist. Davon sind 17 gescheitert und 26 wurden erfolgreich abgeschlossen. Die Anzahl Projektmanager (NoPM) ist direkt mit der Variable PMChange verbunden und weist wenig Informationsgehalt auf. Es gab in der Stichprobe sechs Projekte, bei denen der PM zweimal ausgetauscht wurde, davon sind fünf Projekte gescheitert.
%
% Auswertung Alter und Tenuer PM
%
\newline Das Alter (PMAge2) und die Dienstjahre (PMTen2) des Projektmanagers sind Proxyvariablen für die Lebens- und Berufserfahrung sowie die Kenntnisse der Bühlerwelt. Erfahrenere (ältere) sowie langjährige Mitarbeitend müssten mehr Erfolg im Projektmanagement haben, da sie mehr Praxiserfahrung mit der Bühler-Welt einerseits und dem Projektmanagement anderseits haben sollten. 
\begin{table}[H]
	\centering
	\caption{Durchschnittswerte PMAge und PMTen in Jahren}
	\begin{tabular}{lrr}
		\textbf{Success} & \multicolumn{1}{l}{\textbf{Age}} & \multicolumn{1}{l}{\textbf{Ten}} \\\hline
		FALSE & 41.1 & 12.4 \\
		TRUE  & 39.5 & 11.7 \\
	\end{tabular}%
	\label{ageten}%
\end{table}%
Die Durchschnittswerte des Alter und der Dienstjahre in der Tabelle \ref{ageten} sind für Fail-Projekte ein wenige höher als für Success-Projekte.
% Table generated by Excel2LaTeX from sheet 'fagecat'
\begin{table}[htbp]
	\centering
	\caption{Erfolgsquote pro Alterskategorie}
	\begin{tabular}{lrrrrrr}
		\textbf{Cat\_age} & \multicolumn{1}{l}{\textbf{Erfolgsquote}} & \multicolumn{1}{l}{\textbf{Success}} & \multicolumn{1}{l}{\textbf{Fail}} & \multicolumn{1}{l}{\textbf{Fail [\%]}} & \multicolumn{1}{l}{\textbf{Total}} & \multicolumn{1}{l}{\textbf{Total [\%]}} \\\hline
		20-25 & 2.5   & 15    & 6     & 28.6\% & 21    & 2.2\% \\
		26-30 & 2.3   & 130   & 57    & 30.5\% & 187   & 19.4\% \\
		31-35 & 3.0   & 130   & 44    & 25.3\% & 174   & 18.0\% \\
		36-40 & 2.3   & 97    & 43    & 30.7\% & 140   & 14.5\% \\
		41-45 & \textbf{1.8} & 105   & 57    & 35.2\% & 162   & 16.8\% \\
		46-50 & 2.3   & 65    & 28    & 30.1\% & 93    & 9.6\% \\
		51-55 & \textbf{1.3} & 69    & 52    & 43.0\% & 121   & 12.5\% \\
		56-60 & \textbf{1.4} & 26    & 19    & 42.2\% & 45    & 4.7\% \\
		61-63 & 2.8   & 17    & 6     & 26.1\% & 23    & 2.4\% \\\hline
		\textbf{Total} & \textbf{2.1} & \textbf{654} & \textbf{312} & \textbf{32.3\%} & \textbf{966} & \textbf{100.0\%} \\
	\end{tabular}%
	\label{fagecat}%
\end{table}%
\newline\newline
%
% Auswertung Lead SAS
%
Die Bühler AG unterscheidet zwei Typen von Lead SAS: die Lead SAS für das gesamte Projekt (LeadSASPr) und die Lead SAS für die Projektabwicklung (LeadSASFF). Lead legt in dieser Hinsicht die Verantwortung fest. Da die Verantwortlichkeit auf zwei Gesellschaften aufgeteilt werden kann, wurde zudem eruiert, ob sich die LeadSASFF von der LeadSASPr unterscheidet (LeadSAS.PrFF). Es wird postuliert, dass einige Gesellschaften Projekte erfolgreicher managen. Zudem wird angenommen, dass bei zwei Verantwortungsparteien, die Erfolgsquote höher sein muss, da bei der zugrundeliegende Aufgabenbereich mehr fokussiert werden kann.
% Table frequency Lead SASPr
\begin{table}[H]
	\centering
	\caption{Erfolgsquoten pro LeadSASPr}
	\begin{tabular}{lrrrrrr}
		\textbf{LeadSASPr} & \multicolumn{1}{l}{\textbf{Erfolgsquote}} & \multicolumn{1}{l}{\textbf{Success}} & \multicolumn{1}{l}{\textbf{Fail}} & \multicolumn{1}{l}{\textbf{Fail [\%]}} & \multicolumn{1}{l}{\textbf{Total}} & \multicolumn{1}{l}{\textbf{Total [\textbackslash{}\%]}} \\\hline
		BJHB  & 0.6   & 10    & 17    & 63.0\% & 27    & 2.8\% \\
		\textbf{BBS}   & 0.8   & 27    & 35    & 56.5\% & 62    & 6.4\% \\
		\textbf{BMIL}  & 0.9   & 13    & 15    & 53.6\% & 28    & 2.9\% \\
		BRAL  & 0.9   & 12    & 13    & 52.0\% & 25    & 2.6\% \\
		\textbf{BLOA}  & 1.0   & 19    & 19    & 50.0\% & 38    & 3.9\% \\
		BPRI  & 1.0   & 2     & 2     & 50.0\% & 4     & 0.4\% \\
		\textbf{BBAR}  & 1.1   & 8     & 7     & 46.7\% & 15    & 1.6\% \\
		BJOI  & 1.2   & 19    & 16    & 45.7\% & 35    & 3.6\% \\
		BSSE  & 1.2   & 27    & 22    & 44.9\% & 49    & 5.1\% \\
		BMIN  & 1.2   & 31    & 25    & 44.6\% & 56    & 5.8\% \\
		BBIN  & 1.3   & 20    & 15    & 42.9\% & 35    & 3.6\% \\
		BPAR  & 1.3   & 16    & 12    & 42.9\% & 28    & 2.9\% \\
		BBAI  & 2.0   & 2     & 1     & 33.3\% & 3     & 0.3\% \\
		BMAD  & 2.1   & 15    & 7     & 31.8\% & 22    & 2.3\% \\
		BMEX  & 2.5   & 5     & 2     & 28.6\% & 7     & 0.7\% \\
		BLON  & 3.0   & 3     & 1     & 25.0\% & 4     & 0.4\% \\
		BUZ   & 3.6   & 310   & 86    & 21.7\% & 396   & 41.0\% \\
		BLOC  & 5.5   & 11    & 2     & 15.4\% & 13    & 1.3\% \\
		BYOK  & 5.5   & 11    & 2     & 15.4\% & 13    & 1.3\% \\
		BCHI  & 5.7   & 34    & 6     & 15.0\% & 40    & 4.1\% \\
		BBAN  & 7.3   & 51    & 7     & 12.1\% & 58    & 6.0\% \\
		BDAG  & \multicolumn{1}{r}{NA} & 6     & 0     & 0.0\% & 6     & 0.6\% \\
		BTEH  & \multicolumn{1}{r}{NA} & 2     & 0     & 0.0\% & 2     & 0.2\% \\\hline
		\textbf{Total} &       & \textbf{654} & \textbf{312} & \textbf{32.3\%} & \textbf{966} & \textbf{100.0\%} \\
	\end{tabular}%
	\label{fleadsas}%
\end{table}%
Der Vergleich Erfolgsquoten pro SAS der Tabelle \ref{fleadsas} zeigt, dass die europäischen Gesellschaften im Vergleich tiefere Erfolgsraten haben. BJHB, BBS, BMIL und BRAL, die zusammen 15\% aller Projekte abwickeln, haben eine unterdurchschnittliche Erfolgsbilanz.
%Tabelle LeadSAS.PrFF
\begin{table}[H]
	\centering
	\caption{Häufigkeit LeadSAS.PrFF}
	\begin{tabular}{lrrrrr}
		\textbf{LeadSAS.PrFF} & \multicolumn{1}{l}{\textbf{Success}} & \multicolumn{1}{l}{\textbf{Fail}} & \multicolumn{1}{l}{\textbf{Fail [\%]}} & \multicolumn{1}{l}{\textbf{Total}} & \multicolumn{1}{l}{\textbf{Total [\%]}}
		\\\hline
		identisch    & 569   & 296   & 34.2\% & 865 & 89.5\% \\
		verschieden   & 85    & 16    & 15.8\% & 101 & 10.5\% \\\hline
		\textbf{Total} & \textbf{654} & \textbf{312} &       & \textbf{966} \\
	\end{tabular}%
	\label{fleadsasprff}%
\end{table}%
Die Resultate der Tabelle \ref{fleadsasprff} implizieren, dass der überwiegende Anteil der Projekte eine Verantwortungspartei hatte. Von den 101 Projekten mit geteilter Projektverantwortung wurden lediglich 16\% mit einem prozentualen DB1Act unter Budget abgeschlossen. Das Ergebnis kann folglich als Indiz für die Hypothese gewertet werden.
%%
%%
%%Zeit
%%
%%
\newline\newline\textbf{Zeit:} Die Beurteilung des Zeitmanagement hängt von der Einhaltung des vereinbarten Liefertermins ab. Mehrkosten und Zeitverzug gehen oftmals einher, weshalb unterstellt wird, dass Fail-Projekte den vereinbarten Projektabschluss nicht einhalten konnten. Ferner soll ergründet werden, ab welchem Zeitpunkt respektive Milestone der Zeitverzug üblicherweise eintritt. 
\begin{table}[H]
	\centering
	\caption{Projektlaufzeiten und Zeitverzug [in Monaten]}
	\begin{tabular}{lrrrrrrr}
		\textbf{Success} & \multicolumn{1}{l}{\textbf{Base}} & \multicolumn{1}{l}{\textbf{Act}} & \multicolumn{1}{l}{\textbf{Delay}} & \multicolumn{1}{l}{\textbf{MS2}} & \multicolumn{1}{l}{\textbf{MS8}} & \multicolumn{1}{l}{\textbf{MS10}} & \multicolumn{1}{l}{\textbf{MS11}} \\ \hline
		TRUE  & 11.9  & 17.3  & -5.4  & -0.1  & -2.0  & -5.0  & -5.5 \\
		FALSE & 11.4  & 18.7  & -7.2  & -0.2  & -1.7  & -5.7  & -7.3 \\
	\end{tabular}%
	\label{mtime}%
\end{table}%
Die durchschnittliche budgetierte Projektlaufzeit unterscheidet sich zwischen erfolgreichen und  nicht-erfolgreichen Projekten kaum wohingegen die effektive Projektlaufzeit der Fail-Projekte einen Monat mehr betrug (s. Tabelle \ref{mtime}). Beide Projektgruppen konnten im Durchschnitt den Liefertermin nicht einhalten, wobei die Success-Projekte ca. 2 Monate weniger zeitverzögert waren. Die Termineinhaltung beim MS2 Concept approved bewegt sich im vernachlässigbaren Bereich. Demgegenüber steigt der durchschnittliche Zeitverzug nach MS8 Documented auf zwei, nach MS10 Takeover auf 5-6 Monate und liegt bei Projektabschluss (MS11) zwischen gerundet 6 und 7 Monaten.
\begin{table}[H]
	\centering
	\caption{Anzahl verspäteter ($PrTimeBase-PrTimeAct$) Projekte}
	\begin{tabular}{lrrrrr}
		\textbf{Delay} & \multicolumn{1}{l}{\textbf{Success}} & \multicolumn{1}{l}{\textbf{Fail}} & \multicolumn{1}{l}{\textbf{Fail [\%]}} & \multicolumn{1}{l}{\textbf{Total}} & \multicolumn{1}{l}{\textbf{Total [\%]}} \\ \hline
		FALSE & 139   & 44    & 24.0\% & 183   & 18.9\% \\
		TRUE  & 515   & 268   & 34.2\% & 783   & 81.1\% \\\hline
		\textbf{Total} & \textbf{654} & \textbf{312} & \textbf{32.3\%} & \textbf{966} & \textbf{100.0\%} \\
	\end{tabular}%
	\label{fdelay}%
\end{table}%
Insgesamt wurde der Liefertermin bei 783 Projekten nicht eingehalten (s. Tabelle \ref{fdelay}), wobei das Verhältnis zwischen erfolgreichen und nicht-erfolgreichen zwei zu eins beträgt. Der prozentuale Anteil der Projekte, bei denen die Anlage pünktlich übergeben werden konnte, beträgt 20. 
\begin{table}[H]
	\centering
	\caption{Anzahl zeitverzögerte Projekte pro Milestone [Monate]}
	\begin{tabular}{lrrrr}
		\textbf{Success} & \multicolumn{1}{l}{\textbf{DelayMS2}} & \multicolumn{1}{l}{\textbf{DelayMS8}} & \multicolumn{1}{l}{\textbf{DelayMS10}} & \multicolumn{1}{l}{\textbf{DelayMS11}} \\\hline
		FALSE & 43    & 209   & 275   & 270 \\
		TRUE  & 75    & 408   & 532   & 516 \\\hline
		\textbf{Total} & \textbf{118} & \textbf{717} & \textbf{807} &  \textbf{786}
	\end{tabular}%
	\label{fdelayms}%
\end{table}
Die Tabelle \ref{fdelayms} zeigt, dass die Mehrheit der Projekte bei MS2 noch im Zeitplan agierte und nach MS8 bereits hinter dem vereinbarten Liefertermin lag. Mittels Dummyvariablen pro Milestone wurde ausgewertet, ob sich ein anfängliche Verspätung durch die Projektlaufzeit durchzieht.
% Table generated by Excel2LaTeX from sheet 'time_sukz'
\begin{table}[H]
	\centering
	\caption{Häufigkeit Delay-Projekte}
	\begin{tabular}{lrrrrrr}
		\textbf{Success} & \multicolumn{1}{l}{\textbf{Delay}} & \multicolumn{1}{l}{\textbf{DelayMS2}} & \multicolumn{1}{l}{\textbf{DelayMS8}} & \multicolumn{1}{l}{\textbf{DelayMS10}} & \multicolumn{1}{l}{\textbf{DelayMS11}} & \multicolumn{1}{l}{\textbf{freq}} \\
		FALSE & \multicolumn{1}{l}{TRUE} & 0     & 1     & 1     & 1     & 147 \\
		FALSE & \multicolumn{1}{l}{TRUE} & 0     & 0     & 1     & 1     & 64 \\
		FALSE & \multicolumn{1}{l}{TRUE} & 1     & 1     & 1     & 1     & 33 \\
		FALSE & \multicolumn{1}{l}{TRUE} & 0     & 0     & 0     & 1     & 10 \\
		FALSE & \multicolumn{1}{l}{TRUE} & 0     & 1     & 0     & 1     & 7 \\
		FALSE & \multicolumn{1}{l}{TRUE} & 1     & 0     & 1     & 1     & 5 \\
		FALSE & \multicolumn{1}{l}{TRUE} & 0     & 0     & 0     & 0     & 1 \\
		FALSE & \multicolumn{1}{l}{TRUE} & 1     & 0     & 0     & 1     & 1 \\\hline
		\textbf{Total} &       &       &       &       &       & \textbf{268} \\
		TRUE  & \multicolumn{1}{l}{TRUE} & 0     & 1     & 1     & 1     & 290 \\
		TRUE  & \multicolumn{1}{l}{TRUE} & 0     & 0     & 1     & 1     & 126 \\
		TRUE  & \multicolumn{1}{l}{TRUE} & 1     & 1     & 1     & 1     & 42 \\
		TRUE  & \multicolumn{1}{l}{TRUE} & 0     & 1     & 0     & 1     & 19 \\
		TRUE  & \multicolumn{1}{l}{TRUE} & 1     & 0     & 1     & 1     & 17 \\
		TRUE  & \multicolumn{1}{l}{TRUE} & 0     & 0     & 0     & 1     & 16 \\
		TRUE  & \multicolumn{1}{l}{TRUE} & 1     & 1     & 0     & 1     & 2 \\
		TRUE  & \multicolumn{1}{l}{TRUE} & 0     & 0     & 1     & 0     & 1 \\
		TRUE  & \multicolumn{1}{l}{TRUE} & 1     & 0     & 0     & 0     & 1 \\
		TRUE  & \multicolumn{1}{l}{TRUE} & 1     & 0     & 0     & 1     & 1 \\\hline
		\textbf{Total} &       &       &       &       &       & \textbf{515} \\
	\end{tabular}%
	\label{fdleay}%
\end{table}%
Die meisten Projekte hatten die Eigenschaft, bei einer Verspätung im MS8 ebenso auch im MS11 verspätet zu sein. Die zweithäufigste Gruppe war diejenige, die erstmalig ab dem MS10 den Liefertermin bis zum Projektabschluss nicht mehr einhalten konnte. Erst an dritter Stelle folgt die Gruppe, die bereits ab MS2 bis MS11 hinter dem Zeitplan lag. Auf Basis der Tabellen lassen sich jedoch keine Aussagen machen, ob bei einer Nichteinhaltung des vereinbarten Termin, das Projekt zwangsläufig unter Budget abschliessen wird. 
%%
%%
%% Auswertung SQ
%%
%%
\newline\newline\textbf{SQ:} Die Einflussdeterminanten des SQ-Prozess sind einerseits der Stand im Bezug auf das OR-Budget und die Erfahrung (AMAge2) sowie Dienstjahre (AMTen2) des Verkaufsmanager. Allerdings konnten letztere aufgrund fehlender Datensätze nicht ausgewertet werden. Grundsätzlich wird vermutete, dass ein Budgetdruck im Zeitpunkt des Verkaufsabschlusses, den Verkauf von risikoreicheren Projekten begünstigt. 
% Table generated by Excel2LaTeX from sheet 'sq mean'
\begin{table}[H]
	\centering
	\caption{Budget Gap für Orders Released für BU und Region}
	\begin{tabular}{lrr|rr}
		\textbf{Success} & \multicolumn{1}{l}{\textbf{BU [TCHF]}} & \multicolumn{1}{l}{\textbf{BU [\%]}} & \multicolumn{1}{l}{\textbf{Region [TCHF]}} & \multicolumn{1}{l}{\textbf{Region [\%]}} \\
		FALSE & -7'244.2 & -11.3 & -20'845.5 & -9.3 \\
		TRUE  & -1'207.2 & -4.0  & -14'886.9 & -6.8 \\
	\end{tabular}%
	\label{msq}%
\end{table}%
Die mittlere Abweichung des OR vom Budget des Geschäftseinheit und der Region der Tabelle \ref{msq} waren für Fail-Projekte in absoluten und relativen Grössen höher als für Success-Projekte.  
%%
%%
%% Auswertung Komplexität
%%
%%
\newline\newline\textbf{Komplexität:} Die Komplexität kann anhand unterschiedlicher Dimensionen gemessen werden wie zum Beispiel, dem Inhalt, den Zielen, den Beteiligten oder dem Umfeld. In der nachfolgenden Auswertung wurden die Anzahl Aufträge sowie involvierter Parteien bei den unterschiedlichen Projektphasen und Konsortien als Proxyvariablen für Komplexität untersucht. Die vorherrschende Auffassung bezüglich Komplexität, ist ihr negativer Einfluss auf den Projekterfolg.
%
%Auswertung Konsortium
%
% Table generated by Excel2LaTeX from sheet 'con_part_reg'
\begin{table}[H]
	\centering
	\caption{Anzahl Projekte mit Konsortium pro Region}
	\begin{tabular}{lrrrr}
		\textbf{ConPart} & \multicolumn{1}{l}{\textbf{Region}} & \multicolumn{1}{l}{\textbf{Success}} & \multicolumn{1}{l}{\textbf{Fail}} & \multicolumn{1}{l}{\textbf{Erfolgsquote}} \\\hline
		TRUE  & \multicolumn{1}{l}{East\_Asia} & 1     & 0     & NA \\
		TRUE  & \multicolumn{1}{l}{\textbf{EU}} & 13    & 14    & 0.9 \\
		TRUE  & \multicolumn{1}{l}{MEA\_Afr} & 9     & 7     & 1.3 \\
		TRUE  & \multicolumn{1}{l}{North\_Ame} & 2     & 1     & 2.0 \\
		TRUE  & \multicolumn{1}{l}{\textbf{SAS\_BCHI}} & 17    & 7     & 2.4 \\
		TRUE  & \multicolumn{1}{l}{South\_Ame} & 4     & 2     & 2.0 \\
		TRUE  & \multicolumn{1}{l}{South\_Asia} & 1     & 0     & NA \\\hline
		\textbf{Total} &       & \textbf{47} & \textbf{31} & \textbf{1.5} \\
	\end{tabular}%
	\label{fcons}%
\end{table}%
Die Anzahl Projekte in der Stichprobe, die in einem Konsortium abgewickelt wurden, beträgt 78 , wovon 47 erfolgreich und 31 unter Budget abgeschlossen wurden (s. Tabelle \ref{fcons}). In EU und SAS\_BCHI wurden 51 Projekte im Konsortium abgewickelt. 
% Table generated by Excel2LaTeX from sheet 'cons_par_reg_Ba'
\begin{table}[htbp]
	\centering
	\caption{Häufigkeit Konsortium pro BA für EU, MEA\_Afr und SAS\_BCHI}
	\begin{tabular}{llrrr}
		\textbf{Region} & \textbf{BA} & \multicolumn{1}{l}{\textbf{Success}} & \multicolumn{1}{l}{\textbf{Fail}} & \multicolumn{1}{l}{\textbf{Erfolgsquote}} \\\hline
		EU  & CF    & 5     & 0     & NA \\
		    & DC    & 0     & 1     & 0.0 \\
		    & GL    & 6     & 5     & 1.2 \\
		    & GM    & 1     & 3     & 0.3 \\
		    &VN    & 1     & 5     & 0.2 \\\hline
		\textbf{Total} &  & \textbf{13} & \textbf{14} & \textbf{0.9} \\
			  &  &  & & \\
	    MEA\_Afr &	GL    & 2     & 2     & 1.0 \\
	             & 	GM    & 6     & 5     & 1.2 \\
		         & VN    & 1     & 0     & NA \\\hline
		\textbf{Total} &  & \textbf{9} & \textbf{7} & \textbf{1.3} \\
		& & & & \\
		    SAS\_BCHI & \multicolumn{1}{l}{CF} & 2     & 0     & NA \\
		 & \multicolumn{1}{l}{DC} & 11    & 1     & 11.0 \\
		  & \multicolumn{1}{l}{GM} & 4     & 6     & 0.7 \\\hline
		\textbf{Total} &       &  \textbf{17}   & \textbf{7}     & \textbf{2.4} \\
	\end{tabular}%
	\label{consregba}%
\end{table}%
Die meisten Konsortium-Projekte in Europa stammen vom Geschäftsbereich GL wovon nur jedes zweite erfolgreich zu Ende geführt werden konnte (s. Tabelle \ref{consregba}). Demgegenüber stehen 11 Projekte mit Konsortium in Middle East \& Africa des Geschäftsbereichs GM, wovon ebenso die Hälfte erfolgreich abgeschlossen werden konnte. Die DC-Projekte weisen gegenüber den GM-Projekten in SAS\_BCHI die bessere Erfolgsbilanz aus.
%
%Auswertung Anzahl Aufträge
%
\newline Die Anzahl Aufträge pro Projekt (NoContr) kann ein Indiz für die Anzahl Komponenten der Maschine oder involvierter Parteien sein. 
% Table generated by Excel2LaTeX from sheet 'nocontr'
\begin{table}[H]
	\centering
	\caption{Häufigkeit NoContr}
	\begin{tabular}{lrrrrrr}
		\textbf{NoContr} & \multicolumn{1}{l}{\textbf{Erfolgsquote}} & \multicolumn{1}{l}{\textbf{Success}} & \multicolumn{1}{l}{\textbf{Fail}} & \multicolumn{1}{l}{\textbf{Fail [\%]}} & \multicolumn{1}{l}{\textbf{Total}} & \multicolumn{1}{l}{\textbf{Total [\%]}} \\\hline
		\multicolumn{1}{r}{1} & 2.0   & 449   & 228   & 33.7\% & 677   & 70.1\% \\
		\multicolumn{1}{r}{2} & 2.6   & 136   & 52    & 27.7\% & 188   & 19.5\% \\
		\multicolumn{1}{r}{3} & 1.7   & 38    & 22    & 36.7\% & 60    & 6.2\% \\
		\multicolumn{1}{r}{4} & 3.6   & 18    & 5     & 21.7\% & 23    & 2.4\% \\
		\multicolumn{1}{r}{5} & 4.0   & 4     & 1     & 20.0\% & 5     & 0.5\% \\
		\multicolumn{1}{r}{6} & 3.0   & 6     & 2     & 25.0\% & 8     & 0.8\% \\
		\multicolumn{1}{r}{7} & NA    & 1     & 0     & 0.0\% & 1     & 0.1\% \\
		\multicolumn{1}{r}{8} & NA    & 1     & 0     & 0.0\% & 1     & 0.1\% \\
		\multicolumn{1}{r}{9} & NA    & 1     & 0     & 0.0\% & 1     & 0.1\% \\
		\multicolumn{1}{r}{10} & 0.0   & 0     & 2     & 100.0\% & 2     & 0.2\% \\\hline
		\textbf{Total} & \textbf{2.1} & \textbf{654} & \textbf{312} & \textbf{32.3\%} & \textbf{966} & \textbf{100.0\%} \\
	\end{tabular}%
	\label{fnocontr}%
\end{table}%
In der Stichprobe variiert sie zwischen eins und zehn, wobei die Mehrheit aller Projekte einen einzigen Auftrag pflegte (s. Tabelle \ref{fnocontr}). Weitere 20\% aller Projekte werden mittels zwei Aufträgen abgewickelt. Die Erfolgsquote der zweitgrössten Kategorie liegt leicht über derjenigen von Projekten mit einem Auftrag. Allerdings wird das Verhältnis zwischen Fail- und Success-Projekten der Stichprobe durch Projekte mit einem Vertrag beschrieben.
\newline
%
%Auswertung Anzahl SAS
%

Die Anzahl involvierte SAS (NoSupplSAS) bei der Zulieferung liegt im Bereich null und zehn, wobei null mit Eigenproduktion oder Zulieferung durch Dritte gleichzusetzen ist. In den übrigen Projektphasen sind maximal drei andere Bühler-Gesellschaften involviert. Eine Einzelauswertung pro Projektphase ergibt relativ wenig Aufschluss, weshalb die Häufigkeit der Kombinationen untersucht wurde. 
% Table generated by Excel2LaTeX from sheet 'nosas'
\begin{table}[H]
	\centering
	\caption{Häufigkeit NoSAS-Kombinationen Ausschnitt (eigene Darstellung)}
	\begin{tabular}{lrrrrrrrr}
		\textbf{NoSupplSAS} & \multicolumn{1}{l}{\textbf{MS}} & \multicolumn{1}{l}{\textbf{ME}} & \multicolumn{1}{l}{\textbf{PA}} & \multicolumn{1}{l}{\textbf{IS}} & \multicolumn{1}{l}{\textbf{Success}} & \multicolumn{1}{l}{\textbf{Fail}} & \multicolumn{1}{l}{\textbf{Total}} & \multicolumn{1}{l}{\textbf{Erfolgsquote}} \\ \hline
		\multicolumn{1}{r}{0} & 0     & 0     & 0     & 0     & 180   & 89    & 269 & 2.0 \\
		\multicolumn{1}{r}{1} & 1     & 0     & 0     & 0     & 73    & 26    & 99  & 2.8 \\
		\multicolumn{1}{r}{2} & 2     & 0     & 0     & 0     & 38    & 13    & 51  & 2.9 \\
		\multicolumn{1}{r}{2} & 2     & 1     & 0     & 0     & 23    & 13    & 36  & 1.8 \\
		\multicolumn{1}{r}{1} & 1     & 1     & 1     & 1     & 15    & 12    & 27  & 1.3 \\
		\multicolumn{1}{r}{1} & 1     & 1     & 0     & 0     & 17    & 11    & 28  & 1.5 \\ \hline
		\textbf{Total} &       &       &       &       & \textbf{346} & \textbf{164} & \textbf{510} & 2.1 \\
	\end{tabular}%
	\label{fnosas}%
\end{table}%
Die obige Tabelle \ref{fnosas} zeigt, dass die Eigenfertigung respektive die Zusammenarbeit mit Drittlieferanten während allen Projektphasen das häufigste Charakteristika war. Es fällt auf, dass sich mehr als die Hälfte der Datenpunkte im Bereich mit maximal zwei involvierten SAS-Gesellschaften während der gesamten Projektlaufzeit konzentriert. Aufgrund der Erfolgsquoten kann postuliert werden, dass eine Zusammenarbeit mit mindestens einer weitere Partei im MS, den Projekterfolg begünstigen könnte.
\newpage
\subsection{Kritische Würdigung der Ergebnisse}
\newpage	
	


\newpage
%Theorie und Konzeptteil
\chapter{Ergebnisse}\label{sec: Ergeb}
Die Ergebnisse der finanziellen Analyse und der Untersuchung der Einflussfaktoren werden getrennt dargestellt. Die untersuchte Stichprobe enthält 966 Projekte, wovon 654 erfolgreich abgeschlossen wurden.
%%Ergenbnisse pro Kategorie: Hypothese Nennen, Kommentierung, Ergebnisse präsentieren, Tabellen etc
%%Rahmenbedingungen: Region, BA und allenfalls BU: MS zu feine Gliederung, EquLoc too maany Einzelwerte, CuNo too many Einzelwerte
%%Sales & Quotation: Budget Druck, Rest ging nicht
%%Fullfillment: PMNo to many Einzelwerte, PMAge gemäss Kategorie, PMTen (EVTL.), NoPM not, da Analyse zeigt dass nur einmal 3 PM gab, PMChange,
%%Fulfillment: LeadSASPr, LeadSASPrFF, LeadSAS.PrFF,  NoLeadSASFF nicht, da zu wenig Fälle
%%Fulfillment: CostFCajd alle und HOM
%%Kosten: Tabelle: finanzielle Performanceanalyse: Umsatz, Marge, Erklärung, Erkenntnisse (Bud, Ac, Dev)
%%Kosten: Tabelle: finanzielle Performancenaalyse: Kosten(Bud, Act, Abw und Abw Schlüssel)
%%Kosten: evtl. Tabelle: pro Umsatzkategorie Anz. Cost Dev und: mit wenig, viel Verlust
%%Kosten: Mittelwerte für alle Variablen
%%Zeit: alle, aber ohne Delay
%%Komplexität: alle 

%Tabelle Übersicht Stichprobe
\begin{table}[htbp]
	\centering
	\caption{Übersicht Stichprobe}
	\begin{tabular} {l|r|r}
		\textbf{Erfolgskriterium} & \textbf{absolut} & \textbf{relativ} \\\hline
		SUCCESS & 654 & 68\% \\
		FAIL & 312 & 32\% \\\hline
		\textbf{Total} & \textbf{966} & \textbf{100\%} \\
	\end{tabular}
\end{table}
\section{Finanzielle Performance Analyse}
Zur Bewertung der finanziellen Performance wurden drei verschiedene Auswertungen gemacht: Abweichung Act-Bud, Zusammensetzung der Kosten sowie eine Auswertung pro Umsatzkategorie. Das Ziel besteht darin, den finanziellen Verlust in Abhängigkeit des Erfolgskriterium (DB1BudDev) zu quantifizieren. 
%Bud TO Cost DB1
\begin{table}[H]
	\centering
	\caption{Budget TO, Cost und DB1 [TCHF]}
	\begin{tabular}{lrrrr}
		\textbf{Erfolgskriterium} & \textbf{TO Bud} & \textbf{Cost Bud} &
		\textbf{DB1 Bud} & \textbf{DB1 Bud [\%]} \\\hline
		SUCCESS & 1'552'450 & -1'156'598 & 395'851 & 25.5\% \\
		FAIL  & 618'013 & -465'066 & 152'947 & 24.7\% \\\hline
		\textbf{Total} & \textbf{2'170'463} & \textbf{-1'621'664} & \textbf{548'799} & \textbf{25.3\%}\\
	\end{tabular}%
	\label{bud}%
\end{table}%
%Act TO Cost DB+
\begin{table}[H]
	\centering
	\caption{Actuals TO, Cost und DB1 [TCHF]}
	\begin{tabular}{lrrrr}
		\textbf{Erfolgskriterium} & \textbf{TO Act} & \textbf{Cost Act} & \textbf{DB1 Act}&
		\textbf{DB1 Act [\%]} \\\hline
		SUCCESS & 1'560'001 & -1'041'728 & 518'273 & 33.2\% \\
		FAIL  & 631'346 & -526'600 & 104'746 & 16.6\% \\\hline
		\textbf{Total} & \textbf{2'191'347} & \textbf{-1'568'328} & \textbf{623'018} & \textbf{28.4\%} \\
	\end{tabular}
	\label{act}%
\end{table}%
%Deviation TO Cost DB1
\begin{table}[H]
	\centering
	\caption{Abweichungen ($Act-Bud$) TO, Cost und DB1 [TCHF] }
	\begin{tabular}{lrrrr}
		\textbf{Erfolgskriterium} & \textbf{TO} & \textbf{Cost} & \textbf{DB1}&
		\textbf{DB1 [\%]} \\\hline
		SUCCESS & 7'551 & 114'870 & 122'421 & 7.7\% \\
		FAIL  & 13'333 & -61'534 & -48'202 & -8.2\% \\\hline
		\textbf{Total} & \textbf{20'884} & \textbf{53'336} & \textbf{74'220} & \textbf{3.1\%} \\
	\end{tabular}
	\label{Abw}%
\end{table}%
Die Tabelle \ref{Abw} wurde mittels Tabellen \ref{bud} und \ref{act} berechnet und zeigt, dass der realisierte Umsatz höher war als budgetiert wurde. Diese Abweichung kann auf Zusatzverkäufe oder die Verrechnung allfälliger Mehrkosten an den Kunden zurückgeführt werden. Der kumulierte DB1 der Fail-Projekte lag 48 Mio. CHF (-32\%) unter dem Budget und bei den Success-Projekten 122 Mio. CHF über dem Budget. Die positive Abweichung der Istkosten der Success-Projekte kann mittels der realisierten Kostenreserve, die üblicherweise pro Projekt einkalkuliert wird und je nach Geschäftsbereich zwischen 4\% und 9\% beträgt, zurückgeführt werden. Wenn die Kostenreserve aufgebraucht wird, resultieren Mehrkosten und die Kostenabweichung wird negativ. Da die Reserve in dieser Betrachtung nicht ersichtlich ist, wäre die effektive Differenz für Fail-Projekte (Success-Projekte) tiefer (höher). Die realisierte Marge über alle Success-Projekte beträgt 33\% und liegt 7.7\% über der budgetierten Marge von 25.4\%. Demgegenüber beträgt der DB1Act der Fail-Projekte 16.6\% und liegt 8.2\% unter dem DB1 Bud von 24.7\%. Nachfolgend wird die Kostenperformance näher betrachtet, um herauszufinden, bei welcher Projektphase die Mehrkosten entstehen.
%Aufschlüsselung Kostenabweichung gemäss Projektphase
\begin{table}[H]
	\centering
	\caption{Aufschlüsselung der Kosten nach der Projektphase [TCHF]}
	\begin{tabular}{lrrrrr|r}
		\textbf{Erfolgskriterium} & \multicolumn{1}{l}{\textbf{Total Cost}} & \multicolumn{1}{l}{\textbf{MS}} & \multicolumn{1}{l}{\textbf{ME}} & \multicolumn{1}{l}{\textbf{PA}} & \multicolumn{1}{l}{\textbf{IS}} & \multicolumn{1}{l}{\textbf{Summe}} \\\hline
		SUCCESS & 114'870 & 47'615 & -2'159 & -908  & -7'114 & 37'434 \\
		FAIL  & -61'534 & -20'253 & -12'721 & -7'220 & -22'053 & -62'247 \\\hline
		\textbf{ Total} & \textbf{53'336} & \textbf{ 27'363} & \textbf{ -14'880} & 
		\textbf{ -8'128} & \textbf{ -29'167} & \textbf{ -24'812} \\
	\end{tabular}%
	\label{stocostdb1dev}%
\end{table}%
Die Aufschlüsselung der Kostenabweichung der Tabelle \ref{stocostdb1dev} zeigt, dass die Installationsphase sowohl der erfolgreichen als auch der nicht-erfolgreich Projekte mit Mehrkosten verbunden ist. Die negative Kostenabweichung der Fail-Projekte kann zu einem Drittel auf die IS- und zu einem weiteren Drittel auf die MS-Kosten zurückgeführt werden. Bei den Success-Projekten kann ein gewisser 'Verlust'-Kompensationseffekt durch die positive Kostenabweichung der MS-Kosten festgestellt werden. Die Kostendifferenz zwischen Act und Bud der Fail-Projekte kann fast vollständig durch Summe der Kostenabweichungen der vier Projektphasen MS, ME, PA und IS erklärt werden. Der Unterschied zu 'Total Cost' ist auf fehlende Abbildung der anderen Kostenarten der Projektstruktur zurückzuführen. Dieser Effekt ist bei den Success-Projekten ebenfalls sichtbar und kann zu einem Teil auf die Realisation des Kostenpuffers zurückgeführt werden.
\newline
Als Ergänzung wurde versucht zu eruieren, von welchem Projekttyp in Bezug auf das Umsatzvolumen die Margeneinbusse der Fail-Projekte stammt. Dazu wurde die Häufigkeit und der DB1 Abweichung pro Umsatzkategorie ausgewertet.
% Table generated by Excel2LaTeX from sheet 'TOBud_cat'
\begin{table}[htpb]
	\centering
	\caption{DB1 und Häufigkeit pro Umsatzkategorie (TOBud\_Cat) [TCHF]}
	\begin{tabular}{lrcrrrr}
		\multicolumn{1}{l}{\textbf{Kat.}} & \multicolumn{1}{l}{\textbf{TOBud\_Cat}} & \multicolumn{1}{c}{\textbf{ Anz.}} & \multicolumn{1}{l}{\textbf{DB1 Act}} & \multicolumn{1}{l}{\textbf{DB1 Bud }} & \multicolumn{1}{l}{\textbf{DB1BudDevabs}} &  \multicolumn{1}{c}{\textbf{\%}}\\\hline
		\multicolumn{1}{r}{1} & \multicolumn{1}{l}{[13.2,500)} & 54    & 4'508 & 6'181 & -1'672 & -27\% \\
		\multicolumn{1}{r}{2} & \multicolumn{1}{l}{[500,1e+03)} & 87    & 11'956 & 18'167 & -6'210 & -34\% \\
		\multicolumn{1}{r}{3} & \multicolumn{1}{l}{[1e+03,1.5e+03)} & 54    & 13'510 & 17'476 & -3'966 & -23\% \\
		\multicolumn{1}{r}{4} & \multicolumn{1}{l}{[1.5e+03,2e+03)} & 32    & 11'106 & 14'928 & -3'822 & -26\% \\
		\multicolumn{1}{r}{5} & \multicolumn{1}{l}{[2e+03,2.5e+03)} & 17    & 6'286 & 9'447 & -3'161 & -33\% \\
		\multicolumn{1}{r}{6} & \multicolumn{1}{l}{[2.5e+03,3e+03)} & 12    & 6'987 & 8'276 & -1'289 & -16\% \\
		\multicolumn{1}{r}{7} & \multicolumn{1}{l}{[3e+03,3.5e+03)} & 8     & 5'419 & 6'778 & -1'359 & -20\% \\
		\multicolumn{1}{r}{8} & \multicolumn{1}{l}{[3.5e+03,4e+03)} & 8     & 4'338 & 6'445 & -2'107 & -33\% \\
		\multicolumn{1}{r}{9} & \multicolumn{1}{l}{[4e+03,4.5e+03)} & 4     & 3'191 & 4'551 & -1'360 & -30\% \\
		\multicolumn{1}{r}{10} & \multicolumn{1}{l}{[4.5e+03,5e+03)} & 6     & 2'686 & 4'349 & -1'662 & -38\% \\
		\multicolumn{1}{r}{11} & \multicolumn{1}{l}{[5e+03,1e+04)} & 23    & 23'602 & 35'644 & -12'043 & -34\% \\
		\multicolumn{1}{r}{12} & \multicolumn{1}{l}{[1e+04,3.42e+04)} & 7     & 11'155 & 20'706 & -9'551 & -46\% \\\hline
		\textbf{Total} &       &    \textbf{312}   & \textbf{104'746} & \textbf{152'947} & \textbf{-48'202} & \textbf{-32\%} \\
	\end{tabular}%
	\label{tab:ftobudcat}%
\end{table}%
\newline In der Tabelle \ref{tab:ftobudcat} wird ersichtlich, dass ein Viertel des der DB1-Abweichung auf 23 Projekte mit einem Umsatzvolumen zwischen 5 und 10 Mio. CHF und 20\% auf 7 Projekte mit einem Umsatzvolumen von mehr als 10 Mio. CHF zurückzuführen ist. Die drittgrösste Abweichung stammt von der Umsatzkategorie mit den meisten Projekten.
%%%%%%%%%%%%%%%%%%%%%%%%%%%%%%%%%%%%%%%%%%%%%%%%%%%%%%%%%%%%%%%%%%%%%%%%%
%%%%%%%%%%%%%%%%%%%%%%%%%%%%%%%%%%%%%%%%%%%%%%%%%%%%%%%%%%%%%%%%%%%%%%%%%
%Auswertung Erfolgsfaktoren
%%%%%%%%%%%%%%%%%%%%%%%%%%%%%%%%%%%%%%%%%%%%%%%%%%%%%%%%%%%%%%%%%%%%%%%%%
%%%%%%%%%%%%%%%%%%%%%%%%%%%%%%%%%%%%%%%%%%%%%%%%%%%%%%%%%%%%%%%%%%%%%%%%%
\section{Erfolgsfaktoren}
In diesem Kapitel werden die Ergebnisse pro Variablenkategorie sowie mögliche Erklärungsansätze erläutert. Mittels Histogrammen, Häufigkeitstabellen und Mittelwerten wurde versucht, die Charakteristiken vergangener Fail-Projekte zu ergründen. Die Stichprobe wurde hierfür gemäss Erfolgskriterium in zwei Datensets unterteilt. Zur Evaluation von kategorialen Variablen wurde ein weiteres Kriterium, die Erfolgsquote hinzugezogen, um beispielsweise Geschäftsbereiche oder Region untereinander vergleichen zu können.
\newline
\begin{equation}
\text{Erfolgsquote} = \frac{\text{{Anzahl Success-Projekte}}}{\text{Anzahl Fail-Projekte}}
\end{equation}

\paragraph{Rahmenbedingungen}
Die Analyse der Rahmenbedingungen eines Projekts geben Hinweise darauf, in welchen Geschäftsbereichen und Regionen und mit welchen Kunden nicht-erfolgreiche gemäss dem Erfolgskriterium realisiert wurden. Da die Bühler AG in einer Matrix-Organisation organsiert ist, wurde nebst den Einzelauswertungen für die Region und die Business Area, der Regionen-BA Split für die Häufigkeit der Success- und Fail-Projekte erstellt.
%Tabelle Auswertung Häufigkeit pro Region
\begin{table}[H]
	\centering
	\caption{Erfolgsquote und Häufigkeitsverteilung pro Region}
	\begin{tabular}{lrrrrrr}
		\textbf{Region} & \multicolumn{1}{l}{\textbf{Erfolgsquote}} & \multicolumn{1}{l}{\textbf{Success}} & \multicolumn{1}{l}{\textbf{Fail}} & \multicolumn{1}{l}{\textbf{Fail [\%]}} & \multicolumn{1}{l}{\textbf{Total}} & \multicolumn{1}{l}{\textbf{Total [\%]}} \\ \hline
		East\_Asia & 6.7   & 20    & 3     & 13.0\% & 23    & 2.4\% \\
		EU    & \textbf{1.7}   & 240   & 145   & 37.7\% & 385   & 39.9\% \\
		MEA\_Afr & 2.7   & 112   & 42    & 27.3\% & 154   & 15.9\% \\
		North\_Ame & \textbf{1.4}   & 54    & 38    & 41.3\% & 92    & 9.5\% \\
		SAS\_BCHI & 2.9   & 119   & 41    & 25.6\% & 160   & 16.6\% \\
		South\_Ame & 1.9   & 58    & 31    & 34.8\% & 89    & 9.2\% \\
		South\_Asia & 4.3   & 51    & 12    & 19.0\% & 63    & 6.5\% \\ \hline
		\textbf{Total} & \textbf{2.1} & \textbf{654} & \textbf{312} & \textbf{32.3\%} & \textbf{966} & \textbf{100.0\%} \\
	\end{tabular}%
	\label{freg}%
\end{table}% 
Die Ergebnisse der Tabellen \ref{freg} und \ref{fba} auf der nachfolgenden Seite reflektieren die Tatsache, dass Europa der grösste Absatzmarkt und GM die grösste Business Area der Bühler AG ist. Die niedrigste Erfolgsquote hat NAM als viertgrösste Region (in Abhängigkeit der Anzahl Projekte), gefolgt von Europa. Die Anzahl der Fail-Projekte in den Regionen EU, MEA und SAS\_BCHI beträgt 73\% ($(145+42+41)/312$), weshalb die Erfolgsquote von allen Projekten hauptsächlich durch diese drei Regionen bestimmt wird. Die kleinsten Regionen haben die besten Erfolgsquoten. Die Geschäftsbereichen CF, VN, GL und GM ($(42+68+32+122)/312$) umfassen zusammen 84\% aller Fail-Projekte, wobei die drei letzt genannten zugleich die niedrigsten Erfolgsquoten ausweisen. Die Anzahl untersuchter Projekte der letzten drei Jahre der Geschäftsbereiche CF und VN ist faktisch identisch, allerdings weist VN eine viel tiefere Erfolgsquote aus als CF.
%Auswertung Häufigkeit und Erfolgsquote pro Region
\begin{table}[H]
	\centering
	\caption{Erfolgsquote und Häufigkeitsverteilung pro Geschäftsbereich}
	\begin{tabular}{lrrrrrr}
		\textbf{BA}   & \multicolumn{1}{l}{\textbf{Erfolgsquote}} & \multicolumn{1}{l}{\textbf{Success}} & \multicolumn{1}{l}{\textbf{Fail}} & \multicolumn{1}{l}{\textbf{Fail [\%]}} & \multicolumn{1}{l}{\textbf{Total}} & \multicolumn{1}{l}{\textbf{Total [\%]}} \\ \hline
		CF    & 2.8   & 118   & 42    & 26.3\% & 160   & 16.6\% \\
		DC    & 5.6   & 96    & 17    & 15.0\% & 113   & 11.7\% \\
		GD    & 2.3   & 7     & 3     & 30.0\% & 10    & 1.0\% \\
		GL    & \textbf{1.2}  & 39    & 32    & 45.1\% & 71    & 7.3\% \\
		GM    & \textbf{1.9}   & 226   & 122   & 35.1\% & 348   & 36.0\% \\
		LO    & 1.4   & 30    & 21    & 41.2\% & 51    & 5.3\% \\
		SR    & 5.0   & 35    & 7     & 16.7\% & 42    & 4.3\% \\
		TP    & NA      & 8     & 0     & 0.0\% & 8     & 0.8\% \\
		VN    & \textbf{1.4}   & 95    & 68    & 41.7\% & 163   & 16.9\% \\\hline
		\textbf{Total } & \textbf{2.1} & \textbf{654} & \textbf{312} & \textbf{32.3\%} & \textbf{966} & \textbf{100.0\%} \\
	\end{tabular}%
	\label{fba}%
\end{table}%
Bei der Auswertung der Geschäftseinheiten auf Basis der Tabelle \ref{fbabu} für GL, GM und VN, konnte festgestellt werden, dass die kleineren BU's von GM eine verhältnismässig tiefe Erfolgsquote hatten. Dennoch wird das Verhältnis zwischen erfolgreichen und nicht-erfolgreichen Projekten fast ausschliesslich durch IM, die grösste BU in GM bestimmt. In der Business Area VN, sind die Erfolgsquoten mit Ausnahme von PN und OL grundsätzlich tief. Die Business Unit Grain Storage determiniert die Geschäftsbereichserfolgsquote von GL.
%Auswertung BU und BA
\begin{table}[htpb]
	\centering
	\caption{Erfolgsquote und Häufigkeitsverteilung pro Geschäftseinheit für GL, GM und VN}
	\begin{tabular}{llrrrrr}
		\textbf{BA} & \textbf{BU} & \multicolumn{1}{l}{\textbf{Erfolgsquote}} & \multicolumn{1}{l}{\textbf{Success}} & \multicolumn{1}{l}{\textbf{Fail}} & \multicolumn{1}{l}{\textbf{Fail [\%]}} & \multicolumn{1}{l}{\textbf{Total}} \\\hline
		GL    & GC    & NA    & 1     & 0     & 0.0\% & 1 \\
		GL    & GS    & 1.2   & 36    & 29    & 44.6\% & 65 \\
		GL    & MT    & 0.7   & 2     & 3     & 60.0\% & 5 \\\hline
		\textbf{GL} &  & \textbf{1.2} & \textbf{39} &\textbf{32} & \textbf{45.1\%} & \textbf{71}\\
		&       &       &       &       &        &   \\
		GM    & BA    & 1.5   & 17    & 11    & 39.3\% & 28 \\
		GM    & BR    & 0.9   & 12    & 14    & 53.8\% & 26 \\
		GM    & IM    & 2.1   & 185   & 87    & 32.0\% & 272 \\
		GM    & SM    & 1.2   & 12    & 10    & 45.5\% & 22 \\\hline
		\textbf{GM} &  & \textbf{1.9} & \textbf{226} &\textbf{122} & \textbf{35.1\%} & \textbf{348}\\
		&       &       &       &       &        &   \\
		VN    & AG    & 0.8   & 11    & 14    & 56.0\% & 25 \\
		VN    & FE    & 1.2   & 27    & 22    & 44.9\% & 49 \\
		VN    & NU    & 1.3   & 27    & 21    & 43.8\% & 48 \\
		VN    & OL    & 2.0   & 6     & 3     & 33.3\% & 9 \\
		VN    & PN    & 3.0   & 24    & 8     & 25.0\% & 32 \\\hline
		\textbf{VN} &  & \textbf{1.4} & \textbf{95} &\textbf{68} & \textbf{41.7\%} & \textbf{163}\\
	\end{tabular}%
	\label{fbabu}%
\end{table}%
\newline Im Regionen-BA Split der Tabelle \ref{tab:fregba} sind für diejenigen Regionen mit den niedrigsten Erfolgsquoten, EU und NAM jene BA's mit den niedrigsten Erfolgsquoten zu finden. Die Kombination EU-GM, EU-GL, EU-VN mit den tiefen Erfolgsquoten machen knapp 30\% ($(50+23+26)/312$) aller Fail-Projekte aus. Die niedrige Erfolgsquote von NAM stammt vor allem aus VN- und GM-Projekten, wobei VN noch vor GM weniger gut abschneidet.
% Auswertung BA-Region
\begin{table}[H]
	\centering
	\caption{Erfolgsquote und Häufigkeitsverteilung Regionen-BA für EU und North\_Ame}
	\begin{tabular}{llrrrrr}
		\textbf{Region} & \textbf{BA}    & \multicolumn{1}{l}{\textbf{Erfolgsquote}} & \multicolumn{1}{l}{\textbf{Success}} & \multicolumn{1}{l}{\textbf{Fail}} & \multicolumn{1}{l}{\textbf{Fail [\%]}} & \multicolumn{1}{l}{\textbf{Total}} \\\hline
		EU    & CF    & 1.9   & 58    & 31    & 34.8\% & 89 \\
		EU    & DC    & 5.0   & 45    & 9     & 16.7\% & 54 \\
		EU    & GD    & NA    & 2     & 0     & 0.0\% & 2 \\
		EU    & \textbf{GL}    & 1.0   & 24    & 23    & 48.9\% & 47 \\
		EU    & \textbf{GM}  & 1.2   & 58    & 50    & 46.3\% & 108 \\
		EU    & LO    & 2.5   & 10    & 4     & 28.6\% & 14 \\
		EU    & SR    & 2.5   & 5     & 2     & 28.6\% & 7 \\
		EU    & \textbf{VN}     & 1.5   & 38    & 26    & 40.6\% & 64 \\\hline
		North\_Ame & CF    & 2.0   & 10    & 5     & 33.3\% & 15 \\
		North\_Ame & DC    & 1.0   & 2     & 2     & 50.0\% & 4 \\
		North\_Ame & GL    & 1.0   & 1     & 1     & 50.0\% & 2 \\
		North\_Ame & \textbf{GM}   & 1.5   & 24    & 16    & 40.0\% & 40 \\
		North\_Ame & LO    & 4.0   & 4     & 1     & 20.0\% & 5 \\
		North\_Ame & SR    & 1.0   & 1     & 1     & 50.0\% & 2 \\
		North\_Ame & \textbf{VN}  & 1.0   & 12    & 12    & 50.0\% & 24 \\
	\end{tabular}%
	\label{tab:fregba}%
\end{table}%
Zusammenfassend lässt sich aussagen, dass ungefähr 60\% ($(122+68)/312$ respektive $(145+38)/312$) der Fail-Projekte entweder in den Geschäftsbereichen VN und GM oder in den Regionen EU und NAM liegen. Zudem wird die Erfolgsquote aller Projekte zu 30\% durch europäische Projekte von den Geschäftsbereichen GM, GL und VN bestimmt wird.
%
%Analyse der Kosten
%
\paragraph{Kosten} Das Umsatzvolumen (TOBud) soll Aufschluss über die Grösse und Wichtigkeit eines Projekts geben. Die zugrundeliegende Prämisse postuliert, dass Projekte mit höherem Umsatzvolumen risikoreicher sind und deshalb eher unter Budget beendet werden. Die Gegenhypothese unterstellt, dass grössere Projekte (hoher TOBud) relativ mehr Beachtung erhalten, da sie den Erfolg eines Geschäftsbereich respektive einer Region mehr beeinflussen als kleinere Projekte. Deshalb liege der Fokus auf der Einhaltung der Budgetvorgaben.
%
%Histogram of TOBud_cat
%
\begin{figure}[H]
	\centering
	\includegraphics[width=5cm]{test.pdf}
	\caption{Histogram Umsatzkategorie}
	\label{fig: htobudcat}
	\text{1	= [13.2,500), 2 = [500,1e+03), 3 = [1e+03,1.5e+03), 4 = [1.5e+03,2e+03), 5 = [2e+03,2.5e+03)}
	\text{6 =	[2.5e+03,3e+03), 7 = [3e+03,3.5e+03), 8 = [3.5e+03,4e+03), 9 = [4e+03,4.5e+03)}
	\text{10 = [4.5e+03,5e+03), 11 = [5e+03,1e+04), 12 = [1e+04,3.42e+04)}
\end{figure}
Die Verteilung Umsatzvolumen mit Hilfe der Umsatzkategorie ist linksschief und zeigt dass der Grossteil der Projekte ein Umsatzvolumen von weniger als 10 Mio. CHF haben. Die Anzahl Fail-Projekte konzentriert sich folglich in den vier untersten Kategorien (vgl. auch Tabelle \ref{tab:ftobudcat}).
%%
%%Relative und Asolute Kostenabweichung
%%
\newline\newline\textit{Absolute und relative Kostenabweichung (CostActBud):} Die absoluten und relativen Abweichungen zwischen den aktuellen und den budgetierten Kosten sind direkt mit dem Erfolgskriterium korreliert und sind erwartungsgemäss für Fail-Projekte höher. 
% Mean absolute Kostenabweichungen
\begin{table}[H]
	\centering
	\caption{Arithmetisches Mittel der absoluten Kostenabweichungen pro Kostenart [TCHF]}
	\begin{tabular}{lrrrrr}
		\textbf{Success} & \multicolumn{1}{l}{\textbf{Total Cost}} & \multicolumn{1}{l}{\textbf{MS}} & \multicolumn{1}{l}{\textbf{ME}} & \multicolumn{1}{l}{\textbf{PA}} & \multicolumn{1}{l}{\textbf{IS}}
		\\\hline
		FALSE & -197  & -65   & -41   & -23   & -71 \\
		TRUE  & 176   & 73    & -3    & -1    & -11 \\
	\end{tabular}%
	\label{mcostabs}%
\end{table}%
% Mean relative Kostenabweichung
\begin{table}[H]
	\centering
	\caption{Arithmetisches Mittel der relativen Kostenabweichungen pro Kostenart [\%]}
	\begin{tabular}{lrrrrr}
		\textbf{Success} & \multicolumn{1}{l}{\textbf{Total Cost}} & \multicolumn{1}{l}{\textbf{MS}} & \multicolumn{1}{l}{\textbf{ME}} & \multicolumn{1}{l}{\textbf{PA}} & \multicolumn{1}{l}{\textbf{IS}}
		\\\hline
		FALSE & 12    & 9     & 206   & 79    & 93 \\
		TRUE  & -10   & 4     & 56    & 17    & 28 \\
	\end{tabular}%
	\label{mcostrel}%
\end{table}%
Die Tabellen \ref{mcostabs} und \ref{mcostrel} zeigen, dass die durchschnittlichen Abweichung der Projektkosten von nicht-erfolgreichen Projekten in absoluten und relativen Grössen über derjenigen der Success-Projekte liegt. Dabei fällt vor allem der Durchschnittswert der relativen Differenz Kostenabweichung der ME-Kosten auf.
\newline\newline\textit{Relative Anteile des Kostenbudgets:} Die Zusammensetzung der budgetierten Projektkosten soll Hinweise zur Projektart aufzeigen und ob sie sich zwischen den zwei Projektgruppen unterscheidet.
\begin{table}[htbp]
	\centering
	\caption{Arithmetisches Mittel der relativen Anteile am Gesamtkostenbudget je Kostenart [\%]}
	\begin{tabular}{lrrrr}
		\textbf{Success} & \multicolumn{1}{l}{\textbf{BudMSTot}} & \multicolumn{1}{l}{\textbf{BudMETot}} & \multicolumn{1}{l}{\textbf{BudPATot}} & \multicolumn{1}{l}{\textbf{BudISTot}} \\\hline
		FALSE & 67.1  & 6.2   & 5.9   & 7.7 \\
		TRUE  & 67.9  & 5.4   & 5.1   & 6.8 \\
	\end{tabular}%
	\label{mbudtot}%
\end{table}%
Die dargestellten Mittelwerte in der Tabelle \ref{mbudtot} liegen für nicht-erfolgreiche Projekte bei ME, PA und IS etwas höher. In der vorherigen Auswertung der durchschnittlichen Kostenabweichung wurde exakt bei diesen Projektphasen für Fail-Projekte deutlich höhere Werte festgestellt.
%
%Auswertung Nachlieferung
%
\newline\newline\textit{Nachlieferung SUCostTO:} Nachlieferungen verursachen zusätzliche Kosten, die aufgrund der zeitlichen Verzögerung bei der Fabrikation der Maschine oder Installation entstehen können. Es wird spekuliert, dass der Anteil der Kosten aus Nachlieferungen im Verhältnis zum Umsatzbudget bei Fail-Projekten höher ist als bei Success-Projekten.
%Mean SU Cost
\begin{table}[H]
	\centering
	\caption{Arithmetisches Mittel der SUCostTO [\%]}
	\begin{tabular}{lr}
		\textbf{Success} & \multicolumn{1}{l}{\textbf{SUCostTO}} \\\hline
		FALSE & -0.81 \\
		TRUE  & -0.36 \\
	\end{tabular}%
	\label{msu}%
\end{table}%
Die Auswertung (s. Tabelle \ref{msu}) des arithmetischen Mittels der prozentualen SU-Kosten am Umsatz bestätigt diese erwartete Vermutung.
%Mean SU Cost pro Umsatzkategorie
\begin{table}[H]
	\centering
	\caption{Arithmetisches Mittel der SUCostTO [\%] pro TO-Kategorie}
	\begin{tabular}{llr}
		\textbf{Success} & \textbf{TOBud\_Cat} & \multicolumn{1}{l}{\textbf{SUCostTO}} \\\hline
		FALSE & [13.2,500) & -0.93 \\
		FALSE & [500,1e+03) & -0.69 \\
		FALSE & [1e+03,1.5e+03) & -0.47 \\
		FALSE & [1.5e+03,2e+03) & -0.72 \\
		FALSE & [2e+03,2.5e+03) & -0.57 \\		
		FALSE & [2.5e+03,3e+03) & -0.82 \\
		FALSE & [3e+03,3.5e+03) & -0.31 \\
		FALSE & [3.5e+03,4e+03) & -0.69 \\
		FALSE & [4e+03,4.5e+03) & -0.52 \\
		FALSE & [4.5e+03,5e+03) & -6.88 \\
		FALSE & [5e+03,1e+04) & -0.82 \\
		FALSE & [1e+04,3.42e+04) & -0.36 \\
	\end{tabular}%
	\label{msutocat}%
\end{table}%
Die Mittelwertauswertung von SUCostTO in der Tabelle \ref{msutocat} pro TO-Kategorie zeigt, dass für Projekte mit einem Umsatzvolumen zwischen 13.2 TCHF und 500 TCHF die Nachlieferungskosten in Relation zum Umsatz am höchsten war. Der Wert 6.9\% kann als Anomalie betrachtet werden und ist auf ein Projekt mit einem SUCostTO-Wert von ca. 40\% als Einzelfall zu betrachten.
%
%Auswertung DeltaLastFCAct
%
\newline\newline\textit{Abweichung zwischen dem letzten FC und Act DeltaLastFCAct:} Tendenziell wird die Anpassung des Forecast für die Projektkosten bei erwarteten Mehrkosten möglichst lange hinausgezögert. Einerseits kann mit diesem Vorgehen, die Erklärungsdirektive umgangen werden und anderseits besteht wahrscheinlich die Hoffnung, dass die Projektkosten sich wieder normalisieren. Deshalb wird erwartet, dass die Differenz zwischen der letzten FC-Anpassung und den tatsächlichen Kosten bei Fail-Projekten höher war. 
%Tabelle Mean DeltaLastFCAct  
\begin{table}[H]
	\centering
	\caption{Arithmetisches Mittel DeltaLastFCAct für MS, ME, PA und IS [TCHF]}
	\begin{tabular}{lrrrrr}
		\textbf{Success} & \multicolumn{1}{c}{\textbf{Total FC}} & \multicolumn{1}{c}{\textbf{MS}} & \multicolumn{1}{c}{\textbf{ME}} & \multicolumn{1}{c}{\textbf{PA}} & \multicolumn{1}{c}{\textbf{IS}}
		\\\hline
		FALSE & -490.54 & -445.53 & 7.48 & -12.87 & -14.52 \\
		TRUE  & -436.41 & -454.24 & 7.93 & -13.41 & -6.49\\
	\end{tabular}%
	\label{mdeltalastfcact}%
\end{table}%
Die Ergebnisse der Tabelle \ref{mdeltalastfcact} bestätigen diese Vermutung. Die durchschnittliche Differenz bei den IS-Kosten war für Fail-Projekte doppelt so hoch.
%%
%%
%%Auswertung FF-Variablen
%%
%%
\paragraph{Fulfillment} Der bedeutenste Einflussfaktor im Projektmanagement ist der Projektmanager selbst. Die Evaluation der realisierten Projekte pro Projektmanager inklusive der Erfolgsquote hat ergeben, dass die 966 Projekte von 301 unterschiedlichen Projektmanager abgewickelt wurde. 145 Projektmanager haben ihre Projekte aussschliesslich erfolgreich beendet, wohingegen gerade einmal 45 PM nur unzureichend Projekte abgewickelt hat (s. Tablle \ref{tab:pmno} im Anhang).
%
%Auswertung Projektmanager Change
%
\newline\newline\textit{Projektmanager:} Der Wechsel des Projektmanagers wird mit konfligierende Verhältnisse zwischen den Vertragsparteien assoziiert, weshalb hypothetisch vermutet wird, dass Fail-Projekte eher mit einem PMChange einhergehen. 
\begin{table}[H]
	\centering
	\caption{Häufigkeitsverteilung PMChange}
	\begin{tabular}{lrrrrr}
		\textbf{PMChange} & \multicolumn{1}{l}{\textbf{Success}} & \multicolumn{1}{l}{\textbf{Fail}} & \multicolumn{1}{l}{\textbf{Fail [\%]}} & \multicolumn{1}{l}{\textbf{Total}} &
		\multicolumn{1}{l}{\textbf{Total [\%]}} \\\hline
		no    & 628   & 295   & 32.0\% & 923 & 96\% \\
		yes   & 26    & 17    & 39.5\% & 43  & 4\% \\\hline
		\textbf{Total} & \textbf{654} & \textbf{312} & \textbf{32.3\%} & \textbf{966} & \textbf{100.0\%} \\
	\end{tabular}%
	\label{pmchange}%
\end{table}%
Insgesamt wurden 43 Projekte mit einem Wechsel des Projektmanagers über die letzten drei Jahre abgewickelt, wie der Tabelle \ref{pmchange} zu entnehmen ist. Davon sind 17 gescheitert und 26 wurden erfolgreich abgeschlossen. Die Anzahl Projektmanager (NoPM) ist direkt mit der Variable PMChange verbunden und weist wenig Informationsgehalt auf. Es gab in der Stichprobe sechs Projekte, bei denen der PM zweimal ausgetauscht wurde, davon sind fünf Projekte gescheitert.
%
% Auswertung Alter und Tenuer PM
%
\newline Das Alter (PMAge2) und die Dienstjahre (PMTen2) des Projektmanagers sind Proxyvariablen für die Lebens- und Berufserfahrung sowie die Kenntnisse der Bühlerwelt. Erfahrenere (ältere) sowie langjährige Mitarbeitende müssten mehr Erfolg im Projektmanagement haben, da sie mehr Praxiserfahrung mit der Bühler-Welt einerseits und dem Projektmanagement anderseits haben sollten. Die Durchschnittswerte des Alter und der Dienstjahre in der Tabelle \ref{ageten} im Anhang sind für Fail-Projekte und Success-Projekte faktisch identisch. Sie betragen gerundet 40 beziehungsweise 12 Jahre.
% Table generated by Excel2LaTeX from sheet 'fagecat'
\begin{table}[htbp]
	\centering
	\caption{Erfolgsquote und Häufigkeitsverteilung pro Alterskategorie}
	\begin{tabular}{lrrrrrr}
		\textbf{Cat\_age} & \multicolumn{1}{l}{\textbf{Erfolgsquote}} & \multicolumn{1}{l}{\textbf{Success}} & \multicolumn{1}{l}{\textbf{Fail}} & \multicolumn{1}{l}{\textbf{Fail [\%]}} & \multicolumn{1}{l}{\textbf{Total}} & \multicolumn{1}{l}{\textbf{Total [\%]}} \\\hline
		20-25 & 2.5   & 15    & 6     & 28.6\% & 21    & 2.2\% \\
		26-30 & 2.3   & 130   & 57    & 30.5\% & 187   & 19.4\% \\
		31-35 & 3.0   & 130   & 44    & 25.3\% & 174   & 18.0\% \\
		36-40 & 2.3   & 97    & 43    & 30.7\% & 140   & 14.5\% \\
		41-45 & \textbf{1.8} & 105   & 57    & 35.2\% & 162   & 16.8\% \\
		46-50 & 2.3   & 65    & 28    & 30.1\% & 93    & 9.6\% \\
		51-55 & \textbf{1.3} & 69    & 52    & 43.0\% & 121   & 12.5\% \\
		56-60 & \textbf{1.4} & 26    & 19    & 42.2\% & 45    & 4.7\% \\
		61-63 & 2.8   & 17    & 6     & 26.1\% & 23    & 2.4\% \\\hline
		\textbf{Total} & \textbf{2.1} & \textbf{654} & \textbf{312} & \textbf{32.3\%} & \textbf{966} & \textbf{100.0\%} \\
	\end{tabular}%
	\label{fagecat}%
\end{table}%
Die Tabelle zeigt, dass in den Alterskategorien (51-55) und (56-60) relativ am meisten nicht-erfolgreiche' Projekte. Folglich liegt ein Indiz für die Gegenhypothese vor, wobei angemerkt werden muss, dass Verteilung der Projekte diese Quote beeinflussen kann. Beispielsweise könnte den erfahreneren Mitarbeitenden, die herausfordernden Projekte zugewiesen werden, da sie über mehr fundiertes Wisse im Projektmanagement verfügen.
%
% Auswertung Lead SAS
%
\newline\newline\textit{Lead SAS}
Die Bühler AG unterscheidet zwei Typen von Lead SAS: die Lead SAS für das gesamte Projekt (LeadSASPr) und die Lead SAS für die Projektabwicklung (LeadSASFF). Lead legt in dieser Hinsicht die Verantwortung fest. Da die Verantwortlichkeit auf zwei Gesellschaften aufgeteilt werden kann, wurde zudem eruiert, ob sich die LeadSASFF von der LeadSASPr unterscheidet (LeadSAS.PrFF). Es wird postuliert, dass einige Gesellschaften Projekte erfolgreicher managen. Zudem wird angenommen, dass bei zwei Verantwortungsparteien, die Erfolgsquote höher sein muss, da die jeweiligen Aufgabenbereich besser fokussiert werden kann.
% Table frequency Lead SASPr
\begin{table}[H]
	\centering
	\caption{Erfolgsquoten und Häufigkeitsverteilung pro LeadSASPr}
	\begin{tabular}{lrrrrrr}
		\textbf{LeadSASPr} & \multicolumn{1}{l}{\textbf{Erfolgsquote}} & \multicolumn{1}{l}{\textbf{Success}} & \multicolumn{1}{l}{\textbf{Fail}} & \multicolumn{1}{l}{\textbf{Fail [\%]}} & \multicolumn{1}{l}{\textbf{Total}} & \multicolumn{1}{l}{\textbf{Total [\textbackslash{}\%]}} \\\hline
		BJHB  & 0.6   & 10    & 17    & 63.0\% & 27    & 2.8\% \\
		\textbf{BBS}   & 0.8   & 27    & 35    & 56.5\% & 62    & 6.4\% \\
		\textbf{BMIL}  & 0.9   & 13    & 15    & 53.6\% & 28    & 2.9\% \\
		BRAL  & 0.9   & 12    & 13    & 52.0\% & 25    & 2.6\% \\
		\textbf{BLOA}  & 1.0   & 19    & 19    & 50.0\% & 38    & 3.9\% \\
		BPRI  & 1.0   & 2     & 2     & 50.0\% & 4     & 0.4\% \\
		\textbf{BBAR}  & 1.1   & 8     & 7     & 46.7\% & 15    & 1.6\% \\
		BJOI  & 1.2   & 19    & 16    & 45.7\% & 35    & 3.6\% \\
		BSSE  & 1.2   & 27    & 22    & 44.9\% & 49    & 5.1\% \\
		BMIN  & 1.2   & 31    & 25    & 44.6\% & 56    & 5.8\% \\
		BBIN  & 1.3   & 20    & 15    & 42.9\% & 35    & 3.6\% \\
		BPAR  & 1.3   & 16    & 12    & 42.9\% & 28    & 2.9\% \\
		BBAI  & 2.0   & 2     & 1     & 33.3\% & 3     & 0.3\% \\
		BMAD  & 2.1   & 15    & 7     & 31.8\% & 22    & 2.3\% \\
		BMEX  & 2.5   & 5     & 2     & 28.6\% & 7     & 0.7\% \\
		BLON  & 3.0   & 3     & 1     & 25.0\% & 4     & 0.4\% \\
		BUZ   & 3.6   & 310   & 86    & 21.7\% & 396   & 41.0\% \\
		BLOC  & 5.5   & 11    & 2     & 15.4\% & 13    & 1.3\% \\
		BYOK  & 5.5   & 11    & 2     & 15.4\% & 13    & 1.3\% \\
		BCHI  & 5.7   & 34    & 6     & 15.0\% & 40    & 4.1\% \\
		BBAN  & 7.3   & 51    & 7     & 12.1\% & 58    & 6.0\% \\
		BDAG  & \multicolumn{1}{r}{NA} & 6     & 0     & 0.0\% & 6     & 0.6\% \\
		BTEH  & \multicolumn{1}{r}{NA} & 2     & 0     & 0.0\% & 2     & 0.2\% \\\hline
		\textbf{Total} &       & \textbf{654} & \textbf{312} & \textbf{32.3\%} & \textbf{966} & \textbf{100.0\%} \\
	\end{tabular}%
	\label{fleadsas}%
\end{table}%
Der Vergleich Erfolgsquoten pro SAS der Tabelle \ref{fleadsas} zeigt, dass die europäischen Gesellschaften im Vergleich tiefere Erfolgsraten haben. BJHB, BBS, BMIL und BRAL, die zusammen 15\% aller Projekte abwickeln, haben eine unterdurchschnittliche Erfolgsbilanz.
%Tabelle LeadSAS.PrFF
\begin{table}[H]
	\centering
	\caption{Häufigkeitsverteilung LeadSAS.PrFF}
	\begin{tabular}{lrrrrr}
		\textbf{LeadSAS.PrFF} & \multicolumn{1}{l}{\textbf{Success}} & \multicolumn{1}{l}{\textbf{Fail}} & \multicolumn{1}{l}{\textbf{Fail [\%]}} & \multicolumn{1}{l}{\textbf{Total}} & \multicolumn{1}{l}{\textbf{Total [\%]}}
		\\\hline
		identisch    & 569   & 296   & 34.2\% & 865 & 89.5\% \\
		verschieden   & 85    & 16    & 15.8\% & 101 & 10.5\% \\\hline
		\textbf{Total} & \textbf{654} & \textbf{312} &       & \textbf{966} \\
	\end{tabular}%
	\label{fleadsasprff}%
\end{table}%
Die Resultate der Tabelle \ref{fleadsasprff} implizieren, dass der überwiegende Anteil der Projekte eine Verantwortungspartei hatte. Von den 101 Projekten mit geteilter Projektverantwortung wurden lediglich 16\% mit einem prozentualen DB1Act unter Budget abgeschlossen. Das Ergebnis kann folglich als Indiz zu Gunsten der Hypothese gewertet werden.
%%
%%
%%Zeit
%%
%%
\paragraph{Zeit} Die Beurteilung des Zeitmanagement hängt von der Einhaltung des vereinbarten Liefertermins ab. Mehrkosten und Zeitverzug gehen oftmals einher, weshalb unterstellt wird, dass Fail-Projekte den vereinbarten Projektabschluss nicht einhalten konnten. Ferner soll ergründet werden, ab welchem Zeitpunkt respektive Milestone der Zeitverzug üblicherweise eintritt.
%Table durchschnittlicher PrTimeDelay und pro MS
\begin{table}[H]
	\centering
	\caption{Arithmetisches Mittel der Projektlaufzeit und Zeitverzögerung [Monate]}
	\begin{tabular}{lrrrrrrr}
		\textbf{Success} & \multicolumn{1}{l}{\textbf{Base}} & \multicolumn{1}{l}{\textbf{Act}} & \multicolumn{1}{l}{\textbf{Delay}} & \multicolumn{1}{l}{\textbf{MS2}} & \multicolumn{1}{l}{\textbf{MS8}} & \multicolumn{1}{l}{\textbf{MS10}} & \multicolumn{1}{l}{\textbf{MS11}} \\ \hline
		TRUE  & 11.9  & 17.3  & -5.4  & -0.1  & -2.0  & -5.0  & -5.5 \\
		FALSE & 11.4  & 18.7  & -7.2  & -0.2  & -1.7  & -5.7  & -7.3 \\
	\end{tabular}%
	\label{mtime}%
\end{table}%
Die durchschnittliche budgetierte Projektlaufzeit unterscheidet sich zwischen erfolgreichen und  nicht-erfolgreichen Projekten kaum wohingegen die effektive Projektlaufzeit der Fail-Projekte einen Monat mehr betrug (s. Tabelle \ref{mtime}). Beide Projektgruppen konnten im Durchschnitt den Liefertermin nicht einhalten, wobei die Success-Projekte ca. zwei Monate weniger zeitverzögert waren. Die Termineinhaltung beim MS2 Concept approved bewegt sich im vernachlässigbaren Bereich. Demgegenüber steigt der durchschnittliche Zeitverzug nach MS8 Documented auf zwei, nach MS10 Takeover auf fünf bis sechs Monate und liegt bei Projektabschluss (MS11) zwischen gerundet sechs und sieben Monaten.
%Table Häufigkeit Delay False/Success
\begin{table}[H]
	\centering
	\caption{Häufigkeitsverteilung zeitverzögerter Projekte}
	\begin{tabular}{lrrrrr}
		\textbf{Delay} & \multicolumn{1}{l}{\textbf{Success}} & \multicolumn{1}{l}{\textbf{Fail}} & \multicolumn{1}{l}{\textbf{Fail [\%]}} & \multicolumn{1}{l}{\textbf{Total}} & \multicolumn{1}{l}{\textbf{Total [\%]}} \\ \hline
		TRUE  & 515   & 268   & 34.2\% & 783   & 81.1\% \\
		FALSE & 139   & 44    & 24.0\% & 183   & 18.9\% \\\hline
		\textbf{Total} & \textbf{654} & \textbf{312} & \textbf{32.3\%} & \textbf{966} & \textbf{100.0\%} \\
	\end{tabular}%
	\label{fdelay}%
\end{table}%
Insgesamt wurde gemäss der Tabelle \ref{fdelay} der Liefertermin bei 783 Projekten nicht eingehalten, wobei das Verhältnis zwischen erfolgreichen und nicht-erfolgreichen zwei zu eins beträgt. Der prozentuale Anteil der Projekte, bei denen die Anlage pünktlich an den Kunden übergeben werden konnte, beträgt 20. 
%Table Häufigkeit Delay pro MS
\begin{table}[H]
	\centering
	\caption{Absolute Häufigkeitsverteilung zeitverzögerter Projekte pro Milestone}
	\begin{tabular}{lrrrr}
		\textbf{Success} & \multicolumn{1}{l}{\textbf{DelayMS2}} & \multicolumn{1}{l}{\textbf{DelayMS8}} & \multicolumn{1}{l}{\textbf{DelayMS10}} & \multicolumn{1}{l}{\textbf{DelayMS11}} \\\hline
		FALSE & 43    & 209   & 275   & 270 \\
		TRUE  & 75    & 408   & 532   & 516 \\\hline
		\textbf{Total} & \textbf{118} & \textbf{717} & \textbf{807} &  \textbf{786}
	\end{tabular}%
	\label{fdelayms}%
\end{table}
Die Tabelle \ref{fdelayms} zeigt, dass die Mehrheit der Projekte bei MS2 noch im Zeitplan agierte und nach MS8 bereits hinter dem vereinbarten Liefertermin lag. Mittels Dummyvariablen pro Milestone wurde ausgewertet, ob sich ein anfängliche Verspätung durch die Projektlaufzeit durchzieht.
% Table generated by Excel2LaTeX from sheet 'time_sukz'
\begin{table}[H]
	\centering
	\caption{Absolute Häufigkeitsverteilung von Kombinationen der Zeitverzögerung}
	\begin{tabular}{lrrrrrr}
		\textbf{Success} & \multicolumn{1}{l}{\textbf{Delay}} & \multicolumn{1}{l}{\textbf{DelayMS2}} & \multicolumn{1}{l}{\textbf{DelayMS8}} & \multicolumn{1}{l}{\textbf{DelayMS10}} & \multicolumn{1}{l}{\textbf{DelayMS11}} & \multicolumn{1}{l}{\textbf{freq}} \\
		FALSE & \multicolumn{1}{l}{TRUE} & 0     & 1     & 1     & 1     & 147 \\
		FALSE & \multicolumn{1}{l}{TRUE} & 0     & 0     & 1     & 1     & 64 \\
		FALSE & \multicolumn{1}{l}{TRUE} & 1     & 1     & 1     & 1     & 33 \\
		FALSE & \multicolumn{1}{l}{TRUE} & 0     & 0     & 0     & 1     & 10 \\
		FALSE & \multicolumn{1}{l}{TRUE} & 0     & 1     & 0     & 1     & 7 \\
		FALSE & \multicolumn{1}{l}{TRUE} & 1     & 0     & 1     & 1     & 5 \\
		FALSE & \multicolumn{1}{l}{TRUE} & 0     & 0     & 0     & 0     & 1 \\
		FALSE & \multicolumn{1}{l}{TRUE} & 1     & 0     & 0     & 1     & 1 \\\hline
		\textbf{Total} &       &       &       &       &       & \textbf{268} \\
		TRUE  & \multicolumn{1}{l}{TRUE} & 0     & 1     & 1     & 1     & 290 \\
		TRUE  & \multicolumn{1}{l}{TRUE} & 0     & 0     & 1     & 1     & 126 \\
		TRUE  & \multicolumn{1}{l}{TRUE} & 1     & 1     & 1     & 1     & 42 \\
		TRUE  & \multicolumn{1}{l}{TRUE} & 0     & 1     & 0     & 1     & 19 \\
		TRUE  & \multicolumn{1}{l}{TRUE} & 1     & 0     & 1     & 1     & 17 \\
		TRUE  & \multicolumn{1}{l}{TRUE} & 0     & 0     & 0     & 1     & 16 \\
		TRUE  & \multicolumn{1}{l}{TRUE} & 1     & 1     & 0     & 1     & 2 \\
		TRUE  & \multicolumn{1}{l}{TRUE} & 0     & 0     & 1     & 0     & 1 \\
		TRUE  & \multicolumn{1}{l}{TRUE} & 1     & 0     & 0     & 0     & 1 \\
		TRUE  & \multicolumn{1}{l}{TRUE} & 1     & 0     & 0     & 1     & 1 \\\hline
		\textbf{Total} &       &       &       &       &       & \textbf{515} \\
	\end{tabular}%
	\label{fdleay}%
\end{table}%
Die meisten Projekte hatten die Eigenschaft, bei einer Verspätung im MS8 ebenso auch im MS11 verspätet zu sein. Die zweithäufigste Gruppe war diejenige, die erstmalig ab dem MS10 den Liefertermin bis zum Projektabschluss nicht mehr einhalten konnte. Erst an dritter Stelle folgt die Gruppe, die bereits ab MS2 bis MS11 hinter dem Zeitplan lag. Auf Basis der Tabellen lassen sich jedoch keine Aussagen machen, ob bei einer Nichteinhaltung des vereinbarten Termin, das Projekt zwangsläufig unter Budget abschliessen wird. 
%%
%%
%% Auswertung SQ
%%
%%
\paragraph[short title]{Sales and Quotation} Die Einflussdeterminanten des SQ-Prozess sind einerseits der Stand im Bezug auf das OR-Budget bei Projektabschluss und die Erfahrung (AMAge2) sowie Dienstjahre (AMTen2) des Verkaufsmanager. Allerdings konnten letztere aufgrund fehlender Datensätze nicht ausgewertet werden. Grundsätzlich wird vermutet, dass ein Budgetdruck im Zeitpunkt des Verkaufsabschlusses, den Verkauf von risikoreicheren Projekten begünstigt. 
% Table generated by Excel2LaTeX from sheet 'sq mean'
\begin{table}[H]
	\centering
	\caption{Arithmetisches Mittel des BudGapOR der BU und Region}
	\begin{tabular}{lrr|rr}
		\textbf{Success} & \multicolumn{1}{l}{\textbf{BU [TCHF]}} & \multicolumn{1}{l}{\textbf{BU [\%]}} & \multicolumn{1}{l}{\textbf{Region [TCHF]}} & \multicolumn{1}{l}{\textbf{Region [\%]}} \\\hline
		FALSE & -7'244.2 & -11.3 & -20'845.5 & -9.3 \\
		TRUE  & -1'207.2 & -4.0  & -14'886.9 & -6.8 \\
	\end{tabular}%
	\label{msq}%
\end{table}%
Die mittlere Abweichung des OR vom Budget des Geschäftseinheit und der Region der Tabelle \ref{msq} waren für Fail-Projekte in absoluten und relativen Grössen höher als für Success-Projekte.  
%%
%%
%% Auswertung Komplexität
%%
%%
\newline\newline\textbf{Komplexität:} Die Komplexität kann anhand unterschiedlicher Dimensionen gemessen werden wie zum Beispiel, dem Inhalt, den Zielen, den Beteiligten oder dem Umfeld. In der nachfolgenden Auswertung wurden die Anzahl Aufträge sowie involvierter Parteien bei den unterschiedlichen Projektphasen und Konsortien als Proxyvariablen für Komplexität untersucht. Die vorherrschende Auffassung bezüglich Komplexität, ist ihr negativer Einfluss auf den Projekterfolg.
%
%Auswertung Konsortium
%
Die Anzahl Projekte in der Stichprobe, die in einem Konsortium abgewickelt wurden, beträgt 78 , wovon 47 erfolgreich und 31 unter Budget abgeschlossen wurden, wie der Tabelle \ref{fcons} zu entnehmen ist. In EU und SAS\_BCHI wurden 51 Projekte im Konsortium abgewickelt.
% Table generated by Excel2LaTeX from sheet 'con_part_reg'
\begin{table}[H]
	\centering
	\caption{Absolute Häufigkeitsverteilung der Projekte mit Konsortium pro Region}
	\begin{tabular}{lrrrr}
		\textbf{ConPart} & \multicolumn{1}{l}{\textbf{Region}} & \multicolumn{1}{l}{\textbf{Success}} & \multicolumn{1}{l}{\textbf{Fail}} & \multicolumn{1}{l}{\textbf{Erfolgsquote}} \\\hline
		TRUE  & \multicolumn{1}{l}{East\_Asia} & 1     & 0     & NA \\
		TRUE  & \multicolumn{1}{l}{\textbf{EU}} & 13    & 14    & 0.9 \\
		TRUE  & \multicolumn{1}{l}{MEA\_Afr} & 9     & 7     & 1.3 \\
		TRUE  & \multicolumn{1}{l}{North\_Ame} & 2     & 1     & 2.0 \\
		TRUE  & \multicolumn{1}{l}{\textbf{SAS\_BCHI}} & 17    & 7     & 2.4 \\
		TRUE  & \multicolumn{1}{l}{South\_Ame} & 4     & 2     & 2.0 \\
		TRUE  & \multicolumn{1}{l}{South\_Asia} & 1     & 0     & NA \\\hline
		\textbf{Total} &       & \textbf{47} & \textbf{31} & \textbf{1.5} \\
	\end{tabular}%
	\label{fcons}%
\end{table}%
Die meisten Konsortium-Projekte in Europa stammen vom Geschäftsbereich GL wovon nur jedes zweite erfolgreich zu Ende geführt werden konnte (s. Tabelle \ref{consregba}). Demgegenüber stehen 11 Projekte mit Konsortium in Middle East \& Africa des Geschäftsbereichs GM, wovon ebenso die Hälfte erfolgreich abgeschlossen werden konnte. Die DC-Projekte weisen gegenüber den GM-Projekten in SAS\_BCHI die bessere Erfolgsbilanz aus.
% Table generated by Excel2LaTeX from sheet 'cons_par_reg_Ba'
\begin{table}[htbp]
	\centering
	\caption{Erfolgsquote und absolute Häufigkeit von ConPart für EU, MEA\_Afr und SAS\_BCHI}
	\begin{tabular}{llrrr}
		\textbf{Region} & \textbf{BA} & \multicolumn{1}{l}{\textbf{Success}} & \multicolumn{1}{l}{\textbf{Fail}} & \multicolumn{1}{l}{\textbf{Erfolgsquote}} \\\hline
		EU  & CF    & 5     & 0     & NA \\
		& DC    & 0     & 1     & 0.0 \\
		& GL    & 6     & 5     & 1.2 \\
		& GM    & 1     & 3     & 0.3 \\
		&VN    & 1     & 5     & 0.2 \\\hline
		\textbf{Total} &  & \textbf{13} & \textbf{14} & \textbf{0.9} \\
		&  &  & & \\
		MEA\_Afr &	GL    & 2     & 2     & 1.0 \\
		& 	GM    & 6     & 5     & 1.2 \\
		& VN    & 1     & 0     & NA \\\hline
		\textbf{Total} &  & \textbf{9} & \textbf{7} & \textbf{1.3} \\
		& & & & \\
		SAS\_BCHI & \multicolumn{1}{l}{CF} & 2     & 0     & NA \\
		& \multicolumn{1}{l}{DC} & 11    & 1     & 11.0 \\
		& \multicolumn{1}{l}{GM} & 4     & 6     & 0.7 \\\hline
		\textbf{Total} &       &  \textbf{17}   & \textbf{7}     & \textbf{2.4} \\
	\end{tabular}%
	\label{consregba}%
\end{table}%
%
%Auswertung Anzahl Aufträge
%
\newpage
Die Anzahl Aufträge pro Projekt (NoContr) kann ein Indiz für die Anzahl Komponenten der Maschine oder involvierter Parteien sein. In der Stichprobe variiert sie zwischen eins und zehn, wobei die Mehrheit aller Projekte einen einzigen Auftrag pflegte (s. Tabelle \ref{fnocontr}). Weitere 20\% aller Projekte werden mittels zwei Aufträgen abgewickelt. Die Erfolgsquote der Projekte mit zwei Verträgen liegt leicht über derjenigen von Projekten mit einem Auftrag. Allerdings wird das Verhältnis zwischen Fail- und Success-Projekten der gesamten Stichprobe durch Projekte mit einem Vertrag bestimmt.
% Table generated by Excel2LaTeX from sheet 'nocontr'
\begin{table}[H]
	\centering
	\caption{Erfolgsquote und Häufigkeitsverteilung von NoContr}
	\begin{tabular}{lrrrrrr}
		\textbf{NoContr} & \multicolumn{1}{l}{\textbf{Erfolgsquote}} & \multicolumn{1}{l}{\textbf{Success}} & \multicolumn{1}{l}{\textbf{Fail}} & \multicolumn{1}{l}{\textbf{Fail [\%]}} & \multicolumn{1}{l}{\textbf{Total}} & \multicolumn{1}{l}{\textbf{Total [\%]}} \\\hline
		\multicolumn{1}{r}{1} & 2.0   & 449   & 228   & 33.7\% & 677   & 70.1\% \\
		\multicolumn{1}{r}{2} & 2.6   & 136   & 52    & 27.7\% & 188   & 19.5\% \\
		\multicolumn{1}{r}{3} & 1.7   & 38    & 22    & 36.7\% & 60    & 6.2\% \\
		\multicolumn{1}{r}{4} & 3.6   & 18    & 5     & 21.7\% & 23    & 2.4\% \\
		\multicolumn{1}{r}{5} & 4.0   & 4     & 1     & 20.0\% & 5     & 0.5\% \\
		\multicolumn{1}{r}{6} & 3.0   & 6     & 2     & 25.0\% & 8     & 0.8\% \\
		\multicolumn{1}{r}{7} & NA    & 1     & 0     & 0.0\% & 1     & 0.1\% \\
		\multicolumn{1}{r}{8} & NA    & 1     & 0     & 0.0\% & 1     & 0.1\% \\
		\multicolumn{1}{r}{9} & NA    & 1     & 0     & 0.0\% & 1     & 0.1\% \\
		\multicolumn{1}{r}{10} & 0.0   & 0     & 2     & 100.0\% & 2     & 0.2\% \\\hline
		\textbf{Total} & \textbf{2.1} & \textbf{654} & \textbf{312} & \textbf{32.3\%} & \textbf{966} & \textbf{100.0\%} \\
	\end{tabular}%
	\label{fnocontr}%
\end{table}%
%
%Auswertung Anzahl SAS
%
Die Anzahl involvierte SAS (NoSupplSAS) bei der Zulieferung liegt im Bereich null und zehn, wobei null mit Eigenproduktion oder Zulieferung durch Dritte gleichzusetzen ist. In den übrigen Projektphasen sind maximal drei andere Bühler-Gesellschaften involviert. Eine Einzelauswertung pro Projektphase ergibt relativ wenig Aufschluss, weshalb die Häufigkeit der Kombinationen untersucht wurde. 
% Table generated by Excel2LaTeX from sheet 'nosas'
\begin{table}[H]
	\centering
	\caption{Erfolgsquote und Häufigkeitsverteilung der NoSupplySAS-Kombinationen (Ausschnitt)}
	\begin{tabular}{lrrrrrrrr}
		\textbf{NoSupplSAS} & \multicolumn{1}{l}{\textbf{MS}} & \multicolumn{1}{l}{\textbf{ME}} & \multicolumn{1}{l}{\textbf{PA}} & \multicolumn{1}{l}{\textbf{IS}} & \multicolumn{1}{l}{\textbf{Success}} & \multicolumn{1}{l}{\textbf{Fail}} & \multicolumn{1}{l}{\textbf{Total}} & \multicolumn{1}{l}{\textbf{Erfolgsquote}} \\ \hline
		\multicolumn{1}{r}{0} & 0     & 0     & 0     & 0     & 180   & 89    & 269 & 2.0 \\
		\multicolumn{1}{r}{1} & 1     & 0     & 0     & 0     & 73    & 26    & 99  & 2.8 \\
		\multicolumn{1}{r}{2} & 2     & 0     & 0     & 0     & 38    & 13    & 51  & 2.9 \\
		\multicolumn{1}{r}{2} & 2     & 1     & 0     & 0     & 23    & 13    & 36  & 1.8 \\
		\multicolumn{1}{r}{1} & 1     & 1     & 1     & 1     & 15    & 12    & 27  & 1.3 \\
		\multicolumn{1}{r}{1} & 1     & 1     & 0     & 0     & 17    & 11    & 28  & 1.5 \\ \hline
		\textbf{Total} &       &       &       &       & \textbf{346} & \textbf{164} & \textbf{510} & 2.1 \\
	\end{tabular}%
	\label{fnosas}%
\end{table}%
Die obige Tabelle \ref{fnosas} zeigt, dass die Eigenfertigung respektive die Zusammenarbeit mit Drittlieferanten während allen Projektphasen das häufigste Charakteristika aller Projekte war. Es fällt auf, dass sich mehr als die Hälfte der Datenpunkte im Bereich mit maximal zwei involvierten SAS-Gesellschaften während der gesamten Projektlaufzeit konzentriert. Aufgrund der Erfolgsquoten kann postuliert werden, dass eine Zusammenarbeit mit mindestens einer weiteren Partei im MS, den Projekterfolg begünstigen könnte.
\newpage
%%%%%%%%%%%%%%%%%%%%%%%%%%%%%%%%%%%%%%%%%%%%%%%%%%%%%%%%%%%%%%%%%%%%%%%%%%%%%%%%%%%
%%
%%Beurteilung der Ergebnis
%%
%%%%%%%%%%%%%%%%%%%%%%%%%%%%%%%%%%%%%%%%%%%%%%%%%%%%%%%%%%%%%%%%%%%%%%%%%%%%%%%%%%%
\newpage
%Fazi
% !TEX root = MA.tex
\chapter{Diskussion Ergebnisse}\label{sec:diskerg}

%%Ziel: eigene Ergebnisse interpretieren und praktische Relevanz erläutern
%%Einleitung: Erläutern kurz was gemacht wurde, Zusammenfassung der wichtigsten Ergebnissen, 
%%Erfolgsfaktoren: pro Kategorie, Hypothesen beantworten
%%Erfolgsfaktoren: in Bezug zu Hypothese
%%Erfolgsfaktoren: in Bezug zu Theorie
%%Erfolgsfaktoren: Scheiter und Scheitern lassen, Erfolgsdefiniion = Schwarz/Weiss Denken
%%Erfolgsfaktoren: Theorie sagt was anderes, Vorschläge zur Ergründung, aber Einflussdeterminaten könnten sein.....
%%Schlussfolgerung: Haupterkenntnisse
%%Schlussfolgerung: Relevanz für weitere Forschung und praktische Anwendung, was muss zukünftig geleistet werden, in Planform!
%%Fazit:
\section{Erfolgsfaktoren}\label{sec:diskerf}
%%Ausgangslage: Faktoren können nicht zur Früherkennung dienen
%%Konezptionelle Ansätze: Kritische Stelle im Prozess beleuchten, Hypothetische Vermutungen formulieren, wo Problem liegen könnte
%%Implementierung, Incentivierung
%%Andere Ansätze: Fokus anders legen, 
Nachfolgend werden die Ergebnisse aus Kapitel \ref{sec:ergebnisse} kritisch beurteilt, um einerseits Hinweise für mögliche Erfolgsfaktoren und anderseits Ansätze für weiterführende Analysen zu ergründen.
%% alle vars...vergleich mit theori^^
%% erfolgskriterium
%% pro Kategorie
\newline\newline
Die Analyse von Umsatz, Kosten und DB1 hat ergeben, dass Fail-Projekte in den letzten drei Jahren einen Margenverlust von 48.2 MCHF der insgesamt auf Mehrkosten von 61.5 MCHF im Vergleich zum Budget zurückzuführen ist. Die Kostenabweichungen der Projektphasen Installation und Beschaffung sind dabei am höchsten, womit ein Hinweis vorliegt, welche Prozesse zur Ergründung der Ursachen der Mehrkosten untersucht werden müssen. Die Auswertung der durchschnittlichen Kostenabweichungen zeigt, dass relativ die ME-Kosten am meiste von ihrem Budget abweichen. Daraus folgt, dass pro geplanter Franken an ME-Kosten doppelt soviel Kosten benötigt werden. Das kann entweder daran liegen, dass im Budgetprozess die Kosten zu tief angesetzt wurden, oder aufgrund von Komplikationen erhebliche Mehrkosten entstanden sind. Diese Frage und ob beispielsweise vorgelagerte Fehler in der Prozesskette zu garantierten überhöhten Kosten führen, müssten geklärt werden, um Optimierungspotenziale ausschöpfen zu können.
\newline Der Margenverlust stammt zu einem grossen Teil von verhältnismässig wenig Projekten mit hohem Umsatzvolumen. Da grössere Projekte mit mehr Risiken einhergehen und das Gesamtergebnis eines Geschäftsbereichs erheblich beeinflussen kann, werden solche Projekte oftmals prioritär behandelt. Allerdings darf der kumulierte Verlust gegenüber dem Budget in Bezug auf den Deckungsbeitrag von Projekte mit geringeren Umsatzvolumen nicht vernachlässigt werden. Denn hinsichtlich der Anzahl sind Projekte bis und mit 1 Mio. CHF in der Überzahl. Somit könnte eine Änderung der Prioritätsregel die Verlustverteilung verschieben, wobei allerdings der trade off durchdacht werden sollte. Zur Begründung könnte statistisch evaluiert werden, welche Projektgrösse mehr Einfluss auf die Erfolgswahrscheinlichkeit hat. Obwohl aus finanzieller Sicht eindeutig ist, dass sowohl viele kleine als auch wenig grosse Projekte die negative DB1-Performance bestimmen, kann auf Basis der Ergebnisse kein eindeutiger Rückschluss, ob das Umsatzvolumen eines Projekts als Erfolgsfaktor in Frage käme. Da grosse Projekte tendenziell eher einem Projektassessment nach Ende der Projektlaufzeit unterzogen werden, könnte es lohnend sein die Ursachen des Scheiterns der kleineren Projekte zu eruieren. Daraus könnten sich dann die Erfolgsfaktoren herauskristallisieren, die nicht zwingend mit denjenigen der grossen Projekte übereinstimmen müssen. Zusammenfassend ist darauf hinzuweisen, dass die Kostenabweichungen trotz ihres direktem Zusammenhang mit dem der Auswertung zugrundeliegenden Erfolgskriterium eher der Beurteilung der Projekte und weniger der Einflussnahme auf die Erfolgswahrscheinlichkeit eines Projekts dient. Demzufolge sind sie gemäss der Unterscheidung von (Besteiero, Pinto und Novaski, 2015) als Kriterium und weniger als Erfolgsfaktor zu betrachten (vgl. Kapitel \ref{sec:proj}). 
\newline\newline
Die Analyse der Rahmenbedingungen hat aufgezeigt in welchen Regionen und Geschäftsbereichen, das Volumen abgewickelt wird und wie sich die erfolgreichen zu den nicht-erfolgreichen Projekte verhalten. Typischerweise ist ein nicht-erfolgreiches Projekt in der Region EU, NAM oder SAM in den Geschäftsbereichen GL, VN und GM zu finden. Da die Anzahl der Projekte in EU und GM relativ am höchsten ist, könnte die zusätzliche Auswertung der Ressourcenverteilung aufzeigen, in welchem Masse die Projektmanager ausgelastet sind. Demzufolge könnten beispielsweise die Projektanzahl pro Manager die Priorisierung bestimmter Projekte den Erfolg anderer beeinträchtigen, so dass ein gewisser Kanibalisierungseffekt auftritt. Die Eignung als Erfolgsfaktor, im Sinne einer Erhöhung der Erfolgswahrscheinlichkeit, ist jedoch aufgrund der Konsequenzen entsprechender Massnahmen fragwürdig. Denn die Attribute Region und Geschäftsbereich können über die Steuerung des Verkaufs beeinflusst werden, was zur Folge hätte, dass mehr Projekte in anderen Geschäftsbereichen respektive Regionen verkauft würden. Der Effekt hinsichtlich des Projekterfolgs bleibt dabei ungeklärt. Die Rahmenbedingungen können jedoch ein Argument dafür liefern, welche Projekte beobachtet werden sollen, sodass auftauchende Probleme früher erkannt und entsprechenden Handlungen vorgenommen werden können.
\newline\newline
Da die Abweichungen der Kosten bereits diskutiert wurden, werden nachfolgend die anderen Variablen der Kategorie der Kosten evaluiert. Die durchschnittlichen Kosten aus Nachlieferung für Fail-Projekte waren höher, was mit Mängel im Engineering, zeitverzögerter Beschaffung oder fehlerhaften Konstruktionen zusammenhängen kann. Die Kosten Nachlieferung verursachen Mehrkosten und beurteilen das Projekt in Bezug auf die Einhaltung der Kostenbudgetvorgaben, weshalb sie eher als Erfolgskriterium zu quantifizieren sind (Besteiero, Pinto und Novaski, 2015). Die Zusammensetzung des Budgets in dieser Form hat relativ wenig Aufschluss über den möglichen Einfluss auf den Projekterfolg gegeben. Es könnte beispielsweise eine kategoriale Variable erhoben werden, die in Abhängigkeit eines Schwellenwertes die Projekte in MS-, ME-, PA- und IS-Projekte unterteilt. Dadurch könnte lokalisiert werden, ob sich Fail-Projekte in einem gewissen Projekttyp konzentrieren. Unter der Berücksichtigung der Kundenwünsche ist die Einflussnahmen auf die Zusammensetzung des Budgets eher beschränkt. Obwohl diese Faktoren den Projekterfolg begünstigen können, liegt aufgrund des begrenzten Handlungsspielraum weniger ein Erfolgsfaktor vor. Wie bereits zuvor, kann sich diese Determinante zur Eingrenzung eines Monitoringbereichs eignen. Die Durchschnittswerte der Differenz zwischen dem letzten Forecast und den realisierten Kosten war bei Fail-Projekten für alle Projektphasen höher, insbesondere bei der Installationsphase. Dies kann als Hinweis für erhebliche Mehrkosten kurz vor dem Projektende oder eine zeitverzögerte Kommunikation einer sich verschlechternder Kostenprognose interpretiert werden. 
\newline\newline
Die evaluierten Einflussdeterminanten des Fulfillment-Prozess sind der Projektmanager, das Unternehmen und das Forecastmanagement. Die Verteilung der Anzahl nicht-erfolgreichen Projekte pro Alterskategorie ist nahezu uniform. Ältere im Vergleich zu jüngeren Projektmanager weisen eine tiefere Erfolgsquote aus. Das könnte damit zusammen hängen, dass erfahrenere Projektmanager tendenziell die risikoreicheren Projekte abwickeln. Diese Vermutung müsste zuerst mit Daten belegt werden, um eine abschliessende Beurteilung vornehmen zu können. Der Wechsel des Projektmanagers erfolgte in lediglich 43 Fällen, wovon 17 letztendlich scheiterten. Der überwiegende Teil nicht-erfolgreicher Projekte hatte während der gesamten Laufzeit genau einen Projektmanager. Aufgrund des fehlenden Signifikanztest des Zusammenhangs zwischen Erfolg und Wechsel des Projektmanager kann nicht abschliessend beurteilt werden, ob dieser Indikator als Erfolgsfaktor zu qualifizieren ist. Nichtsdestotrotz kann eingewendet werden, dass der Wechsel des Projektmanagers eher eine ungeplante Konsequenz aus vorgelagertem Handeln ist. Die Definition eines Erfolgsfaktor impliziert jedoch, dass während der Projektlaufzeit proaktiv die Aspekte des Erfolges mittels Entscheidungen für oder gegen eine Handlungsoption gesteuert werden können.
\newline Die Unternehmenskultur der für das Projekt verantwortliche Organisation kann Einfluss auf die Projektumgebung nehmen. Obwohl alle Produktionsstätten und Verkaufsorganisationen demselben Konzern angehören können personelle Unterschiede das Projektmanagement massgeblich beeinflussen. Die Auswertung der Lead SAS des Projekts zeigt, dass BUZ Verantwortungsträgerin für die meisten Projekte ist. Der Erfolgsquote liefert einen Hinweis, dass beispielsweise bei BBS im Vergleich zu BBAN mit ähnlichem Projektvolumen deutlich weniger gut abschneidet und BUZ trotz der häufigsten Fail-Projekte eine relativ gute Erfolgsbilanz ausweist. Die Unterschiede zwischen BBAN und BBS können beispielsweise Projektmanagementprozess induziert, durch Schwierigkeitsgrad der Projekte bedingt sein oder mit dem Personalressourcenmanagement in Verbindung gebracht werden. Ein interessantes Ergebnis liefert die Aufteilung der Gesamtprojektverantwortung und der Projektabwicklungsverantwortung. Obwohl insgesamt nur 101 Projekte mit geteilten Verantwortungsbereiche aufweisen, konnten 88 (84\%) erfolgreich beendet werden. Der Einfluss auf den Projekterfolg der unternehmensspezifischen Charakteristiken scheint intuitiv offensichtlich, wobei der Handlungsspielraum wie bereits bei den Regionen und den Geschäftsbereichen relativ begrenzt ist. Eine weiterführende Analyse beispielsweise der BBS-Projekte oder des Verhältnis bei geteilter Verantwortung könnte mehr Aufschluss über die Erfolgsattribute liefern. 
\newline\newline
Die Evaluation des Zeitmanagement ergab, dass die Mehrheit aller Projekte den Liefertermin nicht einhalten konnte. Das Verhältnis der Anzahl verspäteter Projekte bei MS8 im Vergleich zu MS2 ist genau spiegelverkehrt. Während bei MS2 die durchschnittliche Zeitverzögerung weniger Projekte im Tagesbereich lag, betrug sie für die Mehrheit der Projekte ab MS8 bis zu zwei respektive zwischen fünf und sieben Monate bei Projektende. Dabei lag der gemessene Rückstand bei Fail-Projekten nach dem Meilenstein 11 gegenüber 10 im Mittel um zwei Monate höher. Die Gründe für die Verspätung sind nicht bekannt, könnten aber entscheidende Hinweise für Optimierungspotenziale des Zeitmanagement liefern. Aufgrund der Wechselbeziehung zwischen Zeit und Kosten kann der Effekt auf den Projekterfolg nicht eindeutig bestimmt werden. Zudem wäre gemäss der Definition von Besteiero, Pinto \& Novaski (2015) die Zeitperformance eher als Erfolgskriterium einzuordnen. 
\newline\newline
Die Auswertung der Budgetperformance des Auftragsvolumen zum Zeitpunkt des Projektabschlusses hat ergeben, dass sowohl auf Regions- und Geschäftsbereichsebene das Mittel der nicht-erfolgreichen Projekte relativ höher war. Das Ergebnis sollte jedoch mit Vorsicht beurteilt werden, da erfahrungsgemäss die Zahlen des Auftragsvolumen der ersten gegenüber der zweiten Hälfte des Jahres eher unter den Budgetvorgaben liegen, sodass die Mittelwerte verzerrt sein können. Eine weitere Analyse von zwei Vergleichsgruppen, "Projekte Anfang des Jahres" und "Projekte Ende des Jahres" könnte mehr Erkenntnisse liefern. Dennoch kann der Unterschied zwischen nicht-erfolgreichen und erfolgreichen Projekten hinsichtlich der Erreichung der Budgetvorgaben als Indiz für die Priorisierung des Verkaufsabschluss gegenüber dem Risikopotenzial interpretiert werden. Dabei wird unterstellt, dass im SQ-Prozess beispielsweise Risiken und Kosten unterbewertet werden, sodass verhältnismässig günstiger verkauft werden kann. Zur abschliessenden Beurteilung der Auswirkungen des Budgetdrucks müssten die Offerten und die daraus resultierenden Effekte für den Abwicklungsprozess genauer untersucht werden. Die Eignung als Erfolgsfaktor kann aufgrund des fehlenden Signfikanztests nicht abgeschätzt werden.
\newline\newline 
Die Analyse Komplexitätsfaktoren hat gezeigt, dass die Anzahl involvierter Parteien relativ gering war, der Projektabwicklung mehrheitlich ein Aufträge zu Grunde lag und nur wenige Projekte mit Konsortium abgewickelt wurden. Da die Aufteilung eines Projekts auf mehrere Aufträge an kein Kriterium gebunden, ist die Aussagekraft der Anzahl Aufträge in Bezug auf die Komplexität relativ gering. Projekte mit Konsortium gibt es insgesamt 78, wovon 31 gescheitert sind. Davon wurden 14 in Europa, 7 in MEA\_Afr und 7 in China mehrheitlich von den Geschäftsbereichen GL, GM und VN abgewickelt. Das Merkmal Konsortium erklärt somit einen Teil der nicht-erfolgreichen Projekte der Regionen und Geschäftsbereich, die eine relative niedrige Erfolgsquote aufweisen. 
Die Auswertung der Supplying SAS über alle Projektphasen zeigte dass, Eigenproduktion und die Zusammenarbeit mit Drittlieferanten das häufigste Charakteristika von allen Projekten war. Gewisse Kombinationen hatten eine absolute Häufigkeit von nur einem Projekt. Die Aussagekraft der Unterschiede zwischen erfolgreichen und nicht-erfolgreichen Projekten ist hinsichtlich der Komplexität sehr beschränkt.
\newline\newline
Zusammenfassend kann ausgesagt werde, dass nicht-erfolgreiche Projekte folgenden Attribute aufweisen können:
% Table generated by Excel2LaTeX from sheet 'Sheet1'
\begin{table}[H]
	\centering
	\caption{Mögliche}
	\begin{tabular}{rlrlr}
		\multicolumn{1}{l}{\textbf{Region}} & \textbf{BA} & \multicolumn{1}{l}{\textbf{PMChange}} & \textbf{LeadSASPr} & \multicolumn{1}{l}{\textbf{Konsortium}} \\
		\multicolumn{1}{l}{EU} & GM    & \multicolumn{1}{l}{Yes} & BJHB  & \multicolumn{1}{l}{Yes} \\
		\multicolumn{1}{l}{NM} & GL    &       & BBS   &  \\
		& VN    &       & BMIL  &  \\
	\end{tabular}%
	
	\label{tab:addlabel}%
\end{table}%
Einschränkend muss ausgesagt werden, dass der Einfluss auf den Erfolg eines Projekts mittels einem statistischen Modell geschätzt werden muss.  
\newpage

\section{Frühwarnsystem}\label{sec:diskfru}
Die Erkenntnisse aus der Theorie des vorangehenden Kapitel setzen den Rahmen für die nachfolgende Ausführungen. Die zentralen Anforderungen an Frühwarnsignale sind zum einen, dass sie bereits zu einem frühen Zeitpunkt gemessen werden können. Die Implementierung eines ganzen Frühwarnsystems erfordert allerdings eine strategische Verankerung, da ein konstantes Monitoring und Screening, sowie eine anschliessende Auswertung und Interpretation der Daten notwendig ist. Ohne die Unterstützung des Managements wird die Durchsetzung eines solchen Vorhabens kaum durchsetzbar sein.
%DAten
Die Auswertungen der Einflussfaktoren des Kapitel \ref{drei} geben zwar Hinweise, was mögliche Attribute nicht-erfolgreicher Projekte sein können, allerdings fehlt es an einer statistisch begründeten Signifikanz des Zusammenhangs mit dem Erfolgskriterium. Nichtsdestotrotz können einige Faktoren bereits aufgrund ihrer Natur und des möglichen Erhebungszeitpunktes als mögliche Frühwarnindikatoren ausgeschlossen werden. Denn vorzugsweise sollen sogenannte "leading factors" fokussiert werden, zu denen sämtliche Performanceindikatoren, beispielsweise Kostenabweichungen, Kosten aus Nachlieferung oder Zeitverzögerungen gemäss (Zitat) nicht zählen. Die Rahmenbedingungen sowie auch der Projektmanager oder die Verantwortungsgesellschaften sind zwar zu Beginn des Projektes bekannt, verändern sich jedoch über die Projektlaufzeit kaum. Allerdings könnte nach einer entsprechenden Analyse ihres Einflusses auf die Erfolgswahrscheinlichkeit Projekte mit entsprechenden Attributen eher überwacht werden als andere. Ein solches Vorgehen kann dazuführen, dass andere Signale ausser Acht gelassen werden. Zudem wurden wie bereits gesagt, gewisse Faktoren in der Analyse nicht berücksichtigt, die möglicherweise auch als Frühwarnindikatoren funktionieren könnten. Aus diesen Gründen erscheint es erforderlich, dass neue Faktoren bezüglich ihrer Fähigkeit als Frühwarnsignal zu fungieren, ergründet werden. Es gibt jedoch keinen formalisierten Prozess, zu deren Identifikation. 
\newline Obwohl Projekt Assessments und seine Formen sowie Risikoanalysen bereits viele Hinweise zu möglichen Risiken und Chancen liefern, 
Bühler hat selbst für sogenannte Crash-Projekte Projekt-Assessments durchgeführt, die zu wichtigen Erkenntnissen für zukünftige Projekte geführt hat. Ausserdem wird am Ende jedes Projekts (Milestoen Debriefen) ein sogenanntes Debriefing abgehalten, welches implizierte, das Stärken und Schwächen eines jeden Projekts diskutiert wurden. Der Customer-Project Prozess hat neben anderen eine zentrale Schnittstelle vom Verkaufsprozess zum Fullfillment-Prozess. Basierend auf den Erkenntnissen der Literatur hat sich die Interface Analyse respektive die Beobachtung der
\paragraph{Sales \& Quotation} Schnittstellenthemen als relativ guter Frühwarnindikator erwiesen. Es liegt auf der Hand, dass der Output aus dem SQ-Prozess direkter Input im FF-Prozess bildet. Deshalb entstand die Idee, sozusagen die Frühwarnung für den Projekterfolg ab diesem Zeitpunk zu implementieren. Aus den Prozessabläufen der Bühler AG geht hervor, dass sowohl der SQ-Prozess und der FF-Prozess für Projekte grösser als eine Million umfangreiche Risikoanalysen gemacht werden. Vermutungsweise wird bereits zu diesem Zeitpunkt mögliche Erkenntnisse über zukünftige Herausforderung gewonnen, die wenig Beachtung erhalten. Fehlende Informationen, Assessments oder Dokumentation können als mögliche Frühwarnsignale interpretiert werden (siehe Klaggeg). Deshalb kann es von Nutzen sein eine Kennzahl zu entwickeln, welche auf automatisierte Basis die erforderlichen Dokumente gemäss den Anforderungen des MS1 beobachtet werden, sodass sichergestellt werden kann, dass keine Informationslücken entstehen. Somit wäre bereits früh klar, bei welchen Projekten alle relevanten Informationen zu Verfügung standen und der Übergabeprozess geglückt war. Auf Basis der in Kapitel \ref{sec:drei} erhobenen Daten hat sich gezeigt, dass Volumenmässig der Anteil an Projekte unter oder gleich einer Million fast die Hälfe aller Projekte ausmacht. Es wäre denkbar, dass bei diesen Projekten umfangreiche Risikoprüfungen ausgeblieben sind, da sie nicht priorisiert werden. Diese Annahmen und auch Probleme dieser Projekte müssten genauer untersucht werden, um andere Frühwarnindikatoren zu entdecken. Nichtsdestotrotz ist die Schnittstelle von SQ zu FF auch bei kleinere Projekte wichtig, damit das Projekt erfolgreich abgewickelt werden kann. In diesem Fall würde könnte es sich anbieten, eine Art Interface Analysis, die die Anzahl Interface-Themen und deren Bearbeitung in Bezug auf nur diese Schnittstelle misst, so dass es einerseits eine Plattform gibt, die Interface-Themen erfasst und offene/ungelöste Themen ersichtlich sind. Diese Idee liefert allerdings nur dem Projektabwicklungsprozess nähere Informationen, ob ein Projekte auf die schiefe Bahn gerät. Deshalb müssen auch für den weiteren Projekt-Management Prozess mögliche Ansätze diskutiert werden.
\paragraph{Projektabwicklung:} Im Propjektabwicklungsprozess sind weitere Schnittstellen vorhanden, welche genauer berücksichtigt werden müssen. Gemäss der Datenanalyse ist sowohl für Fail-Projekte und Success-Projekte der Mehrkostenanteil der IS-Kosten am höchsten gewesen. Unabhängig von der Ursache dieses Erscheinungsbild, ist die Installation die letzte Projektphase, so dass es von grossem Nutzen frühzeitig über mögliche Komplikationen Bescheid zu wissen, damit vorbereitende Massnahmen getroffen werden können. Zur Steuerung mittels Frühwarnsystem könnte ein Kombination aus Interfacemanagement und 'Gut Feelings' angewendet werden. Mittels Interfacemanagement soll sichergestellt werden, dass die auftauchenden Themen bearbeitet in nützlicher Frist bearbeitet werden. Der Ansatz der Gut Feelings hat zum Zweck, dass eine breiter Fokus für Variablen existiert, die einerseits während der Projektphase als mögliche Bedrohungen identifiziert werden und anderseits weder in den Risikochecks des SQ noch des FF inkludiert waren. Interne Dokumente belegen, dass der Projektmanager Erkenntnise aus der Risikoanalyse im Projektmanagement in detaillierter Form pflegen muss. Da erfahrungsgemäss ein gewisser Widerwille gegenüber umfangreichen Datenpflege festzustellen ist, sollte es im Tool eine Inputstelle geben für auf Intuition basierende Frühwarnsignale geben. Diese Anlaufstelle soll möglichst wertneutral, frei von Rechtfertigungsanforderung von übergeordneten Parteien, mit effizienter Handhabung und Zugang für sämtliche Projektteilnehmer ausgestattet sein. Dies ermöglicht dem Projektmanager eine Art Radar für zukünftige Herausforderung zu haben. Es sollte möglich sein, ein konstantes Monitoring pro Projekt ohne dabei detaillierte Angaben bereits erfassen zu müssen, sicherzustellen. Aspekte die berücksichtigt werden müssten sind die Strukturierung der Datenmenge sowie die Nutzung und Auswertung der Daten durch die Projektmanager.
\paragraph{Verschuldungsfrage:} Die vorangehende Analyse der Projekte wendet sozusagen ein schwarz-weiss Denken in Bezug auf den Erfolg ab. Allerdings konnte während der Analysephase festgestellt werden, dass es schwierig ist zu unterscheiden, welcher Projekttyp vorliegt. Beispielsweise werden Projekte gemacht, um neue Kunden zu gewinnen oder eine neue Technologie zu fördern, was zur Folge haben kann, dass die Kostenvorgaben relative zum Umsatz ambitiös ausfallen. Es würde jedoch Sinn machen, solche Projekte vom Standardgeschäft abgrenzen zu können, um mögliche Kernkompetenzen respektive Faktoren, die den Projekterfolg beeinflussen zu identifizieren. Zudem ist es wahrscheinlich, dass der Grund für die Mehrkosten nicht immer beim der Bühler AG liegen muss, unter dem Ausschluss, dass die Auswahl der Kunden von ihr beeinflusst wird. In Zusammenhang mit der Identifikation der Erfolgsfaktoren der Bühler AG könnte es folglich von Nutzen sein, Faktoren zu herben, die Aufschluss über kundenseitig induzierte Ursachen geben und wie anschliessend die Mehrkosten gehandhabt wurden. Zudem könnte die Befragung der Projektmanager und Verkaufsmanager mehr Aufschluss über zu berücksichtigenden Erfolgsfaktoren des Projektmanagements der Bühler AG geben. Hinzu kommt, dass aufgrund der Datenqualität wichtige Faktoren, wie zum Beispiel der Zeitpunkt der Forecast-Anpassung nicht ausgewertet werden konnten.
\paragraph{Projektprioritäten:} Aus einer finanziellen Perspektive die Fokussierung der Projekte mit grossem Umsatzvolumen, da ihr Einfluss auf das Ergebnis im Anlagengeschäft relativ gewichtig ist. Dennoch sollte der DB1 Verlust von Projekten mit einem Umsatzvolumen bis maximal 1.0 Mio. CHF nicht vernachläassigt werden, da sie am Projektvolumen eine relativ hohen Anteil haben. Wie bereits erwähnt wurde, werden Projekte unter 1 Mio. CHF von die internen Richtlinien zur vertieften Risikoanalyse der Bühler AG nicht erfasst.
%%% Text fehlt 
\paragraph{Incentivierung:} Die Anwendung von Frühwarnindikatoren bedingt, dass die Unternehmenskultur sowie auch die Projektmanagementstrategie entsprechend verändert respektive ausgerichtet wird. Die Implementierung von Frühwarnsignalen kann keine einmalige Übung darstellen, da es ein laufender Prozess ähnlich dem monatlichen Reporting ist. Die Abstimmung und Ausrichtung der Prozess und involvierten Personen auf ein gemeinsames Ziel "erfolgreiche Projekte" abzuschliessen ist dabei von grosser Wichtigkeit. Der Fokus sollte auf die Ergreifung von Massnahmen zur entsprechenden Gegensteuerung bei Komplikationen gerichtet sein anstatt auf die interne Schuldfrage. Eine sogenannte Fehlerkultur, die den Umgang mit Fehlern, Fehlerfolgen und Fehlerrisiken inkludiert, kann ein konstruktives Lernen aus Fehlern oder die Entdeckung effektiver Massnahmen bei Fehlerrisiken begünstigen. Alam Gühl (2016, S.20-21) plädiert im Rahmen der Projektkultur den positiven Umgang mit Fehlern und  den umfangreichen Austausch von entsprechenden Informationen, das von der Unternehmenskultur begünstigen werden kann. Ein anderes zentraler Einflussfaktor im Zusammenhang mit Frühwarnsignalen, ist auch die Fähigkeit des Projektmanagers zu Eingeständnissen, dass sein Projekt zu scheitern droht. Denn werden drohende Risiken und deren mögliche Realisation verkannt oder verhältnismässig spät kommuniziert, können der Handlungsspielraum eingegrenzt werden. Allerdings muss diese Denkweise aktiv im Unternehmen gefördert werden, damit das Bewusstsein der Fehlerakzeptanz, d.h. die positive Assoziation zu Fehlern und Scheitern gefestigt wird. 


%%Personen abhängig alli müsssen an einem Strang ziehen, Ausrichtung der Menschen an Zielen es Unternehmesn
%% Probleme Installation
%% Projektkategorisierung
%% Verschulden Bühler etc.
 \newpage
%Literaturverzeichnis
\bibliographystyle{apacite} %Wähle Zitierstandard
\bibliography{Literaturverzeichnis}
%\addbibresource{C:/Users/Michèle/Dropbox/Master/MA\06\_MA\_Files/MA\_Latex/Literaturverzeichnis.bib}
\newpage
%Anhang
\renewcommand\appendixtocname{Anhang}
\begin{appendices}
	\input{Anhang}
\end{appendices}

\end{document}

pdflatex MA.tex


kpsewhich lit.bib