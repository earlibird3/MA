\documentclass[11pt]{article}

\usepackage[utf8]{inputenc}
\usepackage[T1]{fontenc}

\usepackage[german]{babel}
\usepackage{nomencl}
\makenomenclature
\renewcommand{\nomname}{Abkürzungsverzeichnis}
\usepackage{hyphenat}
\usepackage{lscape}
\usepackage{array}
\usepackage{longtable}

\usepackage{blindtext}
\usepackage{graphicx}

\usepackage{geometry}
\geometry{a4paper,left=2.5cm,right=2.5cm,top=2.5cm,bottom=2cm}


\title{Einflussfaktoren und Frühwarnsystem im Projektmanagement der Bühler AG}

\author{Michèle Schoch}
\date{21. August 2016}

\begin{document}\selectlanguage{german}

%Titelseite
\begin{titlepage}
\maketitle
\end{titlepage}

%Inhaltsverzeichnis
\tableofcontents

%Beginn des Dockuments
\newpage
\printnomenclature

\nomenclature{DB1}{Deckungsbeitrag}
\nomenclature{Bud}{Budget}
\nomenclature{Act}{Actual}



%http://strobelstefan.org/?p=153



\newpage
\section{Einleitung}
\subsection{Ziel der Arbeit}
\subsection{Methodik und Vorgehen}


\section{Projektmanagement und Erfolgsfaktoren}

Der Erfolg von Projekten und deren Management ist ein in der Forschung viel diskutiertes Thema, weshalb hier in den nachfolgenden Kapiteln zunächst im Allgemeinen auf die Erfolgsdefinition von Projekten respektive Projektmanagement und bereits erforschte Projekterfolgsdeterminanten eingegangen wird. Im Anschluss folgt eine Erläuterung dieser Termini im Kontext mit der Bühler-Welt.

\subsection{Projet, Projektmanagement und Projekterfolg}

Gemäss dem Deutschen Institut für Normung (DIN) ist ein Projekt: " ein Vorhaben, das im Wesentlichen durch Einmaligkeit der Bedingungen in ihrer Gesamtheit gekennzeichnet ist, z.B. Zielvorgabe, zeitliche, finanzielle, personelle und andere Begrenzungen, Abgrenzung gegenüber anderen Vorhaben, projektspezifische Organisation". Daraus folgt, dass Projekte sich bezüglich einzelner Faktoren unterscheiden können, allerdings die Gesamtheit der Faktoren ihre Einzigartigkeit definiert. Beispielsweise begründet bei der Bühler AG die internationale Tätigkeit, das diverse Anlageportfolio und die breite Kundenbasis ein Indiz für einmalige Projekte. Obwohl es unterschiedliche Projekte gibt, beispielsweise im Tiefbau, Hochbau und Ingenieurbau und sich deren Management sowohl durch Differenzen als auch Gemeinsamkeiten charakterisiert, weist ein Projekt gemäss Projektmanagement-Handbuch (ohne Datum) folgende Eigenschaften auf: "komplexe, neuartige Aufgabenstellung, messbare Ziele und Ergebnisse, zeitliche Befristung (Anfang und Ende), begrenzte Ressourcen und die Notwendigkeit von Teamarbeit". Meyer und Rehrer (2012, S.2) sehen die progressive Elaboration, die eine kontinuierliche Konkretiesierung des Projekts während dessen Verlauf als weiteres Merkmal von Projekten. Der exakte Projektbegriff der vorliegenden Arbeit orientiert sich anschliessend an den geschäftsinternen Definitionen der Bühler AG
\newline
Der Managementbegriff wird vom Projektmanagementhandbuch(ohne Datum) als "systematischer Prozess zur Führung komplexer Variablen definiert. Er beinhaltet die Organisation, Planung, Steuerung und Überwachung aller Aufgaben und Ressourcen, die notwendig sind, um die Projektziele zu erreichen". Das Projekt Management Institute (PMI) (PMBOK, 2004) beschreibt Projektmanagement als eine Anwendung von Wissen, Fähigkeiten, Instrumente und Techniken bei Projektaktivitäten, um Projektanforderungen zu erfüllen. Nach Alama und Gühl (2016) wird Projektmanagement als "die Koordination von Menschen und der optimale Einsatz von Ressourcen zum Erreichen Projektzielen dargelegt. Pierce (S.2015, S.2) führt eine generelle Definition aus, gemäss derer Ziele, Prozesse, Planung und Kontrolle den Managementterminus beschreiben. Die Literatur zeigt keine einheitliche Definition, dennoch kann zusammengefasst konstatiert werden, dass Projektmanagement die zielgerichtete Planung, Steuerung und Überwachung von  Ressourcen in den Prozessschritten umfasst. Das Projektmanagementprozess kann trotz inhaltlich je nach Industrie, in folgende Schritte unterteilt werden; Projektinitiierung, Projektplanung, Projektdurchführung und -kontrolle, und Projektabschluss (PMHandbook, ohne Datum). An dieser Stelle wird nicht weiter auf die inhaltlichen Aspekte des Projektmanagements im Analagebau eingegangen, da der Datenanalyse und der Entwicklung von Frühwarnindikatoren der Bühler Projektmanagementprozess zu Grunde liegt, welcher im nächsten Unterkapitel erläutert wird.
\newline
Der Erfolg von Projekten beschäftigt die Forschung seit längerer Zeit. Das vorherrschende Paradigma zur Beurteilung des Projekterfolgs ist das magischen Dreieck, Zeit, Kosten und Qualität. Allerdings ist in der Praxis letztendlich der Kostenaspekt von zentraler Bedeutung, da er mit Geldverlust korreliert ist. Die akutelle Forschung der Erfolgsfaktoren zeigt, dass er Erfolg von Projekten nicht auf ein Faktor reduziert werden kann. Deshalb unterscheidet der Besteiero, Pinto \& Novaski (2015) zwischen kritischen Erfolgsfaktoren und Erfolgskriterien. Die Faktoren erhöhen ihrer Auffassung nach die Wahrscheinlichkeit des Erfolgs wohingegen die Kriterien darüber bestimmen, ob ein Projekt erfolgreich war (Besteiro, Pinto, Novaski, 2015). Hieraus entsteht eine Differenz in der zeitlichen Betrachtung, Faktoren sind während der Projektabwicklung relevant, um das Projekt erfolgreich abzuschliessen wohingegen Kriterien erst nach des Projektabschluss hinzugezogen werden, um über den Erfolg zu bestimmen. Folglich wäre der monetäre Aspekt ein Erfolgskriterium. Es muss jedoch hinterfragt werden, ob sich die Erfolgsbeurteilung von Projekten angesichts ihrer Eigenartigkeit auf ein Kriterium reduzieren lässt. Folglich wird die Vergleichbarkeit deren Erfolg durch die Projektdefinition in Frage gestellt. Damit Projekte verglichen werden können schlägt Autor (Datum) ein gewichtetes Erfolgskriterium vor, welches unterschiedliche Bestandteile der unternehmensinternen Erfolgsdefinition wiederspiegelt. Ebenso werden bei einer reinen Kostenbetrachtung die Einhaltung von Zeitvorgaben und die Qualitätsanforderung ausser Acht gelassen. Kerzner (2014) schlägt vor anstatt der traditionellen Erfolgsbetrachtung vor, den Projekterfolg als die Erreichung des gewünschten Geschäftswertes innerhalb der sich konfligierenden Zielvorgaben zu verstehen. Dabei  unterstellt er eine neue Projektdefintion: Ansammlung nachhaltiger Geschäftswerte, deren Realisierung terminiert ist. Davon ausgehend erkennt Kerzner (2014), dass bei der Erfolgsrealisierung immer ein trade-off erfolgen muss, da beispielsweise bei einer Zeit- und Kostenüberschreitung ein Projekt nicht zwingend ein unerfolgreiches Projekt darstellen muss, da Wissen generiert werden konnte, welches in anderen Projekten oder in anderen Bereichen von Nutzen sein kann. Somit schlägt er einen vierstufige Erfolgsdefinition vor, bei der Projekte in vier Kategorien, gesamtheitlicher Erfolg, Teilerfolg, Teilscheiter und gesamtheitliches Scheitern. Die der Analyse und dieser Arbeit zugrunde liegenden Erfolgsdefinition respektive Erfolgskriterium wird durch die Bühler AG festgelegt. Allerdings lässt sich aus den vorherigen Ausführungen schliessen, dass die Erfolgsdefinition kritisch betrachtet werden muss und zunehmend entwurzelt wird, da die nicht monetären Wertegenerierung eines Projekts bisher kaum Bestandteil des Erfolges war. Zudem ist die Einhaltung der Kosten- und Zeitvorgaben in einem Projekt eine herausfordernde Aufgaben, da sich während des Projekts Änderungen ergeben, die zu einem Kostenanstieg und Zeitverzug führen können. 
\newline
Wie bereits eingangs erwähnt, wurden einige Erfolgsfaktoren mit unterschiedlichen statischen Methoden erforscht. Grundsätzlich liegt dem Konzept der Erfolgsfaktoren die Prämisse, dass Erfolg wiederholbar ist und an bestimmte Faktoren geknüpft ist, zugrunde. Folglich müsste es auch Determinanten geben, welche den Projekterfolg negativ begünstigen. Iyer \& Jha (2006) fanden mittels Expertenbewertungen und anschliessender Faktoranalyse heraus, dass das Engangement der Projektmitarbeiter und die Fähigkeiten des Projekteigners sich positiv auf die Zeitperformance auswirken. Konflikte zwischen dem Projektmanager und Top-Management, dem Projekteigner oder anderen externen Parteien sowie ein Missgunstkultur können zur Überschreitung der vorgegebenen Projektzeit führen. Chan, Ho \& Tam (2001) haben bereits in einer früheren Studie aus 31 möglichen Erfolgsfaktoren mittles der Faktorenanalyse auf fünf relevante Erfolgsfaktoren geschlossen. Basierend auf einer fünfstufigen Erfolgsskala hat ihre Analyse ergeben, dass das Engagement des Projektteams, welches Kooperation, Konfliktlösung, Vertrauenskultur, Verständnis der Projektziele sowie Kommunikation miteinschliesst, als kritischer Erfolgsfaktor zu qualifizieren ist (Chan, Ho \& Tam, 2001). Dieses Ergebnis ist mit den Befunden früheren Studien (Ashley, et al. 1987, Pinto und Slevin, 1988) kompatibel, bei denen das Engagement der Projekteilnehmer ebenso als erfolgskritisch identifiziert wurde. Die Analyse von Chan, Ho \&Tam (2001) hat zudem ergeben, dass sowohl die Fähigkeiten des Auftragsnehmers wie zum Beispiel, die Qualität des Projektmanagement oder die Anwendung innovativer Technologien, als auch die Kompetenz des Kunden, Konstruktionsprojekte abzuwickeln, eine entscheidende Rolle für den Projekterfolg haben. Die Unterstützung der des Managements, Kommunikation oder die Projektmission bei Change- oder IT-Projekten von grosser Bedeutung (Hyvräri, 2006 in Besteiro, Pinto, Novaski, 2015). Varajão, Dominguez \& Ribeiro et al. (2014) untersuchten, ob zwischen Software- und Konstruktionsprojekten Differenzen bezüglich der Erfolgsfaktoren existieren. Mittels einer Likert-Skala und bereits in früheren Studien entdeckten Erfolgsdeterminanten konnte festgestellt werden, dass die Projektplanung und das Verständnis der Projektziele sowie -anforderungen bei beiden Projektarten kritische Erfolgsaspekte sind. Allerdings wird beispielsweise die Effizienz des Projektmanagements und der Miteinbezug aller Projektteilnehmer bei Konstruktionsprojekten als kritischer für den Erfolg erachtet als bei Software-Projekten. Lam, Chan \& Chan (2008) haben ein anderes Erfolgskriterium, welches Kosten, Zeit, Qualität und Funktionalität in einem gewichteten KPI zusammenfasst, zur Bestimmung der kritischen Erfolgsdeterminanten herangezogen. Ihre Analyse hat ergeben, dass die Projektnatur, die Effizienz des Projektmanagement und die Anwendung innovativer Managementtechnologien einen erfolgreichen Projektabschluss begünstigen (Lam, Chan \&Chang, 2008). Gemäss ihrer Aussage würde die Einbringung des Auftragseigners, die Attraktivität und Komplexität des Projekts, kurz die Projektnatur), die Projektmanager dazu veranlassen, mehr Effort für das Projekt zu leisten, da solche Projekte mit Prestige und Selbstverwirklichung verbunden sind (Lam, Chan \& Chan, 2008). Mittels logistischer Regressionsanalyse erforschten Lu, Hua \& Zhang (2017) erforschten die Erfolgsfaktoren aus einer Kostenperspektive. Hierbei wurde festgestellt, die Fähigkeiten des Auftragsnehmers, welche vergleichbare Erfahrungen, Teamfähigkeit und Kostenaffinität, inkludieren, relevant für den Erfolg des Projektes sind. Die obigen Ausführungen fassen die Ergebnisse zahlreicher Studien zu Erfolgsfaktoren zusammen. Trotz unterschiedlicher Erhebungsmethoden und Projekterfolgsdefinitionen wurden sich überschneidende Determinanten identifiziert. Abweichungen können aufgrund unterschiedlicher Projektarten auftreten. Zusammenfassend kann postuliert werden, dass der Erfolg stark mit den jeweils im Projekt involvierten Personen und der Projektumgebung zusammenhängt. Attribute wie, Fehlerkultur, Teamfähigkeit, Konfliktlösen, Vertrauen, gemeinsame Mission oder Wertschätzung stellen nur eine Auswahl dar, um die Projektteilnehmer und das Arbeitsklima zu beschreiben, welche den Projekterfolg begünstigen. Alam \& Gühl (2016) sprechen in diesem Zusammenhang auch von Anforderungen, die während jeder Projektphase gegeben sein müssen, damit Projekte erfolgreich bearbeitet werden können. Ob und wie erfolgreich Projekte letztendlich waren, ist zudem immer von der Erfolgsdefinition abhängig. Erfolgskriterien, die nur auf einer Variablen gründen, erschweren die Vergleichbarkeit von Projekten wohingegen zu komplexe Erfolgskriterien zu Missverständnisen und Unübersichtlichekeit führen können. Im Moment ist ein Trend spürbar, der die Erfolgsperzeption vom magischen Dreieck aufweicht und andere Werte mitberücksichtigt. Dies würde wiederum verschiedene Aspekte wie eine Fehlerkultur fördern. Dieser Abschnitt diente lediglich dazu, bisherige Forschungsergebnisse aufzuzeigen. Im folgenden Kapitel wird näher auf das Untersuchungsobjekt, das Projektmanagement der Bühler AG, eingegangen und die der Analyse zugrunde liegenden Prämissen verdeutlicht. 

\subsection{Bühler Projektmanagement}

Das Bühler Projektmanagement subsumiert in dieser Arbeit sämtliche Prozesse, Performancemessungen und Einflussfaktoren, welche mit dem Verkauf und der Projektabwicklung im Anlagengeschäft der Bühler AG zusammenhängen. Im ersten Unterkapitel wird der C2C-Prozess der Bühler AG erläutert. Dieser Prozess setzt sich aus dem SQ- und FF-Prozess zusammen und bildete die Grundlage für die Identifikation der Einflussfaktoren des Projekterfolges. Anschliessend werden im zweiten Unterkapitel die Einflussfaktoren erläutert. Somit wird in diesem Kapitel auf deduktive Weise die prozesstechnische Grundlage der Analyse erarbeitet.


\subsubsection{Der Projektmanagementprozess}\label{zweieins}

Dieses Kapitel dient dazu das Projektmanagement der Bühler AG zu erläutern. Die folgenden Ausführungen sind insofern relevant, da das Modell zur Bestimmung der Treiber für den Projekterfolg davon abgeleitet wird. 
\newline
Der ganze Prozess vom Verkauf des Projekts respektive Maschine oder Anlage bis zum Projektabschluss, die Übergabe der Anlage an den Kunden, ist in zwei grössere Prozesse unterteilt. Der Sales and Quotation (SQ) und der Project Fullfillment (FF) Prozess. Die beiden Prozesse sind aufeinanderfolgend, wobei der Fokus allerdings auf dem Project Fullfillment Prozess liegt. Nichts desto trotz, soll SQ-Prozess kurz erläutert werden. Der SQ-Prozess gliedert sich in die folgenden vier Phasen: Identify Potential, Set Priorities, Quote and Evaluate Risk und Close Order. Der Prozess wird durch den Area Sales Manager (ASM) sowie durch die Business Unit mit den entsprechendne Centers of Competences betreut, wobei klar geregelt ist, wer welche Verantwortlichkeiten bis zu welchem Zeitpunkt, Miles Stones, hat. In der Phase I geht es darum Geschäftspotenziale und Kundenbedürfnisse zu identifizieren, Kontakte mit den Kunden aufzunehmen und entsprechende Marketingkampagnen zu initiieren. Da nicht aller Geschäftspotenziale realisiert werden können, soll die zweite Phase dazu dienen Prioritäten zu setzen und mittels finanzieller Analysen zur Profitabilität und der konzeptionellen Ausgestaltung des Geschäftsidee unterstützt werden. Das Konzept beinhaltet unter anderem die Konkretisierung der Idee inklusive einer der Analyse der Wertgenerierung für den Kunden sowie den eigenen Ressourcenaufwand, eine Preiskalkulation, ein Check der technologischen, finanziellen, und zeitlichen Durchführbarkeit des Projekts und Tests bezüglich der Integrität und Bonität des Klienten. Die Phase II endet mit dem Entscheid, ob eine Offerte gemacht werden soll oder nicht. Der erste Mile Stone des SQ-Prozess. - Sales Opportunity assessed -  wäre somit erfüllt. Typischerweise ist bis zur Erreichung des ersten Mile Stones der ASM verantwortlich. Der Übergang in die Phase III beginnt mit einer detaillierten Projektuntersuchung, die sowohl technische, kommerzielle und risikobasierte Aspekte umfasst. Der erste Teil der Phase drei endet somit mit der Erreichung des zweiten Mile Stones  - Quotation requested und liegt sowohl in der Verantwortung der Business Unit und des ASM. Bevor es in die Verhandlungsphase (Phase IV) übergeht, liegt die Entwicklung eines Basiskonzepts und dessen Genehmigung im Verantwortungsbereich des Geschäftsbereich (Mile Stone 3 - Basis Concept approved). Zudem müssen vor der Verhandlung mögliche Risiken geprüft werden, um eine Verhandlungsgrenze festlegen zu können (Mile Stone 4 - Negotiation Mandated). Die letzte Phase umfasst die Aktivitäten der Vertragsverhandlungen zwischen dem Kunden, dem Geschäftsbereich und dem ASM sowie die anschliessenden Sicherstellung der Kundenzahlungen. Das Ende des SQ-Prozess bildet der fünfte Mile Stone -  Orders Released, das heisst der Auftrag wird freigegeben! (Prozessportal)
\newline
Der fünfte MS ist der letzte Schritt im SQ-Prozess und zugleich initiiert er den darauffolgenden FF-Prozess. Der Prozess gliedert sich in fünf Phase und zwölf Schritte, der zwölf MS, wovon allerdings nur die für das Modell relevanten Steps und MS ausführlicher erläutert werden sollen.
\newline
\textbf{Phase I: Planning and Basic Engineering}
\newline
Der erste Schritt Hand-over-Meeting (HOM) kennzeichnet die Übergabe des Projekts vom SQ zum FF. An diesem Meeting nehmen typischerweise der AM, der PM und der Teamleiter teil. Es dient dazu, eine reibungslose Übergabe vom Verkauf ans Engineering zu bewirken und allenfalls offene Punkte zu klären sowie sämtliche projektrelevanten Dokumenten zu übergeben. Dieser Schritt endet mit dem Mile Stone HOM fulfilled. Die nachfolgenden Prozessschritte der Phase eins dienen zusammengefasst dazu, das Projekt genauer zu analysieren, das Konzept zu überarbeiten, die Projektstrukturfestzulegen sowie abschliessend das Kick-Off-Meeting durchzuführen. Die Projektanalyse dient dazu sowohl technische als auch kommerzielle Risiken und Chancen zu identifizieren und zu beurteilen sowie entsprechenden Massnahmen daraus abzuleiten. Dieser Schritt hat zum Zweck, die Realisierbarkeit mittels der oben genannten Vorgehensweise zu testen. Im Schritt Concept Approval Internal/Customer geht es darum, das Anlagen- sowie das Maschinenkonzept zu überarbeiten und sowohl intern als auch extern, d.h. vom Kunden genehmigen zu lassen. Dieser Phase ist mit dem zweiten MS Concept approved, das sowohl die interne und externe Genehmigung inkludiert abgeschlossen. Nach der Konzeptannahme kann mit der Strukturierung und der Organisation des Projekts begonnen werden, die mehrheitlich organisatorische und administrative Aufgaben, die dem Projektmanagement dienen, umfasst. Der Abschluss der Phase I bildet das Kick-off-Meeting, welches der Schaffung eines gemeinsamen, identischen Verständnis unter sämtlichen Teilprozessverantwortlichen bildet. Im KOM werden in Abstimmung mit den vertraglichen Bedingungen, verbindliche Vereinbarungen bezüglich, Termine, Kosten, Qualität und Zuständigkeiten getroffen. Ausserdem bietet das KOM Raum zur Diskussion unklarer Punkte. In der Regel findet das KOM nach der kommerziellen respektive Gesamtfreigabe statt.
\newline
\textbf{Phase II: Engineering and Specification}
\newline
Der Prozessschritt sechs, Main Layouts, hat die Ausarbeitung verbindlicher Pläne zur Anlagen- oder Maschinendisposition zum Ziel. In dieser Phase ist vor allem wichtig, dass nochmals Optimierungen stattfinden und das Endergebnis einem internen Review unterzogen wird. Das anschliessende Design Meeting zwischen dem Kunden und den Projektverantwortlichen sowie dem AM bezweckt die Vorstellung der Pläne sowie eine schriftliche Einverständnisklärung des Kunden. Zudem sind ab diesem Zeitpunkt die Änderungen dem Risiko von Mehrkosten, Zeitverzögerung und anderen allfälligen Anpassungen in der Projektstruktur ausgesetzt. Der Abschluss der zweiten FF-Phase bildet der Schritt 8a, Detail Engineering (Mechanical and Automation), das der Detailausarbeitung der Anlagen- und Maschinendispositionen unter der Berücksichtung kundenspezifischer Spezifikationen dient. In dieser Phase wird der MS5 erreicht "Point of now return", welcher voraussetzt, dass Liefertermine mit dem Kunden bestätigt werden und die Kundenzahlung gesichert ist. 
\newline
\textbf{Phase III: Manufacturing, Procurement and Logisic Out}
\newline
Im Prozessschritt 8b geht es darum die Fabrikation mit der eigenen Werkstätte oder Dritten zu koordinieren, sowie die Lieferung an den Kunden sicherzustellen. Es ist hierbei wichtig, die Lieferversprechen einzuhalten und eine vertragskonforme Ablieferungen sicherzustellen. Der nachfolgende Prozessschritt 9 Project Documentation umfasst die Erstellung der Projektdokumentation für den Monteur und den Kunden, welche die Nachvollziehbarkeit der Änderungen garantiert. Das Ende dieser Phase wird durch den MS8 festgehalten, der erreicht wird, wenn die Dokumentation offiziell an den Kunden und den Monteur geschickt wurde. 
\newline
\textbf{Phase IV: Installation and Start up}
\newline
Diese Zeitspanne setzt sich aus den S10, Installation, und S11, Start-up/Take-over, zusammen. Der Zusammenbau einer Anlage respektive die Inbetriebsetzung einer Maschine, erfordert eine Instruktion des Montageteams. Diese Einleitung, möglichst auf Platz, gilt als unabdingbare Voraussetzung, um ein gewisses Qualitätsniveau der Montage und Installation zu gewähren. In diesen beiden Phasen stehen der Projektleiter und der AM in ständigem Austausch mit den Montageverantwortlichen, sodass Unterstützung bei Unklarheiten und Überwachung des Prozesses gewährleistet sind. Nach dem Abschluss der Inbetriebsetzung, erfolgt die Übergabe an den Kunden. Hierbei ist darauf zu achten, dass möglichst alle vertraglich vereinbarten Anforderungen, wie zum Beispiel Tests und Inhalt, Umfang und Darstellung der Übergabedokumente erfüllt werden, da oftmals die letzten Kundenzahlungen an die Leistungserfüllung gekoppelt sind. 
\newline
\textbf{Phase V: Evaluation and Transfer}
\newline
Als letztes folgt das Debriefing, das im Sinne des kontinuierlichen Verbesserungsprozess, werden Rückmeldungen aus der Praxis gesammelt und Verbesserungspotenziale in der Projektabwicklung identifiziert, so dass bei einem nächsten Projekt nicht wieder die gleichen Fehler unterlaufen oder bereits eine geeignetere Methode vorliegt, um auf entsprechende Situationen zu reagieren. Der MS12 ist erreicht, sobald die Garantieperiode von 2 Jahren endet. 
\nomenclature{SQ}{Sales and Quotation}
\nomenclature{MS}{Mile Stone}
\nomenclature{FF}{Fullfillment}
\nomenclature{HOM}{Hand-over-Meeting}
\nomenclature{KOM}{Kick-off-Meeting}

\subsubsection{Einflussfaktoren des Projektmanagementprozesses}\label{zweizwei}
Einleitend zu diesem Kapitel wurde erwähnt, dass der C2C-Prozess die Grundlage für die Identifikation der Einflussfaktoren im Projektmanagement der Bühler AG diente. Das BPM-Team, welche zuständig für das Projektreportings ist, hat aufgrund ihrer Erfahrung mögliche Einflussfaktoren aus finanzieller Perspektive ergründet. Die Idee war, aus einer Finanzperspektive zu erforschen, welche Faktoren, Merkmale erfolgreiche Projekte von nicht erfolgreichen Projekten unterscheiden. Zudem wurde beabsichtige, mittels statistischer Analyse einen signifikanten Kausalzusammenhang zwischen den Einflussfaktoren und dem Projekterfolg zu erforschen. Somit könnten die relevanten Faktoren vor respektive während der Projektphase beobachtet werden, so dass bereits früh erkannt werden kann, ob Projekte erfolgreich abschliessen werden oder nicht. Die nachfolgende Betrachtung der identifizierten Faktoren ist unternehmensspezifisch und berücksichtigt aufgrund der gewählten Perspektive einige Determinanten, welche für den Projekterfolg von Bedeutung sein könnten, nicht. 
\newline
Bühler managet die Projekt mittels SAP und dem eigens konstruierten Projektmanagement-Tool, BPM-Cockpit. Das Cockpit enthält Informationen zu Kosten, Zeit, Verantwortlichkeiten, Risiken und Aktionen und zum Engineering inklusive verschiedener Projektstatus. Das System arbeitet mit einem dreifarbigen Ampelsystem für Kosten, Zeit und Qualität. Die erste beiden werden vom System automatisch gerechnet wohingegen die Qualität auf der subjektiven Einschätzung der Projektmanager beruht, die sie selbst berichten können. Basierend auf der internen Guideline für das BPM-Cockpit ist der Kostenstatus grün, wenn die konsolidierte Abweichung der Projektmarge gemessen in Prozent zwischen dem Forecast und dem Budget grösser als -400 Prozentpunkte beträgt. Dieser Status ändert auf gründ auf gelb, sobald die Abweichung mehr als -4\% beträgt und von gelb auf rot, wenn die Schwelle von -10\% überschritten wird. Die Zeitampel basiert einerseits auf der Differenz zwischen dem realisierten und dem geplanten Termin und anderseits auf der Eintragung im System. Sofern Angaben zum realisierten Termin im System enthalten sind, ist der Status grün, wenn er vor oder auf geplanten Datum liegt und gelb, wenn das Zeitversprechen nicht eingehalten wurde. Die rote Farbe impliziert, dass das BPM-Cockpit keine Informationen zur Erfüllung des Termins enthält und der geplante Termin vor dem heutigen Datum liegt.
\newline
Grafik!!!!!!
\newline
Die Performance eines Bühler-Projekts wird mittels des magischen Dreiecks - Time, Cost und Quality -  ermittelt. Letztendlich hat der Kostenaspekt aus finanzieller Perspektive die relativ gewichtigere Bedeutung als die anderen zwei Dimensionen. Trotz der Anforderung den Fertigstellungstermin des Projekts einzuhalten, liegt der Fokus vor allem auf den Projektkosten und der Projektmarge. Somit wird der Erfolg eines Projektes respektive Geschäftsbereichs der Bühler AG basierend auf der Abweichung zwischen der realisierten und der budgetierten Marge gemessen. Die Bedeutung des finanziellen Performance wird zudem durch die monetäre Incentivierung der Projekt- und Salesmanager sowie Geschäftsbereichsleiter auf Basis der Deckungsbeitragsmarge und realisiertem Auftragsvolumen hervorgehoben. Deshalb wurde für die anschliessenden Erläuterung und Analyse der Faktoren (Kapitel \ref{drei}) das Erfolgskriterium "Deviation DB1-Marge in \%" festgelegt. Der KPI rechnet sich realisierte Marge (DB1 Act) in \% - budgetierte Marge (DB1 Bud) in \% und wird jeweils am Projektende respektive nach Erreichung des MS11 (Kapitel \ref{zweieins}) kalkuliert. Die ursprünglich 70 Determinanten wurden in sechs Kategorien unterteilt, um eine übersichtliche Darstellung vornehmen zu können. Einige Faktoren könnten jedoch in zwei Kategorien unterssteilt werden. 
\newline
\begin{figure}[ht!]
	\centering
	\includegraphics[width=90mm]{Model.jpg}
	\caption{Einflussfaktoren \label{Einflussfaktoren}}
\end{figure}
\newline
Jedes Projekt hat eine Identifikationsnummer, die BPM-ID, die es ermöglicht ein Projekt eindeutig zu identifizieren. Sie zählt jedoch nicht zu den Erfolgsdeterminanten.
\newline
\newline\textbf{Rahmenbedingungen:} In dieser Kategorie sind beschreibende Variablen zusammengefasst, die den eindeutigen Rahmen eines Projekts festlegen. Dazu gehören die Region respektive das Land, in welchem die Anlage gebaut wird, der Kunde und der Geschäftsbereich. Diesen Variablen liegt die Annahme, dass gewisse Charakteristiken den Erfolg eines Projekts begünstigen. Beispielsweise lassen sich die Kunden bezügliche der individuellen Anlagespezifikationen oder ihrer Bonität unterschieden. Ausserdem können kulturelle Differenzen und landesspezifische Regelungen des Handels oder Finanzmarktes die Abwicklung eines Projekts in bestimmten Regionen erschweren. Obwohl bühlerintern der gleiche Leitfaden für das Projektmanagement Gültigkeit hat, können geschäftsbereichbezogene Abweichungen entwickeln. Da jeder Geschäftsbereich seine eigenen Verkaufs- und Projektmanager hat, sind auch die personellen Einflüsse in der Projektphase unterschiedlich. In Bezug auf die Rahmenbedingungen stellt sich die Frage, ob Unterschiede oder Muster in Abhängigkeit vom Projekterfolg vorhanden sind. Die Wichtigkeit eines Projekts, wird auch den Rahmenbedingungen untergeordnet, da sie Aussagen über die Priorisierung und Effort innerhalb eines Geschäftsbereichs machen kann. Hierfür wird das Umsatzbudget des Projekts ins Verhältnis zum Median aller Projektumsätze eines Geschäftsbereichs gesetzt. In diesem Zusammenhang können zwei Betrachtungen vorgenommen werden. Das Umsatzbudget eines Projekts ist bereits während der SQ-Phase bekannt, weshalb ab diesem Zeitpunkt Klarheit über die relative Bedeutung des Projekts für den Geschäftsbereichs gegeben ist. Folglich können die Verkaufsentscheidungen zusätzlich beeinflusst werden, wenn die Erreichung des Budgetziels für den Umsatz unter Druck stehen. 
\newline\newline\textbf{SQ:} Davon ausgehend, dass bereits in der vorgelagerten SQ-Phase Entscheidungen getroffen werden, welchen den Projekterfolg beeinflussen können, wurden personelle Determinanten und Faktoren zum Einfluss von Unternehmensziele ins Modell miteinbezogen. Das Alter des Sales Manager und dessen Betriebszugehörigkeit zu Bühler in Jahren wurden als Proxy-Variablen für die Arbeitserfahrung respektive das Bühler-Know-how herangezogen. Bei der letzteren Variable wird die gesamte Arbeitszeit bei der Bühler AG berücksichtigt, was die Aussagekraft insofern abschwächt, als dass die Erfahrung im Projektmanagement bei der Bühler AG nicht explizit erfasst wird. Die Vorgaben zur Erreichung des Auftragsvolumen können dazu führen, dass Projekte verkauft werden, welche die Profitabilitätserwartungen in ungenügender Weise erfüllen. Intuitiv wird davon ausgegangen, dass solche Projekte eher zum Scheitern verurteilt sein, weshalb die Abweichung von den Unternehmenszielen zum Zeitpunkt der Freigabe des Projekts als möglicher Einflussfaktor auf den Projekterfolg ergründet wurde. Da die Verkaufsmanager in einem geografischen und geschäftsbereichbezogenen Matrix agieren, wurde der Zielstatus sowohl von der Regionen- als auch der BU-Dimension gemessen. Ein weiteres Indiz für Unklarheiten bezüglich des Projektinhalts bildet die Zeitspanne zwischen Projektfreigabe und Projektstart. Die Verzögerung des Projektbeginn kann darauf hindeuten, dass ein komplexes Projekt vorliegt, die Finanzierung noch nicht eindeutig sichergestellt ist oder noch weitere Spezifikationen vorgenommen werden müssen. Die Zeitspanne errechnet sich aus dem Auftragsfreigabe- und Projektstartdatum.
\newline\newline\textbf{FF:} Diese Kategorie subsumiert sämtliche personelle, organisatorische und managementbezogene Faktoren des Fulfillment-Prozesses. Wie bereits bei den Verkaufsmanager werden Alter und Betriebszugehörigkeit in Jahren des Projektmanagers als Erfolgsdeterminanten angesehen. Der Wechsel des Projektmanagers während der Projektlaufzeit kann darauf hindeuten, dass es Differenzen zwischen dem Kunden und Projektmanager gibt oder das Projekt droht eine schlecht Performance zu zeigen. Deshalb wird der Wechsel des Projektmanagers als Indiz für erfolgreiche respektive nicht erfolgreiche Projekte verstanden. Der organisatorische Rahmen für das Projektmanagement gibt die für das Projekt verantwortliche Gesellschaft vor. Es wird unterschieden zwischen Gesamtverantwortung für das Projekt und Verantwortung für die Projektabwicklung (FF), wobei diese Parteien für ein Projekt identisch oder verschieden sein können. Sobald zwei Unternehmen involviert sind, erhöht sich der Komplexitätsgrad eines Projekts, da die Verantwortlichkeiten getrennt sind und eine zusätzliche Schnittstelle zwischen zwei Gesellschaften entsteht. Zudem wirken unter Umständen zwei verschieden Unternehmenskulturen auf die Projektkultur ein, welche diese begünstigen oder beeinträchtigen kann. Finanzielle Aspekte im Projektmanagement beziehen sich bei Bühler auf den Forecast, dessen Anpassung in den Verantwortlichkeitsbereich eines jede Projektmanagers fällt. Die monetäre Komponente einer Forecast-Anpassung kann darauf hindeuten, dass eine Projekt Gefahr läuft mit negativer oder ungenügender Performance abzuschliessen, sofern lediglich Mehrkosten entstehen. Die Ursachen eines Kostenüberschuss sind vielfältig, beispielsweise fehlerhafte Produktion, falsche Spezifikationen oder Mehraufwand bei der Installation. Die Zeitkomponente eines Forecast gibt darüber Auskunft, ab welchem Zeitpunkt die Kosten unter Druck geraten. In der Regel wird davon ausgegangen, dass bei früherer Kenntnis entsprechende Gegenmassnahmen eher Wirkung zeigen, da sie früher ergriffen werden können. Aus diesen Gründen wurde pro Projekt und pro Kostenbudget, Mechanical Supply, Mechnical Engineering, Plant and Automation und Installation, erhoben, ob in welcher Weise der Forecast sich verändert hat und wann er gemacht wurde. Der Fokus liegt dabei auf der Zeitspanne zwischen der negativesten FC-Anpassung und dem Projektende, womit ergründet werden soll, wie viele Monat vor Projektschluss Klarheit über eine einflussreiche Kostenveränderung vorhanden war. Die Bewertung der Ampelstatus wird in dieser Kategorie eingeordnet, da die direkte Projektmanagementperformance vom Ampelstatus sowohl während als auch nach der Projektlaufzeit abhängt. Es wurde für die drei Ampeln ausgewertet, wie viele Monate zwischen HOM und dem Datum des ersten Wechsels von grün auf gelb oder rot liegen. Diese Differenz wird ins Verhältnis zur budgetierten Projektlaufzeit gesetzt, so dass der Indikator eine Aussage darüber macht, nach wieviel Prozent der budgetierten Projektzeit der Status gewechselt hat. Hiermit lässt sich feststellen, ob bereits in einem frühen Stadium die erwarte Performance hätte abgeschätzt werden können. Je früher, beispielsweise der Kostenstatus auf gelb oder rot gewechselt hat, desto früher wurden Forecast-Anpassung gemacht und desto früher hätten Gegenmassenahmen ergriffen werden könne. Ähnliche Überlegungen können für die Zeitbetrachtung gemacht werden, wobei hier eventuell bestimmte Muster auftreten können, beispielsweise sämtliche Zeitverzögerungen erfolgen relative spät. 
\newline\newline\textbf{Monetäre:} Die monetären Aspekte beziehen sich auf Umsatz und Kosten. Das Umsatzvolumen gilt in diesem Modell als Indikator für die Grösse des Projekts, wobei höhere Budgets mit komplexeren und umfangreicheren Projekten assoziert werden. Darüber hinaus besteht die Vermutung das der Budgetmix einen Hinweis auf die Eigenschaften nicht erfolgreicher Projekte könnte.
\newline Nachgelagerte Faktoren sind vor allem auf Abweichungen von den Kostenbudget von Interesse. Wie bereits erwähnt, hat die finanzielle Komponente relativ mehr Fokus für Bühler. Aus diesem Grund wurden als mögliche Einflussfaktoren sämtliche Kostenabweichungen in absoluter und relativer Höhe in das Model miteinbezogen. Diese Grössen sind direkt mit dem finanziellen Erfolg korreliert, so dass es hierbei darum ginge, eine statistische Signifikanz zu ergründen. Die Kosten werden zudem durch sogenannte Subsequent Deliveries (SU), die nach dem offiziellen Liefertermin nachgeliefert werden müssen, beeinflusst. Die Höhe der SU wird ins Verhältnis zum Umsatz gesetzt. Je höher der SU-Anteil desto eher könnte das Projekt weniger erfolgreich abschliessen.
\newline Bühler schrieb während des Betrachtungshorizontes einen Mindestdeckungsbeitrag von 23\% vor. Projekte die ein Budget unter diesem Werte haben, bedurften der Zustimmung der nächst höheren Managementstufe. Es wird davon ausgegangen, dass Projekte mit einem DB1 Bud nahe diesem Grenzwert tendenziell schlechter abschliessen, da sie überbewertet wurden, um dem Genehmigungsprozess zu entgehen. Deshalb ist der DB1 Bud ein Element der Kostenkategorie.
\newline Die letzte monetäre Komponente bildet die Abweichungen der realisierten Kosten vom letzten Kostenforecast. Kleinere Abweichungen können als Indiz gewertet werden, dass 
\newline\newline\textbf{Zeit:} Diese Kategorie fasst sämtliche Indikatoren zur Abweichung der geplanten zur realisierten Projektlaufzeit zusammen. Das Interesse gilt vor allem die Zeitdivergenz zu lokalisieren. Hierfür wurde für fünf ausgewählte Milestones die Zeitdifferenz zwischen der geplanten und aktuellen Laufzeit gemessen. Die Milestones MS2 Concept approved, MS5 Point of no return, MS8 Documented, MS10 Takeover und MS11 Project sowie die  Closed sowie die gesamte Zeitdifferenz wurden hierbei berücksichtigt. 
\newline\newline\textbf{Komplexität:} Unter der Annahme, dass komplexere Projekte eher die Tendenz aufweisen zu scheitern, wurden Proxyvariablen ergründet, um die Komplexität eines Projektes abzubilden. Die Anzahl involvierter Zulieferer sowie der Verträge lässt den Schluss zu, dass das Projekt so komplex war, dass die Aufteilung der Zulieferung respektive die Einteilung in verschiedene Aufträge (Verträge) sinnvoll erschien.
\newline\newline\textbf{Erfolgskriterium:} Das Erfolgskriterium Abweichung von der budgetierten Marge errechnet sich aus realisierter und geplanten Deckungsbeitrag. Nach Abschluss des Projekts (nach MS 11) beginnt die zweijährige Garantieperiode. Die angefallen Kosten dieser Phase werden jedoch nicht direkt dem Projekt belastet, sondern laufen über ein anderes buchhalterisches Konto. Als Risikoprävention wird bei der Budgetierungsphase eines Projekts eine Marge von ca. 8\% zusätzlich einkalkuliert, so dass allfällige Mehrkosten während der Projektbearbeitungszeit abgefedert werden. Daraus folgt, dass sämtliche Projekte, welche diesen Kostenpuffer nicht benötigen beim Projektabschluss eine um diesen Betrag höhere Marge ausweisen. Die in der Analyse verwendeten Projektmargen sind bereits um diesen Faktor bereinigt und sind somit realitätsgetreu.
\nomenclature{SAS}{Sales and Service}
\nomenclature{FF}{Fulfillment}
\nomenclature{TO}{Turnover}
\nomenclature{MS}{Mechanical Supplies}
\nomenclature{MC1}{Mechanical supplies}
\nomenclature{ME}{Mechnical Engineering}
\nomenclature{PA}{Plant Automation}
\nomenclature{IS}{Installation}
\newpage
\section{Analye der Erfolgsfaktoren des Bühler Projektmanagements}\label{drei}
Basierend auf der vorangehende Literaturrecherche kann mit Leichtigkeit der Eindruck gewonnen werden, dass eine weitere Analyse der Faktoren, welche den Erfolg von Projekten beeinflussen, keine neuen Erkenntnisse liefern würde. Für eine Vielzahl der berücksichtigen Studien gründen weitergehende statistische Analysen auf einer anfänglichen Wertung von Erfolgsattributen durch Personen, die in der entsprechenden Industrie oder Projektmanagement tätig sind. Die erforschten Faktoren wurden vorgängig jeweils aus früheren Studien extrahiert. Der Zusammenhang zwischen dem Erfolg von Projekten, der entweder als binäre Ausprägung oder als indexiertes Kriterium repräsentiert war, und den unabhängigen Erfolgsattributen wurde mittels der entsprechender Regressionsanalysen erforscht. Die gewählte Methode der Likert-Skala führte jeweils dazu, dass die erforderlichen Annahmen für eine Regression oder Faktorenanalyse gegeben waren. Da sich die Ergebnisse zu einem Teil überschneiden, kann postuliert werden, dass unabhängig vom gewählten Performancekriterium ein gewisser Konsens bezüglich der Erfolgsfaktoren existiert. Diese Aussage ist mit Vorsicht zu geniessen, da die Studien nicht eins zu eins miteinander verglichen werden können, weshalb sie als Annahme formuliert wurde. Ausserdem lässt sich aus den betrachteten Forschungsberichten schliessen, dass keine unternehmensspezifische Daten respektive unternehmensbezogene Daten zu den Erfolgsattributen erhoben wurden beziehungsweise für die Analyse herangezogen wurde.\newline
Die nachfolgende Analyse wird sich aufgrund der Daten und Fixierung eines bestimmten Performancekriteriums sowie dem Fokus dieser Arbeit von bisherigen Analysen unterscheiden. 
\subsection{Daten und statistische Methoden}
In Kapitel \ref{zweizwei} wurde beschrieben, dass die Bühler AG die zu evaluierenden Faktoren auf Basis bisheriger Erfahrungen aus einer finanziellen Perspektive identifiziert und auch neue Indikatoren geschaffen hat. Der Rahmen für die Datenerhebung bildete die Datenverfügbarkeit des BPM-Cockpits und des SAP. In der Folge wurde für die Datenextraktion ein eigene Query geschaffen, die sämtliche Faktoren pro Projekt abbildet. Trotz mehrfacher Validierung der Daten, konnte nach Erreichung des Fertigstellungstermin keine vollständige Korrektheit vor allem einiger berechneter Indikatoren gewährleistet werden. Die Stichprobe enthält alle abgeschlossenen Projekte im Zeitraum zwischen 2013 und 2015. Das eindeutige Abgrenzungskriterium bilden hierbei der Projektstatus und das Datum des MS11 Start und Übergabe. Zuerst wurden alle Projekte mit einem MS11-Datum zwischen dem 1.1.2013 und dem 31.12.2015 eingegrenzt. Der Projektstatus stellte sicher, dass das Projekt auch aus finanzieller Sicht als abgeschlossen betrachtet werden kann, da gewisse Projekte MS11 bereits erreicht haben können, aber fehlende Rechnungen noch zu verbuchen sind.
\newline
Das ursprüngliche beabsichtigte Analysemodell orientierte sich an den bisherigen Studien und hätte sich aus einer Faktorenanalyse zur Reduktion der Anzahl Faktoren mit anschliessender Regressionsanalyse zur Bestimmung der Abhängigkeiten, zusammengesetzt. Allerdings konnte bei der Prüfung der Modellvoraussetzungen die zwingende Linearitätsannahme zwischen der abhängigen und den unabhängigen Variablen nicht zufriedenstellend erfüllt werden. Selbst eine entsprechende lineare Variablentransformationen hätte die Linearitätsannahme nicht besser erfüllt. Dementsprechend mussten lineare statistische Modelle von den möglichen Analysemethoden ausgeschlossen werden. Aus diesem Grund und der Tatsache, dass Erfolgsfaktoren bereits sehr gut erforscht wurden, hat dazu beigetragen, dass sich die Analyse im Bereich der deskriptiven Statistik bewegt. Darüber hinaus sollen soll mittels explorativer Analysen, ein Teilgebiet der Datenanalyse, Strukturen respektive neue Hypothesen zu möglichen Erfolgsfaktoren formulieren werden. Die Aussagekraft der Ergebnisse wird mit dem Verzicht auf die Anwendung der Inferenzstatistik insofern eingeschränkt, da keine Rückschlüsse auf die Grundgesamtheit (sämtliche Projekte der Bühler AG) gemacht werden können. Allerdings können Aussagen und Vermutungen bezüglich der Stichprobe gemacht werden. Demzufolge kann die nachfolgenden Analyse auch als expost-Analyse betrachtet werden. Ziel dieser expost-Betrachtung aus finanzieller Perspektive ist einerseits einen genauerer Untersuchung des monetären Margenverlusts. Hierbei ist die Ausprägung der Faktoren der nicht erfolgreichen Projekte von zentralem Interesse. Das zweite Ziele ist die Faktoren zu beschreiben, um anschliessend Hypothesen für mögliche Erfolgsfaktoren der Projekte der Bühler AG zu formulieren. Als Basis dienen die Vermutungen und Einschätzungen pro Faktor der Bühler AG, welche zusammen mit den Faktoren ergründet wurden. Diese Vorgehensweise ermöglicht das bisherige Datenmodell zu prüfen und ergänzende Faktoren zu finden. Da diese Analyse die erste ihrer Art für die Bühlerprojekte ist, kann sie zudem wertvolle Hinweise zu Projekten und allenfalls möglichen Problemfeldern liefern.
\newline Wie bereits in Kapitel \ref{zweizwei} erwähnt wurde, bildet das Erfolgskriterium die Abweichung des realisierten vom budgetierten Deckungsbeitrag eines Projekts. Sie stellt die finanzielle Perspektive eines Projekts dar und hat direkten Einfluss auf das Ergebnis eines Geschäftsbereich. Ausserdem hängt die variable Vergütung der Projektmanager und Verkaufsmanager vom realisierten DB1 ab. In erster Linie wurde die Logik Erfolg (über Budget) Fail (unter Budget) angewandt. Allerdings wurde für einige Analysen und Darstellungen die Ampellogik der Kosten des BPM-Cockpits angewandt (s. Kapitel \ref{zweizwei}).
\newline
\newline\textbf{Plausibilität, fehlende Werte und Ausreisser}
\newline
Die Daten wurden vor der Anwendung deskriptiver Statistik auf Plausibilität, fehlende Werte und Ausreisser hin untersucht. Dabei wurde der Fokus darauf gelegt, möglichst viele Datensätze in der Stichprobe zu erhalten. Bei 96 Variablen kann nicht erwartet werden, dass für jedes Projekt alle Werte verfügbar sind. Folglich wurden im Zusammenhang mit fehlenden Werten auch einige Variablen von der Analyse ausgeschlossen. Somit wurde zu Lasten einer Grossen Stichprobe ein gewisser Datenverlust hingenommen. Ausserdem macht es statistisch wenig Sinn, eine Determinante zu evaluieren, wenn für die Hälfte der Datensätze kein interpretierbarer Wert vorhanden ist. Dies würde verzerrte Schlussfolgerungen nach sich ziehen.
\newline Der Plausibilitätstest erfolgte basierend auf der Interpretation des Faktors. Die Stichprobe betrug ursprünglich 1497 Projekte $N = 1497$. Die Anzahl unabhängiger Faktoren belief sich auf 96 $x = 97$, davon wurden 21 aus unterschiedlichen Gründen entfernt. Der ursprüngliche Datensatz enthielt neun Berechnungsspalten, die Bestandteil andere Faktoren waren. Fünf weitere Faktoren wurden entfernt, da sie bereits durch einen andere Determinante im Datensatz enthalten waren oder lediglich der Identifikation des Projektes dienten. Im Anschluss wurde der Plausibilitätstest durchgeführt, der direkten Einfluss auf die Stichprobengrösse hat. Bei der Prüfung stellte sich heraus, dass folgende Anpassungen gemacht werden mussten, damit die Plausibilität der Daten gewährleistet werden konnte. (Tabelle EINFÜGEN).
\newline Nach dem Plausibilitäts betrug $n = 1055$ und die Anzahl Variablen $x = 83$. Im Anschluss an die Plausibilitätsanalyse wurde die Anzahl fehlender Werte pro unabhängiger Variable gemessen. Auf Basis dieser Kalkulation wurde entschieden, einige Variablen von der Analyse systematisch auszuschliessen, da unter deren Berücksichtigung eine erhebliche Anzahl Datensätze verloren gegangen wäre. (Tabelle Einfügen)! Die CostMostnegFCajd-Variablen haben einen weiteren Werte $y = 1'111'111$, der angibt, dass das Projekt nur positive FC-Anpassungen gehabt hat. Dies impliziert, dass der Forecast für die Kosten gesunken sind und somit weniger Kosten erwartet wurden, wobei der Umsatz konstant gehalten wurde. Die HOM-YellowStatus und die HOM-RedStatus haben ebenfalls den Wert $y = 1'111'111$, der in diesem Fall angibt, dass entweder kein Yellow-Status respektive RedStatus gegeben hat. Die Interpretation wäre somit, dass bei den HOMYellow-Status Variablen der Status immer grün war oder zuerst respektive direkt den roten Status hatte. Eine ähnliche Interpretation gilt für HOMRedStatus-Variablen, somit hätte dieses Projekt, den roten Status gar nicht erst erreicht. Da diese Interpretationen valide sind und somit keine fehlende Werte darstellen, wurden sie vor der Plausibilitätsanalyse nicht als NA markiert. Die folgende Tabelle zeigt alle Variablen mit ihrer Anzahl an fehlender Datensätze. Sämtliche Indikatoren mit weniger als 100 fehlender Datensätze wurden in der Analyse mitberücksichtigt. Der restliche Informationsverlust wurde dementsprechend hingenommen. 
\newline
\begin{table}
	\centering
	\caption{Anzahl NA's per variable}
	\begin{tabular} {| l| r | l |}
		\textbf{Variable Code} & \textbf{Anzahl NA's} & \textbf{Action}\\\hline
		AMNo & 98 & von der Analyse ausgeschlossen\\
		CostMostnegFCadj & 126 & von der Analyse ausgeschlossen\\
		CostMostnegFCadjPA & 500 & von der Analyse ausgeschlossen\\
		CostMostnegFCadjIS & 358 & von der Analyse ausgeschlossen\\
		CostFirstadj & 43 &\\
		PrTimeDelayMS2 & 35 &\\
		PrTimeDelayMS5 & 277 & von der Analyse ausgeschlossen\\
		PrTimeDelayMS8 & 15 &\\
		PrTimeDelayMS10 & 37 &\\
		AMAge & 308  & von der Analyse ausgeschlossen\\
		PMAge & 57 & \\		
	\end{tabular}
\end{table}
Nach der Entfernung der fehlenden Datensätze betrug die Stichprobe $n = 900$ und die Anzahl unabhängiger Variablen $ x = 77$. Nachdem 
\newline
Gemäss verschiedener Quellen ist die Identifikation und Eliminierung der Ausreisser ein zentraler Aspekt vor der deskriptiven Analyse, da ansonsten auch die deskriptiven Erhebungsmethoden wenig aussagekräftig bleiben. Der angewendet Ansatz basiert auf dem 'Interquartile Range'. Gemäss diesem Ansatz werden sämtliche Daten von der Analyse ausgeschlossen, welche folgende Werte über- respektive unterschreiten.
\newline\newline
\begin{centering}
	$ all avlues \leq Q1 - 1.5 * IQR$
	\newline
	$ all values \geq Q3 + 1-5 * IQR$
\end{centering}
\newline
\newline
\begin{centering}
	$ all avlues \leq Q1 - 3 * IQR$
	\newline
	$ all values \geq Q3 + 3 * IQR$
\end{centering}
\newline\newline
Obwohl dieser Ansatz als Orientierungshilfe dient, gestalte sich die Entscheidung ob ein Datensatz ein Ausreiser ist, eher schwierig. Denn der Ausreisser ist nur in Relation zu den übrigen Ausprägungen zu bestimmen und folglich vom Datensatz und dem Kontext der Daten abhängige. Beispielsweise ist das Umsatzvolumen in der Bühler AG sehr variable und ab und zu realisiert die Bühler AG auch Projekte, die ein extraordinäres Umsatzvolumen aufweisen. Im Zusammehang mit der Analyse, welche Projekte erfolgreich waren und welche nicht, würde es Sinn machen auch Projekte, welche quasi ein ausserordentliches Umsatzvolumen haben zu analysieren, da dieselben Projektmanagementmethodik zur Anwendung kommt. Deshalb wurde unter dem Gesichtspunkt möglichst viele Datensätze zu erhalten und um die Projektdiversität der Bühler AG zu erfassen sehr sparsam mit der Entfernung von Ausreissern umgegangen. Die vorangehend erwähnte Ansatz wurde als Orientierungshilfe angewandt. Die Nachfolgende Tabelle zeigt alle Variablen, welche um Ausreisser bereinigt wurden an. Der DB1Act entsprechend auf den 3IQR angepasst, da die durchschnittliche Performance des Anlagengeschäft der Bühler AG ca. 28\% beträgt. Die Anpassung dieser Variable hat direkten Einfluss auf DB1BudDev, welche somit nicht mehr zusätzlich bereinigt wurde. Die relativen CostActBud-Variablen für die einzelnen Kostenkategorien wurden jeweils um die höchsten Kostenüberschreitungen relativ zu den nächst höheren Werten bereinigt. Hohe Kostenüberschreitungen in den einzelnen Kostenkategorien treten beispielsweise auf, wenn der Budget-Anteil der entsprechenden Kostenkategorie relativ gross respektive klein ist. Aus diesem Grund ist es schwierig zu identifizieren, ob es sich um einen Ausreisser handelt oder nicht. Es könnte auch ein Fehler in der Budgetierungsphase erfolgt sein, bei der das entsprechende Kostenbudget zu tief eingeschätzt wurde. Allerdings mussten extreme Kostenabweichungen eliminiert werden. Hierbei wurden die maximalen Werte mit den darauffolgende 10 Werten verglichen. Darauf basierend wurde bei allen relativen Kostenvergleichen die höchsen 1-4 Werte eliminiert. Bei der relativen Wichtigkeit eines Projektes für den Geschäftsbereich wurde ein Wert mitunter eliminiert, da der maximale Wert ca. drei mal höher war als der zweithöchste Wert. Dieser Wert hätte die Aussagekraft des Indikators massgebend beeinflusst und zu verzerrten Interpretationen geführt. Nach der Bereinigung der Daten um die Ausreisser beträgt die Stichprobnezahl $N = 883 $ und die Anzahl Variablen $ x = 70$.

\subsection{Finanzieller Impact von nicht erfolgreichen Projekten}
\subsection{Ergebnisse und Interpretation}
\subsection{Kritische Würdigung der Ergebnisse}
\newpage
\section{Frühwarnsysteme und Frühwarnindikatoren als Managementtool}
\subsection{Frühwarnsysteme und Frühwarnindikatoren}
Erlätuerung und Einführung in die "Theorie" der Frühwarnsysteme und Frühwarnindikatoren
\subsection{Frühwarnsystem Bühler AG - Modellansatz}
Entwicklung eines Ansatz für ein Frühwarnsystem bei der Bühler AG
\subsection{Finanzielle Einsparungen durch Implementierung eines Frühwarnsystems}
Retrospektiv: Berechnung der finanziellen Einsparung bei Anwendung des vorgeschlagenen Frühwarnsystemansatzes
\newpage
\subsection{Kritische Würdigung der Ergebnisse im Kontext mit dem Frühwarnsystemansatz}
\newpage
\section{Fazit}


\section*{Annex: Databereinigung}
\newpage\begin{landscape}
\begin{center}
\begin{longtable}{p{6cm}|p{4cm}|p{6cm}|p{3cm}}
	\caption{Variablenschlüssel}\\\hline
	\textbf{Variablename} & \textbf{Variable Code} & {\textbf{Berechnungsformel}} & {\textbf{Masseinheit}} \\\hline\endhead
		Customer Number & CuNo  &       &  \\
		Equipment Location & EquLoc &       &  \\
		Project Manager Number & PMNo  &       &  \\
		Project Manager Change & PMChange &       & \multicolumn{1}{l}{{YES,NO}} \\
		Business Area & BA    &       &  \\
		Business Unit & BU    &       &  \\
		Market Segments & MS    &       &  \\
		Lead SAS Project & LeadSASPr &       &  \\
		Lead SAS Fullifllment is different than Lead SAS Project & LeadSAS.PrFF &       & \multicolumn{1}{l}{{NO,YES}} \\
		Consortial Part & ConPart &       & \multicolumn{1}{l}{{TRUE,FALSE}} \\
		Project Start Date & PrStartDate &       & \multicolumn{1}{l}{DD-MM-YYYY} \\
		Turnover Budget & TOBud &       & \multicolumn{1}{l}{TCHF} \\
		Cost Budget Mechancial Supply & BudMSTot & $\frac{Cost Bud_{MS}}{Cost Bud_{Total}}$ & \multicolumn{1}{l}{\%} \\ [3mm]
		Cost Budget Mechanical Engineering & BudMETot & $\frac{Cost Bud_{ME}}{Cost Bud_{Total}}$ & \multicolumn{1}{l}{\%} \\ [3mm]
		Cost Budget Plant \& Automation & BudPATot & $\frac{Cost Bud_{PA}}{Cost Bud_{Total}}$ & \multicolumn{1}{l}{\%} \\ [3mm]
		Cost Budget Installation  & BudISTot & $\frac{Cost Bud_{IS}}{Cost Bud_{Total}}$ & \multicolumn{1}{l}{\%} \\ [3mm]
		Deckungsbeitrag 1 Budget & DB1Bud &  $\frac{(TOBud - DB1Bud)}{TOBud}$ & \multicolumn{1}{l}{\%} \\ [3mm]
		Deckungsbeitrag 1 Actual & DB1Act & $\frac{(TOAct - DB1Act)}{TOAct}$ & \multicolumn{1}{l}{\%} \\ [3mm]
		Deckungsbeitrag Deviation Actual - Budget & DB1BudDev &  $\frac{(DB1 Act-DB1 Bud)}{DB1 Bud}$     & \multicolumn{1}{l}{\%} \\ [3mm]
		Cost Deviation between Actual and Budget Mechnical Supply & CostActBudMSabs & $((-CostAct_{MS})-(-CostBud_{MS})$ & \multicolumn{1}{l}{TCHF} \\ [3mm]
		Cost Deviation between Actual and Budget Mechnical Engineering & CostActBudMEabs & $((-CostAct_{ME})-(-CostBud_{ME})$ & \multicolumn{1}{l}{TCHF} \\ [3mm]
		Cost Deviation between Actual and Budget Plant Automation & CostActBudPAabs & $((-CostAct_{PA})-(-CostBud_{PA})$ & \multicolumn{1}{l}{TCHF} \\ [3mm]
		Cost Deviation between Actual and Budget Installation & CostActBudISabs & $((-CostAct_{IS})-(-CostBud_{IS})$ & \multicolumn{1}{l}{TCHF} \\
		Subsequent Delivery in Relation to TO & SUCostTO & $\frac{(-SUAct)}{TOAct}$ & \multicolumn{1}{l}{\%} \\ [3mm]
		Cost Deviation between Actual and Budget relative for the whole project & CostActBudRel & $\frac{((-CostAct_{Total}) - (-CostBud_{Total})}{(-CostBud_{Total})}$ & \multicolumn{1}{l}{\%} \\ [3mm]
		Cost Deviation between Actual and Budget relative Mechnical Supply & CostActBudMSRel &  $\frac{((-CostAct_{MS}) - (-CostBud_{MS})}{(-CostBud_{MS})}$ & \multicolumn{1}{l}{\%} \\ [3mm]
		Cost Deviation between Actual and Budget relative Mechnical Engineering & CostActBudMERel & $\frac{((-CostAct_{ME}) - (-CostBud_{ME})}{(-CostBud_{ME})}$ & \multicolumn{1}{l}{\%} \\ [3mm]
		Cost Deviation between Actual and Budget relative Plant Automation & CostActBudPARel & $\frac{((-CostAct_{PA}) - (-CostBud_{PA})}{(-CostBud_{PA})}$ & \multicolumn{1}{l}{\%} \\ [3mm]
		Cost Deviation between Actual and Budget relative Installation & CostActBudISRel & $\frac{((-CostAct_{IS}) - (-CostBud_{IS})}{(-CostBud_{IS})}$ & \multicolumn{1}{l}{\%} \\ [3mm]
		Cost Forecast adjustment Project & CostFCadj &       & \multicolumn{1}{l}{{0,1}} \\
		Cost Forecast adjustment Mechnical Supply & CostFCadjMS &       & \multicolumn{1}{l}{{0,1,2}} \\
		Cost Forecast adjustment Mechnical Engineering & CostFCadjME &       & \multicolumn{1}{l}{{0,1,2}} \\
		Cost Forecast adjustment Plant \& Automation & CostFCadjPA &       & \multicolumn{1}{l}{{0,1,2}} \\
		Cost Forecast adjustment Installation & CostFCadjIS &       & \multicolumn{1}{l}{{0,1,2}} \\
		Months between most negative Forecast adjustment Mechnical Supply and Project Closure & CostMostnegFCadjMS & $(Date(Project Closure))-(Date(most negative FC adj.)_{MS})$ & \multicolumn{1}{l}{Monate} \\ [3mm]
		Months between most negative Forecast adjustment Mechnical Engineerring and Project Closure & CostMostnegFCadjME & $(Date(Project Closure))-(Date(most negative FC adj.)_{ME})$ & \multicolumn{1}{l}{Monate} \\ [3mm]
		Months between Hand-over-Meeting and first Yellow Status Cost & HOMYellCost &$\frac{(Date(first Yellow Status)_{Cost} - (Date(HOM))}{Project baseline}$& \multicolumn{1}{l}{Monate} \\ [3mm]
		Months between Hand-over-Meeting and fist Yellow Status Quality & HOMYellQual & $\frac{(Date(first Yellow Status)_{Quality} - (Date(HOM))}{Project baseline}$ & \multicolumn{1}{l}{Monate} \\ [3mm]
		Months between Hand-over-Meeting and first Yellow Status Time & HOMYellTime &$\frac{(Date(first Yellow Status)_{Time} - (Date(HOM))}{Project baseline}$ & \multicolumn{1}{l}{Monate} \\ [3mm]
		Months between Hand-over-Meeting first Red Status Cost & HOMRedCost & $\frac{(Date(first Red Status)_{Cost} - (Date(HOM))}{Project baseline}$ &\multicolumn{1}{l}{Monate} \\
		Months between Hand-over-Meeting and first Red Status Quality & HOMRedQual & $\frac{(Date(first Red Status)_{Quality} - (Date(HOM))}{Project baseline}$& \multicolumn{1}{l}{Monate} \\
		Months between Hand-over-Meeting and first Red Status Time & HOMRedTime & $\frac{(Date(first Red Status)_{Cost} - (Date(HOM))}{Project baseline}$ & \multicolumn{1}{l}{Monate} \\
		Delta between last Forecast and actual  & DeltaLastFCAct & $Last Cost FC-Cost Act$ & \multicolumn{1}{l}{TCHF} \\
		Delta between last Forecast and actual Mechnical Supply & DeltaLastFCActMS & $Last Cost FC_{MS}-Cost Act_{MS}$ & \multicolumn{1}{l}{TCHF} \\
		Delta between last Forecast and actual Mechnical Engineering & DeltaLastFCActME & $Last Cost FC_{ME}-Cost Act_{ME}$& \multicolumn{1}{l}{TCHF} \\
		Delta between last Forecast and actual Plant Automation & DeltaLastFCActPA & $Last Cost FC_{PA}-Cost Act_{PA}$ & \multicolumn{1}{l}{TCHF} \\
		Delta between last Forecast and actual Installtion & DeltaLastFCActIS & $Last Cost FC_{IS}-Cost Act_{IS}$ & \multicolumn{1}{l}{TCHF} \\
		Business Area Importance of Project & BAImportPr & $\frac{TOBud_{BA}}{Median (TOBud_{BA})}$ & \multicolumn{1}{l}{\%} \\
		Business Unit Importance Project & BUImportPr & $\frac{TOBud_{BU}}{Median (TOBud_{BU})}$ & \multicolumn{1}{l}{\%} \\
		Market Segement Importance  of Project & MSImportPr & $\frac{TOBud_{MS}}{Median (TOBud_{MS})}$ & \multicolumn{1}{l}{\%} \\
		Business Unit Orders Released Budget Gap absolute & BUORBudGapAbs & $OR Act_{BU}-OR Bud_{BU}$ & \multicolumn{1}{l}{TCHF} \\
		Business Unit Orders Released Budget Gap relative & BUORBudGapRel & $\frac{(ORAct_{BU}-ORBud_{BU})}{ORBud_{BU}}$ & \multicolumn{1}{l}{\%} \\
		Region Orders Released Budget Gap absolute & RegiORBudGapAbs & $OR Act_{Region}-OR Bud_{Region}$ & \multicolumn{1}{l}{TCHF} \\
		Region Orders Released Budget Gap relative & RegiORBudGapRel & $\frac{(ORAct_{Region}-ORBud_{Region})}{ORBud_{Region}}$ & \multicolumn{1}{l}{\%} \\
		Project Time Base & PrTimeBase & $Baseline_{MS11}-Baseline_{MS1}$ & \multicolumn{1}{l}{Monate} \\
		Project Time Actual & PrTimeAct & $Act_{MS10}-Act_{MS1}$ & \multicolumn{1}{l}{Monate} \\
		Project Time Delay & PrTimeDelay & $Baseline-Act$ & \multicolumn{1}{l}{Monate} \\
		Project Time Delay for MS2 & PrTimeDelayMS2 & $Baseline_{MS2}-Act_{MS2}$ & \multicolumn{1}{l}{Monate} \\
		Project Time Delay for MS8 & PrTimeDelayMS8 & $Baseline_{MS8}-Act_{MS8}$ & \multicolumn{1}{l}{Monate} \\
		Project Time Delay for MS10 & PrTimeDelayMS10 & $Baseline_{MS10}-Act_{MS10}$ & \multicolumn{1}{l}{Monate} \\
		Project Time Delay for MS11 & PrTimeDelayMS11 & $Baseline_{MS11}-Act_{MS11}$ & \multicolumn{1}{l}{Monate} \\
		Number Project Manager during project life time & NoPM  &       & \multicolumn{1}{l}{Anzahl} \\
		Area Manager Age & AMAge & $Date_{HOM}-Date_{Geburt}$ & \multicolumn{1}{l}{Jahre} \\
		Area Manager Tenure & AMTen & $Date_{HOM}-Date_{Eintritt}$ & \multicolumn{1}{l}{Jahre} \\
		Project Manager Age & PMAge & $Date_{HOM}-(Date_{Geburt}$ & \multicolumn{1}{l}{Jahre} \\
		Project Manager Tenure & PMTen & $Date_{HOM}-Date_{Eintritt}$ & \multicolumn{1}{l}{Jahre} \\
		Number of Lead SAS Fulfillment & NoLeadSASFF &       & \multicolumn{1}{l}{Anzahl} \\
		Number of Supplying SAS & NoSupplSAS &       & \multicolumn{1}{l}{Anzahl} \\
		Number of Supplying SAS Mechnical Supply & NoSupplSASMS &       & \multicolumn{1}{l}{Anzahl} \\
		Number of Supplying SAS Mechnical Engineering & NoSupplSASME &       & \multicolumn{1}{l}{Anzahl} \\
		Number of Supplying SAS Plant \& Automation & NoSupplSASPA &       & \multicolumn{1}{l}{Anzahl} \\
		Number of Supplying SAS Installation & NoSupplSASIS &       & \multicolumn{1}{l}{Anzahl} \\
		Number of Contracts & NoContr &       & \multicolumn{1}{l}{Anzahl} \\
		Turnover Actual & TOAct &       & \multicolumn{1}{l}{TCHF} \\
		Deckungsbeitrag 1 Budget absolut & DB1Budabs & $TOBud-CostBud$ & \multicolumn{1}{l}{TCHF} \\
		Deckungsbeitrag 1 Actual Absolut & DB1Actabs & $TOAct-CostAct$ & \multicolumn{1}{l}{TCHF} \\
		Region & Region &       &  \\
		Success & success &       & {YES,NO} \\
		On Time & onTime &       & {YES,NO} \\
		MMS\_FCadj\_PrAct & MMS\_FCadj\_PrAct & $\frac{CostMostnegFCadjMS}{PrTimeAct}$ & \\
		MME\_FCadj\_PrAct & MME\_FCadj\_PrAct & $\frac{CostMostnegFCadjME}{PrTimeAct}$ &  \\
		Deviation TO Bud from TO Act & TOBudDevabs & $TOAct-TOBud$ & TCHF \\
		Deviation DB1 Bud from DB1 Act & DB1BudDevabs & $DB1Act-DB1Bud$ & TCHF \\
		Cost Budget & CostBud &       & TCHF \\
		Cost Actual & CostAct &       & TCHF \\
		Jahr des Projektstartdatum & PrStartDateY &       & YYYY \\
		Monat des Projektstartdatum & PrStartDatem &       & MM \\
		Tag des Projektstartdatum & PrStartDateD &       & DD \\
		Datum des Projektende & PrEndDate &       & DD-MM-YYY \\
		TO Bud Kategorien & TOBud\_cat &       & {1:70} \\
		Erfolgskriterium gemäss dem Ampelsystem & DB1Dev\_ampel &       & {green, yellow, red} \\
		DB1 Abweichung Kategorien & DB1Dev\_cat &       & {1:6} \\
		Zeiteinhaltung Kategorien & PrTimeDelay\_cat &       & {1:15} \\
		Zeiteinhaltung gemäss Ampelsystem & PrTimeDelay\_ampel &       & {green, yellow, red} \\
		
	\label{tab:addlabel}%
\end{longtable}%
\end{center}
\end{landscape}
\subsection*{Data Collection}
\subsection*{Measures}
\subsection*{Analysis}

\end{document}

pdflatex old_version.tex


