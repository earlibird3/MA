% !TEX root = MA.tex
\chapter{Einleitung}
Sofern Projekte mediale Aufmerksamkeit erhalten, werden deren erheblichen Mehrkosten thematisiert, wie Beispiel zeigen \glqq Sawiris Luxusresort-Projekt fährt Millionen-Verluste ein\grqq{ } oder \glqq Bund verschwendet erneut Millionen für IT-Projekt\grqq{ }\citep{nzz16,fuchs15}. Andere laufende Projekte wie beispielsweise Stuttgart21 oder der Bau des Flughafen Berlin Brandenburg werden trotz erheblicher Kostenüberschüsse geplant fertigzustellen. Die Gründe diesen Kostenabweichungen kann auf geringe Auslastung, unzureichende Kommunikation, mangelhafte Bauarbeiten oder administrative Herausforderungen zurückgeführt werden, wobei die Aufzählung nicht abschliessend ist. Demgegenüber stehen Projekte, die ohne grosse Komplikationen und Mehrkosten zu Ende geführt werden konnten. In diesem Zusammenhang wird auch zwischen erfolgreichen und gescheiterten Projekten unterschieden. Mit der Absicht unterschiedlichen Projektergebnisse zu erklären, wurde bereits mehrfach versucht Faktoren zu ergründen, die den Projekterfolg begünstigen.
\newline\newline
 Projekte beinhalten gemäss PMBOK (ohne Datum, zit. in \citealp[S.~17]{burke10}) die zeitlich bedingte Herstellung eines einzigartigen Produkts. Demzufolge können Projekte eindeutig von der Fabrikation homogener Güter unterschieden werden, was die eingangs erwähnten Beispiele untermauern. Die Abwicklung von Projekten erfolgt mittels dem Projektmanagement, das die Erfüllung der Anspruchsgruppenbedürfnisse durch die wissens- und fähigkeitsbasierte Steuerung der Projektaktivitäten umfasst (ohne Datum, zit. in \citealp[S.~18]{burke10}). Dabei werden klassische von agilen Ansätzen unterschieden, wobei sich erstere durch flexible Anpassung auf verändernde Kundenbedürfnisse und letztere durch getrennte, planbare Phasen, die mittels Meilenstein verbunden sind, charakterisieren lassen (Zitieren). Diese Phasen sind die Initiierung, Durchführung und Kontrolle sowie der Abschluss \citep{pmhodm}. Unabhängig der Methodologien wird der Projekterfolg üblicherweise anhand der primären Ziele, Kosten, Zeit und Qualität gemessen \citep{Atk1999}. Aufgrund des Zusammenhangs der Grössen wird diese Erfolgsdefinition oftmals als Dreieck abgebildet. Diese Darstellung geht auf Martin Barne zurück und wird aufgrund der verbreiteten Anwendung als das magische Dreieck oder die traditionelle Leistungsmessung bezeichnet \citep{kerz14,lock07}. Obwohl von den Projektmanager grundsätzlich die kumulative Erreichung dieser Ziele erwartet wird, kann die Fokussierung eines der Ziele die Erreichung des anderen Ziels beeinträchtigen \citep[S.~21]{lock07}. Daraus folgt, dass die einfache oder mehrfache Zielerreichung die Unterscheidung in erfolgreiche respektive nicht erfolgreiche Projekte ermöglicht. Allerdings ist die Beurteilung letztendlich vom den unternehmensspezifischen Kriterien abhängig. Da der gesamte Projektmanagementprozesses durch unterschiedliche Variablen wie zum Beispiel, Entscheide, involvierte Personen oder der Projektbedingungen beeinflusst wird, liegt die Vermutung nahe, dass einige zentrale Faktoren den Projekterfolg entscheidend beeinflussen können. Die sogenannten kritischen Erfolgsfaktoren müssen gemäss \cite{BeDeNov2015} von den Erfolgskriterien klar abgegrenzt werden, da sie lediglich den Projekterfolg nach dem Projektabschluss beurteilen wohingegen Erfolgsfaktoren zur Erfolgswahrscheinlichkeit eines Projekts beitragen. Diese Faktoren können sich aufgrund der Projektart, der Branche und der gewählten Managementansätze unterschieden. Der Handlungsspielraum eines Projektmanager kann dadurch erweitert werden, da er exakt diese Faktoren so beeinflussen kann, um sein Projekt unter der Berücksichtigung der zugrundeliegenden Erfolgsdefinition erfolgreich abschliessen kann. Trotz dieser Erkenntnis und neuer Projektmanagementmethoden tritt das Projektscheitern relativ häufig auf, wobei die potenzielle Gefährdung der Projektziele relativ lange unentdeckt blieb \citep{WiKlak12,haan13}. Ein möglicher Ansatz dieser Situation entgegenzuwirken, ist die Implementierung eines Frühwarnsystem (im wirtschaftlichen Kontext: Früherkennung) mittels dem Gefahren und Chancen möglichst früh erkannt und entsprechende Gegenmassnahmen ergriffen werden können \citep{haankra13}. Dazu werden mittels unterschiedlicher Techniken versucht, Frühwarnindikatoren, die sich von den Erfolgsfaktoren ableiten können, zu identifizieren. Deren kontinuierliche Überwachung solle die frühzeitige Erkennung von, Gefährdungspotenzial ermöglichen.
\newline\newline
Die vorliegende Arbeit konzentriert sich auf die Einflussfaktoren im Projektemanagement am Unternehmensbeispiel der Bühler AG. Die Bühler AG ist ein Maschinentechnologiekonzern im Familienbesitz mit Hauptsitz in Uzwil. Sie hat eine führende Marktposition in der Herstellung von Maschinen der Nahrungsmittelverarbeitung für Mehl, Pasta, Schokolade, Reis, und auch für die Oberflächenbeschichtung. Die Fabrikation von einzelnen Maschinen oder eines ganzen Maschinenparks wird mittels Projekten unter der Anwendung des vorhin beschriebenen traditionellen Projektmanagementansatzes abgewickelt. Infolge der unzureichenden Performance einiger Projekte und der Erkenntnis, dass eine drohende Verschlechterung der Projektleistung relativ spät im Projektmanagementtool ersichtlich war, entstand die Ambition diejenigen Faktoren, welche den Projektmanagementprozess und letztendlich den Projekterfolg beeinflussen können, zu erfassen. Aus diesen Gründen entstand 2016 eine Liste sämtlicher potenzieller Faktoren, die das Bühler Projektmanagement in Zusammenarbeit mit dem Controlling basierend auf ihren Erfahrungen und Vermutungen erarbeitet. Das Ziel bestand einerseits darin, Unterschiede hinsichtlich der Ausprägung dieser Faktoren zwischen erfolgreichen und nicht-erfolgreichen Projekten zu untersuchen und anderseits ein Zusammenhang zwischen Erfolg und einem oder mehrere Faktoren zu ergründen. Anhand der Ergebnisse wurde die Identifikation der Erfolgsfaktoren sowie die Herleitung möglicher Frühwarnindikatoren beabsichtigt. Der Erfolg der Bühler-Projekte wird anhand des traditionellen Ansatzes Kosten, Zeit und Qualität beurteilt. Im Projektmanagementcockpit der Bühler AG (BPM-Cockpit) wird während der gesamten Projektlaufzeit der Status der Zielerreichung mittels dem dreifarbigen Ampelsystem reflektiert. Dabei ändert die Ampelfarbe der Kosten- und Zeitampel gemäss hinterlegtem Schlüssel automatisch, wohingegen die Qualitätsampel erst mit der Eingabe der Einschätzung des Projektmanagers die Farbe verändert. Obwohl die Zielerreichung aller drei Kriterien gleichermassen vorausgesetzt wird, hat aus einer Finanzperspektive der Kostenaspekt ein wesentlicher Stellenwert. Erhebliche Mehrkosten beeinflussen die Projektmarge \gls{abk:db1} und somit das Ergebnis des Anlagengeschäfts der Bühler AG. Der absolute DB1 eines Projekts berechnet sich aus Umsatz minus Kosten, wobei er in Relation zum Umsatz die prozentuale Marge wiedergibt. Zur Beurteilung des Projekterfolgs wird der realisierte mit dem budgetierten Wert verglichen. Die Relevanz dieses Soll-Ist-Vergleich bei der Bühler AG neben der finanziellen Performance zeigt sich vor allem dadurch, dass er die Grundlage des Incentivierungsmechanismus der Geschäftsbereichsleiter und Projektmanager bildet. 
%%
%%
%%
\section{Ziele der Arbeit}\label{sec:zda}
Ziel dieser Arbeit ist es, die Einflussfaktoren des Projektmanagements der Bühler AG zu analysieren. Mittels deskriptiver statistischer Methoden sollen die charakteristischen und quantitativen Unterschiede hinsichtlich der Faktoren zwischen erfolgreichen und nicht-erfolgreichen Projekten herausgearbeitet werden. Dabei soll die Projekterfolgsdefinition der Bühler AG berücksichtigt werden, da ihr wesentliche Bedeutung innerhalb des Unternehmens beigemessen wird. Daraus folgt, dass der Analyse eine finanzielle Perspektive zu Grunde liegt, da der Kostenaspekt des magischen Dreiecks aufgrund des Erfolgskriteriums der Bühler AG (Abweichung Projektmarge vom Sollwert), fokussiert wird. Der Untersuchungsgegenstand bilden sämtliche Projekte die zwischen 2013 und 2015 abgewickelt und abgeschlossen wurden. Die Projekte sollen anhand des Erfolgskriterium in zwei Gruppen (erfolgreich, nicht-erfolgreich) unterteilt  werden, um anschliessend auf der Basis der Auswertung von  Häufigkeiten, Lagerparametern und finanziellen Performance Differenzen zwischen den Gruppen zu untersuchen. Daraus kann folgende zentrale Fragestellung abgeleitet werden: 
\newline\newline
Welche charakteristischen und quantitativen Unterschiede können auf der Basis der Einflussfaktoren im Projektmanagement der Bühler AG zwischen erfolgreichen und nicht-erfolgreichen Projekten herausgearbeitet werden? 
\newline\newline
Die eingangs erwähnte Problematik des Projektscheiterns und der reaktiven Steuerung des Gefährdungspotenzial eines Projekts legitimieren diese erstmalige Analyse vergangener Projekte der Bühler AG. Ausserdem könnten die Ergebnisse Hinweise für mögliche Erfolgsfaktoren und Frühwarnindikatoren geben. Deshalb soll als untergeordnetes Ziel unter der Berücksichtigung des Projektmanagementprozesse und dem aktuellen Stand der Forschung methodische Ansätze zur Früherkennung im Bühler Projektmanagement diskutiert. In diesem Zusammenhang sind retrospektiv die finanziellen Einsparungen unter der hypothetischen Anwendung des Frühwarnsystems ein weiterer Aspekt der aus der Finanzperspektive von Interesse ist. 
\newline\newline
Der theoretische Rahmen der Arbeit bildet die Theorie der Erfolgsfaktoren im Projektmanagement sowie die Früherkennung. Zum Verständnis der Faktoren soll der Bühler Projektmanagementprozess erklärt werden, da er die Grundlage der Datenerfassung und somit der Analyse bildet. Im methodischen Abschnitt dieser Arbeit soll zunächst auf die Stichprobe, das analytische Vorgehen sowie die Faktoren erläutert werden. Nach der Ergebnispräsentation sollen die Erkenntnisse in Verbindung mit den Aspekten der Forschung, der Zielsetzung dieser Arbeit und den angewandten Methoden diskutiert werden. Die Früherkennungsansätze im Projektmanagement der Bühler AG sollen in einem weiteren Schritt diskursiv erarbeitet und kritisch betrachtet werden.
\newline\newline
Es werden ausschliesslich die von der Bühler AG zur Verfügung gestellten Daten untersucht, was die Erhebung zusätzlicher Daten ausschliesst. Dies hat zur Folge hat, dass Faktoren, welche das Projektmanagement der Bühler AG auch beeinflussen können, im Rahmen dieser Arbeit nicht untersucht werden. Die Aussagekraft der Ergebnisse wird durch die unternehmensspezifische Daten begrenzt, so dass lediglich Rückschlüsse auf die Projekte und den Projektmanagementprozess der Bühler AG gemacht werden können.
%%
%%Methodik
%%
\section{Methodik und Struktur der Arbeit}
Die Gliederung der Arbeit unterscheidet vier Abschnitte: der theoretische und unternehmensspezifische Rahmen, die Methodologie, Ergebnisse sowie die Diskussion. Die Erarbeitung erfolgt dabei auf der Basis einer Kombination von Literaturrecherche, unternehmensspezifischem Wissen sowie statistischer Vorgehensweise.
%Deshalb wird vor der quantitativen Analyse der Projektbegriff und das Projektmanagement eingehender erläutert und in Relation zum Unternehmensbeispiel gesetzt. Zudem sollen bisherige Erkenntnisse aus der Erfolgsfaktorenforschung aufgezeigt sowie die vorherrschende Erfolgsdefinition von Projekten und deren Wandlungstendenz eingehender erläutert werden.
\newline\newline
Im ersten Abschnitt wird eine theoretische Abhandlung zu Projekten, Projektmanagement und bisher erforschten Erfolgsfaktoren dargelegt sowie der Bühler Projektmanagementprozesse erläutert. Die definitorische Abgrenzung dient dazu den Rahmen der Begrifflichkeiten festzulegen und in Verbindung zu den internen Bestimmungen zu setzen. Auf die Unterschiede der Projektarten, verschiedenen Projektmanagementansätze und deren Kategorisierung wird in dieser Arbeit nicht näher eingegangen, da für das Verständnis der Einflussfaktoren vor allem der Bühler Projektmanagementprozess von Bedeutung ist. Die Ergründung der bisher identifizierten Erfolgsfaktoren dient dazu die dominierende Ansicht der Wissenschaft aufzuzeigen, wobei der Fokus auf den Konstruktionsprojekten liegt. Es wurden allerdings Erkenntnisse anderer Projektarten mitberücksichtigt, mit dem Ziel ein breites Spektrum an möglichen Erfolgsfaktoren zu erhalten. Diese Informationsbasis dient als Referenzpunkt für die Erkenntnisse aus den qualitativen Auswertungen und kann ergänzenden Variablen des Projekterfolges im Bühler Projektmanagement enthalten. Dem Erfolgskriteriums ist im Forschungsdesign eine entscheiden Rolle beizumessen, weshalb unterschiedliche Erfolgsdefinitionen diskutiert werden. Die Prozessbeschreibung des Bühler Projektmanagement orientiert sich an den internen Dokumentationen und Darstellungen, wobei diejenigen Bestandteile eingehender erläutert werden, die dem Verständnis der Einflussfaktoren dienen. Anschliessend folgt die theoretische Einführung in die Früherkennung im allgemeinen und im Bezug auf das Projektmanagement, wobei der Fokus auf den methodischen Ansätze der Früherkennung liegt.
\newline\newline
Der zweite Abschnitt umfasst die Beschreibung der Datengrundlage und Datenanalyse, wobei die Erläuterung des analytisches Vorgehen und die Erklärung der Faktoren beinhaltet. Anschliessend folgt im dritten Abschnitt die illustrative und deskriptiv Darlegung der Ergebnisse. Der vierte Abschnitt konzentriert sich auf Diskussion der Ergebnisse der Analyse der Einflussfaktoren, Erfolgsfaktoren und der Früherkennung im Bühler Projektmanagement im Kontext der aktuellen Forschung, der Zielsetzung und dem Methodenansatz dieser Arbeit.
\newline\newline
Abschliesend werden im Fazit die wichtigsten Erkenntnisse der Arbeite zusammenfassend festgehalten und die eingangs erwähnten Forschungsfragen beantwortet. Zudem sollen Empfehlungen für weitere Forschungsthemen und Optimierungspotenziale festgehalten werden.

	




