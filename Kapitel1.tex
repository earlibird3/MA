% !TEX root = MA.tex
\chapter{Einleitung}
Sofern Projekte in den Medien erwähnt werden, betonen die Titelüberschriften wie zum Beispiel \glqq Sawiris Luxusresort-Projekt fährt Millionen-Verluste ein\grqq{ }\cite{nzz16} oder \glqq Bund verschwendet erneut Millionen für IT-Projekt\grqq{ }\cite{fuchs15} deren  erheblichen Mehrkosten. Im deutschen Nachbarland sind ebenso einige Projekte mit hohen Kosten bekannt, beispielsweise Stuttgart21 oder der Bau des Flughafen Berlin Brandenburg. Der gemeinsamer Nenner kann auf unterschiedliche Gründe, zum Beispiel geringe Auslastung, unzureichende Kommunikation, mangelhafte Bauarbeiten oder administrative Herausforderungen zurückgeführt werden, wobei die Aufzählung nicht abschliessend ist. Demgegenüber stehen Projekte, die ohne grosse Komplikationen und Mehrkosten zu Ende geführt werden konnten. In diesem Zusammenhang wird auch zwischen erfolgreichen und gescheiterten Projekten unterschieden. Mit der Absicht unterschiedlichen Projektergebnisse zu erklären, wurde bereits mehrfach versucht Faktoren zu ergründen, die den Projekterfolg begünstigen.
\newline\newline
 Projekte beinhalten gemäss PMBOK (Zitieren) die zeitlich bedingte Herstellung eines einzigartigen Produkts. Demzufolge können Projekte eindeutig von der Fabrikation homogener Güter unterschieden werden, was die eingangs erwähnten Beispiele untermauern. Die Abwicklung von Projekten erfolgt mittels dem Projektmanagement, das die Erfüllung der Anspruchsgruppenbedürfnisse durch die wissens- und fähigkeitsbasierte Steuerung der Projektaktivitäten umfasst (PMBOK in Zitieren). Dabei werden klassische von agilen Ansätzen unterschieden, wobei sich erstere durch flexible Anpassung auf verändernde Kundenbedürfnisse und letztere durch getrennte, planbare Phasen, die mittels Meilenstein verbunden sind, charakterisieren lassen (Zitieren). Diese Phasen sind die Initiierung, Durchführung und Kontrolle sowie der Abschluss (Zitieren PM Handbook online). Unabhängig der Methodologien wird der Projekterfolg üblicherweise anhand der primären Zielen, Kosten, Zeit und Qualität gemessen \cite{Atk1999}. Aufgrund des Zusammenhangs der Grössen wird diese Erfolgsdefinition oftmals als Dreieck abgebildet. Diese Darstellung geht auf Martin Barne zurück und wird aufgrund der verbreiteten Anwendung als das magische Dreieck oder die traditionelle Leistungsmessung bezeichnet (\cite{kerz14} \& \cite{lock07}. Obwohl von den Projektmanager grundsätzlich die kumulative Erreichung dieser Ziele erwartet wird, kann die Fokussierung eines der Ziele auf Kosten der anderen Ziele gehen \cite{lock07}. Daraus folgt, dass die einfache oder mehrfache Ziel-Erreichung respektive Nichterreichung mit Projekterfolg und Projektscheitern gleichgesetzt werden kann. Da der gesamte Projektmanagementprozesses durch unterschiedliche Variablen wie zum Beispiel, Entscheide, involvierte Personen oder der Projektbedingungen beeinflusst wird, wurde vermutet, dass es einige zentrale Faktoren gibt, den Projekterfolg entscheidend beeinflussen können. In diesem Kontext wurde der Begriff der kritischen Erfolgsfaktoren eingeführt. Gemäss \cite{BeDeNov2015} müssten \cite sie von den Erfolgskriterien klar abgegrenzt werden, da sie lediglich den Projekterfolg nach dem Projektabschluss beurteilen wohingegen Erfolgsfaktoren zur Erfolgswahrscheinlichkeit eines Projekts beitragen. Der Handlungsspielraum wird dadurch so verändert, dass der Projektmanager sein Fokus auf diese Faktoren legen kann, um sein Projekt unter der Berücksichtigung der zugrundeliegenden Erfolgsdefinition erfolgreich abschliessen kann. Trotz dieser Erkenntnisse, neuer Methoden und überarbeiteter Erfolgsdefinition, tritt das Projektscheitern relativ häufig auf, wobei die potenzielle Gefährdung der Projektziele relativ lange unentdeckt blieb \cite{WiKlak12} \& \cite{haan13}. Deshalb wird mittels Frühwarnsystem versucht dieser Situation entgegenzuwirken indem  möglichst früh Gefahren und Chancen zu erkennen so dass entsprechenden Gegenmassnahmen ergriffen werden können \cite{haankra13}.
\newline\newline
Die vorliegende Arbeit konzentriert sich auf die Einflussfaktoren im Projektemanagement am Unternehmensbeispiel der Bühler AG. Die Bühler AG ist ein Maschinentechnologiekonzern im Familienbesitz mit Hauptsitz in Uzwil. Sie hat eine führende Marktposition in der Herstellung von Maschinen für die Getreideverarbeitung für Mehl, Pasta, Schokolade, Reis, und auch für die Oberflächenbeschichtung. Die Fabrikation von einzelnen Maschinen oder eines ganzen Maschinenparks wird mittels Projekten unter der Anwendung des vorhin beschriebenen traditionellen Projektmanagementansatzes abgewickelt. Infolge der unzureichenden Performance einiger Projekte und der Erkenntnis, dass eine drohende Verschlechterung der Projektleistung relativ spät im Projektmanagementtool ersichtlich war, entstand die Ambition diejenigen Faktoren, welche den Projektmanagementprozess und letztendlich den Projekterfolg beeinflussen können, zu erfassen. Aus diesen Gründen entstand 2016 eine Liste sämtlicher potenzieller Faktoren, die das Bühler Projektmanagement in Zusammenarbeit mit dem Controlling basierend auf ihren Erfahrungen und Vermutungen erarbeitet. Das Ziel bestand einerseits darin, Unterschiede hinsichtlich der Ausprägung dieser Faktoren zwischen erfolgreichen und nicht-erfolgreichen Projekten zu untersuchen und anderseits ein Zusammenhang zwischen Erfolg und einem oder mehrere Faktoren zu ergründen. Anhand der Ergebnisse wurde die Identifikation der Erfolgsfaktoren sowie die Herleitung möglicher Frühwarnindikatoren beabsichtigt. Der Erfolg der Bühler-Projekte wird anhand des traditionellen Ansatzes Kosten, Zeit und Qualität beurteilt. Im Projektmanagementcockpit der Bühler AG (BPM-Cockpit) wird während der gesamten Projektlaufzeit der Status der Zielerreichung mittels dem dreifarbigen Ampelsystem reflektiert. Dabei ändert die Ampelfarbe der Kosten- und Zeitampel gemäss hinterlegtem Schlüssel automatisch, wohingegen die Qualitätsampel erst mit der Eingabe der Einschätzung des Projektmanagers die Farbe verändert. Obwohl die Zielerreichung aller drei Kriterien gleichermassen vorausgesetzt wird, hat aus einer Finanzperspektive der Kostenaspekt ein wesentlicher Stellenwert. Erhebliche Mehrkosten beeinflussen die Projektmarge \gls{abk:db1} und somit das Ergebnis des Anlagengeschäfts der Bühler AG. Der absolute DB1 eines Projekts berechnet sich aus Umsatz minus Kosten, wobei er in Relation zum Umsatz die prozentuale Marge wiedergibt. Zur Beurteilung des Projekterfolgs wird der realisierte mit dem budgetierten Wert verglichen. Die Relevanz dieses Soll-Ist-Vergleich bei der Bühler AG neben der finanziellen Performance zeigt sich vor allem dadurch, dass er die Grundlage des Incentivierungsmechanismus der Geschäftsbereichsleiter und Projektmanager bildet. 
%%
%%
%%
\section{Ziele der Arbeit}\label{sec:zda}
Ziel dieser Arbeit ist es, die Einflussfaktoren des Projektmanagements der Bühler AG zu analysieren. Mittels deskriptiver statistischer Methoden sollen die charakteristischen und quantitative Unterschiede hinsichtlich der Faktoren zwischen erfolgreichen und nicht-erfolgreichen Projekten herausgearbeitet werden. Dabei soll die Projekterfolgsdefinition der Bühler AG berücksichtigt werden, da ihr wesentliche Bedeutung innerhalb des Unternehmens beigemessen wird. Daraus folgt, dass der Analyse eine finanzielle Perspektive zu Grunde liegt, da der Kostenaspekt des magischen Dreiecks aufgrund des Erfolgskriteriums der Bühler AG, die Abweichung Projektmarge vom Sollwert, fokussiert wird. Der Untersuchungsgegenstand bilden sämtliche Projekte die zwischen 2013 und 2015 abgewickelt und abgeschlossen wurden. Er soll anhand des Erfolgskriterium in zwei Gruppen unterteilt (erfolgreich, nicht-erfolgreich) werden, um anschliessend auf der Basis der Auswertung von  Häufigkeiten, Lagerparametern und finanziellen Performance Divergenzen der Gruppen zu untersuchen. Daraus kann folgende zentrale Fragestellung abgeleitet werden: 
\newline\newline
Welche der Einflussfaktoren im Projektmanagement der Bühler AG charakterisieren unter der Berücksichtigung des Erfolgskriterium \glqq nicht-erfolgreiche\grqq{ } Projekte?
\newline\newline
Die Relevanz dieser Forschungsfragen kann  wird einerseits durch die eingangs 
Darauf aufbauend soll auch evaluiert werden, ob sich gewisse Variablen zur Früherkennung von "bedrohten" Projekten eigen. Das Untersuchungsobjekt der Analyse bilde alle im Zeitraum zwischen 2013 und 2015 abgeschlossenen Projekte. Zudem werden ausschliesslich die von der Bühler AG zur Verfügung gestellten Daten untersucht, was die Erhebung zusätzlicher Daten ausschliesst. Aus den obigen Ausführungen leiten sich für diese Arbeit folgende zentrale Fragestellungen ab, die es zu prüfen gilt:

Der theoretische Rahmen der Arbeit bilden das Projektmanagement und deren Erfolgsfaktoren. Deshalb wird vor 
der quantitativen Analyse der Projektbegriff und das Projektmanagement eingehender erläutert und in Relation zum Unternehmensbeispiel gesetzt. Zudem sollen bisherige Erkenntnisse aus der Erfolgsfaktorenforschung aufgezeigt sowie die vorherrschende Erfolgsdefinition von Projekten und deren Wandlungstendenz erläutert werden. Im Anschluss soll der Bühler Projektmanagementprozess und die Einflussfaktoren aufgezeigt und erklärt werden, da er die Grundlage der Datenerfassung und der darauffolgenden Untersuchung bildet. 
\newline\newline
Nach der Erklärung der analytischen Vorgehensweise und Ergebnispräsentation soll das Thema der Früherkennung und Frühwarnsystem aus wissenschaftlicher Sicht eingeleitet werden. Basierend auf den theoretischen Ausführungen und unter der Berücksichtigung der Ergebnisse sollen konzeptionellen Ansätze zur Früherkennung im Projektmanagement der Bühler AG entwickelt werden. Danach soll abschliessend eine kritische Diskussion der gewonnenen Resultate in Verbindung mit den Aspekten der Forschung und der Zielsetzung dieser Arbeit stattfinden.
\newline\newline 
Der Umfang der Untersuchung wurde bereits vorgängig durch die Datenverfügbarkeit des internen Projektmanagementtool eingegrenzt, was zur Folge hat, dass Faktoren welche das Projektmanagement der Bühler AG auch beeinflussen können, im Rahmen dieser Arbeit nicht untersucht werden. Die Aussagekraft der Ergebnisse wird durch die unternehmensspezifische Daten begrenzt, so dass lediglich Rückschlüsse auf die Projekte und den Projektmanagementprozess der Bühler AG gemacht werden können.
\section{Methodik und Struktur der Arbeit}
Die Gliederung der Arbeit unterscheidet drei Abschnitte: der theoretische und unternehmensspezifische Rahmen, die Methodik und Ergebnisse sowie Anwendung und Diskussion der Erkenntnisse. Die Erarbeitung erfolgt dabei auf der Basis einer Kombination von Literaturrecherche, unternehmensspezifischem Wissen sowie statistischer Vorgehensweise.
\newline\newline
Im ersten Abschnitt wird eine theoretische Abhandlung zu Projekten, Projektmanagement und bisher erforschten Erfolgsfaktoren dargelegt sowie der Bühler Projektmanagementprozesse erläutert. Die definitorische Abgrenzung dient dazu den Rahmen der Begrifflichkeiten festzulegen und in Verbindung zu den internen Bestimmungen zu setzen. Auf die Unterschiede der Projektarten, verschiedenen Projektmanagementansätze und deren Kategorisierung wird in dieser Arbeit nicht näher eingegangen, da für das Verständnis der Einflussfaktoren vor allem der Bühler Projektmanagementprozess von Bedeutung ist. Die Ergründung der bisher identifizierten Erfolgsfaktoren dient dazu die dominierende Ansicht der Wissenschaft aufzuzeigen, wobei der Fokus auf den Konstruktionsprojekten liegt. Es wurden allerdings Erkenntnisse anderer Projektarten mitberücksichtigt, mit dem Ziel ein breites Spektrum an möglichen Erfolgsfaktoren zu erhalten. Diese Informationsbasis dient als Referenzpunkt für die Erkenntnisse aus den qualitativen Auswertungen und kann ergänzenden Variablen des Projekterfolges im Bühler Projektmanagement enthalten. Da sich beispielsweise die Faktoren in Abhängigkeit des Erfolgskriteriums unterscheiden können, ist der gewählten Projekterfolgsdefinition im Forschungsdesign eine entscheiden Rolle beizumessen. Deshalb werden unterschiedliche Betrachtungsweisen des Erfolgskriteriums und die mögliche Abwendung vom traditionellen Erfolgskalkül hervorgehoben. Die Prozessbeschreibung des Bühler Projektmanagement orientiert sich an den internen Dokumentationen und Darstellungen, wobei diejenigen Bestandteile eingehender erläutert werden, die dem Verständnis der Einflussfaktoren dienen.
\newline\newline
Der nächste Abschnitt umfasst die Beschreibung des analytischen Vorgehen, der gewählten Methoden und die Präsentation sowie die kritische Beurteilung der Ergebnisse. Die Analyse stützt sich dabei auf Ansätze der deskriptiven Statistik und der Exploration mittels des finanziellen Verlust bezüglich der Projektmarge, weshalb  die Auswertungen separat dargestellt werden. Die anschliessende Interpreation und Würdigung der Ergebnisse beabsichtigt einerseits den Bezug zum Unternehmen und den Prozessen herzustellen und anderseits Erfolgsfaktoren zu identifizieren. 
%Wahl der Methodik???? verzicht auf Befragungen der Bühler AG...externe Perspektive
\newline\newline
Der letzte Abschnitt konzentriert sich auf die Früherkennung und Frühwarnsysteme im Projektmanagement allgemein und der Bühler AG. Dazu wird das theoretische Wissen auf der Basis von Literatur erarbeitet, wobei die zentralen Aspekte zu den Begrifflichkeiten, Anforderungen und Methoden herausgearbeitet werden sollen. Die ausführliche Erläuterung der methodischen Ansätze der Früherkennung und des Implementierungsprozess sind nicht Teil dieser Arbeit, da der Schwerpunkt auf der konzeptionellen Ausarbeitung von Frühwarnindikatoren für das Bühler Projektmanagement unter der Berücksichtigung des Forschungserkenntnisse liegt. Die Ergebnisse der Analyse sowie die Prozessbeschreibungen aus dem ersten Abschnitt werden hierfür ebenso berücksichtigt, da erstere auf mögliche Frühwarnindikatoren hinweisen kann und letztere den Rahmen der Früherkennung festlegen. Danach folgt eine Diskussion und kritische Betrachtung des Konzepts und der Ergebnisse aus der quantitativen Analyse im Zusammenhang mit wissenschaftlichen Erläuterungen und den Zielen dieser Arbeit.
\newline\newline
Abschliesend werden im Fazit die wichtigsten Erkenntnisse der Arbeite zusammenfassend festgehalten und die eingangs erwähnten Forschungsfragen beantwortet. Zudem sollen Empfehlungen für weitere Forschungsthemen und Optimierungspotenziale festgehalten werden.

	




