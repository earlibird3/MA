% !TEX root = MA.tex
\section{Einleitung}
Erfolg spielt in der gegenwärtigen Gesellschaft eine wichtige Rolle, zum Beispiel bei der Partnersuche, Arbeit, in der Schule, im Sport oder bei der Suche nach Lösungen. Jedes Individuum wird während seiner Lebenszeit zwangsläufig mit Erfolg in Berührung kommen, je nachdem wie Erfolg definiert wird. Die Definition des Duden für Erfolg lautet " ein positives Ergebnis einer Bemühung; Eintreten einer beabsichtigten, erstrebten Wirkung". Daraus folgt, dass für den Erfolg ein positives Ergebnis, ein vorgängig definiertes Ziel und geleisteter Effort vorliegen muss. Diese breite Definition lässt sich auf beliebige Situationen im Leben anwenden. Ausserdem kann das vorausgesetzte positive Ergebnis mittels unterschiedlicher Attributen individualisiert respektive verändert werden. Die Perzeption und Bedeutung von Erfolg ist somit bezüglich seiner Vielfältigkeit unbegrenzt.
\newline\newline Die vorliegende Arbeit befasst sich mit dem Erfolg von Projekten mittels Projektmanagement. Seit den 60-iger Jahren beschäftigt sich die Forschungsgemeinschaft mit dem Projekterfolg, und berücksichtigte dabei unterschiedliche Projektarten, Managementmethoden, Industrien und Erfolgsdefinitionen. Die Vielfalt der Projekte und auch Managementansätze  wird in der Literatur unterschiedlich klassifiziert, so zum Beispiel anhand des Projekttyps oder der Branche. In der Softwareindustrie wird beispielsweise zwischen agilen und traditionellen Ansätzen unterschieden, wobei der traditionelle Ansatz auch in anderen Branchen häufig zur Anwendung kommt. Das eindeutigen Merkmale dieses Ansatzes sind einerseits die Trennung der Projektphasen mittels Meilensteinen und anderseits die Steuerung im Rahmen des magischen Dreiecks: Kosten, Zeit und Qualität. Der Erfolg der Projekte wird demzufolge auf Basis der Abweichungen dieser drei Zielgrössen bestimmt. Zur Projektablauf- und Terminplanung im traditionellen Projektmanagement sind die Netzplantechnik oder der Projektstrukturplan häufig angewendete Ansätze, wobei beide zum Ziel haben das Projekt in Teilaufgaben und Aufgabenpakete zu strukturieren. Grundsätzlich werden mit jeder Managementmethode in Abhängigkeit seiner Vor- und Nachteile erfolgreiche respektive nicht-erfolgreiche Projekte abgewickelt, weshalb der Fokus auf dem Erfolg sowie dessen Definition liegt. Die Untersuchung möglicher Einflussfaktoren ist aber angesichts der Interdisziplinarität des Projektmanagements eine herausfordernde Aufgabe. Die Anzahl Determinanten sowie deren Kombinationen ist beliebig, weshalb oftmals versucht wird auf Basis statistischer Zusammenhänge Erfolgsfaktorenmodelle zu ergründen.
\newline\newline Die vorliegende Arbeit beschränkt sich auf die Untersuchung der Einflussfaktoren im Projektmanagement der Bühler AG. Die Bühler AG ist ein Maschinentechnologiekonzern im Familienbesitz mit Hauptsitz in Uzwil. Sie hat eine führende Marktposition in der Herstellung von Maschinen für die Getreideverarbeitung für Mehl, Pasta, Schokolade, Reis, aber auch für die Oberflächenbeschichtung. Die Herstellung von einzelnen Maschien oder eines ganzen Maschinenparks wird mittels Projekten unter der Anwendung des traditionellen Projektmanagementansatzes abgewickelt. Die Notwendigkeit Projekte und Einflussfaktoren des Projektmanagement näher zu untersuchen entstand infolge der schlechten Projektperformance einiger Projekte und der Tatsache, dass nicht frühzeitig erkannt werden konnte, wann die Projekte auf eine 'schiefe Bahn' geraten. Das Projektmanagementtool (BPM-Cockpit), in welchem sämtliche Daten eines Projekts enthalten sind, lieferte Hinweise, dass die Kommunikation einer Verschlechterung der Erfolgsgrössen relativ spät erfolgte. Aus diesen Gründen erarbeitet das Bühler Projektmanagement (BPM) in Zusammenarbeit mit dem Controlling (CO) eine Liste von Faktoren, die gemäss ihrer subjektiven Einschätzung und Erfahrung der Unterscheidung zwischen nicht erfolgreichen und erfolgreichen dient. Die Erhebung basiert auf einer finanziellen Perspektive, weshalb vorwiegend finanzielle Indikatoren zu Umsatz, Auftragsvolumen, Kosten und Projektergebnis berücksichtigt wurden. Die Liste beinhaltet zudem Faktoren, die sich auf den Projektabwicklungs- und Projektverkaufsprozesse beziehen, wobei ein weiterer Fokus auf Indikatoren zum Forecast-Management und der Performance bezüglich Lieferzeit liegt. Die Evaluation des Projektstatus während der Laufzeit und letztendlich des Projekterfolgs basiert auf dem magischen Dreieck, Zeit, Kosten und Qualität. Ein dreifarbiges Ampelsystem reflektiert die Projektperformance finanzieller, zeitlicher und qualitativer Hinsicht, wobei erstere automatisch gemäss dem intern festgelegten Schlüssel berechnet werden und letzter auf der subjektiven Einschätzung des Projektmanagers beruht. Obwohl vorausgesetzt wird, dass die Bewertung nach Projektabschluss für alle drei Kriterien 'grün' ist, das heisst, pünktliche Lieferung, Projektmarge (DB1) möglichst über dem Budget und hervorragende Qualität, liegt der Fokus letztendlich auf den Finanzen. Die Projektmarge berechnet sich aus Umsatz minus Kosten in Relation zum Umsatz, nimmt Einfluss auf das konsolidierte Unternehmensergebnis und dient intern der Incentivierung von Geschäftsbereichsleitern und Projektmanagern. Sämtliche Auswertungen erfolgen anhand der von der Bühler AG zur Verfügung gestellten Daten und berücksichtigen die zugrundeliegende Finanzperspektive. 
%%
%%Früherkennnunng???
%%
\subsection{Ziel der Arbeit}
Ziel dieser Arbeit ist es, die Einflussfaktoren des Projektmanagements der Anlageprojekte der Bühler AG mittels deskriptiver statistischer Methoden retrospektiv zu beurteilen. Dabei wird die Ergründung der Charakteristiken von erfolgreichen und nicht-erfolgreichen Projekten sowie die Analyse der finanziellen Performance fokussiert. Basierend auf den Ergebnissen soll anschliessend versucht werden Faktoren zur Früherkennung von "bedrohten" Projekten zu identifizieren. Diese Ziele sind direkter Ausfluss der Bedürfnisse der Bühler AG und beabsichtigen interne Hypothesen mit Daten belegen zu können. Das Untersuchungsobjekt der Analyse bilden alle im Zeitraum zwischen 2013 und 2015 abgeschlossenen Anlagenprojekte der Bühler AG einschliesslich der von ihr zur Verfügung gestellten Daten. Basierend auf diesen Ausführungen leiten sich folgende zentrale Fragestellungen ab:
\newline\newline
Welche Eigenschaften unterscheiden vergangene nicht-erfolgreiche von erfolgreichen Projekten? Können auf Basis der Ergebnisse Erfolgsfaktoren und Frühwarnindikatoren der Bühler AG begründet werden? Wie hoch wären retrospektiv die finanziellen Ersparnisse unter Anwendung der Früherkennung gewesen?
\newline\newline
Die Analyse grenzt sich durch den Bezug auf ein einziges Unternehmen von bisherigen Studien zur Identifikation möglicher Erfolgsfaktoren im Projektmanagement ab. Die Auswertungen haben dementsprechend nur Gültigkeit in Bezug auf die bereits abgeschlossenen Projekte der Bühler AG. Der Umfang der Analyse wird durch die festgelegte Stichprobe und vorgängig bestimmten Faktoren begrenzt.
\subsection{Methodik und Vorgehen}
Methodisch baut die Arbeit auf einer Kombination von Literaturrecherche unternehmensspezifischem Wissen sowie statistischer Vorgehensweise auf.
\newline
Anfänglich wird im Kapitel \ref{sec:pmerf} eine theoretische Abhandlung zu Projekten, Projektmanagement und bisher erforschten Erfolgsfaktoren dargelegt sowie der Bühler Projektmanagementprozesse erläutert. Der Fokus im ersten Unterkapitel liegt auf der kritischen Betrachtung der Erfolgsdefinition im Projektmanagement und den bisher erforschten Erfolgsfaktoren für Industrieprojekte, wobei andere Projektarten, mit dem Ziel Unterschiede und Gemeinsamkeiten zu ergründen, hinzugezogen wurden. Im zweite Unterkapitel werden diejenigen Phasen des Projektmanagementprozess der Bühler AG beleuchtet, welche den Rahmen der Analyse festlegen und dem Verständnis der anschliessenden Erläuterung der erhobenen Faktoren dienen.
\newline
Das Kapitel \ref{sec:drei} enthält einerseits eine Übersicht des analytischen Vorgehens (Kapitel \ref{sec:dreieins}), die Ergebnisse der Auswertungen (Kapitel \ref{sec:ergebnisse}) sowie deren kritische Würdigung (Kapitel \ref{sec:krit}). Die Resultate der finanziellen Analyse und der Untersuchung der Einflussfaktoren werden dabei getrennt dargestellt. Sie werden dann im darauffolgenden Unterkapitel in Bezug zueinander gesetzt sowie kritisch betrachtet, um mögliche Schlussfolgerungen zu erfolgreichen respektive nicht-erfolgreichen Projekten der Bühler AG machen zu können.
\newline
Im nächsten Kapitel \ref{sec:vier} folgt zuerst eine theoretische Übersicht über die bisherigen Erkenntnisse von Forschungsberichten zur Früherkennung hinsichtlich des Projektmanagements (Kapitel \ref{viereins}). Danach sollen auf Basis der theoretischen Überlegungen und der Ergebnisse von Kapitel \ref{sec:ergebnisse} Ansätze für mögliche Frühwarnindikatoren erarbeitet werden, mit dem Ziel Einfluss auf die Erfolgsquote nehmen zu können (Kapitel \ref{vierzwei}). Dabei wird auch auf die retrospektive finanzielle Ersparnis im Zusammenhang mit der Früherkennung eingegangen.
\newline
Abschliesend werden im Kapitel \ref{sec:fazit} die wichtigsten Erkenntnis der Arbeit sowie ein Ausblick für weitere Forschungsthemen festgehalten.

	




