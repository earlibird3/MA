% !TEX root = MA.tex
\section{Einleitung}
Der Erfolg, eine erfolgreiche Einleitung schreiben, ist nur eine Perzeption von unzähligen. Erfolg spielt in der gegenwärtigen Gesellschaft eine wichtige Rolle, Erfolg bei der Partnersuche, bei der Arbeit, in der Schule, im Sport oder bei der Suche nach Lösungen. Jedes Individuuzm wird während seiner Lebenszeit zwangsläufig mit Erfolg in Berührung kommen, je nachdem wie Erfolg definiert wird. Wenn Erfolg bedeutet, "Laufen können", sind all jene Menschen, die ohne körperliche Beeinträchtigung in Bezug auf die Gehfähigkeit gelernt haben zu laufen, in ihrem Leben bereits einmal erfolgreich gewesen. Die 'neutrale' Definition des Duden für den Erfolg lautet "positives Ergebnis einer Bemühung; Eintreten einer beabsichtigten, erstrebten Wirkung". Daraus folgt, dass für den Erfolg ein positives Ergebnis, ein vorgängig definiertes Ziel und geleisteter Effort vorliegen muss. Diese breite Definition lässt sich auf beliebige Situationen im Leben anwenden. Ausserdem kann das vorausgesetzte positive Ergebnis mittels unterschiedlicher Attributen individualisiert respektive verändert werden. Die Perzeption und Bedeutung von Erfolg ist somit bezüglich seiner Vielfältigkeit unbegrenzt.
\newline Die vorliegende Arbeit befasst sich mit dem Erfolg von Projekten und diesbezüglich mit dem erfolgreichen Projekt Management. Der Erfolg von Projekten beschäftigt die Forschungsgemeinschaft seit den 60iger Jahren, weshalb eine dichte Anzahl von Studien existiert, welche die Faktoren, die den Projekterfolg beeinflussten. Die Studien untersuchen unterschiedliche Projektarten, beispielsweise Softwareentwicklungsprojekte aber auch Industrieprojekte. Die Unterscheidung ist notwendig, da sich die angewandten Methoden zur Abwicklung der Projekte unterscheidet sowie die Rahmenbedingungen divergieren können. So wird bei Softwareprojekten zwischen agilem Vorgehen und der Wasserfall-Methode unterschieden. Die Wasserfallmethode wäre das Pendant zum traditionellen Managementansatz bei Industrieprojekten. Typischerweise erfolgt das Projektmanagement mit Hilfe sogenannten Milestones, das heisst ein Zwischenziel muss erreicht werden, bevor mit dem nächsten Schritt begonnen werden kann. Ebenso unterscheiden sich die Erfolgsdefinitionen, welche in den Studien unterstellt wurden. Im traditionellen Projektmanagementansatz bewegt sich die Erfolgsdefinition von Projekten im magischen Dreieck; Kosten, Zeit und Qualität. Allerdings liegt letztendlich der Fokus auf dem finanzielle Verlust aus der Projektmanagementtätigkeit, da dieser direkten Einfluss auf das Unternehmensergebnis hat.
\newline In dieser Arbeit beschränkt sich die Untersuchung der Projekte auf die Anlageprojekt der Bühler AG, ein Maschinentechnologiekonzern im Familienbesitz. Die Bühler AG produziert Maschinen für die Nahrungsmittelindustrie. Das Hauptgeschäft der Bühler AG war ursprünglich der Bau von Mühlen, wobei die Herstellung von Mahlwerken noch heute der grösste Geschäftsbereich der Bühler AG ist. Seit der Gründung wurde das Maschinenportfolio laufend ergänzt und beinhaltet heute Maschinen zur Herstellung von Pasta, Schokolade, Reis, Anlagen für die Linsenbeschichtung oder Kugelmühlen für den Mischprozess bei Farben. Gemäss den Angaben der Bühler AG decken sie dem gesamten Wertschöpfungsprozess für die Herstellung der entsprechenden Nahrungsmittel ab, das heisst vom Korn bis zum Endprodukt oder von der Kakaofrucht bis zur Schokolade.
\newline




Das Problem zu erkennen ist wichtiger als die Lösung zu erkennen, denn die genaue Darstellung des Problems führt zur Lösung. 
Albert Einstein, Physiker
	\subsection{Ziel der Arbeit}
	\subsection{Methodik und Vorgehen}
	

	




