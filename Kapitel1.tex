% !TEX root = MA.tex
\section{Einleitung}
Der Erfolg, eine erfolgreiche Einleitung schreiben, ist nur eine Perzeption von unzähligen. Erfolg spielt in der gegenwärtigen Gesellschaft eine wichtige Rolle, Erfolg bei der Partnersuche, bei der Arbeit, in der Schule, im Sport oder bei der Suche nach Lösungen. Jedes Individuuzm wird während seiner Lebenszeit zwangsläufig mit Erfolg in Berührung kommen, je nachdem wie Erfolg definiert wird. Wenn Erfolg bedeutet, "Laufen können", sind all jene Menschen, die ohne körperliche Beeinträchtigung in Bezug auf die Gehfähigkeit gelernt haben zu laufen, in ihrem Leben bereits einmal erfolgreich gewesen. Die 'neutrale' Definition des Duden für den Erfolg lautet "positives Ergebnis einer Bemühung; Eintreten einer beabsichtigten, erstrebten Wirkung". Daraus folgt, dass für den Erfolg ein positives Ergebnis, ein vorgängig definiertes Ziel und geleisteter Effort vorliegen muss. Diese breite Definition lässt sich auf beliebige Situationen im Leben anwenden. Ausserdem kann das vorausgesetzte positive Ergebnis mittels unterschiedlicher Attributen individualisiert respektive verändert werden. Die Perzeption und Bedeutung von Erfolg ist somit bezüglich seiner Vielfältigkeit unbegrenzt.
\newline\newline Die vorliegende Arbeit befasst sich mit dem Erfolg von Projekten und diesbezüglich mit dem erfolgreichen Projekt Management. Der Erfolg von Projekten beschäftigt die Forschungsgemeinschaft seit den 60-iger Jahren, weshalb eine dichte Anzahl von Studien existiert, welche die Faktoren, die den Projekterfolg beeinflussten. Die Studien untersuchen unterschiedliche Projektarten, beispielsweise Softwareentwicklungsprojekte aber auch Industrieprojekte. Die Unterscheidung ist notwendig, da sich die angewandten Methoden zur Abwicklung der Projekte unterscheidet sowie die Rahmenbedingungen divergieren können. So wird bei Softwareprojekten zwischen agilem Vorgehen und der Wasserfall-Methode unterschieden. Die Wasserfallmethode wäre das Pendant zum traditionellen Managementansatz für Industrieprojekten. Charakteristisch für die Wasserfallmethode und den traditionellen Projetmanagementansatz ist, das die Ergebnisse einer Projektphase bilden das direkte Bindeglid zur nächsten Projektphase bilden. Dies impliziert die Erreichung eines Zwischenziels, bevor mit dem nächsten Schritt begonnen werden kann. Ebenso unterscheiden sich die Erfolgsdefinitionen, welche in den Studien unterstellt wurden. Im traditionellen Projektmanagementansatz bewegt sich die Erfolgsdefinition von Projekten im magischen Dreieck; Kosten, Zeit und Qualität. Allerdings liegt letztendlich der Fokus auf dem finanzielle Verlust aus der Projektmanagementtätigkeit, da dieser direkten Einfluss auf das Unternehmensergebnis hat.
\newline\newline Diese Arbeit beschränkt sich die Untersuchung der Projekte auf die Anlageprojekt der Bühler AG, ein Maschinentechnologiekonzern im Familienbesitz. Die Bühler AG produziert Maschinen für die Nahrungsmittelindustrie. Das Hauptgeschäft der Bühler AG war ursprünglich der Bau von Mühlen, wobei die Herstellung von Mahlwerken noch heute der grösste Geschäftsbereich der Bühler AG ist. Seit der Gründung wurde das Maschinenportfolio laufend ergänzt und beinhaltet heute Maschinen zur Herstellung von Pasta, Schokolade, Reis, Anlagen für die Linsenbeschichtung oder Kugelmühlen für den Mischprozess bei Farben. Gemäss den Angaben der Bühler AG decken sie dem gesamten Wertschöpfungsprozess für die Herstellung der entsprechenden Nahrungsmittel ab, das heisst vom Korn bis zum Endprodukt oder von der Kakaofrucht bis zur Schokolade. Die Bühler AG wird mittels einer Matrix-Organisation geführt, das heisst es gibt sowohl Regionen- und Geschäftsbereichsverantwortliche. Die Bühler AG ist in sechs Kontinente in über 197 Länder tätig und hat zwei grosse Business Cluster, Grains \& Foods (GF) mit den fünf Geschäftsbereichen Grain Milling (GM), Value Nutrition (NU), Consumer Foos (CF), Sortex (SR) und Grain Logistics (GL) und Advanced Materials (AM) mit den drei Geschäftsbereichen, Druckguss (DC), Grinding \& Dispersion (GD) und Leybold Optics (LO). Die Projektmanagementorganisation ist in den Geschäftsbereichen, wohingegen das Projektreporting für die Gruppe im Verantwortungsbereich des Bühler Project Management (BPM) Team angesiedelt ist. Das BPM ist Bestandteil der Supportfunktion Coporate Finance. Seine Aufgabe ist das monatliche Projektreporting mittels dem BPM-Cockpit für die Gruppe aufzubereiten. Das BPM-Cockpit ist das Management- und Reporting-Tool, in welchem sämtliche verfügbaren Projektinformationen während des Verlaufs zusammengezogen werden. Es zeigt zugleich die historische und gegenwärtige Sicht aller laufenden und abgeschlossenen Projekte. Allerdings arbeiten nicht alle Gesellschaften mit dem BPM-Cockpit, sondern nur diejenigen, bei welchen sich die Implementierung und die Umstellung in Abhängigkeit der Grösse und des Projektumschlages gelohnt hat. Die Verantwortung für die Vollständigkeit der Daten liegt beim Geschäftsbereich und die Sicherstellung der Funktionsfähigkeit bei der IT. Aufgrund der Erkenntnis, dass gewisse Projekte relativ schlecht abschliessen ohne dass die Verschlechterung der finanziellen Performance frühzeitig erkannt wurde, entstand das Bedürfnis diesem Umstand entgegenwirken zu können. Ausserdem wurde erfahrungsgemäss eine Verschlechterung der Kosten relativ spät kommuniziert, und war somit im BPM-Cockpit lange Zeit nicht ersichtlich. Dieses Phänomen ist auf die Umgehung des Erklärungsdruckes eines Geschäftsbereich aufgrund interner Direktiven zurückzuführen. Mit dem Ziel anhand gewisser Faktoren früher erkennen zu können, wann ein Projekt unter finanziellem Druck steht, erarbeitet das BPM-Team in Zusammenarbeit mit dem Controlling (CO) eine Liste aller Faktoren, die gemäss ihrer subjektiven Einschätzung und Erfahrung zur Identifikation von "schlecht" laufenden Projekten notwendig sind respektive die finanzielle Performance beeinflussen. Die Bühler AG arbeitet mit dem traditionellen Projektmanagementansatz und kennt zwölf Milestones, die sukzessive erreicht werden müssen, bevor die nächste Projektphase beginnen kann. Obwohl im BPM-Cockpit die Projektperformance während des ganzen Projektverlaufs mittels dreistufigem Ampelsystem gemessen wird, liegt der Fokus letztendlich auf der realisierten Marge (DB1) pro Projekt. Einerseits basiert die Incentivierung der Geschäftsbereichsleiter und der Projektmanager auf dem DB1 und anderseits er relevant für das Ergebnisses des Standortes Bühler Uzwil (BUZ) und fliesst in das Gruppenergebnis ein. Der Standort in Uzwil ist der Hauptsitz des Konzerns und wickelt zugleich am meisten Projekte ab, weshalb das Ergebnis auch eine besonderen Bedeutung hat.\textbf{Notwendigkeit der Forschung}
\subsection{Ziel der Arbeit}
Ziel dieser Arbeit ist, in erster Linie die Einflussfaktoren des Projektmanagements der Anlageprojekte der Bühler AG sofern möglich mittels statistischer Methoden zu ergründen. Ausgehend von der Annahme, dass diese Wissen mögliche Hinweise für Faktoren liefert, mit welchen "schlecht" laufende Projekte bereits frühzeitig identifiziert werden können, wäre das darauf aufbauende Ziel dieser Arbeit eine Art Frühwarnsystem zu konzeptionieren. Da bisher wenige Hinweise sondern lediglich intuitive Erklärungsansätze für den Erfolg der Bühler-Projekte verfügbar sind, entstand die Notwendigkeit, diese mit Daten zu belegen. Denn wie bereits Einstein gemeint hat, ist die Erkennung des Problems wichtiger als die Lösung zu erkennen, da die genaue Darstellung des Problems zur Lösung führen würde. Das Objekt der Analyse bilden alle im Zeitraum von 2013 bis 2015 abgeschlossenen Projekte. Für die Datenerhebung der identifizierten Faktoren wurde ein eigene Abfrage im BEX-Analyzer programmiert, da eine Vielzahl der möglichen Einflussfaktoren trotz der Verfügbarkeit der für die Berechnung benötigten Daten nicht im BPM-Cockpit enthalten waren. Basierend auf diesen Ausführungen leiten sich folgende zentrale Fragestellungen ab:
\newline\newline
Welche der ausgearbeiteten Faktoren beeinflussen das Projektmanagement respektive den Projekterfolg? Basierend auf den Ergebnisse, welche Faktoren käme für ein Frühwarnsystem in Frage? Und wie hoch wäre retrospektiv die finanzielle Einsparnis unter Anwendung des Frühwarnsystems gewesen?
\newline\newline
Die Analyse ist retrospektiver Natur und basiert auf projektbezogene Daten. In bisherigen Studien wurden oftmals Experten der Industrie, Projektmanager oder anderer Projektteilnehmer bezüglich ihrer Einschätzung des Einflusses ausgewählter Faktoren auf den Projekterfolg befragt. Die Erfassung der Meinung erfolgte dabei mittels Likert-Skale, welche dann die numerische Datengrundlage bildete. Eine weitere Einschränkung der Analyse bildet die finanzielle Perspektive, mit welcher die Faktoren erhoben wurden. Folglich wurden Determinanten, beispielsweise der Teamgeist eines Projektteams oder das Schnittstellenmanagement nicht erhoben, da einerseits die Messbarkeit relativ schwierig ist und anderseits der Fokus auf dem Forecast-Management und finanziellen Einflussgrössen lag. 
\subsection{Methodik und Vorgehen}
	

	




