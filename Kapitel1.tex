% !TEX root = MA.tex
\chapter{Einleitung}
Erfolg nimmt in der gegenwärtigen Gesellschaft eine zentrale Rolle ein, zum Beispiel im Privatleben bei der Partnersuche, in der Arbeitswelt zur Unternehmensführung oder während der Ausbildung hinsichtlich bestandener Prüfungen. Folglich kann postuliert werden, dass jedes Individuum während seiner Lebenszeit zwangsläufig mit Erfolg konfrontiert. Die Definition des Duden für Erfolg lautet "ein positives Ergebnis einer Bemühung; Eintreten einer beabsichtigten, erstrebten Wirkung". Die Auslegung des Begriffs umfasst somit ein positive Ergebnis einer Anstrengung zur Erreichung eines vorgängig definierten Ziel. Diese breite Definition kann mittels unterschiedlicher Attributen individualisiert werden, so dass sie auf unzählige Situationen anwendbar ist.
\newline\newline
 Die vorliegende Arbeit befasst sich mit dem Erfolg von Projekten und somit mit den Einflussfaktoren im Projektmanagement. Die traditionelle Projekterfolgsdefinition unabhängig der Projektart orientiert sich am magischen Dreieck Zeit, Kosten und Qualität (Atkinson, 1999, S. 337 \& Kerzner, 2014, S. 40), wobei eine minimale Abweichung hinsichtlich der Zielvorgaben dieser drei Grössen angestrebt wird. Zieldivergenzen und die Ungewissheit der Ursachen des Projektscheiterns gaben in Vergangenheit Anlass zur Erforschung der Faktoren, die den Projekterfolg begünstigen. Dabei wurden unterschiedliche Erfolgsdefinitionen, Projektmanagementansätze, Projektarten und Industrien berücksichtigt. Gemäss dem Projektmagazin (2014) kann der herkömmliche Projektmanagementansatz durch die Steuerung des Projektablaufs im magischen Dreieck und die Trennung der Projektphasen mittels Meilensteinen charakterisiert werden. Davon sind die agilen Techniken zu unterscheiden, welche einerseits die strikte Einhaltung von Budget, Zeit und Qualität weniger fokussieren und anderseits Veränderungen des Projektumfeldes oder Leistungsumfangs durch flexiblere Projektdurchführung berücksichtigen (Projektmagazin, 2014). Aufgrund der unterschiedlichen Managementmethoden, Projektarten und Industrien existiert eine beliebige Anzahl von mögliche Erfolgsfaktoren. Projektmanagement kann angesichts seiner Interdisziplinarität mit dem Management einer kleineren Organisation verglichen werden, welche die unbegrenzte Anzahl möglicher Einflussdeterminanten verdeutlicht. Die Herkulesaufgabe ist folglich die Identifizierung der wenigen, relevanten, den sogenannten kritischen Erfolgsfaktoren (Quelle finden). \cite{BeDeNov2015} weisen darauf hin, dass die Erfolgskriterien von den Erfolgsfaktoren abzugrenzen sind. Denn erstere beurteilen, ob das Projekt erfolgreich war, wohingegen letztere mit unabhängige Grössen die zur Erfolgswahrscheinlichkeit eines Projekts beitragen, assoziiert werden (Besteiero, de Souza Pinto \& Novaski, 2015). Demzufolge ermöglicht die Kenntnisse der kritischen Erfolgsdeterminanten den Projektmanagern gezielte Einflussnahme auf die Erfolgswahrscheinlichkeit. Somit kann postuliert werden, dass das übergeordnete Ziel im Projektmanagement letztendlich die Erhöhung der Anzahl erfolgreich abgeschlossener Projekte ist. Variablen des Projekterfolgs beeinflussen den Projektmanagementprozess und können somit auch als Einflussfaktoren im Projektmanagement benannt werden. Deshalb wird nachfolgend Einfluss- und Erfolgsfaktor als Synonyme verwendet. Im engen Zusammenhang mit der Theorie der Erfolgsfaktoren und dem Projekterfolg steht die Früherkennung respektive die Anwendung eines Frühwarnsystems. Wie der Terminus bereits impliziert hat sie zum Ziel, möglichst früh Gefahren und Chancen zu erkennen. Im Projektmanagement bedeutet dies, frühzeitig zu wissen, wann gewisse Projekte hinsichtlich des Erfolgs gefährdet sind, so dass schnellst möglich entsprechenden Gegenmassnahmen ergriffen werden können (Haji-Kazemi, Andersen \& Krane, 2013). Die Fokussierung dieses Konzepts im Projektmanagement wird damit begründet, einerseits trotz stetiger Verbesserung der Projektmanagement-Tools noch sehr viel Projekte misslingen und anders das Projektscheitern oftmals überraschend eintrat (Williams, et. al, 2012 und Haji-Kezmi \& Andersen, 2013). 
\newline\newline
In dieser Arbeit werden die Einflussfaktoren und Ansätze für Frühwarnsysteme im Projektmanagement der Bühler AG untersucht. Die Bühler AG ist ein Maschinentechnologiekonzern im Familienbesitz mit Hauptsitz in Uzwil. Sie hat eine führende Marktposition in der Herstellung von Maschinen für die Getreideverarbeitung für Mehl, Pasta, Schokolade, Reis, und auch für die Oberflächenbeschichtung. Die Herstellung von einzelnen Maschinen oder eines ganzen Maschinenparks wird mittels Projekten unter der Anwendung des vorhin beschriebenen traditionellen Projektmanagementansatzes abgewickelt. Infolge der unzureichenden Performance einiger Projekte und der Erkenntnis, dass eine drohende Verschlechterung der Projektleistung relativ spät im Bühler-Projektmanagement-Cockpit (BPM-Cockpit) ersichtlich war, entstand die Ambition diejenigen Variablen, welche den Projekterfolg beeinflussen können zu erfassen und untersuchen. Aus diesen Gründen erarbeitete 2016 das Bühler Projektmanagement in Zusammenarbeit mit dem Controlling eine Liste von Faktoren, die gemäss ihrer subjektiven Einschätzung und Erfahrung die Unterscheidung zwischen nicht-erfolgreichen und erfolgreichen Projekten ermöglichen. Die Erhebung erfolgte auf der Basis von den verfügbaren Daten des BPM-Cockpits, so dass auch die Berechnung zusätzlicher Indikatoren möglich gewesen ist. Ausserdem wurde der Projektmanagementprozess vom Verkauf bis zum Projektabschluss gesamtheitlich berücksichtigt. Es wurden sowohl finanzielle Faktoren zur Kosten- und Zeitperformance als auch Indikatoren zum Forecast (FC) Management und personelle Determinanten zum Projekt- und Verkaufsmanager ermittelt. Der Erfolg der Bühler-Projekte wird anhand der Zielgrössen Kosten, Zeit und Qualität beurteilt. Im BPM-Cockpit wird während der gesamten Projektlaufzeit der Status bezüglich der Zielerreichung mittels dem dreifarbigen Ampelsystem reflektiert. Dabei ändert die Ampelfarbe der Kosten- und Zeitampel gemäss hinterlegtem Schlüssel automatisch, wohingegen die Qualitätsampel erst mit der Eingabe der subjektiven Einschätzung des Projektmanagers die Farbe verändert. Obwohl alle drei Determinanten evaluiert werden, hat aus finanzieller Perspektive  die monetäre Zielerreichung ein wesentlicher Stellenwert. Der Residualwert aus Umsatz minus Kosten sprich die Projektmarge \gls{abk:db1} hat direkten Einfluss auf das Ergebnis des Anlagengeschäfts der Bühler AG. Ausserdem wird der DB1 in Prozent zur Incentivierung von Geschäftsbereichsleitern und Projektmanagern angewendet. Die Abweichung der prozentualen von der realisierten Projektmarge entscheidet demzufolge über den Erfolg eines Projekts und definiert somit das Erfolgskriterium der nachfolgenden Analyse der Bühler-Projektdaten.  
%%
%%
%%
\section{Ziele der Arbeit}\label{sec:zda}
Ziel dieser Arbeit ist es, die Einflussfaktoren des Projektmanagements der Bühler AG zu analysieren. Unter der Anwendung deskriptiver statistischer Methoden sollen auf Basis der Bühler-Projektdaten retrospektiv charakteristische Unterschiede zwischen erfolgreichen und nicht-erfolgreichen Anlageprojekte herausgearbeitet werden. Dazu sollen die Daten anhand des Erfolgskriterium in zwei Gruppen unterteilt werden und anschliessend auf der Basis von Häufigkeitsverteilungen und Mittelwertauswertungen die Charakteristiken nicht-erfolgreicher Projekte ergründet werden. Durch den Vergleich mit den erfolgreichen Projekten könnten allfällige Differenzen festgestellt werden. Zudem sollen Auswertungen der Finanzdaten zu Kostenabweichungen und Margeneinbussen Attribute von gescheiterten Projekten aufzeigen. Dieses Vorgehen dient vor allem dazu, die Verluste zu \glqq lokalisieren\grqq{} und der Finanzperspektive aufgrund des monetären Erfolgskriterium Rechnung tragen. Basierend auf den Ergebnissen sollen mögliche Erfolgsfaktoren im Bühler Projektmanagement zu erörtern. Darauf aufbauend soll auch evaluiert werden, ob sich gewisse Variablen zur Früherkennung von "bedrohten" Projekten eigen. Das Untersuchungsobjekt der Analyse bilde alle im Zeitraum zwischen 2013 und 2015 abgeschlossenen Projekte. Zudem werden ausschliesslich die von der Bühler AG zur Verfügung gestellten Daten untersucht, was die Erhebung zusätzlicher Daten ausschliesst. Aus den obigen Ausführungen leiten sich für diese Arbeit folgende zentrale Fragestellungen ab, die es zu prüfen gilt:
\newline\newline
Welche Eigenschaften unterscheiden vergangene nicht-erfolgreiche von erfolgreichen Projekten? Können auf Basis der Ergebnisse Erfolgsfaktoren und Frühwarnindikatoren der Bühler AG begründet werden? Wie hoch wären retrospektiv die finanziellen Ersparnisse unter Anwendung der Früherkennung gewesen?
\newline\newline
Der theoretische Rahmen der Arbeit bilden das Projektmanagement und deren Erfolgsfaktoren. Deshalb wird vor 
der quantitativen Analyse der Projektbegriff und das Projektmanagement eingehender erläutert und in Relation zum Unternehmensbeispiel gesetzt. Zudem sollen bisherige Erkenntnisse aus der Erfolgsfaktorenforschung aufgezeigt sowie die vorherrschende Erfolgsdefinition von Projekten und deren Wandlungstendenz erläutert werden. Im Anschluss soll der Bühler Projektmanagementprozess und die Einflussfaktoren aufgezeigt und erklärt werden, da er die Grundlage der Datenerfassung und der darauffolgenden Untersuchung bildet. 
\newline\newline
Nach der Erklärung der analytischen Vorgehensweise und Ergebnispräsentation soll das Thema der Früherkennung und Frühwarnsystem aus wissenschaftlicher Sicht eingeleitet werden. Basierend auf den theoretischen Ausführungen und unter der Berücksichtigung der Ergebnisse sollen konzeptionellen Ansätze zur Früherkennung im Projektmanagement der Bühler AG entwickelt werden. Danach soll abschliessend eine kritische Diskussion der gewonnenen Resultate in Verbindung mit den Aspekten der Forschung und der Zielsetzung dieser Arbeit stattfinden.
\newline\newline 
Der Umfang der Untersuchung wurde bereits vorgängig durch die Datenverfügbarkeit des internen Projektmanagementtool eingegrenzt, was zur Folge hat, dass Faktoren welche das Projektmanagement der Bühler AG auch beeinflussen können, im Rahmen dieser Arbeit nicht untersucht werden. Die Aussagekraft der Ergebnisse wird durch die unternehmensspezifische Daten begrenzt, so dass lediglich Rückschlüsse auf die Projekte und den Projektmanagementprozess der Bühler AG gemacht werden können.
\section{Methodik und Struktur der Arbeit}
Die Gliederung der Arbeit unterscheidet drei Abschnitte: der theoretische und unternehmensspezifische Rahmen, die Methodik und Ergebnisse sowie Anwendung und Diskussion der Erkenntnisse. Die Erarbeitung erfolgt dabei auf der Basis einer Kombination von Literaturrecherche, unternehmensspezifischem Wissen sowie statistischer Vorgehensweise.
\newline\newline
Im ersten Abschnitt wird eine theoretische Abhandlung zu Projekten, Projektmanagement und bisher erforschten Erfolgsfaktoren dargelegt sowie der Bühler Projektmanagementprozesse erläutert. Die definitorische Abgrenzung dient dazu den Rahmen der Begrifflichkeiten festzulegen und in Verbindung zu den internen Bestimmungen zu setzen. Auf die Unterschiede der Projektarten, verschiedenen Projektmanagementansätze und deren Kategorisierung wird in dieser Arbeit nicht näher eingegangen, da für das Verständnis der Einflussfaktoren vor allem der Bühler Projektmanagementprozess von Bedeutung ist. Die Ergründung der bisher identifizierten Erfolgsfaktoren dient dazu die dominierende Ansicht der Wissenschaft aufzuzeigen, wobei der Fokus auf den Konstruktionsprojekten liegt. Es wurden allerdings Erkenntnisse anderer Projektarten mitberücksichtigt, mit dem Ziel ein breites Spektrum an möglichen Erfolgsfaktoren zu erhalten. Diese Informationsbasis dient als Referenzpunkt für die Erkenntnisse aus den qualitativen Auswertungen und kann ergänzenden Variablen des Projekterfolges im Bühler Projektmanagement enthalten. Da sich beispielsweise die Faktoren in Abhängigkeit des Erfolgskriteriums unterscheiden können, ist der gewählten Projekterfolgsdefinition im Forschungsdesign eine entscheiden Rolle beizumessen. Deshalb werden unterschiedliche Betrachtungsweisen des Erfolgskriteriums und die mögliche Abwendung vom traditionellen Erfolgskalkül hervorgehoben. Die Prozessbeschreibung des Bühler Projektmanagement orientiert sich an den internen Dokumentationen und Darstellungen, wobei diejenigen Bestandteile eingehender erläutert werden, die dem Verständnis der Einflussfaktoren dienen.
\newline\newline
Der nächste Abschnitt umfasst die Beschreibung des analytischen Vorgehen, der gewählten Methoden und die Präsentation sowie die kritische Beurteilung der Ergebnisse. Die Analyse stützt sich dabei auf Ansätze der deskriptiven Statistik und der Exploration mittels des finanziellen Verlust bezüglich der Projektmarge, weshalb  die Auswertungen separat dargestellt werden. Die anschliessende Interpreation und Würdigung der Ergebnisse beabsichtigt einerseits den Bezug zum Unternehmen und den Prozessen herzustellen und anderseits Erfolgsfaktoren zu identifizieren. 
%Wahl der Methodik???? verzicht auf Befragungen der Bühler AG...externe Perspektive
\newline\newline
Der letzte Abschnitt konzentriert sich auf die Früherkennung und Frühwarnsysteme im Projektmanagement allgemein und der Bühler AG. Dazu wird das theoretische Wissen auf der Basis von Literatur erarbeitet, wobei die zentralen Aspekte zu den Begrifflichkeiten, Anforderungen und Methoden herausgearbeitet werden sollen. Die ausführliche Erläuterung der methodischen Ansätze der Früherkennung und des Implementierungsprozess sind nicht Teil dieser Arbeit, da der Schwerpunkt auf der konzeptionellen Ausarbeitung von Frühwarnindikatoren für das Bühler Projektmanagement unter der Berücksichtigung des Forschungserkenntnisse liegt. Die Ergebnisse der Analyse sowie die Prozessbeschreibungen aus dem ersten Abschnitt werden hierfür ebenso berücksichtigt, da erstere auf mögliche Frühwarnindikatoren hinweisen kann und letztere den Rahmen der Früherkennung festlegen. Danach folgt eine Diskussion und kritische Betrachtung des Konzepts und der Ergebnisse aus der quantitativen Analyse im Zusammenhang mit wissenschaftlichen Erläuterungen und den Zielen dieser Arbeit.
\newline\newline
Abschliesend werden im Fazit die wichtigsten Erkenntnisse der Arbeite zusammenfassend festgehalten und die eingangs erwähnten Forschungsfragen beantwortet. Zudem sollen Empfehlungen für weitere Forschungsthemen und Optimierungspotenziale festgehalten werden.

	




