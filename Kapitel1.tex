% !TEX root = MA.tex
\chapter{Einleitung}
\glqq Nachhaltiges Wachstum\grqq{} ist eines der Ziele der Bühler AG \citep{Buhl}, ein führender Technologiekonzern in der Nahrungsmittelindustrie. Die Bühler AG \citet{Buhl} bezieht sich dabei auf die starke Eigenfinanzierung, nachhaltiges Umsatzwachstum und das operative Ergebnis, welches den langfristigen Fortbestand des Unternehmens absichern soll. Die Bühler AG leistet gemäss ihren eigenen Angaben einen bedeutenden Beitrag zur Welternährung einer wachsenden Weltbevölkerung (Factsheet). Sie ist in über 140 Länder tätig und führend in der Verarbeitung von Getreide, Reis, Schokolade und Kaffee. Zudem hat sie sich mit dem \glqq Erfolg des Kunden als Massstab\grqq{ }ein ambitiöses Ziel gesetzt \citet{Buhl}, selber erfolgreich zu wirtschaften. Die Herstellung einer Anlage oder eines Maschinenparks wird als Projekt an unterschiedliche Kunden verkauft. Die Projektabwicklung, sprich das Projektmanagement, hat somit in Bezug auf den Projekterfolg einen zentralen Stellenwert. Unvorhergesehene Ereignisse, unzureichende Planung oder Einschätzung der Risiken können schnell in erhebliche Mehrkosten übergehen. Deshalb hat der finanzielle Projekterfolg angesichts der Ziele der Bühler AG eine entscheidende Rolle.
\newline\newline
Die herkömmliche Beurteilung des Projekterfolges inkludiert neben den Kosten die Zeit und Qualität \citep{Atk1999}. Die Darstellung dieser drei Zielgrössen in einer Dreiecksform geht auf Martin Barne zurück und wird aufgrund der verbreiteten Anwendung als das eiserne Dreieck oder die traditionelle Leistungsmessung bezeichnet \citep{kerz14,lock07}. Sie bildet die Tradeoff-Beziehung zwischen den Zielen ab, da bei der Fokussierung eines der Ziele die Erreichung des anderen Ziels beeinträchtigen werden kann \citep[S.~21]{lock07}. Letztendlich entscheidet aber das Unternehmen, welche Kriterien zur Messung des Projekterfolgs herangezogen werden. Da verschiedene Einflussgrössen, wie zum Beispiel die Projektteilnehmer oder die Projektbedingungen, den gesamten Projektmanagementprozess beeinflussen können, wurde bereits mehrfach untersucht, welche Faktoren während dem Projektmanagementprozess eine entscheidende Rolle in Bezug auf den Projekterfolg haben. Die relevanten Faktoren werden im empirischen Fachjargon als kritische Erfolgsfaktoren bezeichnet. Gemäss \citet{BeDeNov2015} sind Erfolgsfaktoren von den Erfolgskriterien abzugrenzen. Die Kriterien beurteilen nach dem Projektabschluss den Projekterfolg, wohingegen Erfolgsfaktoren zur Erfolgswahrscheinlichkeit eines Projekts beitragen \citep*{BeDeNov2015}. Demzufolge sind die Zielgrössen des eisernen Dreiecks als Erfolgskriterien einzuordnen. Die Ergründung der kritischen Erfolgsaktoren ist eine Option, den Projekterfolg aktiv zu managen, da die Unternehmen und Projektmanager dadurch wissen, welche Faktoren sie im Auge behalten müssen. Da jedoch das Projektscheitern häufig überraschend eintrat und potenzielle Gefahren gar nicht oder zu spät erkannt wurden, hat die Implementierung eines Frühwarnsystems (im wirtschaftlichen Kontext: Früherkennung) im Projektmanagement an Bedeutung gewonnen \citep*{WiKlak12,haan13}. Das Ziel besteht darin, mittels Frühwarnindikatoren die möglichen Gefahren und Chancen in einer frühen Projektphase zu erkennen, sodass entsprechende Gegenmassnahmen rechtzeitig ergriffen werden können \citep{haankra13}. Zur Identifikation von Frühwarnindikatoren existiert bisher noch kein formaler Ansatz, weshalb verschiedene Techniken aus anderen Anwendungsbereichen herangezogen werden.
\newline\newline
Die vorliegende Arbeit konzentriert sich auf die Einflussgrössen im Projektmanagement der Bühler AG. Wie bereits einleitend erwähnt wurde, wird Fabrikation von einzelnen Maschinen oder eines ganzen Maschinenparks mittels Projekten unter der Anwendung des Bühler Projektmanagement abgewickelt. Infolge der unzureichenden Performance einiger Projekte und der Erkenntnis, dass eine drohende Verschlechterung der Projektleistung relativ spät im Projektmanagementtool ersichtlich war, entstand die Ambition, diejenigen Faktoren, welche den Projektmanagementprozess und letztendlich den Projekterfolg beeinflussen können, zu erfassen. Aus diesen Gründen entstand 2016 eine Liste sämtlicher potenzieller Faktoren, die das Bühler Projektmanagement in Zusammenarbeit mit dem Controlling basierend auf ihren Erfahrungen und Vermutungen erarbeitet hat. Das Ziel bestand einerseits darin, Unterschiede hinsichtlich der Ausprägung dieser Faktoren zwischen erfolgreichen und nicht-erfolgreichen Projekten zu untersuchen und anderseits einen Zusammenhang zwischen Erfolg und einem oder mehreren Faktoren zu ergründen. Anhand der Ergebnisse wurde die Identifikation der Erfolgsfaktoren sowie die Herleitung möglicher Frühwarnindikatoren beabsichtigt.
\newline \newline Der Erfolg der Bühler-Projekte wird mit dem traditionellen Ansatz Kosten, Zeit und Qualität beurteilt. Im \gls{abk:bpm} wird während der gesamten Projektlaufzeit der prognostizierte Status der Zielerreichung mittels dem dreifarbigen Ampelsystem reflektiert. Dabei ändert die Ampelfarbe der Kosten- und Zeitampel gemäss hinterlegtem Schlüssel automatisch, wohingegen die Qualitätsampel erst mit der Eingabe der Einschätzung des Projektmanagers die Farbe verändert. Obwohl die Zielerreichung aller drei Kriterien gleichermassen vorausgesetzt wird, hat aus einer Finanzperspektive der Kostenaspekt einen wesentlichen Stellenwert. Erhebliche Mehrkosten beeinflussen die Projektmarge \gls{abk:db1} und somit das Ergebnis des Anlagengeschäfts der Bühler AG. Der absolute DB1 eines Projekts berechnet sich aus Umsatz minus Kosten, wobei er in Relation zum Umsatz die prozentuale Marge wiedergibt. Zur Beurteilung des Projekterfolgs wird der realisierte mit dem budgetierten Wert verglichen. Die Relevanz dieses Soll-Ist-Vergleichs bei der Bühler AG zeigt sich neben den finanziellen Performancezielen vor allem dadurch, dass er die Grundlage des Incentivierungsmechanismus der Geschäftsbereichsleiter und Projektmanager bildet. 
%%
%%
%%
\section{Ziele der Arbeit}\label{sec:zda}
Ziel dieser Arbeit ist es, die Faktoren, welche die Bühler AG identifiziert hat, zu analysieren. Mittels deskriptiver statistischer Methoden sollen die Unterschiede auf der Basis der Projektdaten zwischen erfolgreichen und nicht-erfolgreichen Projekten herausgearbeitet werden. Dabei soll das finanzielle Erfolgskriterium der Bühler AG berücksichtigt werden, da ihm wesentliche Bedeutung innerhalb des Unternehmens beigemessen wird. Daraus folgt, dass der Analyse eine finanzielle Perspektive zu Grunde liegt. Den Untersuchungsgegenstand bilden sämtliche Projekte, die zwischen 2013 und 2015 abgewickelt und abgeschlossen wurden. Die Projekte sollen anhand des Erfolgskriteriums in zwei Gruppen (erfolgreich, nicht erfolgreich) unterteilt  werden, um anschliessend auf der Basis der Auswertung von  Häufigkeiten, Lageparametern und finanziellen Performance Differenzen zwischen den Gruppen zu untersuchen. Daraus kann folgende zentrale Fragestellung abgeleitet werden: 
\newline\newline
Welche Unterschiede charakterisieren die erfolgreichen und nicht-erfolgreichen Projekte der Bühler AG?
\newline\newline
Das Ziel des nachhaltigen Wachstums und die Problematik der späten Erkennung von gefährdeten Projekten bilden den Ausgangspunkt dieser Analyse. Die Untersuchung der Projekte bezüglich ihrer Charakteristiken ermöglicht im besten Fall die Typisierung nicht erfolgreicher Projekte. Zudem könnten sich aus den Ergebnissen Hinweise für mögliche Erfolgsfaktoren und Frühwarnindikatoren ergeben. Deshalb sollen als untergeordnetes Ziel unter der Berücksichtigung des Projektmanagementprozesses und dem aktuellen Stand der Forschung methodische Ansätze zur Früherkennung im Bühler Projektmanagement diskutiert werden. In diesem Zusammenhang sind retrospektiv die finanziellen Einsparungen unter der hypothetischen Anwendung des Frühwarnsystems ein weiterer Aspekt, der von der Finanzperspektive her von Interesse ist. 
\newline\newline
Der theoretische Rahmen der Arbeit bildet die Theorie der Erfolgsfaktoren im Projektmanagement sowie der Früherkennung. Zum Verständnis der Faktoren wird der Bühler Projektmanagementprozess erklärt werden, da er die Grundlage der Datenerfassung und somit der Analyse bildet. Im methodischen Abschnitt dieser Arbeit sollen zunächst die Datengrundlage, das analytische Vorgehen sowie die Faktoren erläutert werden. Nach der Ergebnispräsentation sollen die Erkenntnisse in Verbindung mit den Aspekten der Forschung, der Zielsetzung dieser Arbeit und den angewandten Methoden diskutiert werden. Die Früherkennungsansätze im Projektmanagement der Bühler AG sollen in einem weiteren Schritt erarbeitet und kritisch betrachtet werden.
\newline\newline
Es werden ausschliesslich die von der Bühler AG zur Verfügung gestellten Daten untersucht. Die Aussagekraft der Ergebnisse wird durch die unternehmensspezifischen Daten begrenzt, so dass lediglich Rückschlüsse auf die Projekte und den Projektmanagementprozess der Bühler AG gemacht werden können.
%%
%%Methodik
%%
\section{Methodik und Struktur der Arbeit}
Die Gliederung der Arbeit unterscheidet fünf Kapitel: der theoretische und unternehmensspezifische Rahmen, die Methodologie, Ergebnisse, die Diskussion und das Fazit. Die Erarbeitung erfolgt dabei auf der Basis einer Kombination von Literaturrecherche, unternehmensspezifischem Wissen, das während der Anstellung bei der Bühler AG erlangt wurde, sowie statistischer Vorgehensweise.
%Deshalb wird vor der quantitativen Analyse der Projektbegriff und das Projektmanagement eingehender erläutert und in Relation zum Unternehmensbeispiel gesetzt. Zudem sollen bisherige Erkenntnisse aus der Erfolgsfaktorenforschung aufgezeigt sowie die vorherrschende Erfolgsdefinition von Projekten und deren Wandlungstendenz eingehender erläutert werden.
\newline\newline
Im ersten Abschnitt wird eine theoretische Abhandlung zu Projekten, Projektmanagement und bisher erforschten Erfolgsfaktoren dargelegt sowie der Bühler Projektmanagementprozesse erläutert. Die definitorische Abgrenzung dient dazu, den Rahmen der Begrifflichkeiten festzulegen und in Verbindung zu den internen Bestimmungen zu setzen. Auf die Unterschiede der Projektarten, verschiedener Projektmanagementansätze und deren Kategorisierung wird in dieser Arbeit nicht näher eingegangen, da für das Verständnis der Einflussfaktoren vor allem der Bühler Projektmanagementprozess von Bedeutung ist. Im Anschluss werden die bisher erforschten Erfolgsfaktoren im Projektmanagement von Konstruktionsprojekten und weitere Ansätze zur Beurteilung des Projekterfolgs erläutert. Die Prozessbeschreibung des Bühler Projektmanagement orientiert sich an den internen Dokumentationen und Darstellungen, wobei diejenigen Bestandteile eingehender erläutert werden, die dem Verständnis der Faktoren dienen. Anschliessend folgt die theoretische Einführung in die Früherkennung im Allgemeinen und in Bezug auf das Projektmanagement, wobei vorwiegend auf methodischen Ansätze der Früherkennung eingegangen wird.
\newline\newline
Der zweite Abschnitt umfasst die Beschreibung der Datengrundlage und Datenanalyse. Anschliessend folgt im dritten Abschnitt die deskriptive Darlegung der Ergebnisse. Der vierte Abschnitt konzentriert sich auf die Diskussion der Ergebnisse der Analyse und der Früherkennung im Bühler Projektmanagement im Kontext der aktuellen Forschung, der Zielsetzung und dem Methodenansatz dieser Arbeit.
\newline\newline
Abschliessend werden im Fazit die wichtigsten Erkenntnisse der Arbeit zusammengefasst und die eingangs erwähnten Forschungsfragen beantwortet. Zudem sollen Empfehlungen für weitere Forschungsthemen und Optimierungspotenziale festgehalten werden.

	




