% !TEX root = MA.tex
\section{Frühwarnsysteme und Frühwarnindikatoren im Projektmanagement}\label{sec:vier}
Die Notwendigkeit von Früherkennung im Projektmanagement lässt mit dem Auftauchen neuer Herausforderungen während der Projektlaufzeit sowie der sich kontinuierlich verändernden Projektumwelt begründen. Die Anforderungen an die Flexibilität im Projektmanagement und die Fähigkeit zukünftige Ereignisse "vorauszusehen" sind gestiegen. Früherkennung hat zum Ziel aufkommende Gefahren und Chancen frühzeitig zu identifizieren, so dass rechtzeitig entsprechende Massnahmen eingeleitet werden können. In gegenwärtigen Managementsystem der Bühler AG wird ein sich verschlechternder Projektstatus erst bei fortgeschrittener Projektlaufzeit ersichtlich, weshalb beabsichtigt wird mittels der Implementierung eines Frühwarnsystems, dieser Situation entgegenzuwirken. Dadurch sollen mehr Projekte erfolgreich abgeschlossen und das die Marge des Anlagegeschäfts verbessert werden. Anschliessend folgt die Erläuterung der drei Generationen von Frühwarnsystem und sowie deren die Anforderungen. Danach werden Ideen und Ansätze für Früherkennung bei der Bühler AG diskutiert.
\subsection{Frühwarnsysteme und Frühwarnindikatoren}\label{viereins}
Die Voraussetzungen zur Anwendung eines Frühwarnsystems sind gemäss Jacobs, Riegler \& Matter (2012, S. 23), die Möglichkeit ambivalenter Ausgänge, die Gefährdung dominanter Ziele sowie der prozessuale Ablauf einer drohenden "Krise". Obwohl diese Kriterien im Zusammenhang mit Unternehmenskrisen ausgearbeitet wurden, können sie für das Projektmanagement ebenso angewendet werden. Grundsätzlich werden folgende drei Generationen unterschieden.
\begin{description}
	\item[1. Generation:] Die erste Generation orientiert sich an traditionellen Kennzahlen- und Hochrechnungen. Vergleiche der Prognosen mit Sollwerten werden dann als Frühwarnsignale verwendet.
	\item[2. Generation:] Die zweite Generation basiert auf der Anwendung von Indikatoren und Prognosen auf Basis von Faktorenmodelle, die statistisch signifikante Zusammenhänge aufweisen mit dem gewählten Indikator haben.
	\item[3. Generation:] Die dritte Generation stützt sich auf die Theorie der schwachen Signale, die intuitiver und unstrukturierter Natur sind. 
\end{description}
Die erste und zweite Generation haben den Nachteil, dass die Aussagekraft der vergangenheits- und gegenwartsorientierten Datengrundlage relativ beschränkt ist und anderseits durch die Selektion relevante Faktoren unbeachtet bleiben. Aus diesen Gründen sind Diskontinuitäten aus Basis der Hochrechnungen nur schwer erkennbar und an die zugrunde liegenden mathematischen respektive statistischen Modelle gebunden   (Jacobs, Riniger \& Matter, 2012, S. 26- 28). Die dritte Generation versucht die Schwächen der vorangehenden Frühwarnsystem zu kompensieren. Gemäss Ansoff (1967, S.129ff), kündigen sich Diskontinuitäten nicht plötzlich sonder relativ früh durch sogenannte schwache Signale, die als frühe Hinweise zu bevorstehenden einflussreichen Ereignisse zu verstehen sind (Ansoff in Haji-Kazemi \& Anderson, 2013). Diesem Ansatz liegt die Prämisse zugrunde, dass Unternehmen die wahrscheinlichsten Faktoren, welche das Scheitern des Projekts begünstigen sowie die Anzeichen eines bevorstehenden Scheiterns, bereits kennen. Allerdings wird erst in nachgelagerten Projekt Assessments dieses Bewusstsein gefördert. Unter diesem Aspekt erscheint es relativ unverständlich, weshalb diese ignoriert wurden.
\newline
Die Ansätze, wie Projektmanager solche Signale erkennen und zu ihren Gunsten interpretieren können sind vielfältig, wie die Tabelle \ref{tab:Ans} aufzeigt.
%Tabelle mit möglichen Ansätzen
\begin{table}[H]
	\centering
	\caption{Ansätze zur Identifikation von Frühwarnsignalen
		\newline in Anlehnung an Haji-Kazemi, Andersen \& Krane (2013)}\label{tab:Ans}	
	\begin{tabular}{l|l}
		Risikomanagement & Past Project Consultation\\
		Earned Value Management & Cause-Effect-Analyse\\
		Projekt Assessment Ansätze & Gut feelings \\
		Performance Management & Interface Analysis\\
		Stakeholder Analyse & Project Analysis \\
		Maturity Assessment & Project Surrounding Analysis\\
	\end{tabular}		
\end{table}
Die Wahl der Methode ist abhängig von der Projektart und des Unternehmens. Haji-Kazemi, Andersen \& Krane (2013, S. 59) haben mittels mehrere Fallstudien eruiert, dass das Projekt Assessment sowie 'Gut feelings' in der Praxis die bewährt haben, wobei Experten die Überzeugung haben, dass Frühwarnsignale qualitativer Natur sind und eher durch Intuition Erfahrungswissen entdeckt werden. Klakegg, et.al, (2010) haben mittels formalen Assessments die Anzahl fehlender Informationen, fehlende Beurteilungen und Dokumentationen sowie unklare Anforderung der Meilenstein und verspätete Bericht als mögliche Frühwarnsignale ergründet. Missverständnisse bezüglich der Bedürfnisse, sowie mangelnde Offenheit der Unternehmenskultur und Kommunikationsbereitschaft zwischen den Projektteilnehmer, sowie angespannte Projektatomsphäre wurden in Fallstudien mittels der "Gut feelings"-Ansätzen als wichtige Früherkennungshinweise erkannt. Diese Erkenntnisse bestätigen unter anderem die Ansichten der befragten Experten aus anderen Studien. Denn bei der Evaluation von Assessments während der Projektlaufzeit können zwar wichtige Hinweise für nachfolgende Projekte ausgearbeitet, die aber im aktuellen Projekt nicht mehr berücksichtigt werden können. Während diese Indikatoren qualitativen Charakter haben und eher schwierig zu messen sind, konnten Haji-Kazem \& Anderson (2013) im Rahmen des Performance Management die Überwachung der Schnittstellenmassnahmen, die Mitarbeiterzufriedenheit und Risikoüberwachung als effiziente Quellen von Frühwarnsignalen erheben. Ihre Gemeinsamkeiten sind die quantitative und kontinuierliche Messbarkeit sowie die Funktion als sogenannte "leading" Indikatoren, die es ermöglichen in der Ursachen-Wirkungs-kette möglichst früh Hinweise zu möglichen Gefahren zu erhalten. "Lagging" Faktoren liefern dementsprechend eher spät oder zu spät Signale zu möglichen Risiken. Die Herausforderung bei der Identifikation von Frühwarnsignalen mit dem Performancemanagement-Ansatz ist die Selektion der zu überwachenden Faktoren. Ausserdem kann der Einfluss von Drittvariablen unentdeckt respektive unterschätzt werden.
% Faktoren von Klakegg und Kommentieren
% Faktoren von Performance Ansatz, 
% Leading Lagging Faktor,  
% wan ist früh, leading not lagging indicators, IdentifyAc says that performance only measures lagging 38
% possible early warning signs culture, lack of an outsiders perspective on the project, anchoring in the permanent organization, lack of consistency between stakeholders ambition and certain organizations. gut felt signs: detection of unrealism, lack of clarity, misalignment btw qualitative and quantitaive risk analysis 42
%no early signs in later stages, change...not used 43
%problems: difficult to stop projects despite EWS
%need for formalized proces for finding ealry warning signs, outside of the box thinking


\subsection{Ansätze zur Früherkennung für die Bühler AG}\label{vierzwei}
Die Erkenntnisse aus der Theorie des vorangehenden Kapitel setzen den Rahmen für die nachfolgende Ausführungen. Die zentralen Anforderungen an Frühwarnsignale sind zum einen, dass sie bereits zu einem frühen Zeitpunkt gemessen werden können. Die Implementierung eines ganzen Frühwarnsystems erfordert allerdings eine strategische Verankerung, da ein konstantes Monitoring und Screening, sowie eine anschliessende Auswertung und Interpretation der Daten notwendig ist. Ohne die Unterstützung des Managements wird die Durchsetzung eines solchen Vorhabens kaum durchsetzbar sein.
%DAten
Die Auswertungen der Einflussfaktoren des Kapitel \ref{drei} geben zwar Hinweise, was mögliche Attribute nicht-erfolgreicher Projekte sein können, allerdings fehlt es an einer statistisch begründeten Signifikanz des Zusammenhangs mit dem Erfolgskriterium. Nichtsdestotrotz können einige Faktoren bereits aufgrund ihrer Natur und des möglichen Erhebungszeitpunktes als mögliche Frühwarnindikatoren ausgeschlossen werden. Denn vorzugsweise sollen sogenannte "leading factors" fokussiert werden, zu denen sämtliche Performanceindikatoren, beispielsweise Kostenabweichungen, Kosten aus Nachlieferung oder Zeitverzögerungen gemäss (Zitat) nicht zählen. Die Rahmenbedingungen sowie auch der Projektmanager oder die Verantwortungsgesellschaften sind zwar zu Beginn des Projektes bekannt, verändern sich jedoch über die Projektlaufzeit kaum. Allerdings könnte nach einer entsprechenden Analyse ihres Einflusses auf die Erfolgswahrscheinlichkeit Projekte mit entsprechenden Attributen eher überwacht werden als andere. Ein solches Vorgehen kann dazuführen, dass andere Signale ausser Acht gelassen werden. Zudem wurden wie bereits gesagt, gewisse Faktoren in der Analyse nicht berücksichtigt, die möglicherweise auch als Frühwarnindikatoren funktionieren könnten. Aus diesen Gründen erscheint es erforderlich, dass neue Faktoren bezüglich ihrer Fähigkeit als Frühwarnsignal zu fungieren, ergründet werden. Es gibt jedoch keinen formalisierten Prozess, zu deren Identifikation. 
\newline Obwohl Projekt Assessments und seine Formen sowie Risikoanalysen bereits viele Hinweise zu möglichen Risiken und Chancen liefern, 
Bühler hat selbst für sogenannte Crash-Projekte Projekt-Assessments durchgeführt, die zu wichtigen Erkenntnissen für zukünftige Projekte geführt hat. Ausserdem wird am Ende jedes Projekts (Milestoen Debriefen) ein sogenanntes Debriefing abgehalten, welches implizierte, das Stärken und Schwächen eines jeden Projekts diskutiert wurden. Der Customer-Project Prozess hat neben anderen eine zentrale Schnittstelle vom Verkaufsprozess zum Fullfillment-Prozess. Basierend auf den Erkenntnissen der Literatur hat sich die Interface Analyse respektive die Beobachtung der
\paragraph{Sales \& Quotation} Schnittstellenthemen als relativ guter Frühwarnindikator erwiesen. Es liegt auf der Hand, dass der Output aus dem SQ-Prozess direkter Input im FF-Prozess bildet. Deshalb entstand die Idee, sozusagen die Frühwarnung für den Projekterfolg ab diesem Zeitpunk zu implementieren. Aus den Prozessabläufen der Bühler AG geht hervor, dass sowohl der SQ-Prozess und der FF-Prozess für Projekte grösser als eine Million umfangreiche Risikoanalysen gemacht werden. Vermutungsweise wird bereits zu diesem Zeitpunkt mögliche Erkenntnisse über zukünftige Herausforderung gewonnen, die wenig Beachtung erhalten. Fehlende Informationen, Assessments oder Dokumentation können als mögliche Frühwarnsignale interpretiert werden (siehe Klaggeg). Deshalb kann es von Nutzen sein eine Kennzahl zu entwickeln, welche auf automatisierte Basis die erforderlichen Dokumente gemäss den Anforderungen des MS1 beobachtet werden, sodass sichergestellt werden kann, dass keine Informationslücken entstehen. Somit wäre bereits früh klar, bei welchen Projekten alle relevanten Informationen zu Verfügung standen und der Übergabeprozess geglückt war. Auf Basis der in Kapitel \ref{sec:drei} erhobenen Daten hat sich gezeigt, dass Volumenmässig der Anteil an Projekte unter oder gleich einer Million fast die Hälfe aller Projekte ausmacht. Es wäre denkbar, dass bei diesen Projekten umfangreiche Risikoprüfungen ausgeblieben sind, da sie nicht priorisiert werden. Diese Annahmen und auch Probleme dieser Projekte müssten genauer untersucht werden, um andere Frühwarnindikatoren zu entdecken. Nichtsdestotrotz ist die Schnittstelle von SQ zu FF auch bei kleinere Projekte wichtig, damit das Projekt erfolgreich abgewickelt werden kann. In diesem Fall würde könnte es sich anbieten, eine Art Interface Analysis, die die Anzahl Interface-Themen und deren Bearbeitung in Bezug auf nur diese Schnittstelle misst, so dass es einerseits eine Plattform gibt, die Interface-Themen erfasst und offene/ungelöste Themen ersichtlich sind. Diese Idee liefert allerdings nur dem Projektabwicklungsprozess nähere Informationen, ob ein Projekte auf die schiefe Bahn gerät. Deshalb müssen auch für den weiteren Projekt-Management Prozess mögliche Ansätze diskutiert werden.
\paragraph{Projektabwicklung:} Im Propjektabwicklungsprozess sind weitere Schnittstellen vorhanden, welche genauer berücksichtigt werden müssen. Gemäss der Datenanalyse ist sowohl für Fail-Projekte und Success-Projekte der Mehrkostenanteil der IS-Kosten am höchsten gewesen. Unabhängig von der Ursache dieses Erscheinungsbild, ist die Installation die letzte Projektphase, so dass es von grossem Nutzen frühzeitig über mögliche Komplikationen Bescheid zu wissen, damit vorbereitende Massnahmen getroffen werden können. Zur Steuerung mittels Frühwarnsystem könnte ein Kombination aus Interfacemanagement und 'Gut Feelings' angewendet werden. Mittels Interfacemanagement soll sichergestellt werden, dass die auftauchenden Themen bearbeitet in nützlicher Frist bearbeitet werden. Der Ansatz der Gut Feelings hat zum Zweck, dass eine breiter Fokus für Variablen existiert, die einerseits während der Projektphase als mögliche Bedrohungen identifiziert werden und anderseits weder in den Risikochecks des SQ noch des FF inkludiert waren. Interne Dokumente belegen, dass der Projektmanager Erkenntnise aus der Risikoanalyse im Projektmanagement in detaillierter Form pflegen muss. Da erfahrungsgemäss ein gewisser Widerwille gegenüber umfangreichen Datenpflege festzustellen ist, sollte es im Tool eine Inputstelle geben für auf Intuition basierende Frühwarnsignale geben. Diese Anlaufstelle soll möglichst wertneutral, frei von Rechtfertigungsanforderung von übergeordneten Parteien, mit effizienter Handhabung und Zugang für sämtliche Projektteilnehmer ausgestattet sein. Dies ermöglicht dem Projektmanager eine Art Radar für zukünftige Herausforderung zu haben. Es sollte möglich sein, ein konstantes Monitoring pro Projekt ohne dabei detaillierte Angaben bereits erfassen zu müssen, sicherzustellen. Aspekte die berücksichtigt werden müssten sind die Strukturierung der Datenmenge sowie die Nutzung und Auswertung der Daten durch die Projektmanager.
\paragraph{Verschuldungsfrage:} Die vorangehende Analyse der Projekte wendet sozusagen ein schwarz-weiss Denken in Bezug auf den Erfolg ab. Allerdings konnte während der Analysephase festgestellt werden, dass es schwierig ist zu unterscheiden, welcher Projekttyp vorliegt. Beispielsweise werden Projekte gemacht, um neue Kunden zu gewinnen oder eine neue Technologie zu fördern, was zur Folge haben kann, dass die Kostenvorgaben relative zum Umsatz ambitiös ausfallen. Es würde jedoch Sinn machen, solche Projekte vom Standardgeschäft abgrenzen zu können, um mögliche Kernkompetenzen respektive Faktoren, die den Projekterfolg beeinflussen zu identifizieren. Zudem ist es wahrscheinlich, dass der Grund für die Mehrkosten nicht immer beim der Bühler AG liegen muss, unter dem Ausschluss, dass die Auswahl der Kunden von ihr beeinflusst wird. In Zusammenhang mit der Identifikation der Erfolgsfaktoren der Bühler AG könnte es folglich von Nutzen sein, Faktoren zu herben, die Aufschluss über kundenseitig induzierte Ursachen geben und wie anschliessend die Mehrkosten gehandhabt wurden. Zudem könnte die Befragung der Projektmanager und Verkaufsmanager mehr Aufschluss über zu berücksichtigenden Erfolgsfaktoren des Projektmanagements der Bühler AG geben. Hinzu kommt, dass aufgrund der Datenqualität wichtige Faktoren, wie zum Beispiel der Zeitpunkt der Forecast-Anpassung nicht ausgewertet werden konnten.
\paragraph{Projektprioritäten:} Aus einer finanziellen Perspektive die Fokussierung der Projekte mit grossem Umsatzvolumen, da ihr Einfluss auf das Ergebnis im Anlagengeschäft relativ gewichtig ist. Dennoch sollte der DB1 Verlust von Projekten mit einem Umsatzvolumen bis maximal 1.0 Mio. CHF nicht vernachläassigt werden, da sie am Projektvolumen eine relativ hohen Anteil haben. Wie bereits erwähnt wurde, werden Projekte unter 1 Mio. CHF von die internen Richtlinien zur vertieften Risikoanalyse der Bühler AG nicht erfasst.
%%% Text fehlt 
\paragraph{Incentivierung:} Die Anwendung von Frühwarnindikatoren bedingt, dass die Unternehmenskultur sowie auch die Projektmanagementstrategie entsprechend verändert respektive ausgerichtet wird. Die Implementierung von Frühwarnsignalen kann keine einmalige Übung darstellen, da es ein laufender Prozess ähnlich dem monatlichen Reporting ist. Die Abstimmung und Ausrichtung der Prozess und involvierten Personen auf ein gemeinsames Ziel "erfolgreiche Projekte" abzuschliessen ist dabei von grosser Wichtigkeit. Der Fokus sollte auf die Ergreifung von Massnahmen zur entsprechenden Gegensteuerung bei Komplikationen gerichtet sein anstatt auf die interne Schuldfrage. Eine sogenannte Fehlerkultur, die den Umgang mit Fehlern, Fehlerfolgen und Fehlerrisiken inkludiert, kann ein konstruktives Lernen aus Fehlern oder die Entdeckung effektiver Massnahmen bei Fehlerrisiken begünstigen. Alam Gühl (2016, S.20-21) plädiert im Rahmen der Projektkultur den positiven Umgang mit Fehlern und  den umfangreichen Austausch von entsprechenden Informationen, das von der Unternehmenskultur begünstigen werden kann. Ein anderes zentraler Einflussfaktor im Zusammenhang mit Frühwarnsignalen, ist auch die Fähigkeit des Projektmanagers zu Eingeständnissen, dass sein Projekt zu scheitern droht. Denn werden drohende Risiken und deren mögliche Realisation verkannt oder verhältnismässig spät kommuniziert, können der Handlungsspielraum eingegrenzt werden. Allerdings muss diese Denkweise aktiv im Unternehmen gefördert werden, damit das Bewusstsein der Fehlerakzeptanz, d.h. die positive Assoziation zu Fehlern und Scheitern gefestigt wird. 


%%Personen abhängig alli müsssen an einem Strang ziehen, Ausrichtung der Menschen an Zielen es Unternehmesn
%% Probleme Installation
%% Projektkategorisierung
%% Verschulden Bühler etc.