% !TEX root = MA.tex
\section{Frühwarnsysteme und Frühwarnindikatoren im Projektmanagement}
Basierend auf der Annahme, dass einzelne Faktoren den Erfolg eines Projekts beeinflussen können und dass es vermutungsweise ein Erfolgsrezept für das Projektmanagement gibt, gewinnt die Früherkennung von Projekten, die möglicherweise unter der festgelegten Performance abschliessen, an Bedeutung. Das übergeordnete Ziel von Früherkennung respektive von Frühwarnsystemen ist es einerseits aufkommend Gefahren frühzeitig zu erkennen und anderseits, die rechtzeitige Einleitung von Massnahmen zur Gegensteuerung respektive Schadensprävention sicherzustellen. Im betrieblichen Kontext in Bezug auf das Projektmanagement bedeutet dies, drohende Verluste infolge eines Projekts so früh wie möglich zu erkennen und bereits während der Projektlaufzeit Gegenmassnahmen zu ergreifen.  Deshalb werden nachfolgend die Frühwarnsystemansätze zuerst im Allgemeinen und anschliessend in Bezug auf das Projektmanagement erläutert. Im zweiten Unterkapitel sollen mögliche Frühwarnsystemansätze für die Bühler AG diskutiert werden, wobei der zugrundeliegende Fokus auf der grundsätzlichen Beeinflussung der Erfolgsquote und finanzielle Ersparnis liegt.
\subsection{Frühwarnsysteme und Frühwarnindikatoren}
Einleitend sollen der Zweck sowie die drei Generationen von Frühwarnsystem erklärt werden. Anschliessend werden die Anforderungen an Frühwarnsysteme sowie bereits etablierte Ansätze für das Projektmanagement näher betrachtet. 
\newline Der Anwendungsbereich von Frühwarnsysteme ist sehr breit, so finden sie zum Beispiel in Bezug auf Naturkatastrophe oder Brandschutz Verwendung. Im Bereich der Wirtschaft ist die Früherkennung im Zusammenhang mit Unternehmenskrisen von zentraler Bedeutung. Dabei unterstellen Jacobs, Riegler und Matter (2012, S. 23) die Möglichkeit eines ambivalenten Ausgangs sprich der 'Untergang' des Unternehmens sowie die Bewältigung der Krise sind gleichermassen denkbar. Die zusätzlichen Aspekte im Zusammenhang mit einer Unternehmenskrise sind die Existenzgefährdung, die eine nachhaltige Gefährdung unter Einschluss des Existenzverlust, beschreibt, sowie die Gefährdung dominanter Ziele infolge der Krise (Jacobs, Riegler und Matter, 2012, S. 23). Diese Beschreibungen lassen sich im Zusammenhang mit Projekten ebenso anwenden, wie der Prozesscharakter einer Krise und die Steuerungsproblematik. Jacobs, Riegler und Matter (2012, S.23) führen dazu aus, dass die Unternehmenskrise sich als zeitliche Abfolge darstellt und die Option besteht diesen zu beeinflussen. Demzufolge wird vorausgesetzt, dass geeignete Informationen zur Verfügung stehen, um möglichst früh eine drohende Krise zu erkennen, und das Geschehen positiv beeinflussen zu können.
\newline
Grundsätzlich werden drei Generationen von Frühwarnsystem unterschieden, die sukzessive entwickelt wurden. Die erste Generation orientiert sich an traditionellen Kennzahlen- und Hochrechnungen, die zweite Generation an Indikatoren und die dritte an schwachen Signalen. 
\newline
Die erste Generation konzentrierte sich vor allem auf die Verwendung von absoluten und relativen Kennzahlen sowie den darauf basierenden Hochrechnungen, die dann beim Vergleich mit den Sollwerten als Frühwarnindizien gewertet wurden. Dabei wird die zentrale Eigenschaft vorausgesetzt, zukünftig bedeutende Entwicklungen frühzeitig zu erfassen. Allerdings ist die Aussagekraft respektive die Frühwarneigenschaft durch die vergangenheits- und gegenwartsorientierte Natur der zugrundeliegende Daten begrenzt. Hochrechnungen sowie deren laufende Überwachung können die einerseits diese Nachteile etwas aufbessern, allerdings ist gemäss Jacobs, Riegler \& Matter (2012, S. 26) der gewählte Zeitraum für die Planungshochrechnung von Branche zu Branche unterschiedlich sowie die Erkennung von Diskontinuitäten infolge der "Extrapolationsfalle" nicht möglich.
\newline
Die Schwächen der ersten Generation sollen mit der Früherkennung anhand von Indikatoren kompensieren. Das Ziel besteht darin, Veränderungen mittels statistischer Zusammenhänge eines Indikatorenmodelles signalisiert werden, wobei eine signifikante Korrelation zwischen den Indikatoren und der prognostizierenden Grösse sowie einen hohen Vorlaufcharakter der Indikatoren vorausgesetzt wird (Dillerup/Stoi 2008 in Jacobs, Riegler \& Matter, 2008). Nachteile an diesem Konzept sind die Eingrenzung der Faktoren, respektive die Fokussierung von zu beobachtenden Bereichen. Der Ausschluss oder die Nichtberücksichtigung gewisser Faktoren kann dazu führen, dass Strukturbrüche nicht aufgedeckt werden können (Jacobs, Riniger \& Matter, 2012, S. 28).
\newline Deshalb wurde eine die dritte Generation geschaffen, deren Fundament das Konzept der 'schwachen Signale' bildet und im Gegensatz zu den operativen Ansätzen (1. und 2. Generation) auch als strategische Früherkennung genannt wird. Die Grundidee stammt von Ansoff (1967, S.129ff), wonach sich Diskontinuität nicht plötzlich sonder relativ früh durch sogenannte schwache Signale ankündigen. Die relative Unstrukturiertheit der zu evaluierende Daten kann insofern ein Zeitvorsprung generieren, da relativ latente Bedrohungen erkannt werden können und somit auch frühzeitig Handlungsalternativen ausgearbeitet werden können (Jacob, Riniger \& Matter, 2017, S. 29). Charakteristisch für die Daten sind deren intuitive und vage Natur sowie die qualitative und unstrukturierte Art, die mittels Screening und Monitorprozessen gemanaget werden können (Jacob, Riniger \& Matter, 2017, p. 30).n
\newline Dieser grobe Überblick über die drei Generationen dient dazu nachfolgend die bereits identifizierten Ansätze zur Früherkennung im Projektmanagement besser strukturieren zu können. 
\newline Die Ansätze zur Identifikation von Frühwarnindikatoren wurden bisher verhältnismässig wenig erforscht. Risikomanagement-basierte Ansätze, Earned Value Management und verschiedene Variationen von Projekt Assessment zählen zu den meist angewandten Methoden. Daneben existieren die Stakeholder Analyse, Brainstorming, Maturity Measurement, Extrapolation von früheren Projekten sowie Ursachen-Effekt-Analyse, gut feelings und das Interface Management, die zur Identifikation von Frühwarnsignalen herangezogen werden können. Haji-Kazemi, Andersen \& Krane (2013, S. 59) haben mittels mehrere Fallstudien eruiert, dass das Projekt Assessment sowie 'gut feelings' in der Praxis die häufigsten angewandten Ansätze sind. Dabei wurde festgestellt, dass die Signale oftmals qualitativer Natur sind und deshalb formale, standardisierte Prozess zwar zu wichtigen Erkenntnissen führen, aber relevante Gegenmassnahmen zu deren Steuerung nicht ergriffen wurden (Haji-Kazemi, Andersen \& Krane , 2013, S. 59 ). Die fehlenden Definition des gesuchten Objekts, die für die Frühwarnindikatoren der 3. Generation Charakteristisch kann zugleich als Schwäche und Stärke betrachtet werden. Denn mittels dieser Methode kann jegliche Art von Frühwarnsignal ergründet werden, wohingegen bei Projekt Assessments bereits der Betrachtungsbereich bereits vordefiniert ist. Gemäss Haji-Kazemi, Andersen \& Krane (2013, S. 67) bietet das Performance Measurement vielversprechende Instrumenten zur Konzipierung eines Frühwarnsystems an, und Maturity Assessments hat sich bei der Identifikation von Schwachstellen im Projekten relativ gut bewährt. Nichtsdestotrotz ist die Wahl der Methoden stark abhängig von Projektart, der Unternehmenskultur und -kommunikation.
% Ansätze im Projektmanagement 
% wan ist früh, leading not lagging indicators, IdentifyAc says that performance only measures lagging 38
% possible early warning signs culture, lack of an outsiders perspective on the project, anchoring in the permanent organization, lack of consistency between stakeholders ambition and certain organizations. gut felt signs: detection of unrealism, lack of clarity, misalignment btw qualitative and quantitaive risk analysis 42
%no early signs in later stages, change...not used 43
%external project assessment stronger
%problems: difficult to stop projects despite EWS
%need for formalized proces for finding ealry warning signs, outside of the box thinking


\subsection{Ansätze zur Früherkennung für die Bühler AG}
Entwicklung eines Ansatz für ein Frühwarnsystem bei der Bühler AG
Retrospektiv: Berechnung der finanziellen Einsparung bei Anwendung des vorgeschlagenen Frühwarnsystemansatzes
\newpage
\subsection{Kritische Würdigung der Ergebnisse im Kontext mit dem Frühwarnsystemansatz}
\newpage