\chapter{Ergebnisse}\label{sec: Ergeb}
Nachfolgend werden die Ergebnisse pro Variablenkategorie dargestellt und kurz erläutert. Aufgrund der hohen Anzahl der Variablen, mussten gewisse Einschränkungen vorgenommen werden. Es wurden vor allem Resultate von Variablen mit einer zusätzlicher Einteilungsfunktion, beispielsweise das Land der Region, und solche mit ähnlichem Informationsgehalt nicht berücksichtigt. Die Begriff Success und Fail sind mit erfolgreichen respektive nicht erfolgreichen Projekten zu assoziieren.
\newline\newline
%%
%%Übersicht der Stichprobe
%%
Die Stichprobe enthält 966 Projekte von denen 32\% gescheitert sind. Die Erfolgsquote beträgt 2.1, sprich zwei aus drei Projekten können erfolgreich abgeschlossen werden. 
\begin{table}[H]
	\centering
	\caption{Übersicht Stichprobe}
	\begin{tabular} {l|r|r}
		\toprule
		\textbf{Erfolgskriterium} & \textbf{absolut} & \textbf{relativ} \\
		\midrule
		SUCCESS & 654 & 68\% \\
		FAIL & 312 & 32\% \\
		\bottomrule
		\textbf{Total} & \textbf{966} & \textbf{100\%} \\
	\end{tabular}
\end{table}
%%Ergenbnisse pro Kategorie: Hypothese Nennen, Kommentierung, Ergebnisse präsentieren, Tabellen etc
%%Rahmenbedingungen: Region, BA und allenfalls BU: MS zu feine Gliederung, EquLoc too maany Einzelwerte, CuNo too many Einzelwerte
%%Sales & Quotation: Budget Druck, Rest ging nicht
%%Fullfillment: PMNo to many Einzelwerte, PMAge gemäss Kategorie, PMTen (EVTL.), NoPM not, da Analyse zeigt dass nur einmal 3 PM gab, PMChange,
%%Fulfillment: LeadSASPr, LeadSASPrFF, LeadSAS.PrFF,  NoLeadSASFF nicht, da zu wenig Fälle
%%Fulfillment: CostFCajd alle und HOM
%%Kosten: Tabelle: finanzielle Performanceanalyse: Umsatz, Marge, Erklärung, Erkenntnisse (Bud, Ac, Dev)
%%Kosten: Tabelle: finanzielle Performancenaalyse: Kosten(Bud, Act, Abw und Abw Schlüssel)
%%Kosten: evtl. Tabelle: pro Umsatzkategorie Anz. Cost Dev und: mit wenig, viel Verlust
%%Kosten: Mittelwerte für alle Variablen
%%Zeit: alle, aber ohne Delay
%%Komplexität: alle 
\section{Rahmenbedinungen}
Die Häufigkeitsverteilung der Region in der Tabelle \ref{tab:freg} zeigt, dass Europa die grösste Region ist und Mittlere Osten und Afrika sowie China. Die Erfolgsquote der kleinsten Regionen gemessen am Projektvolumen hat die beste Erfolgsquote. Die tiefste Erfolgsquote bei einer mittleren Regionengrösse hatte Nordamerika.
%Tabelle Auswertung Häufigkeit pro Region
\begin{table}[H]
	\centering
	\caption{Erfolgsquote und Häufigkeitsverteilung pro Region}
	\begin{tabular}{lrrrrrr}
		\toprule
		\textbf{Region} & \multicolumn{1}{l}{\textbf{Erfolgsquote}} & \multicolumn{1}{l}{\textbf{Success}} & \multicolumn{1}{l}{\textbf{Fail}} & \multicolumn{1}{l}{\textbf{Fail [\%]}} & \multicolumn{1}{l}{\textbf{Total}} & \multicolumn{1}{l}{\textbf{Total [\%]}} \\
		\midrule
		East Asia & 6.7   & 20    & 3     & 13.0\% & 23    & 2.4\% \\
		EU    & \textbf{1.7}   & 240   & 145   & 37.7\% & 385   & 39.9\% \\
		MEA & 2.7   & 112   & 42    & 27.3\% & 154   & 15.9\% \\
		North\_Ame & \textbf{1.4}   & 54    & 38    & 41.3\% & 92    & 9.5\% \\
		SAS\_BCHI & 2.9   & 119   & 41    & 25.6\% & 160   & 16.6\% \\
		South\_Ame & 1.9   & 58    & 31    & 34.8\% & 89    & 9.2\% \\
		South\_Asia & 4.3   & 51    & 12    & 19.0\% & 63    & 6.5\% \\ 
		\bottomrule
		\textbf{Total} & \textbf{2.1} & \textbf{654} & \textbf{312} & \textbf{32.3\%} & \textbf{966} & \textbf{100.0\%} \\
	\end{tabular}%
	\label{tab:freg}%
\end{table}%
Die Auswertung der Geschäftsbereiche in der Tabelle \ref{tab:fba} hat ergeben, dass GM als grösste BA eine mittlere Erfolgsquote aufweist. GL und LO als mittelgrosse Geschäftsbereiche gemessen am Projektvolumen haben ein kleines Verhältnis zwischen erfolgreichen und nicht erfolgreichen Projekten. VN als zweit grösste BA schneidet gegenüber GM noch schlechter ab und hat aufgrund der höheren Anzahl Projekte mehr Einfluss auf die Erfolgsquote über alle Projekte.

\begin{table}[H]
	\centering
	\caption{Erfolgsquote und Häufigkeitsverteilung pro Geschäftsbereich}
	\begin{tabular}{lrrrrrr}
		\toprule
		\textbf{Geschäftsbereich}   & \multicolumn{1}{l}{\textbf{Erfolgsquote}} & \multicolumn{1}{l}{\textbf{Success}} & \multicolumn{1}{l}{\textbf{Fail}} & \multicolumn{1}{l}{\textbf{Fail [\%]}} & \multicolumn{1}{l}{\textbf{Total}} & \multicolumn{1}{l}{\textbf{Total [\%]}} \\
		\midrule
		CF    & 2.8   & 118   & 42    & 26.3\% & 160   & 16.6\% \\
		DC    & 5.6   & 96    & 17    & 15.0\% & 113   & 11.7\% \\
		GD    & 2.3   & 7     & 3     & 30.0\% & 10    & 1.0\% \\
		GL    & \textbf{1.2}  & 39    & 32    & 45.1\% & 71    & 7.3\% \\
		GM    & \textbf{1.9}   & 226   & 122   & 35.1\% & 348   & 36.0\% \\
		LO    & 1.4   & 30    & 21    & 41.2\% & 51    & 5.3\% \\
		SR    & 5.0   & 35    & 7     & 16.7\% & 42    & 4.3\% \\
		TP    & NA      & 8     & 0     & 0.0\% & 8     & 0.8\% \\
		VN    & \textbf{1.4}   & 95    & 68    & 41.7\% & 163   & 16.9\% \\
		\bottomrule
		\textbf{Total } & \textbf{2.1} & \textbf{654} & \textbf{312} & \textbf{32.3\%} & \textbf{966} & \textbf{100.0\%} \\
	\end{tabular}%
	\label{tab:fba}%
\end{table}%
Die Auswertung des Region in Verbindung mit Geschäftsbereich trägt der Matrixorganisation der Bühler AG Rechnung. In der Tabelle \ref{tab:fregba} wird ersichtlich, dass in Europa die Geschäftsbereiche GM und GL sowie VN unterdurchschnittliche Erfolgsquoten aufweisen. Ein ähnliches Bild ist bei der Region Nordamerika festzustellen. Das Verhältnis von erfolgreichen und nicht erfolgreichen Projekten von CF liegt in beiden Regionen unter demjenigen des gesamten Datensatzes. DC hat in Europa eine überdurchschnittlich hohe Erfolgsquote im Vergleich zu ganzen Region.
% Auswertung BA-Region
\begin{table}[H]
	\centering
	\caption{Erfolgsquote und Häufigkeitsverteilung Regionen-BA für EU und North\_Ame}
	\begin{tabular}{llrrrrr}
		\toprule
		\textbf{Region} & \textbf{BA}    & \multicolumn{1}{l}{\textbf{Erfolgsquote}} & \multicolumn{1}{l}{\textbf{Success}} & \multicolumn{1}{l}{\textbf{Fail}} & \multicolumn{1}{l}{\textbf{Fail [\%]}} & \multicolumn{1}{l}{\textbf{Total}} \\
		\midrule
		EU    & CF    & 1.9   & 58    & 31    & 34.8\% & 89 \\
		EU    & DC    & 5.0   & 45    & 9     & 16.7\% & 54 \\
		EU    & GD    & NA    & 2     & 0     & 0.0\% & 2 \\
		EU    & \textbf{GL}    & 1.0   & 24    & 23    & 48.9\% & 47 \\
		EU    & \textbf{GM}  & 1.2   & 58    & 50    & 46.3\% & 108 \\
		EU    & LO    & 2.5   & 10    & 4     & 28.6\% & 14 \\
		EU    & SR    & 2.5   & 5     & 2     & 28.6\% & 7 \\
		EU    & \textbf{VN}     & 1.5   & 38    & 26    & 40.6\% & 64 \\\hline
		North\_Ame & CF    & 2.0   & 10    & 5     & 33.3\% & 15 \\
		North\_Ame & DC    & 1.0   & 2     & 2     & 50.0\% & 4 \\
		North\_Ame & GL    & 1.0   & 1     & 1     & 50.0\% & 2 \\
		North\_Ame & \textbf{GM}   & 1.5   & 24    & 16    & 40.0\% & 40 \\
		North\_Ame & LO    & 4.0   & 4     & 1     & 20.0\% & 5 \\
		North\_Ame & SR    & 1.0   & 1     & 1     & 50.0\% & 2 \\
		North\_Ame & \textbf{VN}  & 1.0   & 12    & 12    & 50.0\% & 24 \\
		\bottomrule
	\end{tabular}%
	\label{tab:fregba}%
\end{table}%
\clearpage
%%
%%SQ
%%
\section{SQ-Variablen}
Die Tabelle \ref{tab:msq} zeigt, dass die mittlere Abweichung des Auftragsvolumen vom Budget für nicht erfolgreiche Projekte in der Region und im Geschäftsbereich sowohl in relativen als auch absoluten Grössen deutlich höher liegt. Die Parameter zum Area Manager konnten aufgrund fehlender Daten nicht ausgewertet werden.
% Table generated by Excel2LaTeX from sheet 'sq mean'
\begin{table}[H]
	\centering
	\caption{Arithmetisches Mittel des BudGapOR der BU und Region}
	\begin{tabular}{lrr|rr}
		\toprule
		\textbf{Success} & \multicolumn{1}{l}{\textbf{BU [TCHF]}} & \multicolumn{1}{l}{\textbf{BU [\%]}} & \multicolumn{1}{l}{\textbf{Region [TCHF]}} & \multicolumn{1}{l}{\textbf{Region [\%]}} \\\hline
		FALSE & -7'244.2 & -11.3 & -20'845.5 & -9.3 \\
		TRUE  & -1'207.2 & -4.0  & -14'886.9 & -6.8 \\
		\bottomrule
	\end{tabular}%
	\label{tab:msq}%
\end{table}%
\newpage
%%
%%Fulfillment
%%
\section{FF-Variablen}
%
%Auswertung Projektmanager Change
%
\textit{Projektmanager: } Die Auswertung des Projektmanagers mittels einer  Häufigkeitsverteilung hat aufgrund der hohen Anzahl Einzelwerte wenig Informationsgehalt.
\newline 
Der Projektmanager wurde insgesamt bei 43 Projekten ausgewechselt, wovon 17 gescheitert sind. Die Anzahl Projektmanager lieferte keinen zusätzlichen Informationshinweis, da sie im direktem Zusammenhang mit dem PMChange steht. Im Datensatz hatten lediglich sechs Projekte mehr als zwei Projektmanager während der Projektlaufzeit.
\begin{table}[H]
	\centering
	\caption{Erfolgsquote und Häufigkeitsverteilung PMChange}
	\begin{tabular}{lrrrrrr}
		\toprule
		\textbf{PMChange} & \multicolumn{1}{l}{\textbf{Erfolgsquote}} & \multicolumn{1}{l}{\textbf{Success}} & \multicolumn{1}{l}{\textbf{Fail}} & \multicolumn{1}{l}{\textbf{Fail [\%]}} & \multicolumn{1}{l}{\textbf{Total}} &
		\multicolumn{1}{l}{\textbf{Total [\%]}} \\
		\midrule
		no    & 2.1 & 628   & 295   & 32.0\% & 923 & 96\% \\
		yes   & 1.5 & 26    & 17    & 39.5\% & 43  & 4\% \\
		\bottomrule
		\textbf{Total} & \textbf{2.1} & \textbf{654} & \textbf{312} & \textbf{32.3\%} & \textbf{966} & \textbf{100.0\%} \\
	\end{tabular}%
	\label{pmchange}%
\end{table}%
%%
%%Auswertung Alter und Tenuer PM
%%
Das Alter des Projektmanagers (PMAge2) wurde zum Zweck der Auswertung in eine kategoriale Variable mit einer Klassenbreite von fünf Jahren transformiert. Die Durchschnittswerte des Alter und der Dienstjahre in der Tabelle \ref{ageten} im Anhang sind für Fail-Projekte und Success-Projekte faktisch identisch. Sie betragen gerundet 40 beziehungsweise 12 Jahre. Die Tabelle \ref{tab:fagecat} zeigt, dass Projektmanager der Alterskategorien (51-55) und (56-60) eine geringe Erfolgsquote im Vergleich zur Quote des gesamten Datensatzes haben. Die obere und untere Randklasse wickeln jeweils am wenigsten Projekte ab, weisen aber ein relativ gutes Verhältnis zwischen erfolgreiche und nicht erfolgreichen Projekten aus.
% Table generated by Excel2LaTeX from sheet 'fagecat'
\begin{table}[H]
	\centering
	\caption{Erfolgsquote und Häufigkeitsverteilung pro Alterskategorie(Cat\_age)}
	\begin{tabular}{lrrrrrr}
		\toprule
		\textbf{Cat\_age} & \multicolumn{1}{l}{\textbf{Erfolgsquote}} & \multicolumn{1}{l}{\textbf{Success}} & \multicolumn{1}{l}{\textbf{Fail}} & \multicolumn{1}{l}{\textbf{Fail [\%]}} & \multicolumn{1}{l}{\textbf{Total}} & \multicolumn{1}{l}{\textbf{Total [\%]}} \\
		\midrule
		20-25 & 2.5   & 15    & 6     & 28.6\% & 21    & 2.2\% \\
		26-30 & 2.3   & 130   & 57    & 30.5\% & 187   & 19.4\% \\
		31-35 & 3.0   & 130   & 44    & 25.3\% & 174   & 18.0\% \\
		36-40 & 2.3   & 97    & 43    & 30.7\% & 140   & 14.5\% \\
		41-45 & \textbf{1.8} & 105   & 57    & 35.2\% & 162   & 16.8\% \\
		46-50 & 2.3   & 65    & 28    & 30.1\% & 93    & 9.6\% \\
		51-55 & \textbf{1.3} & 69    & 52    & 43.0\% & 121   & 12.5\% \\
		56-60 & \textbf{1.4} & 26    & 19    & 42.2\% & 45    & 4.7\% \\
		61-63 & 2.8   & 17    & 6     & 26.1\% & 23    & 2.4\% \\
		\bottomrule
		\textbf{Total} & \textbf{2.1} & \textbf{654} & \textbf{312} & \textbf{32.3\%} & \textbf{966} & \textbf{100.0\%} \\
	\end{tabular}%
	\label{tab:fagecat}%
\end{table}%
%%
%%Lead SAS Fulfillment
%%
%Tabelle LeadSAS.PrFF
\textit{Lead SAS Projektabwicklung: }Der Verantwortungsbereich für das gesamte Projekt und die Projektabwicklung lagen bei insgesamt 101 Projekten nicht bei derselben Gesellschaft, wie die Tabelle \ref{tab:fleadsasprff} zeigt. Darunter sind 85 erfolgreiche Projekte, so dass die Erfolgsquote sehr hoch ist im Vergleich zu 2.1. Die Anzahl involvierter \gls{abk:sas} lieferte keine zusätzlichen Erkenntnisse, da lediglich bei vier Projekten mehr als zwei \gls{abk:sas} involviert waren.
\begin{table}[H]
	\centering
	\caption{Häufigkeitsverteilung LeadSAS.PrFF}
	\begin{tabular}{lrrrrrr}
		\toprule
		\textbf{LeadSAS.PrFF} & \multicolumn{1}{l}{\textbf{Erfolgsquote}} & \multicolumn{1}{l}{\textbf{Success}} & \multicolumn{1}{l}{\textbf{Fail}} & \multicolumn{1}{l}{\textbf{Fail [\%]}} & \multicolumn{1}{l}{\textbf{Total}} & \multicolumn{1}{l}{\textbf{Total [\%]}}
		\\
		\midrule
		identisch    & 569  & 1.9 & 296   & 34.2\% & 865 & 89.5\% \\
		verschieden   & 85  & 5.3  & 16    & 15.8\% & 101 & 10.5\% \\
		\bottomrule
		\textbf{Total} & \textbf{2.1} &\textbf{654} & \textbf{312} &  \textbf{32.2\%}   & \textbf{966} & \textbf{100\%} \\
	\end{tabular}%
	\label{tab:fleadsasprff}%
\end{table}%
%%
%%Forecast
%%
\textit{Forecast: }Die Parameter, welche die Zeit zwischen dem Projektstart und dem erstmaligen gelben oder roten Ampelstatus messen, konnten nicht ausgewertet werden, da im Nachhinein festgestellt wurde, dass anstatt der realisierten die geplante Projektlaufzeit als Bezugsgrösse gewählt wurde. Die Korrektur der Grösse war nicht mehr möglich.
\clearpage
%%
%%Komplexität
%%
\section{Komplexität:}
Die Komplexität wurde anhand der involvierten zuliefernden Bühler Gesellschaften während den Projektphasen und der Zusammenarbeit mit einer externen Gesellschaft evaluiert.
%
%Auswertung Anzahl SAS
%
Die Anzahl involvierte SAS (NoSupplSAS) im Mechanical Supply liegt im Bereich null und zehn, wobei null mit Eigenproduktion oder Zulieferung durch Dritte gleichzusetzen ist. In den übrigen Projektphasen sind maximal drei andere Bühler-Gesellschaften involviert. Die Komplexität ergibt sich aus der involvierten Anzahl während der gesamten Projektlaufzeit, sodass die Häufigkeit während allen Projektphasen zusammen ausgewertet wurde. Die Tabelle \ref{tab:fnosas} zeigt, dass die Eigenfertigung respektive die Zusammenarbeit mit Drittlieferanten während allen Projektphasen das häufigste Charakteristika aller Projekte war. 
% Table generated by Excel2LaTeX from sheet 'nosas'
\begin{table}[H]
	\centering
	\begin{threeparttable}
	\caption{Erfolgsquote und Häufigkeitsverteilung der NoSupplySAS-Kombinationen}
	\begin{tabular}{lrrrrrrrr}
		\toprule
		\textbf{NoSupplSAS} & \multicolumn{1}{l}{\textbf{MS}} & \multicolumn{1}{l}{\textbf{ME}} & \multicolumn{1}{l}{\textbf{PA}} & \multicolumn{1}{l}{\textbf{IS}} & \multicolumn{1}{l}{\textbf{Success}} & \multicolumn{1}{l}{\textbf{Fail}} & \multicolumn{1}{l}{\textbf{Total}} & \multicolumn{1}{l}{\textbf{Erfolgsquote}} \\
		\midrule
		\multicolumn{1}{r}{0} & 0     & 0     & 0     & 0     & 180   & 89    & 269 & 2.0 \\
		\multicolumn{1}{r}{1} & 1     & 0     & 0     & 0     & 73    & 26    & 99  & 2.8 \\
		\multicolumn{1}{r}{2} & 2     & 0     & 0     & 0     & 38    & 13    & 51  & 2.9 \\
		\multicolumn{1}{r}{2} & 2     & 1     & 0     & 0     & 23    & 13    & 36  & 1.8 \\
		\multicolumn{1}{r}{1} & 1     & 1     & 1     & 1     & 15    & 12    & 27  & 1.3 \\
		\multicolumn{1}{r}{1} & 1     & 1     & 0     & 0     & 17    & 11    & 28  & 1.5 \\ 
		\bottomrule
	\end{tabular}%
	\begin{tablenotes}
	\small
	\item Anmerkung: Die Tabelle zeigt die sechs häufigsten Kombinationen
	\end{tablenotes}
\label{tab:fnosas}%
\end{threeparttable}
\end{table}%
%Auswertung Konsortium
%
Der Parameter Konsortium wurde in Kombination mit der Region und dem Geschäftsbereich ausgewertet, um die Ergebnisse der Rahmenbedingungen zu untersuchen. Die Ergebnisse zu den Geschäftsbereichen GM, GL und VN sind in der Anmerkung der Tabelle \ref{tab:fcons} enthalten. Insgesamt war bei 78 Projekten ein Drittunternehmen involviert. Davon sind 31 gescheitert und wurden in den Regionen Europa sowie Mittlere Osten und Afrika sowie China abgewickelt. 
% Table generated by Excel2LaTeX from sheet 'con_part_reg'
\begin{table}[H]
	\centering
	\begin{threeparttable}	
	\caption{Erfolgsquote und Absolute Häufigkeitsverteilung der Projekte mit Konsortium pro Region}
	\begin{tabular}{lrrrr}
		\toprule
		\textbf{ConPart} & \multicolumn{1}{l}{\textbf{Region}} & \multicolumn{1}{l}{\textbf{Success}} & \multicolumn{1}{l}{\textbf{Fail}} & \multicolumn{1}{l}{\textbf{Erfolgsquote}} \\
		\midrule
		TRUE  & \multicolumn{1}{l}{East\_Asia} & 1     & 0     & NA \\
		TRUE  & \multicolumn{1}{l}{\textbf{EU}} & 13    & 14    & 0.9 \\
		TRUE  & \multicolumn{1}{l}{MEA\_Afr} & 9     & 7     & 1.3 \\
		TRUE  & \multicolumn{1}{l}{North\_Ame} & 2     & 1     & 2.0 \\
		TRUE  & \multicolumn{1}{l}{\textbf{SAS\_BCHI}} & 17    & 7     & 2.4 \\
		TRUE  & \multicolumn{1}{l}{South\_Ame} & 4     & 2     & 2.0 \\
		TRUE  & \multicolumn{1}{l}{South\_Asia} & 1     & 0     & NA \\\hline
		\textbf{Total} &       & \textbf{47} & \textbf{31} & \textbf{1.5} \\
	\end{tabular}%
	\begin{tablenotes}
	\tiny
	\item EU GL: 6 erfolgreiche und 5 nicht erfolgreiche Projekte
	\item EU GM: 1 erfolgreiche und 3 nicht erfolgreiche Projekte
	\item EU VN: 1 erfolgreiche und 5 nicht erfolgreiche Projekte
	\item SAS\_BCHI GM: 4 erfolgreiche und 6 nicht erfolgreiche Projekt
	\end{tablenotes}
	\label{tab:fcons}%
	\end{threeparttable}
\end{table}%
\clearpage
%%
%%Kosten
%%
\section{Kosten}
%\section{Finanzielle Performance Analyse}
Zuerst wurde mittels der Abweichungen zwischen budgetierten und realisierten Zahlen der Umsatz, die Kosten und DB1 der finanzielle Verlust der gescheiterten Projekte quantifiziert. Die erste Tabelle \ref{tab:bud} enthält das Budget, die zweite Tabelle \ref{tab:act} die realisierten Zahlen und die letzte Tabelle \ref{tab:Abw} zeigt die Abweichungen zwischen den ersten beiden. Die tatsächlichen Kosten der erfolgreichen Projekte lagen unter den budgetierten Werten, wohingegen die gescheiterten Projekte erhebliche Mehrkosten generiert haben. Diese Abweichungen wirken sich direkt auf die Projektmarge aus, die bei den nicht erfolgreichen Projekten 8.2\% unter dem Budget von 24.7\% lagen. Demgegenüber zeigt sich bei den erfolgreichen Projekten das gegenteilige Ergebnis, die Projektmarge lag 7.7\% über dem budgetierten Wert von 25.5\%.
\newline 
%Bud TO Cost DB1
\begin{table}[H]
	\centering
	\caption{Budgetierter Umsatz, Kosten und DB1 sämtlicher Projekte[TCHF]}
	\begin{tabular}{lrrrr}
		\toprule
		\textbf{Erfolgskriterium} & \textbf{Umsatz} & \textbf{Kosten} &
		\textbf{DB1} & \textbf{DB1 [\%]} \\
		\midrule
		SUCCESS & 1'552'450 & -1'156'598 & 395'851 & 25.5\% \\
		FAIL  & 618'013 & -465'066 & 152'947 & 24.7\% \\
		\bottomrule
		\textbf{Total} & \textbf{2'170'463} & \textbf{-1'621'664} & \textbf{548'799} & \textbf{25.3\%}\\
	\end{tabular}%
	\label{tab:bud}%
\end{table}%
%Act TO Cost DB+
\begin{table}[H]
	\centering
	\caption{Realisierte(r) Umsatz, Kosten und DB1 sämtlicher Projekte[TCHF]}
	\begin{tabular}{lrrrr}
		\toprule
		\textbf{Erfolgskriterium} & \textbf{TO Act} & \textbf{Cost Act} & \textbf{DB1 Act}&
		\textbf{DB1 Act [\%]} \\
		\midrule
		SUCCESS & 1'560'001 & -1'041'728 & 518'273 & 33.2\% \\
		FAIL  & 631'346 & -526'600 & 104'746 & 16.6\% \\
		\bottomrule
		\textbf{Total} & \textbf{2'191'347} & \textbf{-1'568'328} & \textbf{623'018} & \textbf{28.4\%} \\
	\end{tabular}
	\label{tab:act}%
\end{table}%
%Deviation TO Cost DB1
\begin{table}[H]
	\centering
	\caption{Abweichungen ($Act-Bud$) Umsatz, Kosten und DB1 [TCHF]}
	\begin{tabular}{lrrrr}
		\textbf{Erfolgskriterium} & \textbf{TO} & \textbf{Cost} & \textbf{DB1}&
		\textbf{DB1 [\%]} \\\hline
		SUCCESS & 7'551 & 114'870 & 122'421 & 7.7\% \\
		FAIL  & 13'333 & -61'534 & -48'202 & -8.2\% \\\hline
		\textbf{Total} & \textbf{20'884} & \textbf{53'336} & \textbf{74'220} & \textbf{3.1\%} \\
	\end{tabular}
	\label{tab:Abw}%
\end{table}%
%%
%%Aufschlüsselung Kostenabweichung gemäss Projektphase
%% 
Die Aufschlüsselung der Kosten in der Tabelle \ref{tab:stocostdb1dev} zeigt, dass während den Projektphasen ME und IS sowohl bei den nicht erfolgreichen, als auch der erfolgreichen die budgetierten Kosten erheblich überzogen wurden. Bei den gescheiterten Projekten sind die Mehrkosten von gerundet 62 TCHF zu ungefähr je einem Drittel durch die Kostendifferenz in MS und IS zurückzuführen. Die letzte Spalte sagt aus, dass die gesamte Kostendifferenz der gescheiterten Projekte durch die Mehrkosten der Projektphasen, wohingegen bei den erfolgreichen Projekten nur 30\% ($\frac{37'434}{114'870}$)  erklärt wird.
\begin{table}[H]
	\centering
	\begin{threeparttable}
	\caption{Aufschlüsselung der Kosten nach der Projektphase [TCHF]}
	\begin{tabular}{lrrrrr|r}
		\toprule
		\textbf{Erfolgskriterium} & \multicolumn{1}{l}{\textbf{Total Cost}} & \multicolumn{1}{l}{\textbf{MS}} & \multicolumn{1}{l}{\textbf{ME}} & \multicolumn{1}{l}{\textbf{PA}} & \multicolumn{1}{l}{\textbf{IS}} & \multicolumn{1}{l}{\textbf{Summe}} \\
		\midrule
		SUCCESS & 114'870 & 47'615 & -2'159 & -908  & -7'114 & 37'434 \\
		FAIL  & -61'534 & -20'253 & -12'721 & -7'220 & -22'053 & -62'247 \\
		\bottomrule
		\textbf{ Total} & \textbf{53'336} & \textbf{ 27'363} & \textbf{ -14'880} & 
		\textbf{ -8'128} & \textbf{ -29'167} & \textbf{ -24'812} \\
	\end{tabular}%
	\begin{tablenotes}
		\tiny
		\item Anmerkung Die letzten Spalte kalkuliert die Summe von MeS, ME, PA und IS.
	\end{tablenotes}
	\label{tab:stocostdb1dev}%
	\end{threeparttable}
\end{table}%
%%
%%Projektgrösse
%%
Die Einteilung der Projekte in Umsatzkategorien erlaubt die Analyse, der Anzahl Projekte und -volumen, welche für den Verlust der Projektmarge verantwortlich sind. Die DB1 Einbussen können zu einem Viertel auf 23 Projekte mit Projektvolumen zwischen 5 Mio. und 10. Mio TCHF zurückgeführt werden, wie der Tabelle \ref{tab:ftobudcat}. Die Kategorie mit den höchsten Anteil gescheiterte Projekte, hat einen kumulierten DB1 Verlust gegenüber dem Budget von 6 Mio. TCHF.
% Table generated by Excel2LaTeX from sheet 'TOBud_cat'
\begin{table}[H]
	\centering
	\caption{DB1 und Häufigkeit pro Umsatzkategorie (TOBud\_Cat) [TCHF]}
	\begin{tabular}{lrcrrrr}
		\toprule
		\multicolumn{1}{l}{\textbf{Kat.}} & \multicolumn{1}{l}{\textbf{TOBud\_Cat}} & \multicolumn{1}{c}{\textbf{ Anz.}} & \multicolumn{1}{l}{\textbf{DB1 Act}} & \multicolumn{1}{l}{\textbf{DB1 Bud }} & \multicolumn{1}{l}{\textbf{DB1BudDevabs}} &  \multicolumn{1}{c}{\textbf{\%}}\\
		\midrule
		\multicolumn{1}{r}{1} & \multicolumn{1}{l}{[13.2,500)} & 54    & 4'508 & 6'181 & -1'672 & -27\% \\
		\multicolumn{1}{r}{2} & \multicolumn{1}{l}{[500,1e+03)} & 87    & 11'956 & 18'167 & -6'210 & -34\% \\
		\multicolumn{1}{r}{3} & \multicolumn{1}{l}{[1e+03,1.5e+03)} & 54    & 13'510 & 17'476 & -3'966 & -23\% \\
		\multicolumn{1}{r}{4} & \multicolumn{1}{l}{[1.5e+03,2e+03)} & 32    & 11'106 & 14'928 & -3'822 & -26\% \\
		\multicolumn{1}{r}{5} & \multicolumn{1}{l}{[2e+03,2.5e+03)} & 17    & 6'286 & 9'447 & -3'161 & -33\% \\
		\multicolumn{1}{r}{6} & \multicolumn{1}{l}{[2.5e+03,3e+03)} & 12    & 6'987 & 8'276 & -1'289 & -16\% \\
		\multicolumn{1}{r}{7} & \multicolumn{1}{l}{[3e+03,3.5e+03)} & 8     & 5'419 & 6'778 & -1'359 & -20\% \\
		\multicolumn{1}{r}{8} & \multicolumn{1}{l}{[3.5e+03,4e+03)} & 8     & 4'338 & 6'445 & -2'107 & -33\% \\
		\multicolumn{1}{r}{9} & \multicolumn{1}{l}{[4e+03,4.5e+03)} & 4     & 3'191 & 4'551 & -1'360 & -30\% \\
		\multicolumn{1}{r}{10} & \multicolumn{1}{l}{[4.5e+03,5e+03)} & 6     & 2'686 & 4'349 & -1'662 & -38\% \\
		\multicolumn{1}{r}{11} & \multicolumn{1}{l}{[5e+03,1e+04)} & 23    & 23'602 & 35'644 & -12'043 & -34\% \\
		\multicolumn{1}{r}{12} & \multicolumn{1}{l}{[1e+04,3.42e+04)} & 7     & 11'155 & 20'706 & -9'551 & -46\% \\
		\bottomrule
		\textbf{Total} &       &    \textbf{312}   & \textbf{104'746} & \textbf{152'947} & \textbf{-48'202} & \textbf{-32\%} \\
	\end{tabular}%
	\label{tab:ftobudcat}%
\end{table}%
%%
%%
%%
\textit{Umsatzvolumen: }Das Histogramm der Abbildung \ref{fig:htobudcat} zeigt eine linksschiefe Verteilung der Umsatzvolumen für die erfolgreiche und nicht erfolgreichen Projekte. Die Mehrheit der Projekte ist in den ersten drei Umsatzvolumenkategorien, konzentriert.
%
%Histogram of TOBud_cat
%
\begin{figure}[H]
	\centering
	\includegraphics[width=5cm]{tocat.pdf}
	\caption{Histogramm der Umsatzkategorie}
	\label{fig:htobudcat}
	\tiny
	\text{Erklärung der Umsatzkategorien}
	\text{1	= [13.2,500), 2 = [500,1e+03), 3 = [1e+03,1.5e+03), 4 = [1.5e+03,2e+03), 5 = [2e+03,2.5e+03)}
	\text{6 =	[2.5e+03,3e+03), 7 = [3e+03,3.5e+03), 8 = [3.5e+03,4e+03), 9 = [4e+03,4.5e+03)}
	\text{10 = [4.5e+03,5e+03), 11 = [5e+03,1e+04), 12 = [1e+04,3.42e+04)}
\end{figure}
%%
%%Relative und Asolute Kostenabweichung
%%
\textit{Absolute und relative Kostenabweichung: } Die Werte der Tabelle \ref{tab:stocostdb1dev} implizieren bereits, dass die durchschnittlich absoluten und relativen Kostenabweichungen bei den gescheiterten Projekten grösser sind. ist als bei den erfolgreichen Projekten.
%%
%%Auswertung Nachlieferung
%%
\newline\newline\textit{Kosten infolge Nachlieferung SUCostTO:} Nachlieferungen verursachen zusätzliche Kosten, die aufgrund der zeitlichen Verzögerung bei der Fabrikation der Maschine oder Installation entstehen können. Die nicht erfolgreichen Projekte hatten durchschnittlich höhere Kosten aus Nachlieferung in Relation zum Umsatz, wie der Tabelle \ref{tab:msu} zu entnehmen ist.
%Mean SU Cost
\begin{table}[H]
	\centering
	\caption{Arithmetisches Mittel der SUCostTO [\%]}
	\begin{tabular}{lr}
		\toprule
		\textbf{Success} & \multicolumn{1}{l}{\textbf{SUCostTO}} \\
		\midrule
		FALSE & -0.81 \\
		TRUE  & -0.36 \\
		\bottomrule
	\end{tabular}%
	\label{tab:msu}%
\end{table}%
%
%Auswertung DeltaLastFCAct
%
\textit{Abweichung zwischen dem letzten FC und Act DeltaLastFCAct:} Tendenziell wird die Anpassung des Forecast für die Projektkosten bei erwarteten Mehrkosten möglichst lange hinausgezögert. Das arithmetische Mittel der Differenzen zwischen dem FC und Act war für gescheitere Projekte vor allem bei IS deutlich höher als bei erfolgreichen Projekten.
%Tabelle Mean DeltaLastFCAct  
\begin{table}[H]
	\centering
	\caption{Arithmetisches Mittel DeltaLastFCAct für MS, ME, PA und IS [TCHF]}
	\begin{tabular}{lrrrrr}
		\textbf{Success} & \multicolumn{1}{c}{\textbf{Total FC}} & \multicolumn{1}{c}{\textbf{MS}} & \multicolumn{1}{c}{\textbf{ME}} & \multicolumn{1}{c}{\textbf{PA}} & \multicolumn{1}{c}{\textbf{IS}}
		\\\hline
		FALSE & -490.54 & -445.53 & 7.48 & -12.87 & -14.52 \\
		TRUE  & -436.41 & -454.24 & 7.93 & -13.41 & -6.49\\
	\end{tabular}%
	\label{tab:mdeltalastfcact}%
\end{table}%
\newpage
%%
%%Zeit
%%
\section{Zeit}
Die Beurteilung des Zeitmanagement hängt von der Einhaltung des vereinbarten Liefertermins ab. Sowohl die gescheiterten als auch die erfolgreichen Projekte hielten den Liefertermin nicht ein, wie in der Tabelle \ref{tab:mtime} zu sehen ist. Der Zeitverzug ist im Durchschnitt für gescheiterte Projekte um zwei Monate höher und entsteht während der Übergabe des Projekts  an den Kunden und dem Projektabschluss
%Table durchschnittlicher PrTimeDelay und pro MS
\begin{table}[H]
	\centering
	\caption{Arithmetisches Mittel der Projektlaufzeit und Zeitverzögerung [Monate]}
	\begin{tabular}{lrrrrrrr}
		\toprule
		\textbf{Success} & \multicolumn{1}{l}{\textbf{Base}} & \multicolumn{1}{l}{\textbf{Act}} & \multicolumn{1}{l}{\textbf{Delay}} & \multicolumn{1}{l}{\textbf{MS2}} & \multicolumn{1}{l}{\textbf{MS8}} & \multicolumn{1}{l}{\textbf{MS10}} & \multicolumn{1}{l}{\textbf{MS11}} \\ 
		\midrule
		TRUE  & 11.9  & 17.3  & -5.4  & -0.1  & -2.0  & -5.0  & -5.5 \\
		FALSE & 11.4  & 18.7  & -7.2  & -0.2  & -1.7  & -5.7  & -7.3 \\
		\bottomrule
	\end{tabular}%
	\label{tab:mtime}%
\end{table}%
%%
%%
%%
Die Tabelle \ref{tab:fdelayms} liegt die Mehrheit der Projekte bei MS2 noch im Zeitplan, während sie nach MS8 hinter dem vereinbarten Liefertermin lag.
%Table Häufigkeit Delay pro MS
\begin{table}[H]
	\centering
	\caption{Absolute Häufigkeitsverteilung zeitverzögerter Projekte pro Milestone}
	\begin{tabular}{lrrrr}
		\toprule
		\textbf{Success} & \multicolumn{1}{l}{\textbf{DelayMS2}} & \multicolumn{1}{l}{\textbf{DelayMS8}} & \multicolumn{1}{l}{\textbf{DelayMS10}} & \multicolumn{1}{l}{\textbf{DelayMS11}} \\
		\midrule
		FALSE & 43    & 209   & 275   & 270 \\
		TRUE  & 75    & 408   & 532   & 516 \\
		\bottomrule
		\textbf{Total} & \textbf{118} & \textbf{717} & \textbf{807} &  \textbf{786}
	\end{tabular}%
	\label{fdelayms}%
\end{table}